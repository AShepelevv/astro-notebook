\newpage
\setcounter{page}{2}
\thispagestyle{empty}
	\noindent ББК 22.6\\
%	\hspace*{1.8pc} 
	A 91\\
	УДК 52\\[1pc]

{\small {\itshape Рецензент:}\\ учитель астрономии М.\,В.~Кузнецов (МОУ Гимназия №1~~г.\,о.~Жуковский)}\\[1pc]

{\small {\itshape Редакторы:}\\ 
выпускник Московского государственного университета А.\,В.~Афанасьев,\\
магистр политехнического института Гренобля, бакалавр Московского физико-тех\-ни\-чес\-кого института В.\,А.~Сушко\\[1pc]}


{%\hspace{1pc} 
{\bfseries А.\,С.~Шепелев, Д.\,А.~Долгов, С.\,Д.~Молчанов, С.\,Б.~Борисов}\\ Астрадь~--- краткий сборник теории по астрономии. 2020.~---~64~с: 2-е изд. ISBN 978-5-9909877-1-5\\[2pc]

{\small Астрадь является учебным пособием по астрономии, рекомендованным для школьников 7\,--\,11 класса. Сборник составлен неоднократными призерами международных олимпиад по астрономии, членами астрономического кружка им.~Е.\,П.~Левитана г.\,о.~Жуковский. 

Здесь читатель сможет найти необходимый минимум теории для участия в различных олимпиадах школьников по астрономии. Также Астрадь можно использовать и для освоения школьной программы, потому что наряду со сложными темами освещены и самые базовые вопросы астрономии.}\\[1pc]

{\small{\itshape Вёрстка:} А.\,С.~Шепелев}\\ \vfill


\noindent\begin{minipage}[t]{0.4\tw}
	{\bfseries ISBN 978-5-9909877-1-5}
\end{minipage}
\hfill
\begin{minipage}[t]{0.57\tw}
	\small
	\copyright\,А.\,С.~Шепелев, Д.\,А.~Долгов,\\
	 \hspace*{8pt} С.\,Д.~Молчанов, С.\,Б.~Борисов, 2018\\[3pt]
	\copyright\,Астрономический кружок\\ \hspace*{8pt} им.~Е.\,П.~Левитана г.\,о.~Жуковский, 2018
\end{minipage}
\newpage