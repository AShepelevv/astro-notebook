\subsection{Энергия излучения}
\term{Энергия излучения}~--- энергия, переносимая излучением ($Q_e$).\\
\term{Поток излучения}~--- физическая величина, характеризующая мощность, переносимую излучением,
\begin{equation}
	\Phi_e = \frac{d Q_e}{dt}.
\end{equation}
\imp{Теорема Гаусса}: через любую замкнутую поверхность потоки от одинаковых источников равны.

\term{Спектральная плотность потока излучения}~--- поток излучения, приходящийся на малый единичный интервал спектра,
\begin{equation}
	\Phi_{e, \lambda}(\lambda) = \frac{d\Phi_e(\lambda)}{d\lambda}, \quad\quad \Phi_{e, \nu}(\nu) = \frac{d\Phi_e(\nu)}{d\nu} =  \frac{\lambda^2}{c}\Phi_{e, \lambda}(\lambda).
\end{equation}

\term{Объемная плотность энергии излучения}~--- количество энергии на единицу объема
\begin{equation}
	U_e = \frac{d Q_e}{dV}.
\end{equation}

\term{Светимость}~--- величина, представляющая собой световой поток излучения, испускаемого с малого участка светящейся поверхности единичной площади,
\begin{equation}
	M_e = \frac{d \Phi_e}{dS_1},
\end{equation}
здесь $S_1$~--- площадь объекта, испускающего энергию.

\term{Яркость}~--- световой поток, приходящийся на единичный телесный угол, в расчёте на единичную площадку проекции излучающей поверхности на картинную плоскость,
\begin{equation}
	L_e = \frac{d^2 \Phi_e}{d \Omega\,dS_1 \cos \varepsilon},
\end{equation}
где $\varepsilon$~--- угол между направлением потока излучения и нормалью к плоскости излучающей поверхности.

\term{Интегральная яркость}~--- интеграл яркости по видимой поверхности источника. Показывает количество энергии, пришедшее от источника за единицу времени.
\begin{equation}
	\Lambda_e = \int \limits_S L_e(\vec{r})\,ds.
\end{equation}
\term{Освещенность}~--- величина, равная отношению светового потока, падающего на малый участок поверхности, к его площади~--- поверхностная плотность потока
\begin{equation}
	E_e = \frac{d\Phi_e}{dS_2} \sim \frac{1}{r^2},
\end{equation}
здесь $S_2$~--- площадь поверхности приёмника, $r$~--- расстояние от источника.
