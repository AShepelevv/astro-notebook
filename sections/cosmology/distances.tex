
\subsection{Расстояния в космологии}
Представим, что невдалеке от нас стоит уличный фонарь. Мы хотим узнать, на каком расстоянии он от нас находится. Как же можно это расстояние измерить?
\begin{enumerate}
\item Проще всего измерить расстояние линейкой или каким-то жестким объектом известной длины.
\item Мы можем определить расстояние по времени, которое требуется свету, чтобы дойти от фонаря до нас.
\item Зная реальный размер самого фонаря, можно измерить расстояние до него по его угловому размеру.
\item Можно измерить плотность потока излучения от фонаря и по ней определить расстояние.
\end{enumerate}
В случае с обычным уличным фонарём все эти способы дадут одинаковый результат, равный расстоянию, измеренному линейкой. Пусть теперь наш уличный фонарь~--- это далёкая галактика, не связанная с нами гравитационно и не имеющая пекулярной скорости. Попробуем определить расстояние до неё предложенными выше способами.

\subsubsection{Сопутствующие координаты}
Представим сетку координат, которая расширяется вместе с Вселенной. В таких координатах любая точка Вселенной, не имеющая пекулярной скорости, неподвижна относительно сетки. Следовательно, разность координат объекта и наблюдателя в данной системе не зависит от времени.

\subsubsection*{Сопутствующее расстояние}
По сути, эта разность координат определяется количеством узлов сетки, расположенных между наблюдателем и объектом. Предположим теперь, что в данный момент времени $t_0$ количество узлов равно $N$, а расстояние между соседними~--- $l(t_0) = l_0.$ Следовательно, расстояние между наблюдателем и объектом равно $D = N l_0.$ Поскольку наша Вселенная масштабируема, мы можем вычислить $l$ для любого другого момента времени $t$: $l(t) = l_0 a(t)$ (вспомните уравнение \eqref{eq}). Напомним, что $a(t_0) \equiv 1.$ 

Если расстояние мажду соседними узлами $l(t)$ устремить к нулю, то $D$ будет вычисляться посредством интегрирования по $dl_0.$ Так как $dl_0 = dl(t) / a(t)$, а приращение расстояния $dl(t) = c dt,$ где $c$~--- скорость света,
\begin{equation}
D \equiv D_{\text{C}} = \int dl_0 = \int \frac{c dt}{a}.
\label{com}
\end{equation}
Величина $D_{\text{C}}$ называется сопутствующим расстоянием. Приведём его к более удобному для вычислений виду. Для этого домножим и разделим подынтегральное выражение на $d \ln(a)$.
$$
D_{\text{C}} = c \int_{t}^{t_0} \frac{dt}{d \ln(a)} \frac{d \ln(a)}{a} = c \int_{a}^{1} \frac{d \ln(a)}{a H(a)} = c \int_{a}^{1} \frac{da}{a^2 H(a)}
$$
Вспомним, что $da = - dz / (1+z)^{2}$ (см. \eqref{dadz}) и $a = (1+z)^{-1}$. При замене масштабного фактора на красное смещение интеграл принимает следующий вид:
$$
-c \int_{z}^{0} \frac{dz}{(1+z)^{2}} \frac{(1+z)^{2}}{H(z)} = c \int_{0}^{z} \frac{dz}{H(z)}
$$
Воспользуемся выводами, полученными в прошлом разделе:
\begin{equation}
D_{\text{C}} = D_{\text{H}_0} \int_{0}^{z} \frac{dz}{E(z)}
\label{Dc}
\end{equation}

Поскольку сопутствующее расстояние~--- это расстояние между наблюдателем и объектом в данный момент времени, то взяв интеграл \eqref{Dc} в пределах  от $z = 0$ до $z = \infty,$ мы получим нынешний размер Вселенной. Будем считать, что относительная плотность тёмной энергии $\Omega_{\Lambda,0} \simeq 0.69,$ а вкладом излучения из-за его малости пренебрежём.
\begin{equation}
R_{0,\text{C}} = D_{\text{H}_0} \int_{0}^{\infty} \frac{dz}{E(z)} \approx 3.24D_{\text{H}_0} \approx 13.9 \text{ Gpc}.
\end{equation}

\subsubsection{Световое расстояние}

%Чтобы линейки не были подвержены расширению Вселенной, они должны быть малой длины. Обозначим её за $l.$ Теперь мы можем вычислить расстояние между соседними узлами: $l N(t) = l (N_0 / a(t))$ %При расстояних, значительно меньших размера Вселенной, оно определяется следующим соотношением:
%\begin{equation}
%D_{\text{C}} \equiv \frac{D(t)}{a(t)} = D_0.
%\end{equation}
%мы измеряем расстояние между двумя соседними узлами с помощью жестких линеек одинаковой длины. Понятно, что их количество, требуемое для измерения расстояния, в разные моменты времени будет разным.
%и, прежде, чем перейти к следующему разделу, разберемся с ещё одним уравнением, позволяющим расчитать время, которое требуется свету, излученному в момент времени $t_\text{e}$, чтобы дойти до наблюдателя. Заметим, что всё, относящееся к моменту испускания обозначается с индексом "e"\text{} (от англ. \textit{emit}), а к момету наблюдения~--- с индексом "0",или "o"\text{} (от англ. \textit{observe}). В англоязычной литературе это время называется \text{} "\imp{lookback time}". Обозначим его как $t_\text{L}$.

Пусть мы наблюдаем некоторый объект на красном смнщнении $z$. Расстояние до него равно $D_{\text{C}}$, однако из-за расширения вселенной оно не будет равно тому расстоянию, которое пройдёт свет от объекта да наблюдателя. Световое расстояние складывается из небольших отрезков $c dt$, то есть, чтобы его найти, нужно вычислить следующий интеграл:

$$
D_\text{L} = \int_{t_\text{e}}^{t_\text{o}} c dt = c \int_{a_\text{e}}^{1} \left(\frac{dt}{d \ln(a)}\right)  d \ln(a)
$$
Так как $d \ln (a) \slash dt = H$, то
\begin{equation}
D_\text{L} = c \int_{a_\text{e}}^{1} \frac{da}{aH(a)} = D_{\text{H}_0} \int_{a_\text{e}}^{1} \frac{da}{a E(a)}.
\end{equation}
Или же, выражая через $z$:
\begin{equation}
D_\text{L} = -c \int_{0}^{z_\text{e}} \left(\frac{dt}{dz}\right) dz = D_{\text{H}_0} \int_{0}^{z_\text{e}} \frac{dz}{(1+z)E(z)}.
\end{equation} 

\subsubsection{Расстояние по угловому размеру}

Предположим, что некоторый небольшой объект размера $l$ располагается перпендикулярно нашему лучу зрения на красном смещении $z$ и имеет наблюдаемый угловой размер $\theta \ll 1.$  
Для него можно ввести расстояние: $D_{\text{A}} = l \slash \theta.$ Выясним, как оно связано сосопутствующим. Пока свет шёл от объекта до нас, Вселенная успела расшириться. 
Объект сохранил свой размер, но стал занимать меньше места. Сопутствующее расстояние будет равно $l_0 \slash \theta,$ где $l_0$~--- разность координат между концами объекта на момент испускания видимого света 
(то есть, на эпоху~$z$). Получается, что $l_0 = l (1 + z),$ а расстояние по угловому размеру выражается через сопутствующее следующим образом:
\begin{equation}
D_{\text{A}} = \frac{D_{\text{C}}}{1 + z}.
\end{equation}
 
\subsubsection{Фотометрические расстояние}
Для изотропного источника с известной светимостью $L$ можно определить фотометрическое расстояние:
$$
F = \frac{L}{4 \pi D_{\text{L}}^2},
$$
где $F$~--- измеренная плотность потока излучения от источника. Однако фотометрическое расстояние не равно сопутствующему.
Если наблюдаемый объект находится на красном смещении $z,$ то длина волны каждого принятого фотона не равна длине волны излученного: $\lambda_0 = \lambda_{\text{e}} (1 + z).$ Следовательно, 
энерия фотона ($E = hc \slash \lambda$) тоже будет меняться: $E_0 = E_{\text{e}} \slash (1 + z),$ как и $F.$ Расстояние между фотонами будет увеличиваться пропорционально $1 + z,$ как и $\lambda.$ Увеличение расстояния 
между фотонами уменьшает частоту их детектирования (в сравнении с частотой испускания), а следовательно, и плотность потока в $1 + z$ раз. Поскольку плотность потока, не подверженная этим изменениям,-- это 
просто $L \slash 4 \pi D_{\text{C}}^2,$ можно связать сопутствующее и фотометрическое расстояния:
$$
\frac{L}{4 \pi D_{\text{L}}^2} = \frac{L}{4 \pi D_{\text{C}}^2} \frac{1}{(1 + z)^2},
$$
из чего следует, что
\begin{equation}
    D_{\text{L}} = D_{\text{C}} (1 + z).
\end{equation}
