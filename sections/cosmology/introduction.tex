Будем считать, что наша Вселенная не содержит каких-либо выделенных областей
 и направлений. Её глобальные характеристики одинаковы во всех точках
 пространства в фиксированный момент времени. Это составляет суть
 \term{космологического принципа}. Он подтверждается наблюдениями: распределение
 материи во Вселенной на масштабах более~$100$\,Mpc можно считать однородным
 и изотропным, а относительные флуктуации температуры реликтового фона не
 превышают $ 10^{-5}$. Также введём следующие обозначения: величины без индексов 
 или с индексом <<e>> (от англ. \textit{emit}) обозначают наблюдаемый объект, а с 
 индексами <<$0$>> или <<o>> (от англ. \textit{observe})~--- наблюдателя.