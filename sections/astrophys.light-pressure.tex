\subsection{Давление излучения}
\term{Давление электромагнитного излучения}, падающего на поверхность тела, в отсутствии рассеяния выражается формулой
\begin{equation}
p = \frac{I}{c} \cdot (1 - k + A) \cdot \cos \beta,
\end{equation}
здесь $I$~--- поток падающего излучения, $c$~--- скорость света, $k$~--- коэффициент пропускания, $A$~--- коэффициент отражения, а $\beta$~--- угол падения излучения.

\term{Давление фотонного газа} определяется соотношением
\begin{equation}
p_\text{ф.г.} = \frac{u}{3} = \frac{4 \sigma T^4}{3c},
\end{equation}
где $u$~--- плотность энергии фотонного газа, $T$~--- температура фотонного газа.

Возможными областями применения являются солнечный парус, а в отдалённом будущем~--- фотонный двигатель.