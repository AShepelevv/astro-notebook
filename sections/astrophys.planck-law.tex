\subsection{Формула Планка}
\label{sec:planck-law}
\term{Формула Планка}~--- выражение для спектральной плотности мощности излучения абсолютно чёрного тела на интервале частот $[\nu, \nu + d \nu)$, распространяющейся с телесном угле $d\Omega$, которое было получено Максом Планком в 1900~году. Данное выражение имеет следующий вид:
\begin{equation}
B_\nu(\nu,T)=\frac{2h\nu^3}{c^2}\cdot \frac{1}{\exp\left(\frac{h\nu}{kT}\right)-1} = \left[ \frac{\text{Вт}}{\text{м}^2 \cdot \text{Гц} \cdot \text{ср}}\right],
\label{eq:plancks-law-nu}
\end{equation}
где $\nu$~--- частота излучения, $T$~--- температура АЧТ, $h$~--- постоянная Планка, $k$~--- постоянная Больцмана, $c$~--- скорость света.

Если записать закон излучения Планка \eqref{eq:plancks-law-nu} для длин волн, то
\begin{equation}
B_\lambda(\lambda,T)=\frac{2hc^2}{\lambda^5} \cdot \frac{1}{\exp\left(\frac{hc}{\lambda kT}\right)-1} = \left[ \frac{\text{Вт}}{\text{м}^3 \cdot \text{ср}}\right].
\label{eq:plancks-law-lambda}
\end{equation}
\begin{wrapfigure}[15]{l}{.6\tw}
\centering
\vspace{-.9pc}
 \begin{tikzpicture}
  \begin{axis}[
  				width 	=	.6\tw, 
				height	=	6cm, 
  				ymax	=	1e+14,
  				xmax	=	2000,
  				xmin	=	0,
  				ymin	=	0,
				xlabel	=	{Длина волны $\lambda$,~нм}, 
				ylabel 	= 	{$B_\lambda(\lambda, T)$,~$\text{Вт} \cdot \text{м}^{-3}$}
]
   \addplot+[dashed, thin, black] table[x=l, y=tl] {data/planck.txt};
   \addplot+[black] table[x=l, y=t4] {data/planck.txt} node at (axis cs:870, 1.6e+13) {\tiny{$4500$~K}};
   \addplot+[black] table[x=l, y=t5] {data/planck.txt}node at (axis cs:750, 4.2e+13) {\tiny{$5000$~K}};
   \addplot+[black] table[x=l, y=t58] {data/planck.txt}node at (axis cs:670, 8.5e+13) {\tiny{$5800$~K}};
   \addplot+[black] table[x=l, y=t7] {data/planck.txt}node at (axis cs:1350, 3.5e+13) {\tiny{$7000$~K}};
	%\addplot+[black, smooth] table[x=l, y=t15] {data/planck.txt} node at (axis cs:1670, 5.5e+13) {\tiny{$15000$~K}};
  \end{axis}
 \end{tikzpicture}
\caption{Кривые спектральной плотности мощности изотропного излучения АЧТ с разной температурой}\label{pic:wien-law}
\end{wrapfigure}
Стоит заметить, что при переходе в функции к длинам волн меняется не только частота на длину волны, но и выражение для интервала. 

Формула Планка появилась, когда стало ясно, что формула Рэлея-Джинса удовлетворительно описывает излучение только в области больших длин волн, а~с~убыванием длин волн даёт сильные расхождения с реальными данными. Однако формулу Рэлея-Джинса используют и сейчас для описания кривой Планка на больших длинах волн. 

\change{
Проделаем обратные действия: получим формулу Рэлея-Джинса из формулы Планка. Длинноволновая часть спектра характеризуется соотношением $h\nu \ll kT$, то есть 
\begin{equation*}
	\exp\left( \frac{h\nu}{kT}\right) \approx 1 + \frac{h\nu}{kT}.
\end{equation*}
Подставляя данное выражение в знаменатель \eqref{eq:plancks-law-nu}, получим
\begin{equation*}
	B_\nu(\nu,T) \approx \frac{2h\nu^3}{c^2}\cdot \frac{1}{1 + \frac{h\nu}{kT} - 1} = \frac{2h\nu^3 }{c^2}\cdot \frac{k T}{ h \nu} = \frac{2 \nu^2 k T}{c^2}.
\end{equation*}
}
\change{
	Проделав те же действия для формулы Планка через длину волны, получим:
}
\begin{equation}
	B(\lambda, T) \simeq \frac{2 c k T}{\lambda^4}, \quad\quad B(\nu, T) \simeq \frac{2 \nu^2 k T}{c^2}.
\label{Ray-Jean}
\end{equation}

\change{
	В коротковолновой области, наоборот, $h \nu \gg kT$, следовательно, единица в знаменателе формулы Планка много меньше стоящей там экспоненты, то есть
	\begin{equation*}
		\frac{1}{\exp\left(\frac{h\nu}{kT}\right)-1} \approx \frac{1}{\exp\left(\frac{h\nu}{kT}\right)} = \exp\left(-\frac{h\nu}{kT}\right).
	\end{equation*} 
	Отсюда получаются приближения, называемые приближениями Вина:
}
\begin{equation}
B ( \lambda, T) \simeq \frac{2 h c^2}{\lambda^5} \exp \left( -\frac{h c}{\lambda k T}\right), \quad B( \nu, T ) \simeq \frac{2 h \nu^3}{c^2} \exp \left( -\frac{h \nu}{k T} \right).
\end{equation}