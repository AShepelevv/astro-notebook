\subsection{Движение по орбите}

\term{Закон сохранения момента импульса}:  векторная сумма моментов 
импульса замкнутой системы тел относительно выбранной оси остается 
постоянной, если суммарный момент $\vec{M}_\Sigma$ внешних сил, действующих на систему, равен нулю. Иначе,
\begin{equation}
\vec{L}_\Sigma \equiv \sum\limits_{i=1}^{n} \left[ \vec{r} \times m \vec{v} \right] = \const.
\end{equation}
Закон сохранения момента импульса справедлив как для движения по эллипсу, так и по гиперболе и параболе. Следствием этого закона и закона сохранения энергии является \imp{интеграл энергии}~--- формула для скорости тела в точке орбиты, удалённой на расстояние~$r$ от центрального тела с массой $M$:
\begin{equation}
v = \sqrt{ GM \left( \frac{2}{r} - \frac{1}{a} \right)}.
\label{eq:int-energy}
\end{equation}
Из \eqref{eq:int-energy} для апоцентра и перицентра можно заключить следующее:
\begin{equation}v_{\text{ап}}=\sqrt{\frac{GM}{a}} \sqrt{\frac{1-e}{1+e}},
\quad\quad\quad v_{\text{пер}}=\sqrt{\frac{GM}{a}}\sqrt{\frac{1+e}{1-e}}.
\end{equation}
Согласно \eqref{eq:int-energy} и \eqref{eq:ellipse-pol-eq} для скорости тела в произвольной точке орбиты справедливо выражение
\begin{equation}
v = \sqrt{\frac{GM}{p}\cdot(1 + 2 e \cos \nu + e^2)},
\end{equation}
где $\nu$~--- истинная аномалия, а $p$~--- фокальный параметр.
