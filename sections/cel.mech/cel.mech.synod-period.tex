\subsection{Синодический период}

\term{Синодический период} --- промежуток времени между двумя 
последовательными одноимёнными конфигурациями планеты или Луны.

\imp{Относительная угловая скорость $\mathit{\omega_s}$} планеты равна 
разности скоростей углового перемещения планеты ($360^\circ/T$) и Земли ($360^\circ/E$) по 
орбите. Из определения относительной угловой скорости вытекает общая формула 
для синодического периода: \begin{equation}
\frac1S=\left| \frac1E-\frac1T \right|
\end{equation}

Для внешних и внутренних планет соответственно выражения принимает следующий вид: \begin{equation} \frac{1}{S} = \frac{1}{E} - \frac{1}{T} \quad \text{и} \quad \frac{1}{S} = \frac{1}{T} - \frac{1}{E},
\end{equation}
где $S$~--- синодический период, $T$~--- сидерический период, $E$~--- 
сидерический период обращения Земли.

В случае, если тело обращается по орбите в противоположную сторону, то связь 
между синодическим и сидерическим периодами тела принимает следующий вид:
\begin{equation}\frac1S=\frac1E+\frac1T
\end{equation}
Иногда синодический период планет или их спутников называют {\itshape периодом смены фаз}. 
