\subsection{Фазы планет и спутников}

\term{Фаза} планеты (спутника)~--- отношение площади освещённой  части видимого диска ко всей его площади.
Фаза рассчитывается по следующей формуле:

\begin{minipage}{0.65\tw}
	\begin{equation}
\Phi = \frac{1 + \cos \phi}{2} = \cos^2 \frac{\phi}{2},
\end{equation}
где $\phi$~--- \term{фазовый угол} --- угол между лучом света, падающим от Солнца на планету, и лучом, отразившимся от неё в сторону наблюдателя (см.~Рис.\,\ref{fig:phase-angel-scheme}). Фаза объекта может принимать значения от 0 до 1.
\end{minipage}
\begin{minipage}{0.35\tw}
	\hfill
	\includegraphics[width = 0.9\tw]{phase-angle}
	\captionof{figure}{Фазовый угол}
	\label{fig:phase-angel-scheme}
\end{minipage}

Видимая границы между освещенной и неосвещенной частями поверхности объекта называется \term{терминатором}. В зоне терминатора для наблюдателя на объекте источник пересекает горизонт.