\subsection{Фазы планет и спутников}

\term{Фаза} планеты (спутника)~--- отношение площади освещённой  части видимого диска ко всей его площади.
Фаза рассчитывается по формуле
\begin{equation}
\Phi = \frac{1 + \cos \phi}{2} = \cos^2 \frac{\phi}{2},
\end{equation}
\begin{minipage}{0.67\tw}
где $\phi$~--- \term{фазовый угол} --- угол между лучом света, падающим от Солнца на планету, и лучом, отразившимся от неё в сторону наблюдателя (см.~Рис.\,\ref{fig:phase-angel-scheme}). Фаза объекта может принимать значения от 0 до 1.

Видимая граница между освещенной и неосвещенной частями поверхности объекта называется \term{терминатором}. В зоне терминатора для наблюдателя на объекте источник пересекает горизонт.
\end{minipage}
\hfill
\begin{minipage}{0.31\tw}
	\hfill
	\vspace{-.5pc}
	\includegraphics[width = \tw]{phase-angle}
	\captionof{figure}{Фазовый угол}
	\label{fig:phase-angel-scheme}
\end{minipage}

