\section{Увеличение и разрешающая способность телескопа}
\textit{Увеличение телескопа} --- это увеличение поля зрения телескопа в определённое количество раз. Увеличение рассчитывается по следующей формуле:
\begin{equation}
\text{Г}=\frac{F}{f}=\frac{D}{d},
\end{equation}
где $F$ --- фокусное расстояние телескопа, $f$ --- фокусное расстояние окуляра, $D$ --- диаметр входного зрачка (телескопа), $d$ --- диаметр выходного зрачка (окуляра).

Также стоит заметить, что диаметры выходного и входного зрачка являются диаметрами пучка света.

Увеличение является \textit{равнозрачковым}, если диаметр выходного зрачка принимается за диаметр зрачка наблюдателя ($d=d_{\text{З}}$):
\begin{equation}
\text{Г}_{\text{Р.З.}}=\frac{D}{d_{\text{З}}},
\end{equation}
где $d_{\text{З}}$ --- диаметр зрачка, обычно в тёмное время суток принимается за 6 мм.

\textit{Разрешающая способность} --- это наименьшее угловое расстояние между двумя объектами, при котором телескоп может различить их раздельно. Разрешние телескопа вычисляется таким образом:
\begin{equation}
\beta=\frac{1.22\lambda}{D},
\end{equation}
где $\lambda$ --- длина волны, при наблюдении глазом $\lambda\approx 550$~нм.