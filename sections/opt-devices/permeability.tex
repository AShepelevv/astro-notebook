\subsection{Проницающая способность. Поле зрения}
Зависимость \textit{поле зрения телескопа} от \textit{поля зрения окуляра} представлена таким образом:
\begin{equation}
W=\frac{\alpha}{\text{Г}},
\end{equation}
где $\text{Г}$ --- увеличение телескопа, $W$ --- поле зрения телескопа, $\alpha$ --- поле зрения окуляра.

Поле зрения телескопа можно вычислить, зная время прохождения звезды через него и склонение звезды:
\begin{equation}
W'=\frac{\tau\cos\delta}{4},
\end{equation}
где $\tau$ --- время прохождения звезды со сколонением $\delta$ через поле зрения.

Зная поле зрения и фокусное расстояние телескопа, можно вычислить размер изображения в фокальной плоскости:
\begin{equation}
r=2F\tg\frac{W}{2}\approx\rho F
\end{equation}
где $\rho$ --- угловой размер объекта, $F$ --- фокусное расстояние телескопа.

Масштаб изображения рассчитывется таким образом:
\begin{equation}
\mu=\frac{\rho}{r}
\end{equation}

\textit{Относительное отверстие} является отношением диаметра телескопа к его фокусному расстоянию:
\begin{equation}
\forall=\frac{D}{F},
\end{equation}
где $D$ --- диаметр объектива телескопа.

\textit{Светосила} равна квадрату относительного отверстия:
\begin{equation}
A=\forall^2=\frac{D^2}{F^2}
\end{equation}

Причём известно, что светосила пропорциональна освещённости, создаваемой объективом в фокальной плоскости:
\begin{equation}
S\sim A
\end{equation}

\textit{Проницающая способность телескопа} --- это предельная звёздная величина объектов, которые могут быть зарегистрированными данным телескопом:
\begin{equation}
m_{max}=5.5^{\text{m}}+2.5\lg{D}+2.5\lg{\text{Г}},
\end{equation}
где $m_{max}$ --- передельная звёздная величина.

Упрощённая формула проницающей способности выглядит так:
\begin{equation}
m_{max}=2.1^m+5\lg{D}
\end{equation}

Световой поток, собираемый объективом:
\begin{equation}
J=E\frac{\pi D^2}{4},
\end{equation}
где $E$ --- освещённость объектива.