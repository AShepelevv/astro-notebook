\subsection{Матрица}
На практике часто оказывается полезным матричное исчисление. \term{Матрицей} размера $m \times n$ над полем $F$ называется $nm$ элементов из $F$, которые удобно представлять в виде 
	\begin{equation}
		\underset{m \times n}{A} = 
		\begin{pmatrix}
			a_{11} & a_{12} & \cdots & a_{1n}\\
			a_{21} & a_{22} & \cdots & a_{2n}\\
			\vdots & \vdots & \ddots & \vdots\\
			a_{m1} & a_{m2} & \cdots & a_{mn}
		\end{pmatrix}.
	\end{equation}
	К матрицам применимы следующие простейшие операции:
	\begin{equation}
	\underset{m \times n}{A} + \underset{m \times n}{B} = 
	\begin{pmatrix}
		a_{11} + b_{11} & a_{12} + b_{12} & \cdots & a_{1n} +  b_{1n}\\
		a_{21} + b_{21} & a_{22} + b_{22}& \cdots & a_{2n} + b_{2n}\\
		\vdots & \vdots & \ddots & \vdots\\
		a_{m1} + b_{m1} & a_{m2} + b_{m2} & \cdots & a_{mn} +  b_{mn}
	\end{pmatrix};
	\end{equation}
	\begin{multline}
		\underset{m \times n}{A} \cdot \underset{n \times m}{B} = 
		\begin{pmatrix}
			a_{11} & a_{12} & \cdots & a_{1n}\\
			a_{21} & a_{22} & \cdots & a_{2n}\\
			\vdots & \vdots & \ddots & \vdots\\
			a_{m1} & a_{m2} & \cdots & a_{mn}
		\end{pmatrix}
		\begin{pmatrix}
			b_{11} & b_{12} & \cdots & b_{1m}\\
			b_{21} & b_{22} & \cdots & b_{2m}\\
			\vdots & \vdots & \ddots & \vdots\\
			b_{n1} & b_{n2} & \cdots & b_{nm}
		\end{pmatrix} = \\
		= \begin{pmatrix}
			\sum\limits_{i=1}^{n} a_{1i} b_{i1} & \sum\limits_{i=1}^{n} a_{1i} b_{i2} & \cdots & \sum\limits_{i=1}^{n} a_{1i} b_{im}\\
			\sum\limits_{i=1}^{n} a_{2i} b_{i1} & \sum\limits_{i=1}^{n} a_{2i} b_{i2} & \cdots & \sum\limits_{i=1}^{n} a_{2i} b_{im}\\
			\vdots & \vdots & \ddots & \vdots\\
			\sum\limits_{i=1}^{n} a_{mi} b_{i1} & \sum\limits_{i=1}^{n} a_{mi} b_{i2} & \cdots & \sum\limits_{i=1}^{n} a_{m	i} b_{im}\\
		\end{pmatrix},
	\end{multline}
	легко видеть, $A B \not = B A$.
	
	\term{Определитель}~--- функция $\det(X):\underset{n \times n}{\text{Mat}} \rightarrow \R$, вычисляемая по формуле:
	\begin{equation}
		\det A = \det
		\begin{pmatrix}
			a_{11} & a_{12} & \cdots & a_{1n}\\
			a_{21} & a_{22} & \cdots & a_{2n}\\
			\vdots & \vdots & \ddots & \vdots\\
			a_{m1} & a_{m2} & \cdots & a_{mn}
		\end{pmatrix} = 
		\sum\limits_{k = 1}^n (-1)^{k+1} a_{hk} \det M_{hk},
		\label{eq:det-def1}
	\end{equation}
	где $M_{hk}$~--- дополнительный минор~--- матрица, полученная из $A$ вычеркиванием $h$-й строки и $k$-го столбца. Данное рекурсивное выражение для определителя матрица $A$ называется разложением по $h$-й строке, аналогично определяется разложение по столбцу. Для формальной полноты определения скажем, что определитель матрицы порядка единицы равен единственному элементу матрицы.
	
	Раскрывая выражение \eqref{eq:det-def1}, получим эквивалентное ему:
	\begin{equation}
		\det A = \sum\limits_{(\alpha_1, \ldots, \alpha_n)} (-1)^{\pi(\alpha_1, \ldots, \alpha_n)} a_{1\alpha_1} a_{2\alpha_2} \ldots a_{(n-1)\alpha_{n-1}} a_{n\alpha_n},	
		\label{eq:det_def2}
	\end{equation}
	где суммирование производится по всевозможным перестановкам индексов $(\alpha_1, \ldots, \alpha_n)$, а $\pi(\alpha_1, \ldots, \alpha_n)$~--- число инверсий в перестановке. Отсюда следует полилинейность определителя. Перегруппировкой членов получим, что суммирование может производиться не только по перестановкам строк, но и по перестановкам столбцов.
	
	Матрицу можно представить, как
	\begin{equation}
		A = \begin{pmatrix}
			\vec{a}_1 & \cdots & \vec{a}_n
		\end{pmatrix},
	\end{equation}
	где $\vec{a}_1, \ldots, \vec{a}_n$~--- вектора размерности $m$. А также можно считать, что 
	\begin{equation}
		A = \begin{pmatrix}
			\vec{a}_1^{\T}\\
			\vdots\\
			\vec{a}_m^{\T}
		\end{pmatrix}
	\end{equation}
	где $\vec{a}_1, \ldots, \vec{a}_n$~--- вектора размерности $n$.
	
	Из выражения \eqref{eq:det_def2} для определителя следует, что 
	\begin{equation}
		\det \begin{pmatrix}
			\vec{a}_1 & \cdots & \vec{a}_i & \vec{a}_{i+1} & \cdots & \vec{a}_n
		\end{pmatrix} = 
	-\det \begin{pmatrix}
			\vec{a}_1 & \cdots & \vec{a}_{i+1} & \vec{a}_i & \cdots & \vec{a}_n
		\end{pmatrix},
	\end{equation}
	так как число инверсий для соответствующей перестановки изменяется на единицу изменяется на единицу. Если же поменять местами не соседние строки, а две произвольные, то знак определителя также изменится на противоположны, так как <<перетаскиваение>> $i$-ой строки по $i+1, \ldots, j - 1$ строчкам дает столько же инверсий, сколько <<перетаскивание>> $j$-ой по этим же строчкам, плюс еще одна инверсия при смене порядка $i$-ой и $j$-ой строк.
	
	Отсюда вытекает еще одно свойство: если две строки (столбца) матрицы равны, то определитель равен нулю. Проверим это, при перестановке равных строк местами матрица не изменяется, но про предыдущему свойству изменяется знак определителя, то есть $\det A = - \det A$, значит, $\det A = 0$.
	
	Из линейности по каждой строке и каждому столбцу вытекает, что общий множитель элементов какой-либо строки (столбца) определителя можно вынести за знак определителя. Следовательно, если хотя бы одна строка (столбец) матрицы нулевая, то определитель равен нулю.

	Если две (или несколько) строки (столбца) матрицы линейно зависимы, то её определитель равен нулю. Достаточно разложить одну из строк (столбцов) по остальным, воспользоваться линейность, а потом вынести множитель, получим две одинаковые строки, то есть определитель каждой такой матрицы будет равен нулю.
	
	Из последнего вытекает утверждение, что при добавлении к любой строке (столбцу) линейной комбинации других строк (столбцов) определитель не изменится.
	
	Встает вопрос, как можно применить полученные знания об определителе квадратной матрицы? Очень просто! Составив матрицу из $n$ векторов в пространстве с размерностью $n$ и посчитав ее определитель, можно выяснить, является ли система из данных векторов линейно зависимой.