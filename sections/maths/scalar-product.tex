\subsection{Скалярное произведение}
\term{Скалярным произведением} двух векторов называется билинейная операция над ними, зависящая только от длин этих векторов и угла между ними, результатом которой является скаляр. Скалярное произведение векторов $\vec{a}$ и  $\vec{b}$ выражается следующим образом:
\begin{equation}
	\scalar{a}{b} = |\vec{a}||\vec{b}| \cos \widehat{\vec{a}\vec{b}}. \label{eq:scalar-prod1}
\end{equation}
\begin{equation}
	\scalar{a}{b} = \vec{a}^{\T} \vec{b} =
	\begin{pmatrix}
		a_1 & \cdots & a_n
	\end{pmatrix}
	\begin{pmatrix}
		b_1\\
		\vdots\\
		b_n
	\end{pmatrix}
	= a_1 b_1 + \ldots + a_n b_n.
	\label{eq:scalar-prod2}
\end{equation}

Докажем эквивалентность \eqref{eq:scalar-prod1} и \eqref{eq:scalar-prod2}. Пусть в ортонормированном базисе $\{\vec{e}_1, \ldots, \vec{e}_2\}$ векторы имеют следующие представления:
\begin{equation}
	\vec{a} = \sum\limits_{i = 1}^n a_i \vec{e}_i, \qquad \vec{b} = \sum\limits_{i = 1}^n b_i \vec{e}_i.
\end{equation}
Заметим, что $(\vec{a} \cdot \vec{e}_i) = |\vec{a}||\vec{e}_i| \cos \theta_i = |\vec{a}| \cos \theta_i = a_i$, где $\theta_i$~--- угол вектора $\vec{a}$ с $i$-м базисным вектором $\vec{e}_i$. Тогда
\begin{equation}
	\scalar{a}{b} = \left( \vec{a} \cdot \sum\limits_{i=1}^n b_i\vec{e}_i \right) = \sum\limits_{i=1}^n b_i(\vec{a} \cdot \vec{e}_i) = \sum\limits_{i=1}^n a_i b_i = a^{\T}b.
\end{equation}

Из \eqref{eq:scalar-prod1} и \eqref{eq:scalar-prod2} очевидна билинейность и симметричность скалярного произведение, то есть
\begin{equation}
	\scalar{(a + b)}{(c + d)} = \scalar{(c + d)}{(a + b)} = \scalar{a}{c} + \scalar{a}{d} + \scalar{b}{c} + \scalar{b}{d}.
\end{equation}

Практическое пременение скалярное произведение находит в вопросах проверки ортогональности векторов (как частный случай нахождения угла между векторами), потому что $\vec{a} \perp \vec{b}$ тогда и только тогда, когда $\scalar{a}{b} = 0$, так как
\begin{equation}
	\cos \widehat{\vec{a}\vec{b}} = \frac{\scalar{a}{b}}{|\vec{a}||\vec{b}|}.
\end{equation}

Отсюда также получается выражение для проекции вектора $\vec{a}$ на прямую с направляющим вектором $\vec{l}$:
\begin{equation}
	\pr_\vec{l} \vec{a} = \frac{\scalar{a}{l}}{|\vec{l}|^2} \vec{l}.
\end{equation}

Используя скалярное произведение, получим важную утверждение теоремы косинусов из планиметрии: пусть $\vec{c} = \vec{b} - \vec{a}$, то есть имеем треугольник со сторонами $|\vec{a}|$, $|\vec{b}|$ и $|\vec{c}|$. Рассмотрим скалярное произведение вектора $\vec{c}$ самого на себя:
\begin{multline}
	\scalar{c}{c} = \scalar{(b - a)}{(b - a)} = \scalar{b}{(b-a)} - \scalar{a}{(b - a)} = \\
	= \scalar{b}{b} - 2\scalar{b}{a} + \scalar{a}{a}
\end{multline}
\begin{equation}
	c^2 = b^2 + a^2 - 2ab\cos \widehat{\vec{a}\vec{b}}
\end{equation}
