\subsection{Производная}

\term{Производная} $f'(x_0)$ функции $f$ в точке~$x_0$~--- предел отношения приращения функции $f(x_0 + \Delta x) - f(x_0)$ к приращению её аргумента $\Delta x$ при $\Delta x \rightarrow 0$, если такой предел существует. Иначе, 
\begin{equation}
	f'(x_0) = \lim_{\Delta x \to 0}\frac{f(x_0 + \Delta x) - f(x_0)}{\Delta x}
\end{equation}

\begin{wrapfigure}[10]{r}{0.45\tw}
	\centering
	\vspace{-1.3pc}
	\centering
	\begin{tikzpicture}[yscale=0.97]
		\def\MIN{0.3}
		\def\MAX{3}
		
		\def\xo{1.4}
		\def\dx{2.3}
		
		\def\yo{1.2}
		\def\dy{2.5}
		
        \def\al{45}
		
		\tkzDefPoint(0,0){O}
		
		\tkzDefPoint(-\MIN,0){X'}
		\tkzDefPoint(\MAX,0){X}
		\tkzDefPoint(\xo,0){X0}
		\tkzDefPoint(\dx,0){DX}
		
		\tkzDefPoint(0,-\MIN){Y'}
		\tkzDefPoint(0,\MAX){Y}
		\tkzDefPoint(0,\yo){Y0}
		\tkzDefPoint(0,\dy){DY}
		
		\tkzDefPointBy[translation=from O to Y0](X0) \tkzGetPoint{X0Y0}
		\tkzDefPointBy[translation=from O to Y0](DX) \tkzGetPoint{DXY0}
		\tkzDefPointBy[translation=from O to DY](X0) \tkzGetPoint{X0DY}
		\tkzDefPointBy[translation=from O to DY](DX) \tkzGetPoint{DXDY}
		
		
		\tkzDefPointBy[rotation=center X0Y0 angle \al](DXY0) \tkzGetPoint{x}
		\tkzInterLL(DX,DXDY)(X0Y0,x) \tkzGetPoint{Df}
		
		\tkzDrawSegments[semithick, -latex](X',X Y',Y)	
		
		\draw [thick] (-0.3, 0.5) .. controls (2, 0.5) and (2, 3) .. (\MAX, \MAX);
		
		\tkzDrawLines(X0Y0,Df)
		\draw [dashes] (Y0) -- (DXY0) coordinate[pos=1.2](x) -- (x);
		\draw [dashes] (DY) -- (DXDY) coordinate[pos=1.2](x) -- (x);
		\draw [dashes] (X0) -- (X0DY) coordinate[pos=1.15](x) -- (x);
		\draw [dashes] (DX) -- (DXDY) coordinate[pos=1.15](x) -- (x);
		
		\tkzDrawSegment[latex-latex](DXY0,Df)
		
		\tkzMarkAngle[size=0.3](DXY0,X0Y0,Df)
		\tkzLabelAngle[pos=0.5](DXY0,X0Y0,Df){\footnotesize$\alpha$}
		
		\tkzLabelPoint[above](X){$f(x)$};
		\tkzLabelPoint[right](Y){$f(x)$};
		\tkzLabelPoint[below](X0){$x_0$};
		\tkzLabelPoint[below](DX){$x + \Delta x$};
		\tkzLabelPoint[left](Y0){$f(x_0)$};
		\tkzLabelPoint[left](DY){$f(x + \Delta x)$};
		
		\tkzLabelSegment[right](DXY0,Df){$df(x_0)$};
		
		\tkzDrawPoints(X0, DX, Y0, DY, X0Y0, DXDY, DXY0, Df)
	\end{tikzpicture}
	\caption{}
	\label{pic:math-div}
\end{wrapfigure}
Значение производной в точке равно тангенсу угла наклона касательной к графику функции в этой точке. Следовательно, точки, где производная обнуляется, являются локальными минимумами и максимумами функции.

Общепринятые обозначения для производной функции $f$ в точке $x_0$:
\begin{equation}
	f'(x_0) = f'_x(x_0) = D f(x_0) = \frac{d f}{d x}(x_0) = \dot{f} (x_0).
\end{equation}

Используя определение и операции с пределами, несложно получить следующие правила дифференцирования:\\[-0.5pc]
\begin{minipage}{0.5\textwidth}
	\begin{align*}
		(f+g)' &= f' + g';\\
		(Cf)' &= Cf';\\
		(fg)' &= f' g + f g';
	\end{align*}
\end{minipage}
\begin{minipage}{0.5\textwidth}
	\begin{align*}
		\left(\dfrac{f}{g}\right)' &= \dfrac{f' g - f g'}{g^2};\\
		\dfrac{d}{dx}f\bigl(g(x)\bigr) &= \dfrac{df(g)}{dg}\dfrac{dg(x)}{dx}.
	\end{align*}
\end{minipage}\\[0.5pc]
И таблицу производных наиболее распространенных функций:
\begin{gather*}
    \begin{aligned}
	   (x^a)' &= a x^{a-1};
	   & (\cos x)' &= - \sin x;
	   & (\arccos x)' &= - \dfrac{1}{\sqrt{1 - x^2}};\\
	   (a^x)' &= a^x \ln a;
	   & (\log_a x)' &= \dfrac{1}{x \ln a};
	   & (\arctg x)' &= \dfrac{1}{1 + x^2};\\
	   (\sin x)' &= \cos x;
	   & (\arcsin x)' &= \dfrac{1}{\sqrt{1 - x^2}};
	   &  (\arcctg x)' &= - \dfrac{1}{1 + x^2};\\
    \end{aligned}\\
    \scalar{a}{b}' = \scalar{a'}{b} + \scalar{a}{b'};\qquad
	\cross{a}{b}' = \cross{a'}{b} + \cross{a}{b'}; \\
	\triple{a}{b}{c}' = \triple{a'}{b}{c} + \triple{a}{b'}{c} + \triple{a}{b}{c'}.
\end{gather*}
