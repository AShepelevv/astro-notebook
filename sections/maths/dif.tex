\subsection{Производная}

\begin{wrapfigure}[11]{r}{0.46\tw}
	\centering
	\vspace{-1.1pc}
	\centering
	\begin{tikzpicture}[scale=1.1]
		\footnotesize
		
%		\foreach \x in {-0.5, -0.4,...,3} {
%			\draw [line width=.1pt] (\x, -0.5) -- (\x, 3);
%		};
%		
%		\foreach \x in {0, 1,...,3} {
%			\draw [line width=.4pt] (\x , -0.5) -- (\x , 3);
%		};
%		
%		\foreach \y in {0, 1,...,3} {
%			\draw [line width=.4pt] (-0.5, \y) -- (3, \y);
%		};
%		
%		\foreach \y in {-0.5, -0.4,...,3} {
%			\draw [line width=.1pt] (-0.5, \y) -- (3, \y);
%		};
		
		\draw[semithick, -latex] (-0.5, 0) -- (3, 0); 		
		\draw[semithick, -latex] (0, -0.5) -- (0, 3); 	
		
		\draw [thick] (-0.3, 0.5) .. controls (2, 0.5) and (2, 3) .. (3, 3);	
		
		\draw (0.5, 0.28) -- (2.5, 2.28);
		\draw [dashes] (0, 1.2) -- (2.6, 1.2);
		\draw [dashes] (0, 2.5) -- (2.6, 2.5);
		\draw [dashes] (1.4, 0) -- (1.4, 1.6);
		\draw [dashes] (2.3, 0) -- (2.3, 2.9);
		\draw [latex-latex] (2.3, 1.2) -- (2.3, 2.08);
		
		\draw (1.8, 1.2) arc(0:43:0.4);
		
		\draw (1.8, 1.4) node[anchor=west]{$\alpha$};
		\draw (3, 0) node[anchor=south]{$x$};
		\draw (0, 3) node[anchor=west]{$f(x)$};
		\draw (1.4, -0.06) node[anchor=north]{$x_0$};
		\draw (2.3, 0) node[anchor=north]{$x_0 + \Delta x$};
		\draw (0, 1.2) node[anchor=east]{$f(x_0)$};
		\draw (0, 2.5) node[anchor=east]{$f(x + \Delta x)$};
		\draw (2.3, 1.6) node[anchor=west]{$df(x_0)$};
		
		
		\point (1.4, 1.2);
		\point (2.3, 1.2);
		\point (2.3, 2.5);
		\point (1.4, 0);
		\point (2.3, 0);
		\point (2.3, 2.08);
		\point (0, 1.2);
		\point (0, 2.5);
	\end{tikzpicture}
	\caption{}
	\label{pic:math-div}
\end{wrapfigure}
\term{Производная в точке}~--- предел отношения приращения функции к приращению её аргумента при стремлении приращения аргумента к нулю, если такой предел существует. \imp{Геометрический смысл производной}: значение производной в точке численно равно тангенсу угла наклона касательной к графику функции в данной точке. Следовательно, точки, где производная обнуляется, соответствуют локальным минимумам и максимумам функции.
\begin{equation}
	f^\prime(x_0) = \lim_{\Delta x \to 0}\frac{f(x_0 + \Delta x) - f(x_0)}{\Delta x}
\end{equation}
Общепринятые обозначения для производной функции $f(x)$ в точке $x_0$:
\begin{equation}
	f^\prime(x_0) = f^\prime_x(x_0) = D f(x_0) = \frac{d f}{d x}(x_0) = \dot{f} (x_0).
\end{equation}
Правила дифференцирования:\\[-0.5pc]
\begin{minipage}{0.5\textwidth}
	\begin{align*}
		(f+g)^\prime &= f^\prime + g^\prime;\\
		(Cf)^\prime &= Cf^\prime;\\
		(fg)^\prime &= f^\prime g + f g^\prime;
	\end{align*}
\end{minipage}
\begin{minipage}{0.5\textwidth}
	\begin{align*}
		\left(\dfrac{f}{g}\right)^\prime &= \dfrac{f^\prime g - f g^\prime}{g^2};\\
		\dfrac{d}{dx}f\bigl(g(x)\bigr) &= \dfrac{df(g)}{dg}\dfrac{dg(x)}{dx}.
	\end{align*}
\end{minipage}\\[0.5pc]
Таблица производных:
\begin{align*}
	(x^a)^\prime &= a x^{a-1};
	& (\cos x)^\prime &= - \sin x;
	& (\arccos x)^\prime &= - \dfrac{1}{\sqrt{1 - x^2}};\\
	(a^x)^\prime &= a^x \ln a;
	& (\log_a x)^\prime &= \dfrac{1}{x \ln a};
	& (\arctg x)^\prime &= \dfrac{1}{1 + x^2};\\
	(\sin x)^\prime &= \cos x;
	& (\arcsin x)^\prime &=
	\dfrac{1}{\sqrt{1 - x^2}};
	&  (\arcctg x)^\prime &= - \dfrac{1}{1 + x^2}.
\end{align*}
