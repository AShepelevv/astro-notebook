\subsection{Гармонические колебания}
\term{Гармонические колебания}~--- колебания, при которых физическая величина изменяется с течением времени по гармоническому закону. Уравнение гармонических колебаний представляет собой дифференциальное уравнение второго порядка, вида
\begin{equation}
	\ddot{x} + \omega^2 x = 0.
\end{equation}
Его общее решение имеет вид
\begin{equation}
	x(t) = C_1 \sin \omega t + C_2 \cos \omega t = C_0 \sin (\omega t + \varphi_0),
\end{equation}
где~$C_0$, $C_1$~и~$C_2$~--- некоторые константы, определяющие амплитуду колебаний,~а~$\varphi_0$~--- начальная фаза колебаний. Отсюда видно, что для периода колебаний справедливо соотношение
\begin{equation}
	T = \frac{2 \pi}{\omega}.
\end{equation}
Гармонические колебания материальной точки совершаются под действием сил, пропорциональных смещению колеблющейся точки от положения равновесия и направленной противоположно этому смещению. Примерами гармонических колебаний могут служить математический и пружинный маятники.