\subsection{Телесный угол}
\label{subsec:solid-angle}

\begin{wrapfigure}{r}{0.46\tw}
	\centering
	\vspace{-1.1pc}
	\centering
	\begin{tikzpicture}[scale=1.1]
		\footnotesize
%		\clip(-2.1, -2.1) rectangle (2.1, 2.1);
%		
%		\foreach \x in {-5, -4.9,...,5} {
%			\draw [line width=.1pt] (\x, -5) -- (\x, 5);
%		};
%		
%		\foreach \x in {-5, -4,..., 5} {
%			\draw [line width=.4pt] (\x , -5) -- (\x , 5);
%		};
%		
%		\foreach \y in {-5, -4,..., 5} {
%			\draw [line width=.4pt] (-5, \y) -- (5, \y);
%		};
%		
%		\foreach \y in {-5, -4.9,..., 5} {
%			\draw [line width=.1pt] (-5, \y) -- (5, \y);
%		};

		
		\draw [semithick] (0, 0) circle (2);
		\draw [semithick] (-2, 0) arc(-180:0:2 and 0.5);
		\draw [semithick, dashes] (-2, 0) arc(180:0:2 and 0.5);
		
		\draw [top color=white!80!black, bottom color=white!50!black] (0, 0.8) .. controls +(270:0.1) 
		and ++(0:-0.1) .. (0.2, 0.6) .. controls ++(0:0.2) 
		and ++(0:-0.2) .. (0.6, 0.8) .. controls ++(0:0.1) 
		and ++(0:-0.1) .. (0.9, 0.6) .. controls +(0:0.1) 
		and ++(270:0.2) .. (1.1, 0.8) .. controls +(90:0.2) 
		and ++(0:0.1) .. (0.9, 1.2) .. controls +(180:0.1) 
		and ++(0:0.1) .. (0.6, 1.1) .. controls +(180:0.1) 
		and ++(0:0.1) .. (0.3, 1.4) .. controls +(180:0.2) 
		and ++(90:0.2) .. (0, 0.8) -- cycle;

		
		\draw [-latex] (0, 0) -- (-2, 0);
		\draw [top color=white!80!black, bottom color=white!50!black] (0, 0.8) .. controls +(270:0.1) 
		and ++(0:-0.1) .. (0.2, 0.6) .. controls ++(0:0.2) 
		and ++(0:-0.2) .. (0.6, 0.8) .. controls ++(0:0.1) 
		and ++(0:-0.1) .. (0.9, 0.6) .. controls ++(0:0.1)
		and ++(225:0.01) ..(1.03, 0.62) -- (0, 0) -- cycle;

		
		\draw (0.7, 0.95) node{$S$};
		\draw (-1, 0) node[anchor=south]{$R$};
		\draw (0, 0) node[anchor=north west]{$O$};
		\draw [fill=white] (0, 0) circle(0.03);
		
	\end{tikzpicture}
	\caption{}
	\label{pic:math-solid-angle}
\end{wrapfigure}
\term{Телесный угол}~--- часть пространства, которая является объединением всех лучей, выходящих из данной точки (вершины угла) и пересекающих некоторую поверхность (которая называется поверхностью, стягивающей данный телесный угол). Телесный угол измеряется в стерадианах и равен отношению площади сферы, которую вырезает данный угол, к квадрату радиуса данной сферы.
\begin{equation}
	\Omega = \frac{S}{R^2}
\end{equation}
Телесный угол полной сферы равен $4\pi$. Величину телесного угла, образованного конусом с углом раствора $\alpha$ можно определить по формуле
\begin{equation}
	\Omega = 2 \pi \left(1 - \cos \frac{\alpha}{2}\right).
\end{equation}

Телесный угол, соответствующий сегменту высоты~$h$ сферы радиуса~$R$, равен
\begin{equation}
	\Omega = 2 \pi \left(1 - \dfrac{h}{\sqrt{R^2 + h^2}}\right).
\end{equation}

Для сферического треугольника с углами $\alpha$, $\beta$ и $\gamma$ справедливо соотношение
\begin{equation}
	\Omega = \alpha + \beta + \gamma - \pi. \label{eq:solid-angle-sphere-tri}
\end{equation}

Докажем формулу~\eqref{eq:solid-angle-sphere-tri}. Для этого сначала определим формулу для телесного угла \imp{двуугольника}~--- одной из четырёх частей сферы, образованных делением последней двумя различными большими кругами. Как легко догадаться из названия, у двуугольника два угла, равных между собой в силу свойств больших кругов сферы, также отсюда следует, что углы двуугольника лежат в противоположных точках сферы. Иначе, двуугольник~--- это <<долька>> сферы. 

Из определения очевидно, что площадь двуугольника это $\alpha/ 2\pi$ площади сферы, которой он принадлежит, здесь $\alpha$~--- величина углов двуугольника. Следовальтельно, телесный угол двуугольника с углом $\alpha$ равен \begin{equation*}
     \Omega = \Omega_\text{сф} \cdot \frac{\alpha}{2 \pi} = 2 \alpha.
 \end{equation*}
 
 Заметим, что сферический треугольник с углами $\alpha$, $\beta$ и $\gamma$ образован пересечением трёх двуугольников с углами, соответственно, $\alpha$, $\beta$ и $\gamma$. При этом вертикальные, по углами, двуугольники этим двуугольникам в пересечении образуют равный исходному сферический треугольник. 
 
 Можно заметить, что все шесть двуугольников полностью покрывают сферу, пересекаясь лишь на двух равных сферических треугольниках. Учитывая тройной счёт сферических треугольников, запишем выражение для телесного угла сферы:
 \begin{equation*}
     4\pi = 2 \cdot (2 \alpha +2 \beta + 2 \gamma) - 2 \cdot 2 \Omega_\triangle \quad \Rightarrow \quad \Omega_\triangle = \alpha + \beta + \gamma - \pi.
 \end{equation*}













