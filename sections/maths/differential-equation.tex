\subsection{Дифференциальные уравнения}
Под \term{дифференциальными уравнениями} понимаются любые алгебраические или трансцендентные нетривиальные равенства, содержащие дифференциалы или производные. Существует большое число различных видов дифференциальных уравнений, изучение которых занимает не один семестр университетской программы. В рамках текущего повествования рассмотрим только два простейших из них для создания общего представления о дифференциальных уравнениях.

\subsubsection{Уравнения с разделяющимися переменным}

Рассмотрим уравнение $ P(t,x)\,dt + Q(t,x)\,dx = 0$, где $P$ и $Q$~--- некоторые непрерывные в некоторой области $\Omega \subseteq \R^2$ функции. Если их можно представить в виде $P(t,x) = T_p(t)X_p(x)$ и $Q(t,x) = T_q(t)X_q(x)$, то это уравнение с \term{разделяющимися переменными}. Его можно записать как
\begin{equation*}
    T_p(t)X_p(x)\,dt + T_q(t)X_q(x)\,dx = 0.
\end{equation*}

Отдельно необходимо рассмотреть случаи, когда $X_p(x) = 0$ или $T_q(t) = 0$, а общее решение можно найти, разделив обе части уравнения на $X_p(x)T_q(t)$. В таком случае уравнение примет вид
\begin{equation*}
    \frac{T_p(t)}{T_q(t)} \, dt = - \frac{X_q(x)}{X_p(x)} \, dx,
\end{equation*}
интегрируя левую и правую часть, получим,
\begin{equation*}
     \int \frac{T_p(t)}{T_q(t)} \, dt = - \int \frac{X_q(x)}{X_p(x)} \, dx + C,
\end{equation*}
где константа интегрирования $C$ определяется из начальных условий. Примеры использования и решения таких уравнений можно найти в разделах~\ref{sec:gravity-law} и ~\ref{sec:earth-atmosphere}.

\subsubsection{Уравнения гармонического консервативного осциллятора}

\term{Гармонические колебания} материальной точки совершаются под действием сил, пропорциональных смещению колеблющейся точки от положения равновесия и направленной противоположно этому смещению. Примерами гармонических колебаний могут служить математический и пружинный маятники.

Рассмотрим материальную точку массы $m$ и возвращающую силу $F = -kx$, где $x$~--- смещение материальной точки от точки равновесия, где сила $F$ на неё не действует, а $k$~--- некая постоянная, определяющая величину силы $F$.

Тогда по второму закону Ньютона ускорение $a$ материальной точки можно записать как
\begin{equation*}
    \frac{d^2 x}{dt^2} \equiv a = - \frac{k}{m} x.
\end{equation*}
Обозначим $k/m$ как $\omega^2$, а вторую производную смещения по времени как $\ddot x$, получим уравнение
\begin{equation}
    \ddot x + \omega^2 x = 0.
\end{equation}

Это уравнение описывает поведение консервативного гармонического осциллятора (\imp{гармонических колебаний} без затухания). Его общее решение имеет вид
\begin{equation}
	x(t) = C_1 \sin \omega t + C_2 \cos \omega t = C_0 \sin (\omega t + \varphi_0),
	\label{eq:harmonic-oscillation}
\end{equation}
где~$C_0$, $C_1$~и~$C_2$~--- некоторые константы, определяющие амплитуду колебаний,~а~$\varphi_0$~--- начальная фаза колебаний. Все параметры колебаний устанавливаются из начальных условий. 

Из~\eqref{eq:harmonic-oscillation} видно, что для периода колебаний справедливо выражение
\begin{equation}
	T = \frac{2 \pi}{\omega}.
\end{equation}
Величину $\omega$ называют круговой частотой, так как измеряется она в радианах в секунду. Частота $f$ в Гц связана в круговой частотой соотношением $\omega = 2 \pi f$.




