\subsection{Формулы приближенного вычисления}
\label{sec:form}

Пусть функция $f: \R \rightarrow \R$ имеет производную $n$-ого порядка в точке~$x_0$~--- $f^{(n)}(x_0) \in \R$, тогда равенство
\begin{equation}
    f(x) = \sum\limits_{k=0}^{n} \frac{f^{(k)}(x_0)}{k!} (x - x_0)^k + r_n(x)
    \label{eq:taylor-formula}
\end{equation}
называется \term{формулой Тейлора} функции $f$ в точке $x_0$. При этом первое слагаемое называется \imp{многочленом Тейлора}, а второе~--- \imp{остаточным членом}.

Докажем, что остаточный член $r_n(x) = o\big( (x - x_0)^n \big)$ при $x \rightarrow x_0$.\footnote{$g(x - x_0) = o(h(x - x_0))$ при $x \rightarrow x_0$ означает, что $\lim\limits_{x \rightarrow x_0} g(x - x_0) / h(x - x_0) = 0$.} Пусть существует $f^{(n)} (x_0) \in \R$, пусть также $P_n(x)$~--- многочлен Тейлора функции $f$. Тогда $P^{(k)}(x_0) = f^{(k)}(x_0), ~ 0 \leqslant k \leqslant n$.\footnote{Подробное доказательство этого факта оставим читателю} Поэтому для остаточного члена $r_n(x) = f(x) - P_n(x)$ выполняется $r_n(x_0) = r'_n(x_0) =$ $= \ldots = r_n^{(n)}(x_0) = 0$. Используя правило Лопиталя, получим
\begin{equation*}
    \lim_{x \rightarrow x_0} \frac{r_n(x)}{(x - x_0)^n}
        = \lim_{x \rightarrow x_0} \frac{r'_n(x)}{n (x - x_0)^{n-1}}
        = \ldots
        = \lim_{x \rightarrow x_0} \frac{r^{(n-1)}_n(x)}{n!(x - x_0)},
\end{equation*}
а по определению производной в точке, последнее выражение равно $r^{(n)}(x_0) / n! = 0$. Следовательно, $r_n(x) = o\big( (x - x_0)^n \big)$ при $x \rightarrow x_0$.

Если функция $f$ имеет в точке $x_0$ производные всех порядков, то
\begin{equation}
    f(x) = \sum\limits_{k=0}^{\infty} \frac{f^{(k)}(x_0)}{k!} (x - x_0)^k
    \label{eq:taylor-series}
\end{equation}
называется \term{рядом Тейлора} функции $f$ в точке $x_0$. При $x_0 = 0$ этот ряд также называют \imp{рядом Маклорена}.

В реальных задачах астрономии и физики редко возникает необходимость в использовании ряда Тейлора~\eqref{eq:taylor-series}. Наиболее применима формула Тейлора~\eqref{eq:taylor-formula} при $x_0 = 0$. Откуда можно получить следующие формулы приближенного вычисления при $x \ll 1$:
\begin{align*}
    \sin x &\simeq x - \frac{x^3}{6} \simeq x, &
    \cos x &\simeq 1 - \frac{x^2}{2} + \frac{x^4}{4!} \simeq 1 - \frac{x^2}{2}, \\
    \tg x &\simeq x + \frac{x^3}{3} \simeq x, &
    \ln(1+x) &\simeq x - \frac{x^2}{2} + \frac{x^3}{3} \simeq x, \\
    (1 + x)^\alpha &\simeq 1 + \alpha x, &
    e^x &\simeq 1 + x + \frac{x^2}{2} \simeq 1 + x, \\
    \sin (\theta + x) &\simeq \sin \theta + x \cos \theta, &
    \cos(\theta + x) &\simeq \cos \theta - x \sin \theta.
\end{align*}
\begin{wrapfigure}{r}{0.47\tw}
    \centering
    \vspace{-1pc}
    \tikzsetnextfilename{math-approx}
    \begin{tikzpicture}
        \begin{axis}[
            width    =    .5\tw,
            height    =    4.5cm,
            xlabel    =    {$x$},
            ylabel    =    {$y$},
            ymax    =    1,
            ymin    =    0,
            xmax    =    1.571,
            xmin    =    0,
            legend cell align=left,
            legend style={
                row sep = 1mm,
                draw=none,
                fill=none,
                font=\scriptsize,
                at={(axis cs:1.5, 0.1)}, anchor=south east,
            },
        ]
            \addplot [domain=0:2, samples=100] {sin(deg(x))};
            \addplot [domain=0:2, samples=100, dashed] {x};
            \addplot [domain=0:2, samples=100, dotted] {x - (x^3)/6};
            \legend{
                $\sin(x)$,
                $x$,
                $x - x^3\!/6$
            }
        \end{axis}
    \end{tikzpicture}
    \caption{}
    \label{pic:math-approx}
\end{wrapfigure}
При необходимости более точного приближения всегда можно выписать следующие несколько членов многочлена Тейлора. 

По аналогии, последовательно применяя формулу Тейлора и принимая различные переменные за аргумент функции, можно получить формулу для нескольких малых переменных:
\begin{equation*}
    (1 + a)^\alpha (1 + b)^\beta \ldots \simeq 1 + \alpha a + \beta b + \ldots
\end{equation*}
