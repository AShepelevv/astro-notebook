\subsection{Синодический период}

\textbf{Синодический период} --- промежуток времени между двумя последовательными одноимёнными конфигурациями планеты или Луны. Для Луны можно ввести другое определение --- промежуток времени между двумя последовательными одинаковыми фазами.

$S$ --- синодический период.

$T$ --- сидерический период.

$E$ --- сидерический период обращения Земли.

Для внешних планет:
$$\frac1S=\frac1E-\frac1T$$
Для внутренних планет:
$$\frac1S=\frac1T-\frac1E$$

В случае, если тело обращается по орбите в протвоположную сторону, то связь между синодическим и сидерическим периодами тела выглядит так:
$$\frac1S=\frac1E+\frac1T$$