\subsection{Фотометрия. Показатель цвета}
\term{Фотометрическая система UBV}~--- система, разработанная для классификации звезд в зависимости от их цвета. В этой системе измеряются звездные величины звезд на трех участках спектра: $U$ (максимум чувствительности $\lambda_\text{макс}^U = 350$~нм), $B$ ($\lambda_\text{макс}^B = 430$~нм) и $V$ ($\lambda_\text{макс}^V = 550$~нм). Для звёзд спектрального класса A0V все три значения равны, поэтому такие звёзды являются фотометрическим стандартом. Одним из таких стандартов является яркая звезда северного неба Вега.

\term{Показатель цвета} $(B-V)$~--- разница блеска объекта в фильтрах $B$ и $V$, измеряется в звёздных величинах. Существует эмпирическая зависимость, связывающая показатель цвета $(B-V)$ и температуру  объекта,
\begin{equation}
T = 4600 \cdot\left(\frac{1}{0.92\left(B-V\right) + 1.7} + \frac{1}{0.92\left(B-V\right) + 0.62}\right).
\end{equation}

\term{Избыток цвета} $E\left(B - V\right)$~--- разность между наблюдаемым показателем цвета звезды и нормальным, свойственным ее спектральному классу.
\begin{equation}
E\left(B - V\right) = \left(B - V\right) - \left(B - V\right)_0 = \left(B - B_0\right) - \left(V - V_0\right) = A_B - A_V,
\end{equation}
величины $A_B$ и $A_V$~--- поглощения на голубом и видимом участках спектра.

\term{Межзвёздное поглощение}~--- суммарный эффект рассеяния и истинного поглощения света пылевыми частицами в межзвёздной среде. Известно соотношение, связывающее межзвёздное поглощение с избытком цвета,
\begin{equation}
\frac{A_V}{E\left(B - V\right)} = R \approx 3.1.
\end{equation}
%Поглощение света в среде зависит от длины волны, и установлена эмпирическая зависимость
%\begin{equation}
%A_B - A_V = 1.086 \bigl(\tau\left(B\right) - \tau\left(V\right)\bigr),
%\end{equation}
%где $\tau\left(B\right)$ и $\tau\left(V\right)$~---  оптические толщины в голубом и видимом диапазоне.