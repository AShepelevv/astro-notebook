\subsection{Телесный угол}
\term{Телесный угол}~--- часть пространства, которая является объединением всех лучей, выходящих из данной точки (вершины угла) и пересекающих некоторую поверхность (которая называется поверхностью, стягивающей данный телесный угол). Телесный угол измеряется в стерадианах и равен отношению площади сферы, которую вырезает данный угол, к квадрату радиуса данной сферы.
\begin{equation}
\Omega = \frac{S}{R^2}
\end{equation}
Телесный угол полной сферы равен $4\pi$. Величину телесного угла, образованного конусом с углом раствора $\alpha$ можно определить по формуле:
\begin{equation}
\Omega = 2 \pi \left(1 - \cos \frac{\alpha}{2}\right)
\end{equation}
При наличии радиуса $R$ и высоты сегмента $H$ телесный угол выражается:
\begin{equation}
\Omega = 2 \pi \left(1 - \dfrac{H}{\sqrt{R^2 + H^2}}\right)
\end{equation}
Для сферического треугольника с углами $\alpha$, $\beta$ и $\gamma$ справедливо соотношение:
\begin{equation}
\Omega = \alpha + \beta + \gamma - \pi
\end{equation}