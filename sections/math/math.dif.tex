\subsection{Производная}
\textbf{Производная в точке} - предел отношения приращения функции к приращению её аргумента при стремлении приращения аргумента к нулю, если такой предел существует.
\begin{equation}
f^\prime(x_0) = \lim_{\Delta x \to 0}\frac{f(x_0 + \Delta x) - f(x_0)}{\Delta x}
\end{equation}
Общепринятые обозначения для производной функции $y = f(x)$ в точке $x_0$:
\begin{equation}
f^\prime(x_0) = f^\prime_x(x_0) = D f(x_0) = \frac{d f}{d x}(x_0) = \dot{y} (x_0)
\end{equation}
Таблица производных:
\begin{center}
\begin{tabular}{|c|c|c|}
\hline
$(c)^\prime = 0$ & $(\sin x)^\prime = \cos x$ & $(\arcsin x)^\prime = \frac{1}{\sqrt{1 - x^2}}$ \\
\hline
$(x^a)^\prime = a x^{a-1}$ & $(\cos x)^\prime = - \sin x$ & $(\arccos x)^\prime = - \frac{1}{\sqrt{1 - x^2}}$ \\
\hline
$(a^x)^\prime = a^x \ln a$ & $(\tan x)^\prime = \frac{1}{\cos^2 x}$ & $(\arctg x)^\prime = \frac{1}{1 + x^2}$ \\
\hline
$(\log_a x)^\prime = \frac{1}{x \ln a}$ & $(\cot x)^\prime = - \frac{1}{\sin^2 x}$ & $(\arcctg x)^\prime = - \frac{1}{1 + x^2}$ \\
\hline
\end{tabular}
\end{center}
Геометрический смысл производной. Значение производной в точке численно равно тангенсу угла наклона касательной к функции в данной точке. Точки, где производная равна 0, соответствуют локальным минимумам и максимумам функции.
