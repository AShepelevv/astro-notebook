\subsection{Производная}
\textbf{Производная в точке}~--- предел отношения приращения функции к приращению её аргумента при стремлении приращения аргумента к нулю, если такой предел существует. Геометрический смысл производной: значение производной в точке численно равно тангенсу угла наклона касательной к функции в данной точке. Точки, где производная равна 0, соответствуют локальным минимумам и максимумам функции.
\begin{equation}
f^\prime(x_0) = \lim_{\Delta x \to 0}\frac{f(x_0 + \Delta x) - f(x_0)}{\Delta x}
\end{equation}
Общепринятые обозначения для производной функции $y = f(x)$ в точке $x_0$:
\begin{equation}
f^\prime(x_0) = f^\prime_x(x_0) = D f(x_0) = \frac{d f}{d x}(x_0) = \dot{y} (x_0)
\end{equation}
Правила дифференцирования:
\begin{align*}
(f+g)^\prime &= f^\prime + g^\prime\\
(Cf)^\prime &= Cf^\prime\\
(fg)^\prime &= f^\prime g + f g^\prime\\
\left(\dfrac{f}{g}\right)^\prime &= \dfrac{f^\prime g - f g^\prime}{g^2}\\
\dfrac{d}{dx}f(g(x)) &= \dfrac{df(g)}{dg}\dfrac{dg(x)}{dx}
\end{align*}
Таблица производных:
\begin{align*}
(c)^\prime &= 0 & (\sin x)^\prime &= \cos x & (\arcsin x)^\prime &= \dfrac{1}{\sqrt{1 - x^2}} \\
(x^a)^\prime &= a x^{a-1} & (\cos x)^\prime &= - \sin x & (\arccos x)^\prime &= - \dfrac{1}{\sqrt{1 - x^2}} \\
(a^x)^\prime &= a^x \ln a & (\tan x)^\prime &= \dfrac{1}{\cos^2 x} & (\arctg x)^\prime &= \dfrac{1}{1 + x^2} \\
(\log_a x)^\prime &= \dfrac{1}{x \ln a} & (\cot x)^\prime &= - \dfrac{1}{\sin^2 x} & (\arcctg x)^\prime &= - \dfrac{1}{1 + x^2} 
\end{align*}

