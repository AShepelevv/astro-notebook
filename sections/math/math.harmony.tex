\subsection{Гармонические колебания}
\term{Гармонические колебания}~--- колебания, при которых физическая величина изменяется с течением времени по гармоническому (синусоидальному, косинусоидальному) закону.
\begin{equation}
x(t) = A \sin (\omega t + \phi) \text { или } x(t) = A \cos (\omega t + \phi)
\end{equation}
Уравнение гармонический колебаний:
\begin{equation}
\frac{d^2 x}{d t^2} + \omega x = 0
\end{equation}
Гармонические колебания совершаются под действием упругих или квазиупругих сил - сил, значение которых пропорционально отклонению со знаком минус. Примерами гармонических колебаний могут служить математический и пружинный маятники.