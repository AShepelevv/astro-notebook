\subsection{Гармонические колебания}
\term{Гармонические колебания}~--- колебания, при которых физическая величина изменяется с течением времени по гармоническому (синусоидальному, косинусоидальному) закону.
\begin{equation}
x(t) = A \sin (\omega t + \varphi) \text { или } x(t) = A \cos (\omega t + \varphi)
\end{equation}
$\varphi$~--- начальная фаза колебаний, которая определяет значение полной фазы колебания (и самой величины $x$) в момент времени $t = 0$.\\
Уравнение гармонических колебаний:
\begin{equation}
\ddot{x} + \omega^2 x = 0
\end{equation}
Период колебаний с частотой $\omega$ выражается:
\begin{equation}
T = \frac{2 \pi}{\omega}
\end{equation}
Гармонические колебания совершаются под действием упругих или квазиупругих сил - сил, значение которых пропорционально отклонению со знаком минус. Примерами гармонических колебаний могут служить математический и пружинный маятники.