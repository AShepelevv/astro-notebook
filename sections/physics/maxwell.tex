\subsection{Распределение Максвелла}

Рассмотрим однородный изотропный газ в некотором объеме. Это означает, что направления скоростей частиц в любом элементе объ­ёма газа равновероятны. Пусть число молекул в единице объёма, имеющих скорости
\begin{equation*}
	\vec{v} \in [v_x, v_x + d v_x] \times [v_y, v_y + d v_y] \times [v_z, v_z + d z_z],
\end{equation*}
где $\times$~--- знак декартова произведения множеств, иначе, в элементе объема пространства скоростей $d^3 v = d v_x \, d v_y \, d v_z$, равно
\begin{equation*}
	d n(\vec{v}) = n f(\vec{v}) \, d^3 v,
\end{equation*}
где $n$ — концентрация частиц, а $f(\vec{v})$ — некоторая функция распределения.
Представим вероятность того, что $x$-компонента скорости имеет значение в интервале $[v_x, v_x + d v_x]$, как
\begin{equation*}
	d W (v_x) = \varphi(v_x) \, d v_x,
\end{equation*}
где $\varphi(x)$~--- функция распределения одной компоненты скорости. Вследствие изотропности газа аналогичные распределения вероятностей должны быть и для других компонент скорости:
\begin{equation*}
	d W(v_y) = \varphi(v_y) \, d v_y, \quad 
	d W(v_z) =\varphi(v_z) \, d v_z.
\end{equation*}

Предполагая, что компоненты $\{v_x, v_y, v_z\}$~--- независимые случайные величины, запишем вероятность некоторого значения вектора скорости~$\vec{v}$:
\begin{equation*}
	d W(v_x, v_y, v_z)
	   = \varphi(v_x) \varphi(v_y) \varphi(v_z) \,d v_x \, d v_y \, d v_z
\end{equation*}
С другой стороны,
\begin{equation*}
	d W\left(v_x, v_y, v_z\right)
	   = \frac{d n(\vec{v})}{n} = f(\vec{v}) \, d v_x \,d v_y \,d v_z
\end{equation*}
Таким образом, получаем,
\begin{equation*}
	f(\vec{v}) = \varphi(v_x) \varphi(v_y) \varphi(v_z),
\end{equation*}
прологарифмируем обе части полученного равенства:
\begin{equation}
    \ln f(\vec{v}) = \ln \varphi(v_x) + \ln \varphi(v_y) + \ln \varphi(v_z).
    \label{eq:ln-maxwll}
\end{equation}
Это функциональное уравнение должно решаться совместно с уравнением
\begin{equation*}
    v^2 = v_x^2 + v_y^2 + v_z^2
\end{equation*}
Продифференцируем уравнение \eqref{eq:ln-maxwll} по переменной $v_x$,
\begin{equation}
    \frac{f'_{v_x}(\vec{v})}{f(\vec{v})} \frac{\partial v}{\partial v_x}
        = \frac{\varphi'(v_x)}{\varphi(v_x)}.
    \label{eq:maxwell-distribution-function-derivation}
\end{equation}
Так как
\begin{equation*}
    \frac{\partial v}{\partial v_x}
        = \frac{\partial}{\partial v_x} \sqrt{v_x^2+v_y^2+v_z^2}
        = \frac{1}{2\sqrt{v_x^2+v_y^2+v_z^2}} \frac{\partial(v_x^2+v_y^2+v_z^2)}{\partial v_x}
        = \frac{v_x}{v},
\end{equation*}
то \eqref{eq:maxwell-distribution-function-derivation} можно записать как
\begin{equation*}
\frac{1}{v} \frac{f^{\prime}(\vec{v})}{f(\vec{v})}=\frac{1}{v_x} \frac{\varphi^{\prime}\left(v_x\right)}{\varphi\left(v_x\right)}.
\end{equation*}
Правая часть этого равенства не зависит от $v_y$ и $v_z$, тогда как левая часть содержит эти переменные. Следовательно, обе части равенства должны быть постоянными. Пусть они равны $-2\alpha$, тогда
\begin{gather*}
	\frac{1}{v_x} \frac{\varphi'(v_x)}{\varphi(v_x)} = -2 \alpha,\\
	\frac{d\varphi(v_x)}{\varphi(v_x)} = -2\alpha v_x \, dv_x,\\
	\int \frac{d\varphi(v_x)}{\varphi(v_x)} = -2\alpha \int v_x \, dv_x,\\
	\ln |\varphi(v_x)| = -\alpha v_x^2 + C,\\
	\varphi(v_x) = e^{-\alpha v_x^2 + C} \equiv A e^{-\alpha v_x^2}.
\end{gather*}
Аналогично
\begin{gather*}
    \varphi(v_y) = A e^{-\alpha v_y^2}, \quad \varphi(v_z)=A e^{-\alpha v_z^2}, \\
    f(\vec{v}) = \varphi(v_x) \varphi(v_y) \varphi(v_z) = A^3 e^{-\alpha v^2}.
\end{gather*}
Константа $A$ определяется из условия нормировки 
\begin{gather*}
    \int \varphi(x) \, d x = 1,\\
    A \int\limits_{-\infty}^{\infty} e^{-\alpha x^2} \, dx = 1, \quad t = \sqrt{\alpha}{x},\\ 
    A \sqrt{\frac{\pi}{\alpha}} \cdot \frac{2}{\sqrt{\pi}} \int\limits_0^\infty e^{-t^2} \, dt \equiv  A \sqrt{\frac{\pi}{\alpha}} \erf \infty = A \sqrt{\frac{\pi}{\alpha}} = 1.
\end{gather*}
откуда следует, что $A = \sqrt{\alpha / \pi}$.\footnote{$\displaystyle \erf x =  \int\limits_{0}^x e^{-t^2} \, dt$~--- функций ошибок Гаусса.}

Таким образом, для одной компоненты скорости:
\begin{equation*}
    d W(v_x) = \sqrt{\frac{\alpha}{\pi}} e^{-\alpha v_x^2} \, d v_x,
\end{equation*}
и для вектора скорости:
\begin{equation*}
    d n(\vec{v}) = n \left(\frac{\alpha}{\pi}\right)^{3 / 2} e^{-\alpha v^2} d^3 v
\end{equation*}
Для объяснения смысла параметра $\alpha$ найдем среднюю кинетическую энергию молекул:
\begin{equation*}
    \bar{\varepsilon}_{\text{кин}} 
    = \frac{\overline{m v^2}}{2}
    = \frac{1}{n} \int \frac{m v^2}{2} n\left(\frac{a}{\pi}\right)^{3 / 2} e^{-\alpha v^2} d^3 v.
\end{equation*}
Так как подынтегральное выражение зависит только от модуля скорости, можно сделать замену  $d^3 v =  4 \pi v^2 d v$, получим
\begin{multline*}
    \bar{\varepsilon}_{\text{кин}}
        =  \frac{2m \alpha^{3 / 2}}{\sqrt{\pi}} \int_0^{\infty} v^4  e^{-\alpha v^2} \, d v 
        \overset{t = \alpha v^2}{=}  \frac{m}{\alpha \sqrt{\pi}} \int_0^{\infty} t^{3/2} e^{-t} \, dt = \\
        =  -\frac{m}{\alpha \sqrt{\pi}} \left.\Gamma \left(\frac{5}{2}, t\right) \right|_0^\infty 
        = -\frac{m}{\alpha \sqrt{\pi}} \left(0 - \frac{3 \sqrt{\pi}}{4} \right)
        = \frac{3 m}{4 \alpha}. 
\end{multline*}
Эта величина должна быть равна $3 k T / 2$, откуда $\alpha = m / (2 k T)$.

Итак, в однородном по пространству идеальном газе со средней концентрацией $n$ в равновесном состоянии число молекул, обладающих скоростями
\begin{equation*}
    \vec{v} \in [v_x, v_x + d v_x] \times [v_y, v_y + d v_y] \times [v_z, v_z + d z_z],
\end{equation*}
определяется распределением Максвелла
\begin{equation}
    d n(\vec{v})
        = n \left(\frac{m}{2 \pi k T}\right)^{3 / 2} \exp \left(-\frac{m v^2}{2 k T}\right) \, d^3 v.
    \label{eq:maxwell-distribution}
\end{equation}

\subsubsection{Распределение по модулю скорости}
Для того чтобы найти распределение по модулю скорости, от единичных клеток можно перейти с сферическим слоям радиуса $v$, как это было сделано ранее, и применить замену $dv^3 =  4 \pi v^2 \, dv$. Таким образом в данном элементе пространства будут находиться все векторы с модулем $v$
\begin{gather}
d n(v)=n \Phi(v) d v, \nonumber\\ 
\Phi(v)=4 \pi v^2 f(v)=4 \pi\left(\frac{m}{2 \pi k T}\right)^{3 / 2} v^2 \exp \left(-\frac{m v^2}{2 k T}\right).
\label{eq:maxwell-distribution-module}
\end{gather}

\subsubsection{Распределение по энергиям}
В некоторых случаях удобно перейти от распределения частиц по скоростям к распределению по кинетическим энергиям. Рассмотрим замену~$v = \sqrt{2 \varepsilon / m}$, тогда~$dv = d\varepsilon / \sqrt{2 m \varepsilon}$. Применим её к распределению Максвелла по модулю скорости~\eqref{eq:maxwell-distribution-module}, получим,
\begin{equation}
d n(\varepsilon)=n F(\varepsilon) d \varepsilon, \quad F(\varepsilon)=\frac{2}{\sqrt{\pi(k T)^3}} \exp \left(-\frac{\varepsilon}{k T}\right) \sqrt{\varepsilon}.
\end{equation}

\subsubsection{Средние значения}

Найдём среднее значение скорости частиц. По определению среднего
\begin{equation*}
    \overline{\vec{v}} = \int \vec{v} \, dn(\vec{v}),
\end{equation*}
заметим, что функция $dn(\vec{v})$~--- чётная, в то время как $\vec{v}$~--- нечётная, следовательно, подынтегральное выражение также нечётная функция, а значит, её интеграл по всему пространству равен нулю. Другими словами, $\vec{v} = 0$, то все направления скорости равновероятны.

Значение остальных интегралов вычисляется аналогично действиям, проделанным выше,~--- путём соответствующих замен. Так что оставим это упражнение читателю, а здесь приведет лишь результат. 

Итак, \term{средняя квадратичная скорость} 
\begin{equation}
    v_{\text {с.к. }} 
        = \sqrt{\overline{v^2}}
        = \sqrt{\int\limits_0^\infty v^2 \Phi(v) \, dv} 
        = \sqrt{3 k T / m};
\end{equation}
\term{средняя величина} (модуль) \term{скорости}
\begin{equation}
     \overline{v}
         = \int\limits_0^{\infty} v \Phi(v) \, dv
         = \sqrt{\frac{8 k T}{\pi m}};
\end{equation}
\term{средняя энергия}
\begin{equation}
    \bar{\varepsilon} 
        = \frac{1}{n} \int\limits_{\varepsilon = 0}^{\infty} \varepsilon \, d n(\varepsilon) 
        = \frac{3}{2} k T.
\end{equation}

\subsubsection*{Среднее число ударов молекул о стенку}

\begin{wrapfigure}[9]{r}{0.35\tw}
    \vspace{-1pc}
    \centering
    \tikzsetnextfilename{mean-particles}
    \begin{tikzpicture}
        \tkzDefPoint(0, 0){A}
        \tkzDefPoint(0, 2){B}
        \tkzDefPoint(3, 1){C}
        \tkzDefPoint(3, -1){D}
        \tkzDefPoint(1.5, 0.5){O}
        \tkzDefPoint(-0.5, 0.5){V}
        \tkzDefPoint(1.5, 1){dV1}
        \tkzDefPoint(1.5, 1.2){dV2}

        \foreach \r in {0.5, 0.7} {
            \tkzDefShiftPoint[O](\r,0){r}
            \tkzDrawCircle[gray!40, line width=0.4pt](O,r)
        }

        \tkzDrawSegments[](A,B B,C C,D D,A)
        \tkzDrawSegments[-latex](V,O V,dV1 V,dV2)
        \tkzLabelSegments[below, font=\scriptsize](O,V){$\vec{v}_z$}
        \tkzLabelSegments[above, font=\scriptsize](V,dV2){$\vec{v}$}

        \tkzMarkAngle[size=0.8, arc=l, mksize=2pt](O,V,dV1)
        \tkzLabelAngle[font=\scriptsize, pos=1.05](O,V,dV1){$\theta$}
    \end{tikzpicture}
    \caption{Столкновение частиц со стенкой}
    \label{pic:maxwell-mean-particle-collision}
\end{wrapfigure}
Рассмотрим столкновения молекул газа с неподвижной стенкой. Выделим группу молекул, имеющих скорость $\vec v$. Число таких молекул обозначим как $d n(\vec{v})$. Вследствие изотропии газа в телесный угол $d \Omega = 2 \pi \sin \theta \, d \theta$, \lookPicRef{pic:maxwell-mean-particle-collision}, летит доля молекул, равная $d \Omega / 4 \pi$. Соответственно число этих молекул равна $dn(\vec{v}) \, d \Omega / 4 \pi$. 

За время $dt$ до поверхности долетят молекулы, удаленные от нее на расстояние $v_z \, dt$, где $v_z = v \cos \theta$. Bсего в площадку $dS$ попадут молекулы, находящиеся в цилиндре объемом $v_z \, dt \, dS$, содержащем $v_z \, dt \, dS \, dn(\vec{v}) \, d\Omega / (4 \pi)$ молекул выделенной группы. Суммируя результат по всем допустимым углам $\theta \in (0, \nicefrac{\pi}{2})$ и скоростям $v \in (0, \infty)$ и деля результат на $d t \, d S$, получаем
\begin{multline}
    j 
        = \int d n(v) v \cos \theta \, \frac{d \Omega}{4 \pi} = \\
        = \frac{n}{4 \pi} \underbrace{\int\limits_0^{\infty} v \Phi(v) \, d v}_{\text{средняя скорость}} \underbrace{\int\limits_0^{\pi / 2} 2 \pi \cos \theta  \sin \theta \, d \theta}_{\pi / 2}
        = \frac{1}{4} n \bar{v} .
    \label{eq:mean-count-mxwl}
\end{multline}
Величина $j$ представляет собой плотность потока частиц газа, то есть число частиц, пересекающих единичную площадку в одну сторону в единицу времени, $[j]= \text{частиц}/\left(\text{м}^2 \cdot \text{c}\right)$.

