\subsection{Магнитное поле}

Вектор магнитной индукции в данной точке вводится как \textit{векторное произведение} вектора направления электрического тока и радиус-вектора данной точки. 
\begin{equation}
	d \vec{B} = \frac{\mu_0}{4\pi} \cdot \frac{I[d \vec{l}\times\vec{r}]}{r^3},
	\label{eq:BSL-law}
\end{equation}
где константа $\mu_{0} = 4 \pi \cdot 10^{-7} \, \frac{\text{Тл} \, \text{м}}{\text{А}}$. Линии магнитного поля идут от северного полюса к южному. Магнит поворачивается во внешнем магнитном поле так, чтобы его собственное поле совпадало с внешним. Соответственно стрелка компаса поворачивается в соответствии с магнитным полем Земли. В общем-то говоря точная причина симметрии земного магнитного поля до сих пор неизвестна.

Поток магнитного поля вводится как:
\begin{equation}
	d\Phi = \vec{B}d\vec{S}.
\end{equation}
В силу отсутствия в мире магнитных зарядов можно записать упрощенный вариант теоремы Гаусса:
\begin{equation}
    \Phi = \oiint_{S}{\Vec{B} \, d\Vec{S}} = 0
\end{equation}
Из курса электростатики известно, что электростатическое поле консервативно. То есть интергал поля по произвольному замкнутому контуру равен 0:
\begin{equation*}
    \oint{\Vec{E} \cdot d \Vec{l}} = 0
\end{equation*}
Для произвольной геометрической кривой с выбранным направлением обхода и произвольного контура тока справедлива теорема о циркуляции:
\begin{center}
\begin{tikzpicture}
    \begin{scope}[very thick,decoration={
    markings,
    mark=between positions 0.05 and 1 step 2cm with {\arrow{latex}}}]
    \draw[postaction={decorate}] plot [smooth cycle] coordinates {(0,0) (1,-1) (2, -1.3) (3, -2)(4, -2) (3,0.5) (1,1)};
    \draw[postaction={decorate}] plot [smooth cycle] coordinates {(2.5,-0.5) (1.5,-2) (2, -2.8) (3, -3)(4, -3) (5,-0.5) (4,1)};
    \end{scope}
    \draw (7, -0.5) node[right, draw] {
$\begin{aligned}
    \oint_{\Omega}{\Vec{B} \cdot d \Vec{l}} = \mu_0 I
\end{aligned}$};
\draw (6.5, -2) node[right] {
Теорема о циркуляции};
    \node[] at (4.5, -3) {$\Omega$};
    \node[] at (0.5, -1) {$I$};
\end{tikzpicture}
\end{center}
Знак $I$ выбирается по знаку скалярного произведения $\Vec{n}$ и направления под которым ток пересекает плоскоть контура. Пользоваться теоремой следует аналогично теореме Гаусса, выбирая удобный для решения контур. 