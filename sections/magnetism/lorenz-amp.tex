\subsection{Сила Ампера и сила Лоренца}

Связывающим компонентом между магнитными полями и токами является сила Ампера:
\begin{equation}
	d\vec{F}_\text{A} = I [ d \vec{l} \times \vec{B} ].
\end{equation}
Записав выражение силы Ампера для электрона можно получить так называемую силу Лоренца
\begin{equation*}
	\vec{F}_\text{A} = e n v S [ d\vec{l} \times \vec{B} ].
\end{equation*}
Однако учитывая единичность электрона $nSdl=1$ и что векторы $d\vec{l}$ и~$\vec{v}$ сонаправлены, получим:
\begin{equation}
	\vec{F}_\text{л} = e [ \vec{v} \times \vec{B} ].
\end{equation}
Отсюда следует, что в однородном магнитном поле заряженные частицы будут двигаться по окружностям. Получим радиус кривизны траектории, приравняв силу Лоренца к центробежной:
\begin{gather}
	\frac{m_e v^2}{R} = e v B \nonumber,\\
	R = \frac{m_e v}{e B}.
\end{gather}
Отсюда период период движения электрона
\begin{equation}
    T = \frac{2 \pi R}{v} = \frac{2 \pi m_e}{e B},
\end{equation}
откуда частота
\begin{equation}
	f = \frac{e B}{2 \pi m_e},
\end{equation}
данная частота называется \term{циклотронной}. А излучение, создаваемое заряженными частицами, движущимися в магнитном поле~--- \term{цикло\-тронным}.
