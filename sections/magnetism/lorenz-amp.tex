\subsection{Сила Ампера и сила Лоренца}
Связывающим компонентом между магнитными полями и токами является сила Ампера:
\begin{equation}
	d\vec{F}_A = I[d\vec{l}\times\vec{B}].
\end{equation}
Записав выражение силы Ампера для электрона можно получить так называемую силу Лоренца
\begin{equation*}
	\vec{F}_A = envS[d\vec{l}\times\vec{B}].
\end{equation*}
Однако учитывая единичность электрона $nSdl=1$ и что векторы $d\vec{l}$ и $\vec{v}$ сонаправлены, получим:
\begin{equation}
	\vec{F}_{\text{л}} = e[\vec{v}\times\vec{B}].
\end{equation}
Отсюда следует, что в однородном магнитном поле заряженные частицы будут двигаться по окружностям. Радиус кривизны такой траектории:
\begin{equation}
	\frac{m_e v^2}{R} = evB \, \Rightarrow \, R = \frac{m_e v}{eB}.
\end{equation}
Если же поделить радиус кривизны на скорость и умножить на $2\pi$ можно получить период облета по такой окружности. Обратная к данному периоду величина называется циклотронной частотой.
\begin{equation}
	f = \frac{eB}{2\pi m_e}.
\end{equation}