\subsection{Движение по эллиптической орбите}

(Численно) Для тел солнечной системы.$$T^2_{\text{год}}=a^3_{\text{а.е.}}$$
Средняя скорость планет Солнечной системы.$$v_{\text{орб}}=\sqrt{\frac{GM}{a}}\approx \frac{29,8}{\sqrt{a}}$$
Скорость в апоцентре, $e$ --- эксцентриситет.$$v_{\text{аф}}=v_{\text{орб}}\sqrt{\frac{1-e}{1+e}}$$
Скорость в перицентре, $e$ --- эксцентриситет.$$v_{\text{пер}}=v_{\text{орб}}\sqrt{\frac{1+e}{1-e}}$$
Скорость в точке орбиты, удалённой на расстояние $r$ от центрального тела, $M$ --- масса центрального тела.$$v=\sqrt{GM\left(\frac2r - \frac1a\right)}$$
Скорость в точке орбиты, для которой истинная аномалия $\nu$, $p$ --- фокальный параметр.$$v=\sqrt{\frac{GM}{p}\cdot(1+2e\nu+e^2)}$$