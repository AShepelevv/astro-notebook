\subsection{Годичное движение Солнца}

\begin{wrapfigure}[23]{r}{0.62\tw}
    \vspace{-1pc}
    \centering
    \tikzsetnextfilename{analemma}
    \begin{tikzpicture}
        \footnotesize
        \begin{axis}[
            width    =    7.27cm,
            height    =    10cm,
            xmax     =    20,
            xmin    =    -20,
            ymax    =    27.5,
            ymin     =     -27.5,
            xlabel    =    {$\eta$, мин},
            ylabel     =     {$\delta_\odot$, $~^\circ$}
        ]
            \addplot[smooth] table[x=eta, y=delta, col sep = comma] {data/analemma.csv};
            \addplot[only marks, mark = o,mark options={scale=0.4, black}] table[x=eta, y=delta, col sep = comma] {data/analemma-months.csv}; %
            \draw (axis cs:-4.31549,-22.7642) node[anchor=south west] {01.01};
            \draw (axis cs:-13.7937,-15.9581) node[anchor=north east] {01.02};
            \draw (axis cs:-12.2946,-5.49806) node[anchor=south east] {01.03};
            \draw (axis cs:-3.41919,5.51223) node[anchor=south east] {01.04};
            \draw (axis cs:3.16321,15.5839) node[anchor=west] {01.05};
            \draw (axis cs:1.62501,22.3907) node[anchor=south west] {01.06};
            \draw (axis cs:-4.38004,22.7234) node[anchor=south east] {01.07};
            \draw (axis cs:-5.9518,16.4302) node[anchor=east] {01.08};
            \draw (axis cs:1.35013,6.67371) node[anchor=south west] {01.09};
            \draw (axis cs:12.2793,-4.12177) node[anchor=south west] {01.10};
            \draw (axis cs:16.6081,-14.6495) node[anchor=east] {01.11};
            \draw (axis cs:9.1938,-22.1884) node[anchor=south east] {01.12};
%            \draw [-latex, gray] (axis cs:5,-5) .. controls (axis cs:18,-20) and (axis cs:-15,-20) .. (axis cs:-5,-5);
%            \draw [-latex, gray] (axis cs:1,15) .. controls (axis cs:5,24) and (axis cs:-8,23) .. (axis cs:-3,15);
        \end{axis}
    \end{tikzpicture}
    \caption{Аналемма}
    \label{fig:analemma}
\end{wrapfigure}
В течение сидерического года Земля совершает полный оборот вокруг Солнца. Вследствие этого Солнце движется относительно далёких звёзд для наблюдателя на Земле. Это движение совершается по большому кругу небесной сферы~---  \term{эклиптике}~--- проекции орбиты Земли на небесную сферу. 

Однако, в силу прецессии земной оси, период такого движения равен \term{тропическому году}, который короче сидерического года примерно на 20~мин~25~сек.

%\begin{wrapfigure}[12]{r}{0.5\tw}
%    \centering
%    \vspace{-.9pc}
%    \tikzsetnextfilename{sun-path}
%    \begin{tikzpicture}
%        \begin{axis}[
%            width    =    .5\tw,
%            height    =    4.5cm,
%            xlabel    =    {Прямое восхождение $\alpha^h$},
%            ylabel    =    {Склонение $\delta^{\circ}$},
%            extra y ticks    =    {23.44, -23.44},
%            ytick = {-20, -10, 0, 10, 20},
%            ymax    =    25,
%            ymin    =    -25,
%            xmax    =    24,
%            xmin    =    0,
%            xtick    =    {0, 4, 8, 12, 16, 20, 24},
%            x dir = reverse
%            ]
%            \addplot [domain=0:24, samples=100] {atan(sin(x*15)*tan(23.44))};
%        \end{axis}
%    \end{tikzpicture}
%    \caption{График зависимости склонения Солнца от его прямого восхождения}
%\end{wrapfigure}
В моменты, когда Солнце находится в \imp{точке весеннего равноденствия}~$\aries$  (20~марта, реже~21) его координаты: $\alpha=0^h$, $\delta=0^{\circ}$. Во время прохождения этой точки обе координаты Солнца растут. Так происходит до момента, пока Солнце не пройдет \imp{точку летнего солнцестояния} (21~июня, реже~20), координаты которой~--- $\alpha=6^h$ и $\delta=\varepsilon$. После этого склонение Солнца начинает убывать. В момент прохождения \imp{точки осеннего равноденствия}~$\libra$ (22~или 23~сентября), координаты Солнца составляют $\alpha=12^h$, $\delta=0^{\circ}$. После прохождения \imp{точки зимнего солнцестояния} с координатами $\alpha=18^h$, $\delta=-\varepsilon$ (22~или 21~декабря) склонение Солнца начинает увеличиваться.

\begin{wrapfigure}[5]{r}{0.4\tw}
    \centering
    \vspace{-1pc}
    \tikzsetnextfilename{sun-lambda-alpha}
    \tdplotsetmaincoords{70}{-70}
    \begin{tikzpicture}[tdplot_main_coords]
        \footnotesize

        \def\R{5}
        \def\EPS{23.5}
        \def\ALPHA{45}
        \def\LAMBDA{atan(tan(\ALPHA)/cos(\EPS))}
        \def\DELTA{asin(sin(\LAMBDA)*sin(\EPS))}

        % Draw triangle
        \tdplotsetrotatedcoords{0}{0}{0};
        \draw[tdplot_rotated_coords, semithick] (\R,0,0) arc (0:{\ALPHA}:\R);
        \tdplotsetrotatedcoords{90}{-\EPS}{-90};
        \draw[tdplot_rotated_coords, semithick] (\R,0,0) arc (0:\LAMBDA:\R);
        \tdplotsetrotatedcoords{90+\ALPHA}{-90}{-90};
        \draw[tdplot_rotated_coords, semithick] (\R,0,0) arc (0:\DELTA:\R);

        % Draw points
        \tdplotCsDrawPoint{\R}{180}{-90}
        \tdplotCsDrawPoint{\R}{180 + \ALPHA}{-90 + \DELTA}
        \tdplotCsDrawPoint{\R}{180 + \ALPHA}{-90}

        % Label arcs
        \tdplotCsLabelPoint{\R}{180 + \ALPHA / 2}{-90}{$\alpha$}{below}
        \tdplotCsLabelPoint{\R}{180 + \ALPHA / 2}{-90 + \DELTA / 2}{$\lambda$}{above right}
        \tdplotCsLabelPoint{\R}{180 + \ALPHA}{-90 + \DELTA / 2}{$\delta$}{left}

        % Mark right angle
        \tdplotsetrotatedcoords{\ALPHA}{0}{0};
        \draw [tdplot_rotated_coords](\R,-0.2,0.2) coordinate (c) (\R,0,0.2) coordinate (a1) -- (c) (\R,-0.2,0) coordinate (a2) -- (c) pic [angle radius=0.2cm] {right angle=a1--c--a2};

        % Mark and label angle t near P
        \def\angleRadius{0.6}
        \tdplotsetrotatedcoords{0}{0}{0};
        \draw [tdplot_rotated_coords, canvas is yz plane at x = \R]({\angleRadius * cos(\EPS)},0) arc (0:{\EPS - 3}:\angleRadius);
        \tdplotCsLabelPoint{\R}{6}{88.3}{$\varepsilon$}{left}
    \end{tikzpicture}
    \caption{}
    \label{pic:sun-lambda-alpha}
\end{wrapfigure}
Найдём выражение для прямого восхождения Солнца $\alpha$ через переменные $\lambda$ и $\varepsilon$, где $\varepsilon = 23.44^\circ$~--- угол наклона экватора Земли к эклиптике. Для этого рассмотрим формулу пяти элементов~\eqref{eq:formula-5-elem} для прямоугольного сферического треугольника, представленного на рисунке~\picRef{pic:sun-lambda-alpha}:
\begin{gather}
    \sin \delta \cos 90^\circ = \cos \lambda \sin \alpha - \sin \lambda \cos \alpha \cos \varepsilon,\notag\\
    \cos \lambda \sin \alpha = \sin \lambda \cos \alpha \cos \varepsilon,\notag\\
    \frac{\tg \alpha }{\tg \lambda} = \cos \varepsilon.
    \label{eq:tgAlpha/tgLambda}
\end{gather}

Из аналогичных рассуждений несложно получить, что прямое восхождение Солнца связано со склонением формулой. Рассмотрим формулу пяти элементов~\eqref{eq:formula-5-elem},
\begin{gather*}
    \cos \delta \sin \alpha = \sin \delta \cos \alpha \cos 90^\circ + \sin \lambda \cos \varepsilon,\\
    \cos \delta \sin \alpha = \sin \lambda \cos \varepsilon.
\end{gather*}
Выразим $\sin \lambda$ из теоремы синусов~\eqref{eq:sphere-th-sinus}:
\begin{equation*}    
    \sin \lambda = \frac{\sin \delta}{\sin \varepsilon}.
\end{equation*}
и подставим в полученное выше равенство:
\begin{equation}
    \sin\alpha = \frac{\tg\delta}{\tg\varepsilon}.  
    \label{eq:sin-alpha}
\end{equation}




