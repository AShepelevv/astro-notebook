\subsection{Солнечное время. Уравнение времени}
\term{Истинные солнечные сутки}~--- промежуток времени между двумя последовательными одноимёнными кульминациями Солнца.

\term{Истинное солнечное время}~--- промежуток времени между нижней кульминацией Солнца и его текущим положением. Рассчитывается по формуле
\begin{equation}
	T_{\text{ист}} = t_{\text{сол}}+12^h,
\end{equation}
где $t_{\text{сол}}$~--- часовой угол Солнца.

\begin{wrapfigure}[14]{r}{0.6\tw}
	\centering
%	\vspace{-1pc}
	\begin{tikzpicture}
		\begin{axis}[
			width	=	.63\tw,
			height	=	5.7cm,
			xmax 	=	365,
			xmin	=	0,
			ymax	=	20,
			ymin 	= 	-20,
			ylabel	=	{Уравнение времени $\eta$, мин},
			xlabel 	= 	{Суток от 1~января 00:00}
			]
			\addplot [domain=0:365, samples = 100, black, smooth] {7.53 * cos(360*(x - 81)/365) + 1.5 * sin(360*(x - 81)/365) - 9.87 * sin(2*360*(x - 81)/365)};
		\end{axis}
	\end{tikzpicture}
	\caption{График уравнения времени}
	\label{pic:time-eq}
\end{wrapfigure}
\term{Среднее солнечное время} ($T_\text{ср}$)~--- промежуток времени между нижней кульминацией \imp{среднего Солнца} и текущим его положением. \term{Среднее Солнце}~--- точка небесной сферы, которая равномерно движется по небесному экватору, совершая полный оборот за тропический год. Зная долготу наблюдателя, нетрудно вычислить среднее солнечное время:
\begin{equation*}
	T_\text{ср} = \text{UTC} + \frac{\lambda}{15^\circ/\text{час}},
\end{equation*}
где UTC~--- \imp{всемирное время}~--- среднее (или, что тоже самое, истинное) солнечное время на Гринвиче (меридиан с долготой $\lambda = 0^\circ$).

\term{Поясное} или \term{гражданское время}~--- среднее солнечное время на срединном меридиане географического часового пояса. В России также установлено декретное время, которое на 1 час больше поясного.

\term{Уравнение времени}~--- разница между истинным солнечным временем и средним солнечным временем, возникающая по причине неравномерности движения Земли по орбите и наклона земного экватора к плоскости эклиптики (см.~Рис.\,\ref{pic:time-eq}). Пусть $B \equiv 2 \pi (x - 81)/T_\oplus$, где $x$~--- количество суток, прошедшее с 1~января~00:00, тогда
\begin{equation}
	\eta = T_{\text{ист}} - T_\text{ср} = 7.53 \cos B + 1.5 \sin B - 9.87 \sin 2 B.
\end{equation}
