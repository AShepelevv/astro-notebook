\subsection{Солнечное время. Уравнение времени}
\term{Истинные солнечные сутки}~--- промежуток времени между двумя последовательными одноимёнными кульминациями Солнца.

\term{Истинное солнечное время}~--- промежуток времени между нижней кульминацией Солнца и его текущим положением. Рассчитывается по формуле
\begin{equation}
	T_{\text{ист}} = t_{\text{сол}}+12^h,
\end{equation}
где $t_{\text{сол}}$~--- часовой угол Солнца.

\term{Среднее солнечное время} ($T_\text{ср}$)~--- промежуток времени между нижней кульминацией \imp{среднего Солнца} и текущим его положением. \term{Среднее Солнце}~--- точка небесной сферы, которая равномерно движется по небесному экватору, совершая полный оборот за тропический год. Зная долготу наблюдателя, нетрудно вычислить среднее солнечное время:
\begin{equation*}
	T_\text{ср} = \text{UTC} + \frac{\lambda}{15^\circ/\text{час}},
\end{equation*}
где UTC~--- \imp{всемирное время}~--- среднее (или, что тоже самое, истинное) солнечное время на Гринвиче (меридиан с долготой $\lambda = 0^\circ$).

\term{Поясное} или \term{гражданское время}~--- среднее солнечное время на срединном меридиане географического часового пояса. В России также установлено декретное время, которое на 1 час больше поясного.

\begin{wrapfigure}[12]{r}{0.55\tw}
	\centering
	\vspace{-0.7pc}
	\begin{tikzpicture}
		\begin{axis}[
			width	=	6.5cm,
			height	=	4.7cm,
			xmax 	=	365,
			xmin	=	0,
			ymax	=	20,
			ymin 	= 	-20,
			ylabel	=	{$\eta$, мин},
			xlabel 	= 	{$d$, сут}
		]
			\addplot [domain=0:365.25, samples = 100, black, smooth]{-7.65 * sin(360*x/365.25) + 9.86 * sin(2 * ( 102.9 + 360*x/365.25 ))};
		\end{axis}
	\end{tikzpicture}
	\caption{График уравнения времени}
	\label{pic:time-eq}
\end{wrapfigure}
\term{Уравнение времени}~--- разница между истинным солнечным временем и средним солнечным временем, возникающая по причине неравномерности движения Земли по орбите и наклона земного экватора к плоскости эклиптики (см.~Рис.\,\ref{pic:time-eq}). Получим приближенное выражение для него. Для это этого вспомним уравнение Кеплера~\eqref{eq:kepler-eq} и величину эксцентриситета орбиты Земли $e_\oplus = 0.017 \ll 1$, откуда следует, что можно применять формулы приближенного вычисления, см.~Раздел~\ref{sec:form}. Рассмотрим выражение
\begin{multline*}
    \sin E 
        = \sin(E - M + M) =\\
        = \underbrace{\sin (E - M)}_{\simeq E - M} \cos M + \underbrace{\cos(E - M)}_{\simeq 1} \sin M \simeq\\
        \simeq (E - M) \cos M + \sin M,
\end{multline*}
так как $E - M \sim e$. Отсюда можно сделать первое приближение
\begin{equation*}
    E_1 
        = M + e \sin E 
        = M + e \big( (E - M) \cos M + \sin M \big) 
        \simeq M + e \sin M.
\end{equation*}
Воспользуемся полученным приближением, чтобы точнее оценить $E$.
\begin{equation*}
    E
        = M + e \big( (E_1 - M) \cos M + \sin M \big) 
        = M + e \sin M + \frac{e^2}{2} \sin 2M.
\end{equation*}

Теперь запишем первые три члена многочлена Тейлора формулы перехода от истинной аномалии к эксцентрической~\eqref{eq:kepler-eq-E-nu-2}:
\begin{equation*}
    \nu 
        = 2 \arctg \left(\sqrt{\frac{1+e}{1-e}} \tg \frac{E}{2} \right) 
        \simeq E + e \sin E + \frac{e^2}{4} \sin 2E.
\end{equation*}
Подставим сюда выражение для $E(M)$ и воспользуемся формулой для разложения синуса суммы:
\begin{multline*}
    \nu 
        \simeq M + e \sin M + \frac{e^2}{2} \sin 2M + \\
        + e \bigg( \sin M \cdot \underbrace{\cos (e \sin M + \ldots)}_{\simeq 1} + \cos M \cdot \underbrace{\sin ( e \sin M  + \ldots )}_{\simeq e \sin M} \bigg) + \\
        + \frac{e^2}{4} \bigg( \sin 2M \cdot \underbrace{\cos (2e \sin M + \ldots)}_{\simeq 1} + \cos 2M \cdot \underbrace{\sin (2e \sin M + \ldots)}_{\ll 1}\bigg)  \simeq \\
        \simeq M + 2e \sin M + \frac{5e^2}{4} \sin 2M.
\end{multline*}

Обозначим как $\omega = 103^\circ$ эклиптическую долготу перицентра, тогда эклиптическая долгота Солнца $\lambda = \nu + \omega$. 

\begin{wrapfigure}[5]{r}{0.4\tw}
    \centering
    \vspace{-1pc}
    \tikzsetnextfilename{sun-lambda-alpha}
    \tdplotsetmaincoords{70}{-70}
    \begin{tikzpicture}[tdplot_main_coords]
        \footnotesize
        
        \def\R{5}
        \def\EPS{23.5}
        \def\ALPHA{45}
        \def\LAMBDA{atan(tan(\ALPHA)/cos(\EPS))}
        \def\DELTA{asin(sin(\LAMBDA)*sin(\EPS))}
            
        % Draw triangle
        \tdplotsetrotatedcoords{0}{0}{0};
        \draw[tdplot_rotated_coords, semithick] (\R,0,0) arc (0:{\ALPHA}:\R);
        \tdplotsetrotatedcoords{90}{-\EPS}{-90};
        \draw[tdplot_rotated_coords, semithick] (\R,0,0) arc (0:\LAMBDA:\R);
        \tdplotsetrotatedcoords{90+\ALPHA}{-90}{-90};
        \draw[tdplot_rotated_coords, semithick] (\R,0,0) arc (0:\DELTA:\R);
        
        % Draw points
        \tdplotCsDrawPoint{\R}{180}{-90}
        \tdplotCsDrawPoint{\R}{180 + \ALPHA}{-90 + \DELTA}
        \tdplotCsDrawPoint{\R}{180 + \ALPHA}{-90}
        
        % Label arcs
        \tdplotCsLabelPoint{\R}{180 + \ALPHA / 2}{-90}{$\alpha$}{below}
        \tdplotCsLabelPoint{\R}{180 + \ALPHA / 2}{-90 + \DELTA / 2}{$\lambda$}{above right}
        \tdplotCsLabelPoint{\R}{180 + \ALPHA}{-90 + \DELTA / 2}{$\delta$}{left}
        
        % Mark right angle
        \tdplotsetrotatedcoords{\ALPHA}{0}{0};
        \draw [tdplot_rotated_coords](\R,-0.2,0.2) coordinate (c) (\R,0,0.2) coordinate (a1) -- (c) (\R,-0.2,0) coordinate (a2) -- (c) pic [angle radius=0.2cm] {right angle=a1--c--a2};
        
        % Mark and label angle t near P
        \def\angleRadius{0.6}
        \tdplotsetrotatedcoords{0}{0}{0};
        \draw [tdplot_rotated_coords, canvas is yz plane at x = \R]({\angleRadius * cos(\EPS)},0) arc (0:{\EPS - 3}:\angleRadius);
        \tdplotCsLabelPoint{\R}{6}{88.3}{$\varepsilon$}{left}
    \end{tikzpicture}    
    \caption{}
    \label{pic:sun-lamdda-alpha}
\end{wrapfigure}

Далее найдём выражение для прямого восхождения Солнца $\alpha$ через переменные $\lambda$ и $\varepsilon$, где $\varepsilon = 23.44^\circ$~--- угол наклона экватора Земли к эклиптике. Для этого рассмотрим формулу пяти элементов для прямоугольного сферического треугольника, представленного на рисунке~\picRef{pic:sun-lamdda-alpha}:
\begin{gather}
    \sin \delta \cos 90^\circ = \cos \lambda \sin \alpha - \sin \lambda \cos \alpha \cos \varepsilon,\notag\\
    \cos \lambda \sin \alpha = \sin \lambda \cos \alpha \cos \varepsilon,\notag\\
    \frac{\tg \alpha }{\tg \lambda} = \cos \varepsilon.
\end{gather}
Теперь запишем формулу тангенса половинного угла для $\tg \sfrac{\varepsilon}{2}$ и воспользуемся полученным выражением для $\cos \varepsilon$:
\begin{gather*}
    \tg^2 \frac{\varepsilon}{2} = \frac{1 - \cos \varepsilon}{1 + \cos \varepsilon} \equiv y ,\\
    1 - \frac{\tg \alpha}{\tg \lambda} = y \cdot \left( 1 + \frac{\tg \alpha}{\tg \lambda} \right),\\
    \sin \lambda \cos \alpha - \cos \lambda \sin \alpha = y \cdot \left( \sin \lambda \cos \alpha + \cos \lambda \sin \alpha \right),\\
    \sin ( \lambda - \alpha ) = y \sin(\alpha + \lambda),\\
    \alpha = \lambda - \arcsin \left( y \sin( \alpha + \lambda) \right).
\end{gather*}
Отметим, при $\varepsilon = 0$ выполняется $\alpha = \lambda$. Следовательно, можно сделать нулевое приближение $\alpha_0 = \lambda$. Воспользуемся методом последовательных приближений для получения более точного выражения для $\alpha(\lambda, \varepsilon)$.
\begin{gather}
    \alpha_1 = \lambda - \arcsin \left( y \sin (\alpha_0 + \lambda)  \right) \overset{\varepsilon \ll 1}{\simeq} \lambda - y \sin 2 \lambda,\nonumber\\ 
    \alpha_2 
        = \lambda - \arcsin \left( y \sin (\alpha_1 + \lambda) \right)
        \overset{\varepsilon \ll 1}{\simeq} \lambda - y \sin 2 \lambda + \frac{y^2}{2} \sin 4 \lambda. \label{eq:second-approx-alpha-lambda}
\end{gather}
Используем~\eqref{eq:second-approx-alpha-lambda} для записи уравнения времени:
\begin{multline}
    \nu    
        = t_\text{ист} - t_\text{ср}
        = \alpha_2 - (M + \omega) = \\
        = \lambda - y \sin 2 \lambda + \frac{y^2}{2} \sin 4 \lambda - M - \omega = \\
        = \nu - y \sin (2\nu + 2\omega)  + \frac{y^2}{2} \sin (4\nu + 4\omega)  - M \simeq \\
        \overset{\varepsilon \ll 1,\, e \ll 1}{\simeq} \!\!\underbracket[0.5pt]{~2e \sin M\,}_\text{эксц-ть орб.} - \underbracket[0.5pt]{\tg^2 \frac{\varepsilon}{2} \sin (2M + 2\omega)}_\text{наклон орбиты}.
        \label{eq:time-eq-M}
\end{multline}
Подставим~в~\eqref{eq:time-eq-M} параметры орбиты Земли:
\begin{equation*}
    e = 0.0167,~\varepsilon = 23.44^\circ,~\omega = 102.9^\circ,
\end{equation*}
чтобы получить уравнение времени в минутах,~\lookPicRef{pic:time-eq}:
\begin{equation}
    \eta = t_\text{ист} - t_\text{ср} =  -7.65 \sin \frac{2\pi d}{P} + 9.86 \sin 2 \left( 1.80 + \frac{2\pi d}{P} \right)~\text{мин},
\end{equation}
где $T_\text{сид}$~--- сидерический год (здесь не учитываются поправки, связанные с прецессией Земной оси), а $d$~--- время от момента прохождения точки перицентра (в современную эпоху это происходит в период со 2 по 5 января).



