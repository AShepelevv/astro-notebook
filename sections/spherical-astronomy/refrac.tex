\subsection{Рефракция}
\term{Рефракция}~--- явление преломления световых лучей, приходящих от небесных светил, в атмосфере планеты. Для наблюдателя на поверхности планеты с атмосферой положение светила будет отличаться от истинного на некоторый угол.

Для зенитного расстояния $z < 70^\circ$ величины рефракции можно определить по формуле
\begin{equation}
	\rho = 60.25'' \cdot \tg z' \cdot \frac{p}{760} \frac{273^{\circ}}{273^{\circ}+ t^{\circ}},
	\label{eq:refrac}
\end{equation}
где $t^{\circ}$~--- температура воздуха в$~^\circ$C, $p$~--- атмосферное давление в мм~рт.\,ст., $z'$~--- видимое зенитное расстояние. При н.~у.: $p = 760$ мм~рт.\,ст. и $t = 0^{\circ}$C, формула \eqref{eq:refrac} принимает вид
\begin{equation}
	\rho = 60.25'' \cdot \tg z'.
\end{equation}

\begin{wrapfigure}{l}{0.35\tw}
	\centering
	\vspace{-1pc}
	\begin{tikzpicture}[scale=1]
		\footnotesize
		\coordinate (O) at (0, 0) {};
		
		\draw (2, 0) arc(0:110:2);
		\draw (3, 0) arc(0:105:3);
		
		\draw (0, .3) arc(90:41.8:.3);
		\draw [double, line cap=butt](1.94, 2) arc(180:221.8:.3);
		\draw [decoration={snake, segment length=1mm, amplitude=0.3mm}, decorate](2.52, 1.89) arc(-21.5:41.8:.3);
		\draw (0, 1.8) -- (.2, 1.8) -- (.2, 2);
		
		\draw (O) -- (0, 3);
		\draw (0,2) -- (2.24, 2) -- (3, 1.7);
		\draw [-latex] (0, 2) -- (1.24, 2);
		\draw [-latex] (2.24, 2) -- (2.848, 1.76);
		\draw [dashes] (O) -- (2.98, 2.67);
		
		\draw (O) node[anchor=north east] {$C$};
		\draw (0, 3) node[anchor=south east] {$Z$};
		\draw (0, 2) node[anchor=south east] {$O$};
		\draw (0, 1) node[anchor=east] {$R_\oplus$};
		\draw (0.75, 0.72) node[anchor=north west] {$R_\oplus$};
		\draw (1.85, 1.72) node[anchor=north west] {$h$};
		
		\draw (.17, .25) node[anchor=south] {$\alpha$};
		\draw (1.95, 2.05) node[anchor=north east] {$\beta$};
		\draw (2.55, 2.07) node[anchor= west] {$\beta'$};
		
		\draw [fill=white] (O) circle (.03);
		\draw [fill=white] (0, 2) circle (.03);
		\draw [fill=white] (0, 3) circle (.03);
	\end{tikzpicture}
	\caption{}
	\label{pic:refraction}
\end{wrapfigure}

Однако для расчета рефракции у горизонта данная формула не подходит. Получим оценку на величину рефракции у горизонта, считая атмосферу Земли однородной, положив её высоту $h$ равной 8~км. Для этого рассмотрим луч зрения, лежащий в плоскости математического горизонта наблюдателя. Найдем угол между лучом и касательной плоскостью к верхней границе атмосферы в точке пересечения этой границы лучом (см.~Рис.\,\ref{pic:refraction}):
\begin{equation*}
	\beta = 90^\circ - \alpha = 90^\circ - \arccos \frac{R_\oplus}{R_\oplus + h} = 87.13^\circ.
\end{equation*}
Для любой другой высоты атмосферы и радиуса планеты расчёт производится ровно также с соответствующими параметрами.

Коэффициент преломления воздуха $n_\text{в}$ при давлении $p = 1$~атм и температуре $t=0^\circ\text{C}$ равен $1 + 2.9\times 10^{-4}$. Следовательно угол преломления будет равен $\beta' = \arcsin n_\text{в} \sin \beta = 87.48^\circ$. Величина отклонения луча и есть рефракция, то есть $\rho = \beta' - \beta = 0.35^\circ$.

Реальное же значение рефракции $\rho_0$ у горизонта составляет около $0.5^\circ$.

