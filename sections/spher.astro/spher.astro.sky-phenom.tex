\subsection{Явления, связанные с суточным вращением небесной сферы}
\term{Верхняя} и \term{нижняя кульминация}~--- моменты пересечения светилом небесного меридиана, причём при верхней кульминации светило имеет наибольшую высоту, а при нижней~--- наименьшую.

Высота светила в верхней и нижней кульминации со склонением $\delta<\varphi$ (кульминирует к югу от зенита, азимут $A=0^{\circ}$) вычисляется следующим образом соответственно:
\begin{equation}
h_{\text{В}}= 90-\varphi+\delta\text{   }
h_{\text{Н}}= \varphi-90+\delta
\end{equation}

Если светило имеет склонение $\delta>\varphi$ (кульминирует к северу от зенита, азимут $A=180^{\circ}$), то высота в верхней и нежней кульминации вычисляется так:
\begin{equation}
h_{\text{В}}= 90+\varphi-\delta\text{   }
h_{\text{Н}}= -90-\varphi-\delta
\end{equation}

Условие, при котором светило восходит или заходит:
\begin{equation}
|\delta|<(90^{\circ}-|\varphi|)
\end{equation}

Для вычисления часового угла светила используются следующая формула:
\begin{equation}
\cos t=\frac{\cos z-\sin\varphi\sin\delta}{\cos\varphi\cos\delta}
\end{equation}
Отсюда следует, что часовой угол захода и восхода светила: $\cos t=-\tg\varphi\cdot\tg\delta$

Для вычисления азимута светила используется формула:
\begin{equation}
\cos A=\frac{\cos\delta\cos t-\cos\varphi\cos z}{\sin\varphi\sin z}
\end{equation}

Следовательно, азимут точки восхода~--- $A_{\text{В}}=A=\arccos(-\sin\delta/\cos \varphi)$, а точки захода~--- $A_{\text{Н}}=360-A$.

\term{Звёздное время}~--- часовой угол точки весеннего равноденствия. Звёздное время светила вычисляется так:
\begin{equation}
s_{\text{в}}=\alpha+t
\end{equation}