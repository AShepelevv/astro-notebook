\subsection{Рефракция}

До того, как лучи света от объекта на небесной сфере попадут в глаз наблюдателя, они проходят через атмосферу Земли и преломляются в ней. Вследствие этого светило нам видится чуть выше, чем оно было бы в отсутствия атмосферы. Это явление называется \term{рефракцией}~--- преломление в атмосфере световых лучей от небесных светил. Величина рефракции пропорциональна зенитному расстоянию светила. Средняя величина рефракции у горизонта равна $35'$.

Для $z<70$ действует приближённая формула для величины рефракции:
\begin{equation}
\rho=60.25''\cdot\tg z'\cdot \frac{p}{760}\frac{273^{\circ}}{273^{\circ}+t^{\circ}},
\end{equation}
где $t^{\circ}$~--- температура воздуха, $p$~--- атмосферное давление в мм рт. ст., $z'$~--- видимое зенитное расстояние. При н.у. ($p=760$ мм рт. ст. и $t=0^{\circ}$) формула принимает следующий вид:
\begin{equation}
\rho=60.25''\cdot\tg z'
\end{equation}