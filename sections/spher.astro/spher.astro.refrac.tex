\subsection{Рефракция}
\term{Рефракция}~--- явление преломления световых лучей, приходящих от небесных светил, в атмосфере планеты. Для наблюдателя на поверхности планеты с атмосферой положение светила будет отличаться от истинного на некоторый угол. Средняя величина рефракции у горизонта для земной атмосферы равна $35'$.

Для зенитного расстояния $z < 70^\circ$ величины рефракции можно определить по формуле
\begin{equation}
\rho = 60.25'' \cdot \tg z' \cdot \frac{p}{760} \frac{273^{\circ}}{273^{\circ}+ t^{\circ}},
	\label{eq:refrac}
\end{equation}
где $t^{\circ}$~--- температура воздуха в$~^\circ$C, $p$~--- атмосферное давление в мм~рт.\,ст., $z'$~--- видимое зенитное расстояние. При н.~у.: $p = 760$ мм~рт.\,ст. и $t = 0^{\circ}$C, формула \eqref{eq:refrac} принимает вид
\begin{equation}
	\rho = 60.25'' \cdot \tg z'.
\end{equation}