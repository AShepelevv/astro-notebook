\subsection{Сферическая тригонометрия}
Некоторые задачи астрономии, связанные с видимыми положениями небесных тел, сводятся к решениею \term{сферических треугольников}~--- фигур на поверхности сферы, состоящие из трёх точек и трёх дуг больших кругов, соединяющих эти точки.

Свойства сферических треугольников:
\begin{enumerate}
\item Помимо трёх признаков равенства плоских треугольников, для сферических треугольников верен ещё один: два сферических треугольника равны, если их соответствующие углы равны.
\item Два сферических треугольника равны, если они подобны.
\item Для сторон сферического треугольника выполняются 3 неравенства треугольника: каждая сторона меньше суммы двух других сторон и больше их разности.
\item Сумма всех сторон $a+b+c$ всегда меньше $360^{\circ}$.
\item Сумма углов сферического треугольника $s=\alpha +\beta +\gamma$ всегда меньше $540^{\circ}$  и больше $180^{\circ}$
\item Если от двух углов сферического треугольника отнимем третий, получим угол, меньший $180^{\circ}$
\item Площадь сферического треугольника определяется по формуле:
\begin{equation}
s=\sigma\frac{\pi R^2}{180^{\circ}},
\end{equation}
где $\sigma$~--- \term{сферический избыток}, равный разности суммы трёх углов сферического треугольника и $180^{\circ}$.
\end{enumerate}

Возьмём сферический треугольник $ABC$, образованный на сфере радиуса $R$ и с центром в точке $O$. Тогда \term{сферическая теорема} косинусов будет иметь сдедующий вид:

