\subsection{Солнечное время. Уравнение времени}
\term{Истинные солнечные сутки}~--- промежуток времени между двумя последовательными одноимёнными кульминациями Солнца.

\term{Истинное солнечное время}~--- промежуток времени между нижней кульминацией Солнца и любым другим его положением. Рассчитывается по следующей формуле:
\begin{equation}
T_{\text{ист}}=t_{\text{сол}}+12^h,
\end{equation}
где $t_{\text{сол}}$~--- часовой угол Солнца.

\term{Среднее солнечное время} ($T_m$)~--- промежуток времени между нижней
кульминацией среднего Солнца и любым другим его положением. \imp{Среднее Солнце}~--- точка небесной сферы, которая равномерно движется по небесному экватору, совершая полный оборот вокруг точки весеннего равноденствия за тропический год. Зная долготу наблюдателя, нетрудно вычислить среднее солнечное время:
\begin{equation}
T_m = \text{UTC} + \lambda,
\end{equation}
где UTC~--- \imp{всемирное время}~--- среднее солнечное время в Гринвиче.

\term{Поясное время}~--- местное среднее солнечное время на срединном меридиане географического часового пояса. В России также установлено декретное время, которое на 1 час больше поясного.


\term{Уравнение времени}~--- разница между истинным солнечным временем и средним солнечным временем:
\begin{equation}
\eta=T_{\text{ист}}-T_m
\end{equation}
\begin{figure}[h!]
	\centering
	\vspace{-.7pc}
	\begin{tikzpicture}
 		\begin{axis}[
 						width	=	.6\tw, 
 		 				height	=	6cm,
 		 				xmax 	=	1, 
 		 				xmin	=	0, 
 		 				ymax	=	20, 
 		 				ymin 	= 	-20,
 		  			]
  			\addplot [domain=0:1, samples = 100, black, smooth] {7.53 * cos(360*(x - 81/365)) + 1.5 * sin(360*(x - 81/365)) - 9.87 * sin(2*360*(x - 81/365))};
		\end{axis}
	\end{tikzpicture}
	\caption{График уравнения времени}
\end{figure}