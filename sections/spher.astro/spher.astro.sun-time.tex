\section{Солнечное время. Уравнение времени}
\term{Истинные солнечные сутки}~--- промежуток времени между двумя последовательными одноимёнными кульминациями Солнца.

\term{Истинное солнечное время}~--- промежуток времени между нижней кульминацией Солнца и любым другим его положением. Рассчитывается по следующей формуле:
\begin{equation}
T_{\text{ист}}=t_{\text{сол}}+12^h,
\end{equation}
где $t_{\text{сол}}$~--- часовой угол Солнца.

\term{Среднее солнечное время} ($T_m$)~--- промежуток времени между нижней кульминацией среднего Солнца и любым другим его положением. \imp{Среднее Солнце}~--- точка небесной сферы, которая равномерно движется по небесному экватору, совершая полный оборот вокруг точки весеннего равноденствия за тропический год. Зная долготу наблюдателя, нетрудно вычислить среднее солнечное время:
\begin{equation}
T_m=UTC+\lambda,
\end{equation}
где $UTC$~--- всемирное время равное среднему солнечному времени в Гринвиче.

\term{Поясное время}~--- местное среднее солнечное время на срединном меридиане географического часового пояса. В России также установлено декретное время, которое на 1 час больше поясного.


\term{Уравнение времени}~--- разница между истинным солнечным временем и средним солнечным временем:
\begin{equation}
\eta=T_{\text{ист}}-T_m
\end{equation}

\centering
 \begin{figure}[!h]
  \centering
  \includegraphics[width=0.87\textwidth]{sun-time.pdf}
  \caption{График уравнения времени}
 \end{figure}