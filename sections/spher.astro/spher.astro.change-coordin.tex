\subsection{Изменение экваториальных координт}
В моменты, когда Солнце находится в \imp{точке весеннего равноденствия} (20 марта, реже 21) его координаты $\alpha=0^h$, $\delta=0^{\circ}$. Во время прохождения этой точки обе координаты Солнца растут. Так происходит до момента, пока Солнце не пересечёт \imp{точку летнего солнцестояния} (21 июня, реже 20). Слонение Солнца начинает уменьшаться. В момент прохождения \imp{точки осеннего равноденствия} (22 или 23 сентября), координаты Солнца $\alpha=12^h$, $\delta=0^{\circ}$. После прохождения \imp{точки зимнего солнцестояния} (22 или 21 декабря) склонение Солнца начинает увеличиваться.

Годичный путь Солнца по небесной сфере можно аппроксимировать синусоидой, откуда очевидным образом получается следующая приближенная формула для расчёта склонения Солнца в заданный момент времени:
\begin{equation}
\delta=\varepsilon\cdot\sin \left(\frac{d}{T}360^{\circ}\right),
\end{equation}
где $\varepsilon$~--- наклон эклиптики к плоскости небесного экватора, $d$~--- порядковый номер дня после весеннего равноденствия, $T$~--- тропический год.

Более точная формула и выглядит следующим образом:
\begin{equation}
\delta=\arcsin\left(\sin\varepsilon\cdot\sin \left(\frac{d}{T}360^{\circ}\right)\right)
\end{equation}

Известно, что движение Солнца по эклиптике происходит неравномерно, поэтому данные формулы не являются верными для точного расчёта.

Прямое восхождение Cолнца связано со склонением данной формулой:
\begin{equation}
\sin\alpha=\frac{\tg\delta}{\tg\varepsilon}
\end{equation}

Большинство формул, представленных выше, следуют из формул перехода между системами координат. 


\begin{figure}[!h]
\centering
 \begin{tikzpicture}
 \begin{axis}[
 no markers,
 xlabel={Прямое восхождение $\alpha^h$}, 
 ylabel={Склонение $\delta^{\circ}$},
 minor x tick num = 1,
 minor y tick num = 1,
 grid = both,
 line width=.7pt, 
 extra y ticks={23.44, -23.44},
 ymax=25,
 ymin=-25,
 xmax=24,
 xmin=0,
 xtick={0,4,8,12,16,20,24}
 ]
 \addplot [domain=0:24, samples=100]{atan(sin(x*15)*tan(23.44))}; 
 \end{axis}
 \end{tikzpicture}
 \caption{График зависимости склонения от прямого восхождения Солнца}
\end{figure}