\subsection{Изменение экваториальных координт}
\begin{wrapfigure}[12]{r}{0.5\tw}
	\centering
	\vspace{-.7pc}
 	\begin{tikzpicture}
 		\begin{axis}[
 						width	=	.5\tw,
 						height	=	4.5cm,
 						xlabel	=	{Прямое восхождение $\alpha^h$}, 
 						ylabel	=	{Склонение $\delta^{\circ}$},
 						extra y ticks	=	{23.44, -23.44},
 						ytick = {-20, -10, 0, 10, 20},
 						ymax	=	25,
 						ymin	=	-25,
 						xmax	=	24,
 						xmin	=	0,
 						xtick	=	{0, 4, 8, 12, 16, 20, 24}
 					]
 			\addplot [domain=0:24, samples=100] {atan(sin(x*15)*tan(23.44))}; 
 		\end{axis}
 	\end{tikzpicture}
 	\caption{График зависимости склонения от прямого восхождения Солнца}
\end{wrapfigure}
В моменты, когда Солнце находится в \imp{точке весеннего равноденствия}  (20 марта, реже 21) его координаты $\alpha=0^h$, $\delta=0^{\circ}$. Во время прохождения этой точки обе координаты Солнца растут. Так происходит до момента, пока Солнце не пройдет \imp{точку летнегосолнцестояния} (21 июня, реже 20), после этого склонение Солнца начинает уменьшаться. В момент прохождения \imp{точки осеннего равноденствия} (22 или 23 сентября), координаты Солнца составляют $\alpha=12^h$, $\delta=0^{\circ}$. После прохождения \imp{точки зимнего солнцестояния} (22 или 21 декабря) склонение Солнца начинает увеличиваться.

Годичный путь Солнца по небесной сфере можно аппроксимировать синусоидой, откуда очевидным образом получается следующая приближенная формула для расчёта склонения Солнца в заданный момент времени:
\begin{equation}
\delta=\varepsilon\cdot\sin \left(\frac{2 \pi d}{T}\right),
\end{equation}
где $\varepsilon$~--- наклон эклиптики к плоскости небесного экватора, $d$~--- порядковый номер дня после весеннего равноденствия, $T$~--- тропический год.

Более точная формула следует из сферической тригонометрии и имеет вид
\begin{equation}
\delta=\arcsin\left(\sin\varepsilon\cdot\sin \left(\frac{2 \pi d}{T}\right)\right)
\end{equation}

Известно, что движение Солнца по эклиптике происходит неравномерно, поэтому данные формулы не являются верными для точного расчёта.

Прямое восхождение Cолнца связано со склонением данной формулой:
\begin{equation}
\sin\alpha=\frac{\tg\delta}{\tg\varepsilon}
\end{equation}

Большинство формул, представленных выше, следуют из формул перехода между системами координат. 
