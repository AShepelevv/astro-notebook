\subsection{Чёрные дыры}
\term{Чёрная дыра}~(ЧД)~--- область пространства-времени с массой $M$, гравитационное притяжение которой настолько велико, что покинуть её не могут даже объекты, движущиеся со скоростью света $c$. Граница этой области называется \imp{горизонтом событий}, а её характерный размер~$R_G$~--- \imp{гравитационным радиусом}, для величины которого справедливо равенство
\begin{equation}
    R_G = \frac{2 G M}{c^2}.
\end{equation}

Минимальная масса ЧД составляет около $2.5M_{\odot}$. А плотность ЧД определяется отношением ее массы~$M$ к~объему~$V$, следовательно
\begin{equation}
    \rho = \frac{M}{V} = \frac{3c^6}{32\pi M^2G^3}.
\end{equation}

\term{Эффект излучения} (испарения) \term{Хокинга}~--- эффект, при котором гравитационное поле черной дыры поляризует вакуум, в результате чего возможно образование не только виртуальных, но и реальных пар частица~--античастица. Одна из частиц, оказавшаяся чуть ниже горизонта событий, падает внутрь чёрной дыры, а другая, оказавшаяся чуть выше горизонта, улетает, унося энергию (то есть часть массы) чёрной дыры. Для мощности излучения ЧД справедлива формула
\begin{equation}
    L = \frac{h c^6}{30720 \pi^2 G^2 M^2},
\end{equation}
где $h$ --- постоянная Планка. Спектр хокинговского излучения для безмассовых полей оказался строго совпадающим с излучением абсолютно чёрного тела, что позволило приписать ЧД температуру, равную
\begin{equation}
    T = \frac{h c^3}{16 \pi^2 k G M},
\end{equation}
где $k$ --- постоянная Больцмана.
