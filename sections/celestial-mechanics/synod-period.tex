\subsection{Синодический период}

\term{Синодический период} (период смены фаз) небесного тела~--- период относительного кругового движения.

В большинстве случаев имеется в виду относительное движение планет вокруг Солнца (А). Также может рассматриваться период смены фаз спутника планеты (Б) или период между последовательными пролетами спутника над одним и тем же меридианом планеты (В).

Важным понятием в определении синодического периода является \imp{относительная угловая скорость}~--- скорость увеличения углового разделения двух рассматриваемых объектов. Так в случае (A) это скорость изменения угла <<{\slshape одна планета\,--\,Солнце\,--\,другая планета}>>, в случае (Б)~--- угла <<{\slshape спутник\,--\,планета\,--\,Солнца}>>, наконец, в случае (В)~--- угла между отрезком <<{\slshape центр планеты\,--\,спутника}>> и меридианом планеты.

Численно относительная скорость равна разности угловых скоростей двух рассматриваемых движений.
\begin{equation*}
    \boldsymbol\omega_\text{отн} = \boldsymbol{\omega}_1 - \boldsymbol\omega_2.
\end{equation*} 
Обозначим за $S$ синодический период, тогда
\begin{equation*}
    \frac{2\pi}{S} = |\boldsymbol\omega_\text{отн}| = |\vec\omega_1 - \vec\omega_2| = \left| \frac{2\pi}{T_1} \pm \frac{2\pi}{T_2} \right|,
\end{equation*}
где выбор знака в последнем выражении зависит от взаимного направления векторов угловых скоростей, а значит от направления каждого из движений. Окончательно, для синодического периода верно,
\begin{equation}
    \frac{1}{S} = \left| \frac{1}{T_1} \pm \frac{1}{T_2} \right|.
\end{equation}
    
Для внешних и внутренних планет, соответственно, выражения принимают вид:
\begin{equation} \frac{1}{S} = \frac{1}{T_\oplus} - \frac{1}{T_\text{пл}} \quad \text{и} \quad \frac{1}{S} = \frac{1}{T_\text{пл}} - \frac{1}{T_\oplus},
\end{equation}
где $S$~--- синодический период, $T_\text{пл}$~--- сидерический период планеты, $T_\oplus$~--- сидерический период обращения Земли.

В случае, если тела обращаются в противоположные стороны, связь
их синодического периода с сидерическими очевидным образом принимает вид:
\begin{equation}
	\frac{1}{S} = \frac{1}{T_1} + \frac{1}{T_2}.
\end{equation}

{\footnotesize
Рассмотрим подробнее описанные выше случаи на конкретных примерах.\newline
{\bfseries (A) Относительное движение двух планет вокруг Солнца.} Для анализа возьмем пару Земля и Венера. Период Земли $T_\oplus = 1$~год, Венеры~--- $T_{\venus} = 0.62$~года. Следовательно, синодический период (период смены фаз) Венера для земного наблюдателя, также синодический период Земли для наблюдателя на Венера, определяется как
\begin{equation*}
    \frac{1}{S_{\oplus\venus}} = \left| \frac{1}{T_\oplus} - \frac{1}{T_{\venus}} \right| = \frac{1}{T_{\venus}} - \frac{1}{T_\oplus} = \frac{1}{0.62~\text{года}} - \frac{1}{1~\text{год}} = 0.61~\frac{1}{\text{год}}~~\Rightarrow~~S_{\oplus\venus} = 1.63~\text{года},
\end{equation*}
важно отметить, что здесь имеется в виду не календарный, а сидерический год Земли.\newline
{\bfseries (Б) Период смены фаз спутника планеты.} Найдём период смены фаз естественного спутника Земли~--- Луны. Движения, рассматриваемые в данном примере, это обращение Луны вокруг Земли с сидерическим периодом Луны $T_{\rightmoon} = 27.3$~суток и обращение Земли вокруг Солнца с сидерическим периодом Земли $T_\oplus \simeq 365$~суток. Следовательно, период смены фаз Луны
\begin{equation*}
    S_{\rightmoon} = \left| \frac{1}{T_\oplus} - \frac{1}{T_{\rightmoon}}\right|^{-1} = \left(\frac{1}{T_{\rightmoon}} - \frac{1}{T_\oplus}\right)^{-1} = \left(\frac{1}{27.3} - \frac{1}{365}\right)^{-1} = 29.5~\text{суток}.
\end{equation*}
{\bfseries (В) Орбитальное движение и вращение вокруг своей оси.} Несколько переиначим пример описанный выше: рассмотрим движение Земли вокруг Солнца и суточное вращение Земли вокруг своей оси, найдём продолжительность солнечных суток на Земле. Сидерический период Земли $T_\text{орб} \simeq 365$~суток (солнечных), а звёздные сутки $T_\text{ос} = 23^\text{h}\,56^\text{m}\,4^\text{s}$. Отсюда,
\begin{equation*}
    T_\text{сут} = \left(\frac{1}{T_\text{ос}} - \frac{1}{T_\text{орб}}\right)^{-1} = 24~\text{часа}.
\end{equation*}
}


