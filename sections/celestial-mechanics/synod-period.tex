\subsection{Синодический период}

\term{Синодический период} (период смены фаз)~--- время, прошедшее между двумя последовательными одноимёнными конфигурациями одного тела при наблюдении с другого.

\imp{Относительная угловая скорость} планет равна
разности скоростей углового перемещения одной планеты ($2\pi/T_1$) и другой ($2\pi/T_2 $) по орбите. Из определения относительной угловой скорости вытекает общая формула для продолжительности синодического периода:
\begin{equation}
	\frac1S=\left| \frac{1}{T_1}-\frac{1}{T_2} \right|.
\end{equation}
Для внешних и внутренних планет соответственно выражения принимают следующий вид:
\begin{equation} \frac{1}{S} = \frac{1}{T_\oplus} - \frac{1}{T_\text{пл}} \quad \text{и} \quad \frac{1}{S} = \frac{1}{T_\text{пл}} - \frac{1}{T_\oplus},
\end{equation}
где $S$~--- синодический период, $T_\text{пл}$~--- сидерический период планеты, $T_\oplus$~--- сидерический период обращения Земли.

В случае, если тела обращаются в противоположные стороны, то связь
их синодического периода с сидерическими очевидным образом принимает вид:
\begin{equation}
	\frac{1}{S} = \frac{1}{T_1} + \frac{1}{T_2}.
\end{equation}
