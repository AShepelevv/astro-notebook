\subsection{Гравитационный маневр}
\label{sec:grav-assist}

\term{Гравитационный маневр}~--- изменение направления и скорости движения космического аппарата, а как следствие и параметров орбиты, в результате гравитационного взаимодействия с массивным третьим телом. Может осуществляться с применением тяги двигателя и без.

В общем случае рассматривается незамкнутое движение аппарата вокруг третьего тела, иначе  аппарат становится его спутником. Рассмотрим гравитационный маневр без двигателя подробнее и найдем максимальное достижимое в этой случае изменение скорости.

Из закона сохранения энергии ясно, что максимальная скорость достигается на минимальном расстоянии от гравитирующего тела, а из минимальности расстояния следует перпендикулярность вектора скорости к радиус-вектору аппарата в этот момент. Исходя из этого запишем законы сохранения для момента пересечения сферы влияния тела и момента максимального сближения с ним:
\begin{align*}
    \text{ЗСМИ:}&\quad V_\infty b = q V_\text{макс},\\
    \text{ЗСЭ:}&\quad \frac{V_\infty^2}{2} = \frac{V_\text{макс}^2}{2} - \frac{GM}{q}.
\end{align*}
Откуда выразим максимальную скорость:
\begin{equation*}
    V_\text{макс} = \sqrt{V_\infty^2 + \frac{2GM}{q}} = \sqrt{V_\infty^2 + \frac{2 V_1^2}{ \rho}},\\ \quad \rho \equiv \frac{q}{R_\text{пл}} > 1,
\end{equation*}
где $V_1$~--- первая космическая скорость на поверхности тела, и прицельный параметр гиперболической орбиты аппарата
\begin{equation*}
     b = q \frac{V_\text{макс}}{V_\infty} = a(e - 1)\sqrt{1 + \frac{2 V_1^2}{\rho V_\infty^2}} = a\sqrt{e^2 - 1},
\end{equation*}
отсюда получаем уравнение на эксцентриситет:
\begin{gather*}
    \sqrt{\frac{e + 1}{e - 1}} 
        = \sqrt{1 + \frac{2 V_1^2 }{ \rho V_\infty^2}}
        = \sqrt{1 + \frac{2 }{\rho\nu^2}}, 
        \quad \nu \equiv \frac{V_\infty}{V_1} > 0;\\
    e(\rho, \nu) = \rho\nu^2 + 1.
\end{gather*}

Далее найдем зависимость угла поворота, что то же самое, угла между асимптотами орбиты, от входных параметров:
\begin{gather*}
    \alpha = 2 \arctg \frac{b}{a} = 2 \arctg \frac{a \sqrt{e^2 - 1}}{a} = 2 \arctg \sqrt{e^2 - 1},\\
    \alpha(\rho, \nu) = 2 \arctg \sqrt{(\rho\nu^2 + 1)^2 - 1},\\
    \varphi(\rho, \nu) = \pi - \alpha( \rho, \nu).
\end{gather*}

Остается с помощью теоремы косинусов определить приращение скорости, получаемое при гравитационном маневре, 
\begin{gather*}
    (\Delta V)^2 = 2 V_\infty^2 - 2 V_\infty^2 \cos \phi = 2 V_\infty^2(1 + \cos \alpha),\\
    \frac{\Delta V}{V_1} = \nu \sqrt{2 \left[ 1 + \cos \left(2 \arctg \sqrt{(\rho\nu^2 + 1)^2 - 1} \right) \right]}. 
\end{gather*}

Для нахождения максимального приращения скорости необходимо решить задачу оптимизации, решение которой выходит за рамки этой книги, поэтому сразу приведем ответ:
\begin{equation*}
    (\rho_0, \nu_0)
        = \underset{\rho \geqslant 1,~\nu \geqslant 0} \argmax\, \frac{\Delta V (\rho, \nu)}{V_1} 
        = (1, 1).
\end{equation*}
Следовательно, максимальное приращение скорости при гравитационном маневре с выключенным двигателем составляет
\begin{equation*}
    (\Delta V)_\text{макс}
%        = \nu_0 V_1 \sqrt{2 \left[ 1 + \cos \left(2 \arctg \sqrt{(\rho_0\nu_0^2 + 1)^2 - 1} \right) \right]}
        = \sqrt{\frac{G M}{R}}.
\end{equation*}

\begin{wrapfigure}[11]{r}{0.45\tw}
    \vspace{-1.2pc}
    \begin{tikzpicture}
		\begin{axis} [
			width    = 4cm,
			height   = 4cm,
			colormap = {GS}{rgb(0cm)=(0.3, 0.3, 0.3) rgb(1cm)=(1, 1, 1)},
			xlabel 	 = {$\rho$},
			ylabel 	 = {$\nu$},
			zlabel 	 = {$I/I_0$},
			view	 = {0}{90},
			ytick    = {0,1,2,3,4},
			xtick	 = {1,2,3,4,5},
			colorbar,
			colorbar style = {ytick = {0, .2, .4, .6, .8, 1.}}
		]
				
			\addplot3[surf,shader=interp] table[x=r, y=v, z=D] {data/grav-assist.txt};
		\end{axis}
	\end{tikzpicture}
	\caption{Зависимость величины приращения скорости $\frac{\Delta V}{V_1}$ от параметров $\rho$ и $\nu$}
\end{wrapfigure}
Важно понимать, полученная величина достигается только при определенной скорости входа в зону гравитационного влияния тела, а также при сверхблизком по меркам современной космонавтики пролете. Поэтому в действительности гравитационные маневры добавляют лишь некоторую часть этой скорости, которая используется для ускорения, торможения или поворота.

Отличительной особенность гравитационного маневра с включением двигателя является возможное различие скорости входа и выхода из зоны влияния тела. Тем самым можно достичь произвольного изменения скорости, если это позволяют запасы топлива на борту аппарата.

