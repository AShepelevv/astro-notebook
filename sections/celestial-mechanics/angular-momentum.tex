\subsection{Закон сохранения момента импульса}
\term{Момент импульса} ($\vec{L}$)~--- \imp{псевдовекторная} величина, определяющая момент количества движения (импульса). Согласно этому определению момент импульса $\vec{L}$ пробной массы зависит от её радиус-вектора и импульса следующий образом:
\begin{equation}
	\vec{L} = \cross{r}{p}.
\end{equation}

Из геометрического смысла векторного произведения ясно, что $\vec{L}$ ортогонален векторам $\vec{r}$ и $\vec{p}$. Отсюда же $|\vec{L}| \equiv L$ равен площади параллелограмма, построенного на векторах $\vec{r}$ и $\vec p$. Иными словами $L = rp_\perp$, что в свою очередь можно записать как $ L = r p \sin \alpha$, где $\alpha$~--- угол поворота от $\vec{r}$ к $\vec p$, учитывая направление. Согласно свойству векторного произведения, направление $\vec L$ определяется правилом правой руки~--- тройка векторов $\vec r$, $\vec p$ и $\vec L$ должна быть правой.

Для системы пробных масс\footnote{Здесь и далее в этом разделе под пробной массой имеется в виду точечная масса, чтобы не брать в рассмотрение момент импульса вращения тел.}, суммарный момент действия внешних сил на которую равен нулю, полный момент импульса сохраняется. Данное утверждение называется \term{законом сохранения момента импульса}.

Докажем его сначала в случае движения постоянной пробной массы~$m$ в гравитационном поле неподвижного массивного тела с массой~$M$. Рассмотрим для этого производную по времени момента импульса данной пробной массы:
\begin{multline}
	\frac{d\vec{L}}{dt} = \frac{d\cross{r}{p}}{dt} = \left[ \frac{d \vec{r}}{dt} \times \vec{p} \right] + \left[\vec{r} \times \frac{d\vec{p}}{dt} \right] = m \underbrace{\cross{v}{v}}_{\vec{0}} + \left[\vec{r} \times m \frac{d\vec{v}}{dt} \right] = \\
	= m \cross{r}{a} = m \cross{r}{g(r)} =  -\frac{G M m}{r^3} \underbrace{ \left[\vec{r} \times \vec{r}\right]}_{\vec{0}} = 0.
\end{multline}
Равенство нулю производной момента импульса доказывает его постоянство в рассматриваемом случае.

Обобщим теперь на случай произвольного конечного числа $n$ пробных масс $m_1, \ldots, m_n$:
\begin{multline*}
	\frac{d\vec{L}}{dt} =  \sum\limits_{i=1}^n \frac{d[\vec{r}_i \times \vec{p}_i]}{dt} \overset{m_i=\const}{=} \sum\limits_{i=1}^n \left[\frac{d\vec{r}_i}{dt} \times m_i\vec{v}_i \right] + \sum\limits_{i=1}^n \left[\vec{r}_i \times m_i \frac{d\vec{v}_i}{dt} \right] = \\
	= \sum\limits_{i=1}^n \underbrace{\left[\vec{v}_i \times m_i\vec{v}_i \right]}_{\vec{0}} + \sum\limits_{i=1}^n \left[\vec{r}_i \times m_i \vec{a}_i \right] = \\
	= \sum\limits_{i=1}^n \left[ \vec{r}_i \times m_i \sum\limits_{j=1}^n \frac{G m_j}{|\vec{r}_j - \vec{r}_i|^3} (\vec{r}_j - \vec{r}_i) \right] = \\
	= \sum\limits_{i,j=1}^n \frac{G m_i m_j}{|\vec{r}_j - \vec{r}_i|^3} \big[\vec{r}_i \times (\vec{r}_j - \vec{r}_i) \big] = \sum\limits_{i,j=1}^n \frac{G m_i m_j}{|\vec{r}_j - \vec{r}_i|^3} [\vec{r}_i \times \vec{r}_j] \equiv \sum\limits_{i,j=1}^n \vec{x}_{ij}.
\end{multline*}
Можно заметить, что $\vec{x}_{ij} = -\vec{x}_{ji}$, так как
\begin{equation*}
	\vec{x}_{ij} \equiv \frac{G m_i m_j}{|\vec{r}_j - \vec{r}_i|^3} [\vec{r}_i \times \vec{r}_j] = - \frac{G m_j m_i}{|\vec{r}_i - \vec{r}_j|^3} [\vec{r}_j \times \vec{r}_i] \equiv -\vec{x}_{ji}.
\end{equation*}
Отсюда сразу следует равенство нулю производной по времени полного момента импульса, что завершает доказательство его сохранения.

\begin{figure}[t]
	\begin{subfigure}[b]{0.47\tw}
		\centering
		\begin{tikzpicture}[scale=1.3]
			\footnotesize
			
			\draw [-latex] (0, 0) -- (-2, 0);
			\draw [-latex] (-2, 0) -- (-3, -1);
			
			\draw [dashes, -latex] (-2, 0) -- (-2, -1);
			\draw [dashes] (-2, 0) -- (-3, 0);
			
			\draw (-2.3, 0) arc(180:225:.3);
			\draw (-1.8, 0) -- (-1.8, -.2) -- (-2, -.2);
			
			\draw (0, 0) node {$\odot$};
			
			\point (-2, 0)
			
			\draw (-1, 0) node[anchor=south west] {$\vec{r}$};
			\draw (-2.95, -.9) node[anchor=south] {$\vec{p}$};
			\draw (-2, -.75) node[anchor=west] {$\vec{p}_\perp$};
			\draw (-2.25, -.17) node[anchor=east] {$\alpha$};
			\draw (0, 0) node[anchor=south west] {$\vec{L}$};
			
		\end{tikzpicture}
		\caption{}
	\end{subfigure}
	\hfill
	\begin{subfigure}[b]{0.47\tw}
		\centering
		\begin{tikzpicture}[scale=1]
			\footnotesize
			
			\draw [-latex] (0, 0) -- (0, 2);
			\draw [-latex] (0, 0) -- (2.2, -.6);
			\draw [-latex] (2.2, -.6) -- (3.6, 0);
			\draw [dashes, -latex] (0, 0) -- (1.4, 0.6);
			
			%	\draw [dashes, -latex] (-2, 0) -- (-2, -1);
			\draw [dashes] (2.2, -.6) -- (3, -.82);
			\draw (0, -1) .. controls (2, -1) and (3, -.25) .. (3.5, .4);
			
			\draw (0, .2) -- (.2, 0.15) -- (0.2, -.05);
			\draw (0, .25) -- (.25, 0.35) -- (0.25, .1);
			
			\point (2.2, -.6);
			
			\draw (2.5, -.68) .. controls (2.65, -.6) and (2.65, -.5) .. (2.5, -.47);
			
			\draw (1.1, -.3) node[anchor=south] {$\vec{r}$};
			\draw (3.4, -.1) node[anchor=north] {$\vec{p}$};
			\draw (1.15, 0.55) node[anchor=south] {$\vec{p}$};
			%	\draw (-2, -.75) node[anchor=west] {$\vec{p}_\perp$};
			\draw (2.6, -.55) node[anchor=west] {$\alpha$};
			\draw (0, 1) node[anchor=east] {$\vec{L}$};
			
		\end{tikzpicture}
		\caption{}
	\end{subfigure}
	\caption{}
\end{figure}

