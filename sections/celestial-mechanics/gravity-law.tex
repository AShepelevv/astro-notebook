\subsection{Закон всемирного тяготения}
Согласно \imp{закону всемирного тяготения}, опубликованному в 1687 году Ньютоном в его книге <<Математические начала натуральной философии>>, сила притяжения между двумя точечными телами с массами $M$ и $m$, находящимися на расстоянии $r$, равна
\begin{equation}
	F=\frac{GMm}{r^2}, 
	\label{eq:grav-law-1}
\end{equation}\nopagebreak где $G\simeq 6.67\cdot 10^{-11}~\text{м}^3 /\left( \text{кг} \cdot \text{с}^2 \right)$~---  \term{гравитационная постоянная}, действует вдоль прямой, соединяющей тела, и направлена в сторону гравитирующего тела.

Пусть радиус-вектор $\vec r$ одного тела откладывается от второго, тогда в векторной форме сила, действующая на первое тело,
\begin{equation}
	\vec F (\vec r) = -\frac{GMm}{r^3} \vec r. 
	\label{eq:grav-law-2}
\end{equation}

Векторное поле $\vec V (\vec r)$ называется \term{потенциальным}, если существует такая функция $\varphi(\vec r)$, что $\vec V (\vec r) = - \nabla \varphi(\vec r)$. Рассмотрим векторное поле гравитационых сил~\eqref{eq:grav-law-2}, оно является сферически симметричным, так как зависит только от расстояния до гравитирующего тела, следовательно
\begin{equation*}
	\vec F (\vec r) = -m \frac{d\varphi(r)}{dr} \, \frac{\vec{r}}{r}
\end{equation*} 
%\begin{multline*}
%    \vec F(\vec r) = - m \nabla \varphi(\vec r) = - m \left( \frac{\partial \varphi(\vec r)}{\partial x} \vec e_x + \frac{\partial \varphi(\vec r)}{\partial y} \vec e_y + \frac{\partial \varphi(\vec r)}{\partial z} \vec e_z \right) = \\
%    - m \left( \frac{\,d \varphi(\vec r)}{\,d r} \frac{\partial r}{\partial x} \vec e_x + \frac{\partial \varphi(\vec r)}{\partial y} \vec e_y + \frac{\partial \varphi(\vec r)}{\partial z} \vec e_z \right) =
%\end{multline*}

Перейдём к скалярным величинам, разделим переменные и проинтегрируем левую и правую часть полученного равенства.
%При интегрировании на интервале $(0, r]$ интеграл расходится, поэтому интегрирование будем вести на интервале $(+\infty, r]$, на котором интеграл сходится.
\begin{gather*}
	\int\limits_{+\infty}^{r} \frac{ F(r) }{m} \,d r = \int\limits_{+\infty}^{r} d\varphi(r),\\
%	\int\limits_{+\infty}^{r} \frac{GM }{r^2} \,d r = \int\limits_{+\infty}^{r} d\varphi(r),\\
	-\left.\frac{GM}{r}\right|_{+\infty}^r = \varphi(r)|_{+\infty}^r,\\
	-\frac{GM}{r} + 0 = \varphi(r) - \varphi(+\infty),\\
	\varphi(r) = -\frac{GM}{r} + \varphi(+\infty).
\end{gather*} 
Видно, что потенциал гравитационного поля определен с точностью до постоянной величины $\varphi(+\infty)$, которую для удобства полагают равной нулю. Окончательно, \term{потенциал} гравитационного поля
\begin{equation}
	\varphi(r) = -\frac{GM}{r}.
\end{equation}

\term{Потенциальная энергия} $U(r)$ тела с массой $m$ в гравитационном поле тела с массой $M$~--- энергия, необходимая, чтобы переместить первое тело с бесконечности на расстояние $r$ от второго. Исходя из определения, потенциальная энергия определяется, как
\begin{equation}
	U(\vec r)
	= \oint\limits_{+\infty}^\vec{r} \big(\vec F( \vec r),d \vec r\big) 
	= \int\limits_{+\infty}^r \frac{GMm}{r^2} \,d r 
%	= GMm \left(-\frac{1}{r} + 0 \right) 
	= -\frac{GMm}{r} = U(r).	
\end{equation}

Напряженность гравитационного поля $g$ называют \term[ускорение свободного падения]{ускорением свободного падения} $g$, вычисляется по формуле
\begin{equation}
	\vec g(\vec r) = -\nabla \varphi(\vec r) = \frac{\vec F( \vec r)}{m} = - \frac{GM}{r^3} \vec r.
	\label{eq:g}
\end{equation} 
Следовательно, (\ref{eq:grav-law-1}) можно записать как
\begin{equation}
	\vec F (\vec r) = m \vec g(\vec r).
\end{equation}
