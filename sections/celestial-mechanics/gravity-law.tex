\subsection{Закон всемирного тяготения}
Согласно \imp{закону всемирного тяготения}, опубликованному в 1687 году Ньютоном в его книге <<Математические начала натуральной философии>>\,\cite{newton1687philosophiae}\cite{newton1846philosophiaeEnglish}, сила притяжения между двумя точечными телами с массами $M$ и $m$, находящимися на расстоянии $r$, равна
\begin{equation}
	F=\frac{GMm}{r^2}, 
	\label{eq:grav-law-1}
\end{equation}\nopagebreak где $G\simeq 6.67\cdot 10^{-11}~\text{м}^3 /\left( \text{кг} \cdot \text{с}^2 \right)$~---  \term{гравитационная постоянная}, действует вдоль прямой, соединяющей тела, и направлена в сторону гравитирующего тела.

Пусть радиус-вектор $\vec r$ одного тела откладывается от второго, тогда в векторной форме сила, действующая на первое тело,
\begin{equation}
	\vec F (\vec r) = -\frac{GMm}{r^3} \vec r. 
	\label{eq:grav-law-2}
\end{equation}

Векторное поле $\vec V (\vec r)$ называется \term{потенциальным}, если существует такая функция $\varphi(\vec r)$, что $\vec V (\vec r) = - \nabla \varphi(\vec r)$. Рассмотрим векторное поле гравитационых сил~\eqref{eq:grav-law-2}, оно является сферически симметричным, так как зависит только от расстояния до гравитирующего тела, следовательно
\begin{equation*}
	\vec F (\vec r) = -m \frac{d\varphi(r)}{dr} \, \frac{\vec{r}}{r}
\end{equation*} 
%\begin{multline*}
%    \vec F(\vec r) = - m \nabla \varphi(\vec r) = - m \left( \frac{\partial \varphi(\vec r)}{\partial x} \vec e_x + \frac{\partial \varphi(\vec r)}{\partial y} \vec e_y + \frac{\partial \varphi(\vec r)}{\partial z} \vec e_z \right) = \\
%    - m \left( \frac{\,d \varphi(\vec r)}{\,d r} \frac{\partial r}{\partial x} \vec e_x + \frac{\partial \varphi(\vec r)}{\partial y} \vec e_y + \frac{\partial \varphi(\vec r)}{\partial z} \vec e_z \right) =
%\end{multline*}

Перейдём к скалярным величинам, разделим переменные и проинтегрируем левую и правую часть полученного равенства.
%При интегрировании на интервале $(0, r]$ интеграл расходится, поэтому интегрирование будем вести на интервале $(+\infty, r]$, на котором интеграл сходится.
\begin{gather*}
	\int\limits_{+\infty}^{r} \frac{ F(r) }{m} \,d r = \int\limits_{+\infty}^{r} d\varphi(r),\\
%	\int\limits_{+\infty}^{r} \frac{GM }{r^2} \,d r = \int\limits_{+\infty}^{r} d\varphi(r),\\
	-\left.\frac{GM}{r}\right|_{+\infty}^r = \varphi(r)|_{+\infty}^r,\\
	-\frac{GM}{r} + 0 = \varphi(r) - \varphi(+\infty),\\
	\varphi(r) = -\frac{GM}{r} + \varphi(+\infty).
\end{gather*} 
Видно, что потенциал гравитационного поля определен с точностью до постоянной величины $\varphi(+\infty)$, которую для удобства полагают равной нулю. Окончательно, \term{потенциал} гравитационного поля
\begin{equation}
	\varphi(r) = -\frac{GM}{r}.
\end{equation}

\begin{wrapfigure}[12]{r}{0.47\tw}
    \centering
    \vspace{-1pc}
    \tikzsetnextfilename{potential-energy}
    \begin{tikzpicture}[scale=0.5]
        \begin{scope}[yscale=0.5]
            \draw [black, semithick] plot [smooth, tension=1] coordinates { (0,0) (1,2) (3,0) (6,1.5) (3,6) (6,10) (9,8) (8,11) (10,12)};
        \end{scope}
        
        \tkzDefPoint(0, 0){R0}
        \tkzDefPoint(2, -2){M}
        \tkzDefPoint(6, 5){R}
        \tkzDefShiftPoint[R](-2.1, -0.35){D}
        \tkzDefPoint(10,6){I}
        \tkzDefPointBy[homothety=center M ratio 0.7](R) \tkzGetPoint{F}
        \tkzDefPointBy[projection=onto R--D](F) \tkzGetPoint{F'}
        
        
        \tkzLabelPoint[above](I){$\infty$}
        \tkzLabelPoint[right](M){$M$}
        \tkzLabelPoint[above right=-1pt](R){$m$}
        \tkzLabelSegment[below left=-3pt](M,R0){$\vec{R}$}
        \tkzLabelSegment[below right=-3pt, pos=0.45](M,R){$\vec{r}$}
        \tkzLabelSegment[right](R,F){$\vec{F}(\vec{r})$}
        \tkzLabelSegment[above](R,D){$d\vec{r}$}
        
        \tkzDrawSegments[-latex](M,R0 M,R)
        \tkzDrawSegments[thick, -latex](R,D R,F)
        \tkzDrawSegments[dashed](F,F')
        
        \tkzMarkRightAngles[size=0.3](F,F',R)
        
        \tkzDrawPoints(M, R, R0, I)
    \end{tikzpicture}
    \caption{К выводу формулы потенциальной энергии}
    \label{pic:potential-energy}    
\end{wrapfigure}

\term{Потенциальная энергия} $U(\vec{R})$ тела с массой $m$ в гравитационном поле тела с массой $M$~--- энергия, необходимая, чтобы переместить первое тело с бесконечности на вектор $\vec{R}$ от второго. Исходя из определения, потенциальная энергия определяется, как
\begin{equation}
    U(\vec R) =  \oint\limits_{+\infty}^\vec{r} \big(\vec F( \vec r),d \vec r\big).
\end{equation}
Здесь интегрирование ведется вдоль пути перемещения тела с бесконечности в точку $\vec{r}$. Снова воспользуемся сферической симметрией гравитационного поля:
\begin{multline}
    U(\vec R) 
    = - \oint\limits_{+\infty}^\vec{R} \big(m \nabla \varphi (\vec{r}), d \vec r\big)
    = - \oint\limits_{+\infty}^\vec{R} m \frac{d \varphi(r)}{d r} \left( \frac{\vec r}{r}, d \vec r \right) = \\
    = \int\limits_{+\infty}^R m \frac{d \varphi(r)}{d r} \,d r 
    = \int\limits_{+\infty}^R m \, d \varphi(r) = -\frac{GMm}{R} = U(R).
\end{multline}

Напряженность гравитационного поля~--- отношение силы, действующей на точную пробное тело к массе этого тела, также называют \term[ускорение свободного падения]{ускорением свободного падения} и обозначают за $g$. Ускорение свободного падения вычисляется по формуле
\begin{equation}
	\vec g(\vec r) = -\nabla \varphi(\vec r) = \frac{\vec F( \vec r)}{m} = - \frac{GM}{r^3} \vec r.
	\label{eq:g}
\end{equation} 
Следовательно, (\ref{eq:grav-law-1}) можно записать как
\begin{equation}
	\vec F (\vec r) = m \vec g(\vec r).
\end{equation}
