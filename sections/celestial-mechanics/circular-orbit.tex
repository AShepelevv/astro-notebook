\subsection{Круговое движение. Первая космическая скорость}

Пусть тело движется по окружности радиуса~$R$ с~постоянной скоростью~$v$. Найдём, какой ускорение $\vec a$ оно испытывает в ходе такого движения. Рассмотрим для этого малый промежуток времени $dt$. За это время тело проходит по орбите $d\alpha = \omega \, dt$~радиан, где $\omega$~--- угловая скорость движения тела по окружности, определяемая выражением
\begin{equation*}
    \omega = \frac{2 \pi}{T} = \frac{ 2\pi \cdot v}{2\pi R} = \frac{v}{R},
\end{equation*}
где за $T$ обозначен период движения тела по окружности.

С другой стороны за время $dt$ вектор скорости $\vec{v}$ изменяется на $d \vec{v}$, причем~$d \vec{v}$ направлен вдоль радиуса окружности в сторону центра. Из~\picRef{pic:circular-velocity} видно, что $d v = v \omega \, dt$, следовательно, модуль ускорения
\begin{equation*}
    a = \frac{dv}{dt} = \omega v = \frac{v^2}{R},
\end{equation*}
а вектор $\vec a$ направлен радиально к центру.

При круговом движении малого тела в гравитационном поле массивного тела массы $M$ первое испытывает силу всемирного тяготения со стороны второго. А значит, оно испытает ускорение свободного падения~$\vec{g}$~\eqref{eq:g}. Приравняв которое к центростремительному ускорению~$\vec{a}$, получим выражение для \term{первой космической} или \term{круговой скорости}.
\begin{gather}
    g = \frac{G M}{R^2} = \frac{v_1^2}{R},\nonumber\\
    v_1 = \sqrt{\frac{GM}{R}},
\end{gather}
где $v_1$~--- первая космическая скорость~--- скорость малого тела на круговой орбите радиуса $R$ вокруг тела массы $M$.
