\subsection{Круговое движение. Первая космическая скорость}

\begin{wrapfigure}[9]{r}{0.35\tw}
    \vspace{-1pc}
    \centering
    \tikzsetnextfilename{circular-movement}
    \begin{tikzpicture}
        \footnotesize
        \tkzDefPoint(0,0){C}
        \def\R{2.5}
        \def\V{1.5}
        \def\ALPHA{40}
        
        \tkzDefShiftPoint[C](0,\R){V1b}
        \tkzDefShiftPoint[V1b](-\V,0){V1e}
        \tkzDefPointBy[rotation=center C angle \ALPHA](V1b) \tkzGetPoint{V2b}
        \tkzDefShiftPoint[V2b](-\V,0){V2e}
        
        \tkzDefPointBy[rotation=center V2b angle \ALPHA](V2e) \tkzGetPoint{V'}
        
%        \tkzDrawSegments[line width=2pt, lightgray](V2b,V2e V2b,V' V2e,V' C,V1b C,V2b)
%        \tkzDrawArc[line width=2pt, lightgray](C,V1b)(V2b)
        
        \tkzDrawSegments[-latex, thick ](V1b,V1e V2b,V2e V2b,V' V2e,V')
        \tkzDrawSegments(C,V1b C,V2b)
        
        \tkzLabelSegment[right](C,V1b){$R$}
        \tkzLabelSegment[above, pos=0.6](V1b,V1e){$\vec{v}$}
        \tkzLabelSegment[above](V2b,V2e){$\vec{v}$}
        \tkzLabelSegment[below right=-2pt, pos=0.6](V2b,V'){$\vec{v}'$}
        \tkzLabelSegment[left](V2e,V'){$d\vec{v}$}
        
        \tkzLabelAngle[pos=0.6](V1b,C,V2b){\footnotesize$\alpha$}
        \tkzLabelAngle[pos=0.6](V2e,V2b,V'){\footnotesize$\alpha$}
                
        \tkzMarkAngles[size=0.4](V1b,C,V2b V2e,V2b,V')
        \tkzMarkRightAngles[size=0.2](V',V2b,C V1e,V1b,C)
        
        \tkzDrawArc[angles, semithick](C,V1b)(80,145)
        
        
        \tkzLabelPoint[right](C){$C$}
        \tkzLabelPoint[above](V1b){$A$}
        \tkzLabelPoint[above](V2b){$B$}
        \tkzLabelPoint[above](V2e){$D$}
        \tkzLabelPoint[below](V'){$E$}
        
        \tkzDrawPoints(C, V1b, V2b, V2e, V')
    \end{tikzpicture}
    \caption{}
    \label{pic:circular-motion}
\end{wrapfigure}

Пусть тело движется по окружности радиуса~$R$ с~постоянной скоростью~$v$. Найдём, какое ускорение $\vec a$ оно испытывает в ходе такого движения. Рассмотрим для этого малый промежуток времени $dt$, за который оно перемещается из точки~$A$ в точку~$B$,~\lookPicRef{pic:circular-motion}. За это время тело проходит по орбите угол $\alpha = \omega \, dt$, где $\omega$~--- угловая скорость движения тела по окружности, определяемая выражением
\begin{equation*}
    \omega = \frac{2 \pi}{T} = \frac{ 2\pi \cdot v}{2\pi R} = \frac{v}{R},
\end{equation*}
где за $T$ обозначен период движения тела по окружности.

С другой стороны за время $dt$ вектор скорости $\vec{v}$ изменяется на~$d \vec{v}$ и становится равным $\vec{v}'$, причем $|\vec{v}| = |\vec{v}'| = v$.  Треугольники $\triangle ABC$ и $\triangle BDE$ подобны, следовательно,
\begin{equation*}
    \frac{dv}{v} = \frac{|AB|}{R} \rightarrow \frac{v\,dt}{R}~\text{при}~dt \rightarrow 0
\end{equation*}
 А значит, 
\begin{equation*}
    a = \frac{dv}{dt} = \frac{v^2}{R},
\end{equation*}
причём вектор $\vec{a}$ направлен вдоль радиуса окружности в сторону центра. 

При круговом движении малого тела в гравитационном поле массивного тела массы $M$ первое испытывает силу всемирного тяготения со стороны второго. А значит, оно испытает ускорение свободного падения~$\vec{g}$~\eqref{eq:g}. Приравняв которое к центростремительному ускорению~$\vec{a}$, получим выражение для \term{первой космической} или \term{круговой скорости}.
\begin{gather}
    g = \frac{G M}{R^2} = \frac{v_1^2}{R},\nonumber\\
    v_1 = \sqrt{\frac{GM}{R}},
\end{gather}
где $v_1$~--- первая космическая скорость~--- скорость малого тела на круговой орбите радиуса $R$ вокруг тела массы $M$.
