\subsection{Круговое движение}

\subsubsection{Первая космическая скорость}

\begin{wrapfigure}[9]{r}{0.35\tw}
    \vspace{-1pc}
    \centering
    \tikzsetnextfilename{circular-movement}
    \begin{tikzpicture}
        \footnotesize
        \tkzDefPoint(0,0){C}
        \def\R{2.5}
        \def\V{1.5}
        \def\ALPHA{40}
        
        \tkzDefShiftPoint[C](0,\R){V1b}
        \tkzDefShiftPoint[V1b](-\V,0){V1e}
        \tkzDefPointBy[rotation=center C angle \ALPHA](V1b) \tkzGetPoint{V2b}
        \tkzDefShiftPoint[V2b](-\V,0){V2e}
        
        \tkzDefPointBy[rotation=center V2b angle \ALPHA](V2e) \tkzGetPoint{V'}
        
%        \tkzDrawSegments[line width=2pt, lightgray](V2b,V2e V2b,V' V2e,V' C,V1b C,V2b)
%        \tkzDrawArc[line width=2pt, lightgray](C,V1b)(V2b)
        
        \tkzDrawSegments[-latex, thick ](V1b,V1e V2b,V2e V2b,V' V2e,V')
        \tkzDrawSegments(C,V1b C,V2b)
        
        \tkzLabelSegment[right](C,V1b){$R$}
        \tkzLabelSegment[above, pos=0.6](V1b,V1e){$\vec{v}$}
        \tkzLabelSegment[above](V2b,V2e){$\vec{v}$}
        \tkzLabelSegment[below right=-2pt, pos=0.6](V2b,V'){$\vec{v}'$}
        \tkzLabelSegment[left](V2e,V'){$d\vec{v}$}
        
        \tkzLabelAngle[pos=0.6](V1b,C,V2b){\footnotesize$\alpha$}
        \tkzLabelAngle[pos=0.6](V2e,V2b,V'){\footnotesize$\alpha$}
                
        \tkzMarkAngles[size=0.4](V1b,C,V2b V2e,V2b,V')
        \tkzMarkRightAngles[size=0.2](V',V2b,C V1e,V1b,C)
        
        \tkzDrawArc[angles, semithick](C,V1b)(80,145)
        
        
        \tkzLabelPoint[right](C){$C$}
        \tkzLabelPoint[above](V1b){$A$}
        \tkzLabelPoint[above](V2b){$B$}
        \tkzLabelPoint[above](V2e){$D$}
        \tkzLabelPoint[below](V'){$E$}
        
        \tkzDrawPoints(C, V1b, V2b, V2e, V')
    \end{tikzpicture}
    \caption{}
    \label{pic:circular-motion}
\end{wrapfigure}

Пусть тело движется по окружности радиуса~$R$ с~постоянной скоростью~$v$. Найдём, какое ускорение $\vec a$ оно испытывает в ходе такого движения. Рассмотрим для этого малый промежуток времени $dt$, за который оно перемещается из точки~$A$ в точку~$B$,~\lookPicRef{pic:circular-motion}. За это время тело проходит по орбите угол $\alpha = \omega \, dt$, где $\omega$~--- угловая скорость движения тела по окружности, определяемая выражением
\begin{equation*}
    \omega = \frac{2 \pi}{T} = \frac{ 2\pi \cdot v}{2\pi R} = \frac{v}{R},
\end{equation*}
где за $T$ обозначен период движения тела по окружности.

С другой стороны за время $dt$ вектор скорости $\vec{v}$ изменяется на~$d \vec{v}$ и становится равным $\vec{v}'$, причем $|\vec{v}| = |\vec{v}'| = v$.  Треугольники $\triangle ABC$ и $\triangle BDE$ подобны, следовательно,
\begin{equation*}
    \frac{dv}{v} = \frac{|AB|}{R} \rightarrow \frac{v\,dt}{R}~\text{при}~dt \rightarrow 0
\end{equation*}
 А значит, 
\begin{equation*}
    a = \frac{dv}{dt} = \frac{v^2}{R},
\end{equation*}
причём вектор $\vec{a}$ направлен вдоль радиуса окружности в сторону центра. 

При круговом движении малого тела в гравитационном поле массивного тела массы $M$ первое испытывает силу всемирного тяготения со стороны второго. А значит, оно испытает ускорение свободного падения~$\vec{g}$~\eqref{eq:g}. Приравняв которое к центростремительному ускорению~$\vec{a}$, получим выражение для \term{первой космической} или \term{круговой скорости}.
\begin{gather}
    g = \frac{G M}{R^2} = \frac{v_1^2}{R},\nonumber\\
    v_1 = \sqrt{\frac{GM}{R}},
\end{gather}
где $v_1$~--- первая космическая скорость~--- скорость малого тела на круговой орбите радиуса $R$ вокруг тела массы $M$.

\subsubsection{Геостационарные и геосинхронные спутники}
    Рассмотрим массивное тело массы $M$, вращающееся вокруг своей оси с периодом $T$. \term{Геостационарным} спутником такого тела называется спутник, находящийся постоянно над одной точкой поверхности этого тела. Другими словами, такой спутник, что его радиус-вектор является продолжением радиус-вектора одной из точек поверхности этого тела. 
    
    Заметим, что плоскость орбиты спутника всегда проходит через центр масс гравитирующего тела, а значит, совпадает с одним из больших кругов с центом в центре масс гравитирующего тела. Из этого факта и определения геостационарного спутника следует, что точка поверхности центрального тела также должна находиться в плоскости этого большого круга.
    
    С другой стороны, только точки на экваторе тела движутся в плоскости большого круга, траектории других точек являются малыми кругами, так как их плоскость не содержит центра масс тела.
    
    Из данных утверждений вытекает ограничение на расположение орбиты геостационарного спутника~--- она всегда находится в плоскости экватора центрального тела.
    
    Кроме этого вращение тела происходит равномерно, следовательно, движение спутника также должно быть равномерно. Значит, орбита должна быть круговой. Далее будет показано, что угловая скорость тел на иных орбитах не постоянна. 
    
    Отсюда легко получить радиус орбиты геостационарного спутника: нужно лишь приравнять период вращения центрального тела к периоду обращения спутника по орбите с искомым радиусом $R_\text{г.\,стац.}$:
    \begin{gather}
        T = \frac{2 \pi R_\text{г.\,стац.}}{v_1} = 2 \pi \sqrt{\frac{R^3_\text{г.\,стац.}}{G M}}, \nonumber \\
        R_\text{г.\,стац.} = \sqrt[3]{\frac{G M T^2}{4 \pi^2}}.
    \end{gather}
    
    Расширением понятия геостационарности является \term{гео\-синх\-рон\-ность}, когда спутник в один и тот же момент периода своего обращения находится над одной и той же точкой поверхности центрального тела. 
    
    Отсюда, также как и в случае геостационарности, следует, что период обращение по орбите совпадает с периодом вращения центрального тела. Значит, \imp{большая полуось}\footnote{\lookSecRef{sec:ellips}} орбиты $a_\text{г.\,синх.} = R_\text{г.\,стац.}$ согласно третьему закону Кеплера\footnote{\lookSecRef{sec:first-kepler-law}}. При этом орбита может быть как наклоненной к экватору, так и вытянутой~--- эллиптичной.
    
    В заключение данного раздела найдём большую полуось геосинхронной орбиты спутника Земли. Здесь важно учесть, что сидерический период вращения Земли вокруг своей оси составляет одни звёздные, а не солнечные сутки\footnote{\lookSecRef{sec:synodic-period}}, продолжительность которых $
    T_\oplus 
%        = 23\,\text{ч}~56\,\text{мин}~4\,\text{c} 
        = 86164\,\text{c}
    $. Отсюда,
    \begin{equation*}
        a_\text{г.\,синх.\,$\oplus$} = \sqrt[3]{\frac{GM_\oplus T_\oplus}{4 \pi^2}} \simeq 42164~\text{км}.
    \end{equation*}
