\subsection{Уравнение Баркера}
Для того, чтобы связать истинную аномалию с некоторым временем при движении тела по параболе используют \term{уравнение Баркера}.

Вспомним выражение для периода тела на орбите из третьего закона Кеплера (\ref{eq:kepler-third-law}):
\begin{equation*}
	T = 2\pi \sqrt{\frac{a^3}{GM}},
\end{equation*}
Однако с учётом того, что расстояние до перицентра  $q = a(1-e)$ и также вспомнив определение средней аномалии можно записать:
\begin{equation*}
	M=t \sqrt{\frac{GM(1-e)^3}{q^3}}.
\end{equation*}
С другой же стороны из уравнения Кеплера (\ref{eq:kepler-eq}) и разложения синуса в ряд Тейлора (%вставить ссылку на формулу Тейлора
) средняя аномалия это:
\begin{gather*}
	M=E - e \sin E = E - e \left(E - \frac{E^3}{3!} + \frac{E^5}{5!} - \dots\right) \\
	M = E(1-e) + e\left(\frac{E^3}{3!} - \frac{E^5}{5!} + \dots \right).
\end{gather*}
Теперь запишем формулу перехода от истинной аномалии к эксцентрической (%вставить ссылку на формулу с тангенсами
):
\begin{equation*}
	\tan \frac{E}{2} = \sqrt{\frac{1-e}{1+e}} \tan \frac{\nu}{2}.
\end{equation*}
Заметим, что при всех $\nu$, исключая $\pi$, мы можем считать, что $E \ll 1$, так как для параболы мы можем считать $1-e \ll 1$. Следовательно, поскольку $\tan x \simeq x$:
\begin{equation*}
	E = 2 \sqrt{\frac{1-e}{1+e}} \tan \frac{\nu}{2}.
\end{equation*}
И, подставляя это в предыдущее уравнение, получим:
\begin{equation*}
	M = t \sqrt{\frac{GM(1-e)^3}{q^3}} = \frac{2(1-e)^{3/2}}{\sqrt{1+e}}\tan\frac{\nu}{2} + \frac{8(1-e)^{3/2}}{3! \cdot (1+e)^{3/2}}\tan^3 \frac{\nu}{2} + \dots
\end{equation*}
Сократив на $(1-e)^{3/2}$ и приведя выражение, можно получить \imp{уравнение Баркера}
\begin{equation}
	t = \sqrt{\frac{2q^3}{GM}}\left(\tan \frac{\nu}{2 } + \frac{1}{3}\tan^3 \frac{\nu}{2}\right).
\end{equation}