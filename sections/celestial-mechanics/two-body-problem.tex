\subsection{Задача двух тел}
В случае сравнимых масс законы Кеплера нуждаются в уточнении. Данная проблема называется \term{задачей двух тел}. Пусть в пространстве в некоторой произвольной системе отсчета существуют два тела с массми $M_1$ и $M_2$ c данными начальными координатами $\vec{r}_1$ и $\vec{r}_2$, а также начальными скоростями $\vec{v}_1$ и $\vec{v}_2$. Решение задачи двух тел заключается в поиске законов движения $\vec{r}_1(t)$ и $\vec{r}_2(t)$.

\begin{wrapfigure}[7]{r}{0.47\tw}
    \centering
    \vspace{-1pc}
    \tikzsetnextfilename{reduced-mass}
    \begin{tikzpicture}[]
        
        \tkzDefPoint(0, 0){BC}
        \tkzDefPoint(-2, 1){M1}
        \tkzDefPoint(1, -1/2){M2}
        \tkzDefPoint(-1, -1){O}
        
        \tkzLabelPoint[above right](BC){ц.м.}
        \tkzLabelPoint[left](M1){$M_1$}
        \tkzLabelPoint[right](M2){$M_2$}
        \tkzLabelSegment[below left=-3pt](O,M1){$\vec{r_1}$}
        \tkzLabelSegment[below right=-3pt, pos=0.45](O,M2){$\vec{r_2}$}
        \tkzLabelSegment[above, pos=0.45](M1,M2){$\vec{r}$}
        \tkzLabelSegment[above left=-3pt](O,BC){$\vec{R}$}
        
        \tkzDrawSegments[latex-](M1,O M2,O M1,M2 BC,O)
        
        \tkzDrawPoints(M1,M2,O,BC)
    \end{tikzpicture}
    \caption{}
    \label{pic:reduced-mass}    
\end{wrapfigure}

Второй закон Ньютона в данной СО записывается как:
\begin{equation}
\begin{aligned}
	\vec{F}_{12} &= M_1 \ddot{\vec{r}}_1,\\
	\vec{F}_{21} &= M_2 \ddot{\vec{r}}_2.
\end{aligned}
\label{eq:tbp-newton-second-laws}
\end{equation}
Если сложить два данных уравнения, учитывая что по третьему закону Ньютона $\vec{F}_{12} = -\vec{F}_{21}$, можно получить частный случай \imp{теоремы о движении центра масс}:
\begin{equation*}
	M_1 \ddot{\vec{r}}_1 + M_2 \ddot{\vec{r}}_2 = (M_1 + M_2) \ddot{\vec{R}} = 0,
\end{equation*}
где $\ddot{\vec{R}}$~--- ускорение центра масс системы. Данное равенство следует напрямую из определения центра масс. % добавить \ref
Введем вектор $\vec{r}$ как $\vec{r}_1-\vec{r}_2$, тогда общее решение задачи можно найти как:
\begin{equation}
\begin{aligned}
	\vec{r_1}(t) &= \vec{R}(t) + \frac{M_2}{M_1 + M_2} \vec{r}(t), \\
	\vec{r_2}(t) &= \vec{R}(t) - \frac{M_1}{M_1 + M_2} \vec{r}(t).
\end{aligned}
\end{equation}
Как уже было показано, в данной системе ускорение центра масс равно нулю, а его координаты и скорость находятся напрямую из начальных условий, поэтому сконцентрируемся на поиске выражения для $\vec{r}(t)$.

Поделим каждое из уравнений (\ref{eq:tbp-newton-second-laws}) на соответствующую массу и вычтем их друг из друга:
\begin{equation}
	\ddot{\vec{r}}_1 - \ddot{\vec{r}}_2 = \ddot{\vec{r}} = \frac{\vec{F}_{12}}{M_1}-\frac{\vec{F}_{21}}{M_2} = \left(\frac{1}{M_1}+\frac{1}{M_2}\right) \vec{F} = \frac{\vec{F}}{\mu}.
	\label{eq:tbp-relative-second-law}
\end{equation}
Тут мы опять воспользовались третьим законом Ньютона, чтобы обозначить $\vec{F} = \vec{F}_12 = -\vec{F}_21$. Выражение с обратной суммой обратных масс называется \term{приведённой массой} системы.
\begin{equation}
	\mu = \frac{1}{\frac{1}{M_1} + \frac{1}{M_2}} = \frac{M_1 M_2}{M_1 + M_2}.
\end{equation}
Выражение (\ref{eq:tbp-relative-second-law}) является аналогом второго закона Ньютона для задачи двух тел. Это уравнение можно решить аналогично обычной задаче Кеплера и свести его к аналогу уравнения (\ref{eq:first-kepler-law-eq}):
\begin{equation}
	\frac{d^2 u}{d \theta^2} + u = \frac{GM_1 M_2}{\mu l^2}.
\end{equation}
Решение данного уравнения~--- эллипс с фокальным параметром
\begin{equation*}
	p = \frac{\mu l^2}{GM_1 M_2} = \frac{l^2}{G(M_1 + M_2)}.
\end{equation*}
Таким образом тела будут двигаться по подобным эллипсам с коэффициентом подобия $M_1/M_2$ так, что центр масс системы будет двигаться без ускорения.

Найдем выражение для полной энергии в системе двух тел. По определению полная энергия есть сумма кинетической и потенциальной:
\begin{equation*}
	T + U = E_0
\end{equation*}
Из теоремы о вириале известно соотношение
\begin{equation*}
	2\langle T \rangle_\tau = -\langle U \rangle_\tau
\end{equation*}
Таким образом можно получить выражение для полной энергии через среднюю величину потенциальной энергии:
\begin{equation*}
	E_0 = \left\langle \frac{U}{2} \right\rangle_\tau
\end{equation*}
Гравитационный потенциал имеет форму $1/r$. Найдем по определению среднее по времени (средней аномалии) значение величины $1/r$ для тела на орбите:
\begin{equation*}
	\left\langle \frac{1}{r} \right\rangle_\tau = \frac{1}{2 \pi} \int\limits_0^{2\pi}\frac{dM}{r} = \frac{1}{2 \pi} \int\limits_0^{2\pi}\frac{dE}{a} = \frac{1}{a}.
\end{equation*}
Тут мы воспользовались соотношением $a dM = r dE$. Отсюда величина полной энергии
\begin{equation*}
	E_0 = -\frac{G M_1 M_2}{2a},
\end{equation*}
где $a$ есть полуось эллипса образованного вектором $\vec{r}$.





