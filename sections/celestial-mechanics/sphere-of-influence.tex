\subsection{Сфера действия, сфера Хилла}
\term{Сфера действия}~--- область пространства внутри которой возмущение от внешнего тела относительно планеты меньше, чем возмущение от планеты относительно внешнего тела. То есть лёгкое отклонение от положения на орбите не должно приводить к уводу внешним телом. В первом приближении данная область~--- сфера. Приведем  классический вывод её радиуса:

Для этого рассмотрим две точечные массы~$A$ и $B$ расположенные в точках $\vec{r}_A$ и $\vec{r}_B$, с массами $M_A$ и $M_B$ соответственно. Расстояние между объектами $R=\left|r_B-r_A\right|$. Введем третью безмассовую частицу $C$ в точке $\vec{r}_{\text{C}}$. Далее для анализа динамики движения точки $C$ мы можем рассматривать задачу как в системе отсчета точки $A$, так и точки $B$.

Рассмотрим систему отсчета связанную с точкой~$A$. Точка~$B$ с силой гравитации которую мы обозначим как~$g_B$ будет возмущать точку~$C$ относительно гравитации~$g_A$ точки~$A$. Вследствие закона всемирного тяготения, точка~$A$ будет притягиваться к точке~$B$ с ускорением 
\begin{equation*}
	\vec{a}_A=\frac{G M_B}{R^3}\left(\vec{r}_B-\vec{r}_A\right),
\end{equation*}
а значит данная система отсчета неинерциальна. Для того чтобы оценить эффекты возмущений в данной системе отсчета можно рассмотреть отношение величины данных возмущений к основному гравитирующему телу, то есть:
\begin{equation*}
	\chi_A=\frac{\left|g_B-a_A\right|}{\left|g_A\right|}.
\end{equation*}
Возмущающая сила $g_B-a_A$ также известна как приливная (\ref{Ebb_flow}). Аналогичным образом строится $\chi_B$ для системы отсчета точки $B$, в выводе требуется лишь провести замену $A \leftrightarrow B$.

По приближении точки $C$ к $A, \chi_A \rightarrow 0$ и $\chi_B \rightarrow \infty$, и в обратную сторону. Главным считается тот объект относительно которого отношение возмущающих сил меньше, чем относительно другого. Поверхность для которой $\chi_A=\chi_B$ разграничивает пространство на области влияния того или иного тела. В общем случае эта поверхность имеет сложную форму, однако в случае когда масса одного тела много больше другого, скажем $M_A \ll M_B$, возможно найти хорошую аппроксимацию этой разграничивающей поверхности. В этом случае поверхность будет располагаться вокруг точки~$A$, обозначим $r$ расстояние от точки $A$ до разграничивающей поверхности.

Главное ускорение в системе отсчета точки $A$ будет равно $g_A$:
\begin{equation*}
	g_A = \frac{GM_A}{r^2}.
\end{equation*}
Ускорение системы отсчета $a_A$ по модулю равно:
\begin{equation*}
	a_A = \frac{GM_B}{R^2}.
\end{equation*}
Вторичное же ускорение равно:
\begin{equation*}
	g_B\simeq\frac{GM_B}{R^2}+\frac{GM_B}{R^3}r.
\end{equation*}
Приливные силы тогда как разность вторичного ускорения и ускорения системы отсчёта:
\begin{equation*}
	g_B-a_A \simeq \frac{G M_B}{R^3} r.
\end{equation*} 
Заключая находим отношение $\chi_A$:
\begin{equation*}
	\chi_A\simeq\frac{M_B}{M_A} \frac{r^3}{R^3}.
\end{equation*}
Аналогично для системы отсчёта точки $B$. Главным ускорением тут будет $g_B$, а вторичным $g_A$. Вследстивие массивности тела $B$ ($M_A~\ll~M_B$) ускорение системы отсчета можно принять равным 0:
\begin{equation*}
	a_B=\frac{G M_A}{R^2} \simeq 0,
\end{equation*}
тогда приливные силы равные $g_A-a_B$ будут численно равны $g_A$. Отсюда отношение $\chi_B$ равно:
\begin{equation*}
	\chi_B \simeq \frac{M_A}{M_B} \frac{R^2}{r^2}.
\end{equation*}
Расстояние до сферы действия должно удволетворять равенству:
\begin{equation*}
	\frac{M_B}{M_A} \frac{r^3}{R^3}=\frac{M_A}{M_B} \frac{R^2}{r^2},
\end{equation*}
а отсюда мы можем записать выражение для радиуса сферы действия точки $A$:
\begin{equation}
	\frac{r}{R}=\left(\frac{M_A}{M_B}\right)^{2 / 5}.
\end{equation} 
\term{Сфера Хилла}~--- одна из моделей рассчета радиуса \imp{сферы действия}. Имеет физический смысл области пространства вокруг некоторого объекта (планеты) в которой его собственное гравитационное влияние на пробную массу больше, чем от внешних тел (Солнца). В первом приближении данная область~--- сфера, содержащая на себе точку $L_1$ \eqref{eq:lagrange-12} интересующей системы. В обозначениях предыдущей задачи её радиус будет выражаться как:
\begin{equation}
	\frac{r}{R} = \sqrt[3]{\frac{M_A}{3M_B}}.
\end{equation}
\begin{wrapfigure}[15]{r}{0.50\tw}
    \centering
    \vspace{-1pc}
    \tikzsetnextfilename{hill-influence-sphere-plot}
    \begin{tikzpicture}
		\begin{axis}[
			width	=	6cm,
			height	=	6cm,
			xlabel	=	{$\lg(M_B / M_A)$},
			ylabel	=	{$r/R$},
			ymax	=	0.3,
			ymin	=	0,
			xmax	=	5,
			xmin	=	1.5,
			legend cell align=left,
			legend style={
				row sep = 0.8pc,
				draw=none,
				fill=none,
				font=\scriptsize,
				at={(axis cs:2.65, 0.285)}, anchor=north west,
			},
		]
			\addplot [domain=1.5:5, samples=100, black] {10 ^ (-2 * x / 5)};
			\addplot [domain=1.5:5, samples=100, dash pattern=on 6pt off 2pt on 1pt off 2pt] {1 / (3 * 10^x)^(1/3)};
			
			\legend{
			    Сфера действия,
				Сфера Хилла
			}
		\end{axis}
	\end{tikzpicture}
	\caption{Сравнение радиусов сферы действия и сферы Хилла}
	\label{pic:hill-influence-sphere-plot}    
\end{wrapfigure} 
Часто про сферу Хилла можно услышать что она является областью пространства в которой объект может иметь собственный спутник, однако в реальности же на расстоянии равной сфере Хилла любое внешнее возмущение будет уводить спутник с орбиты и по-настоящему стабильные обриты имеют радиусы в 2-3 раза меньшие, чем сфера Хилла.
 График сравнения радиусов сферы действия и сферы Хилла для разных отношений масс $M_A$ и $M_B$ приведен правее. При отношении масс тел порядка 250 размеры сфер сравнимы между собой, однако же для малых тел для которых параметр $\lg(M_{\astrosun} / M)>6$ сфера действия будет уже вдвое меньше сферы Хилла. 
\begin{figure}[h!]
\tikzsetnextfilename{hill-influence-sphere-bar-plot2}
\begin{tikzpicture}
\begin{axis}[
    ybar,
    ymode=log,
    enlargelimits=0.15,
    legend cell align=left,
    legend style={at={(0.03,0.97)},
      			  row sep = 0.8pc,
      			  font=\scriptsize,
      			  anchor=north west,
      			  draw=none,},
    ylabel={млн. км},
    symbolic x coords={Меркурий,Венера,Земля,Марс,Юпитер,Сатурн,Уран,Нептун},
    xtick=data,
    width	=	\tw,
	height	=	6cm,
    ]
\addplot[pattern=dots] coordinates {(Меркурий,0.117) (Венера, 0.616) (Земля, 0.929) (Марс, 0.578) (Юпитер, 48.2) (Сатурн, 54.5) (Уран, 51.9) (Нептун, 86.2)};
\addplot[pattern=grid] coordinates {(Меркурий, 0.1753) (Венера, 1.0042) (Земля, 1.4714) (Марс, 0.9827) (Юпитер, 50.5736) (Сатурн, 61.6340) (Уран, 66.7831) (Нептун, 115.0307)};
\legend{Cфера действия,
	    Сфера хилла}
\end{axis}
\end{tikzpicture}
\caption{Радиусы сферы действия и сферы Хилла для разных планет}
	\label{pic:hill-influence-sphere-bar-plot2}   
\end{figure}