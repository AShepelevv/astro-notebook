\subsection{Движение по орбите}

\begin{wrapfigure}[8]{r}{0.4\tw}
    \centering
    \vspace{-0.75pc}
    \tikzsetnextfilename{orbit-motion}
    \begin{tikzpicture}
        \def\R{4}
        \def\w{1}
        \def\s{1.5}
        \def\r{2.5}
        \def\eps{0.1}
        
        \tkzDefPoint(0,0){O}
        \tkzDefPoint(0,\w){W}
        \tkzDefPoint(0,\s){S}
        \tkzDefPoint(2.4,-0.6){R1}
        \tkzDefPoint(3,1){R2}
        
        \tkzGetPointCoord(R1){r}
        \tkzGetPointCoord(R2){rr}
        
        \draw [black] plot [smooth, tension=1] coordinates { (0.5,-1) (\rx,\ry) (\rrx,\rry) (4,1.2)};
        
        \tkzDrawSegments[-latex,semithick](O,W O,S O,R1 O,R2 R1,R2)
        
        \tkzLabelPoint[below left = -1pt](W){$\boldsymbol{\omega}$}
        \tkzLabelPoint[below left = -1pt](S){$\vec{s}$}
        \tkzLabelSegment[below](O,R1){$\vec{r}$}
        \tkzLabelSegment[above=1pt](O,R2){$\vec{r}(t + dt)$}
        \tkzLabelSegment[left](R1,R2){$d\vec{r}(t)$}
        
        \tkzMarkRightAngle[size=0.2](R1,O,W)
        \tkzMarkRightAngle[size=0.3](R2,O,W)
        
        \tkzMarkAngle[size=0.4, angle eccentricity=2](R1,O,R2)
        \tkzLabelAngle[pos=0.8](R1,O,R2){\footnotesize$\boldsymbol{\omega} \, dt$}
        
        \tkzDrawPoints(O, R1, R2)
    \end{tikzpicture}
    \caption{}
\end{wrapfigure}
Рассмотрим такую физическую величину, как \term{секториальная ско\-рость}~--- это векторная величина, описывающая ориентированную площадь, заметаемую радиус вектором тела за единицу времени. Пусть в момент времени $t$ тело находилось в точке $\vec{r}(t)$, а через промежуток времени $dt$~--- в точке $\vec{r}(t + dt)$. Обозначим перемещение тела за этот промежуток времени как $d\vec{r}$. Его можно выразить через скорость тела в момент времени $t$, считая её постоянной на промежутке от $t$ до $t + dt$: $d\vec{r} = \vec{v} \,d t$. Площадь, которую заметает радиус-вектор тела $\vec{r}(t)$ равна половине параллелограмма, построенного на векторах $\vec{r}(t)$ и $d\vec{r}$, то есть
\begin{equation*}
    \vec{s} = \frac{1}{2} [\vec{r} \times \vec{v} dt],
\end{equation*}
следовательно, секториальная скорость
\begin{equation*}
    \boldsymbol{\sigma} = \frac{d \vec{s}}{dt} = \frac{1}{2} [\vec{r} \times \vec{v}]  = \frac{\vec{L}}{2m} = \frac{\vec{l}}{2},
\end{equation*}
где $\vec{l}$~--- удельный (на единицу массы) момент импульса. Полученное выражение доказывает \imp{второй закон Кеплера}.

С другой стороны, перемещение $d\vec{r}$ можно выразить через угловую скорость $\boldsymbol{\omega}$, как $d \vec{r} = [\vec{r} \times \boldsymbol{\omega} \,d t]$. Тогда
\begin{equation*}
    \boldsymbol{\sigma}
    = \frac{1}{2} \big[ \vec{r} \times [\vec{r} \times \boldsymbol{\omega} ]\big]
    = \frac{1}{2} \left(\vec{r} \underbrace{(\vec{r}, \boldsymbol{\omega})}_0 - \boldsymbol{\omega} ( \vec{r}, \vec{r} ) \right)
    = \frac{r^2 \boldsymbol{\omega}}{2}.
\end{equation*}

Получим еще одно важное соотношение~--- \term{интеграл энергии}~--- формулу для скорости тела на орбите с большой полуосью $a$ в точке, удалённой на расстояние~$r$ от центрального тела с массой $M$. Для этого запишем ЗСЭ, а для полной энергии запишем выражение (\ref{eq:total-orbit-energy}):
\begin{equation*}
	\frac{mv^2}{2} - \frac{GMm}{r} = -\frac{GMm}{2a}.
\end{equation*}
Данное выражение можно переписать в следующем виде:
%Для этого рассмотрим  сначала точку перицентра ($q$, <<п>>) и апоцентра ($Q$, <<a>>) данной орбиты. Запишем для них закон сохранения энергии и закон сохранения момента импульса:
%\begin{gather*}
%    -\frac{GMm}{q} + \frac{m v^2_\text{п}}{2} = -\frac{GMm}{Q} + \frac{m v^2_\text{а}}{2},\\
%    mv_\text{п}q = mv_\text{a}Q.
%\end{gather*}
%Из ЗСМИ и выражений для перицентрического~$q$ и апоцентрического~$Q$ расстояний через большую полуось $a$ и эксцентриситет $e$ имеем:
%\begin{equation*}
%    \frac{v_\text{а}}{v_\text{п}} = \frac{1 - e}{1 + e}.
%\end{equation*}
%Используя это соотношения, преобразуем ЗСЭ:
%\begin{gather}
%    \frac{v_\text{п}^2}{2} \left( 1 - \frac{(1 -e)^2}{(1 + e)^2} \right) = GM \left( \frac{1}{a(1-e)} - \frac{1}{a(1+e)} \right),\\
%    \frac{v_\text{п}^2}{2} \cdot \frac{ 1 + 2e + e^2 - 1 + 2e - e^2}{(1+e)^2} = \frac{GM}{a} \cdot \frac{1 + e - 1 +  e}{(1+e)(1-e)},\\
%    v_\text{п} = \sqrt{\frac{GM}{a}}\sqrt{\frac{1+e}{1-e}}, \quad \quad v_\text{a} = \sqrt{\frac{GM}{a}}\sqrt{\frac{1-e}{1+e}}.
%\end{gather}
%Запишем теперь ЗСЭ для перицентра и произвольной точки орбиты на расстоянии $r$:
%\begin{gather*}
%    -\frac{GMm}{q} + \frac{m v^2_\text{п}}{2} = -\frac{GMm}{r} + \frac{m v^2}{2},\\
%    -\frac{GMm}{q} + \frac{GMm}{2a} \cdot \frac{1+e}{1-e} = -\frac{GMm}{r} + \frac{m v^2}{2},\\
%    v^2 = GM \left( \frac{2}{r} - \frac{2}{a(1 - e)} + \frac{1+e}{a (1-e) }\right) = GM \left( \frac{2}{r} - \frac{1}{a} \right),
%\end{gather*}
\begin{equation}
    v = \sqrt{ GM \left( \frac{2}{r} - \frac{1}{a} \right)}.
    \label{eq:int-energy}
\end{equation}
Полученное выражение и называется интегралом энергии. Согласно \eqref{eq:int-energy} и \eqref{eq:ell-eq-pol} для скорости тела в произвольной точке орбиты также справедливо выражение
\begin{equation}
    v = \sqrt{\frac{GM}{p}\cdot(1 + 2 e \cos \nu + e^2)},
\end{equation}
где $\nu$~--- истинная аномалия, а $p$~--- фокальный параметр.

Установим зависимость скорости от эксцентрической аномалии~$E$\footnote{\lookSecRef{sec:kepler-eq}}, для этого воспользуемся выражением~\eqref{eq:kepler-eq-r-E} и подставим его в интеграл энергии~\eqref{eq:int-energy}, 
\begin{equation}
	v = \sqrt{\frac{GM}{a}}\sqrt{\frac{1 + e \cos E}{1 - e \cos E}}.
	\label{eq:orbit-motion-eccentcic-int-energy}
\end{equation}

%Найдем величину момента импульса пробной массы $m$ на эллиптической орбите. В силу постоянства данной величины, можно выбрать любую точку орбиты для её поиска. Проще всего рассмотреть перицентр или апоцентр, рассмотрим первый.
%\begin{multline*}
%    L
%    = m v_q q
%    = m \sqrt{\frac{GM}{a} \frac{1+e}{1-e}} \cdot a(1-e) =\\
%    = m \sqrt{GMa (1 + e)(1-e)}
%    = m \sqrt{GMa(1-e^2)}
%    = m \sqrt{GMp}.
%\end{multline*}

Найдем величину момента импульса пробной массы $m$ на параболической орбите. В силу постоянства данной величины, можно выбрать любую точку орбиты для её поиска. Проще всего рассмотреть перицентр:
\begin{multline*}
    L
    = m v_q q
    = m v_2(q) q
    = m \sqrt{\frac{2GM}{q}} \cdot q =\\
    = m \sqrt{2GMq}
    = m \sqrt{2GM \cdot \frac{p}{2}}
    = m \sqrt{GMp}.
\end{multline*}
Момент импульса также можно записать в терминах эксцентрической аномалии. Из такого равенства можно получить зависимость угла между радиус-вектором и вектором скорости от эксцентрической аномалии. Записав $L$ как $m r v \sin \alpha$, а далее подставив (\ref{eq:kepler-eq-r-E}) и (\ref{eq:orbit-motion-eccentcic-int-energy})
\begin{gather}
	m a (1 - e \cos E) \cdot \sqrt{\frac{GM}{a}}\sqrt{\frac{1 + e \cos E}{1 - e \cos E}} \sin \alpha = m \sqrt{GMa(1-e^2)}, \nonumber\\
	 \sin \alpha = \sqrt{\frac{1-e^2}{1- e^2 \cos^2 E}}.
\end{gather}

