Разберем также второй способ найти координаты точек $L_{4}$ и $L_5$. Рассмотрим \picRef{pic:larg-points-4-5_2}. Пусть положение тел с массами $M_1$ и $M_2$, а также точки $L_4$, относительно центра масс задается радиус-векторами $\vec{r}_1$, $\vec{r}_2$ и $\vec{r}_3$ соответственно. Выпишем координаты этих векторов, приняв расстояние между массивными телами за $R$:
\begin{equation*}
    \vec{r}_1
        = \begin{pmatrix}
            -R \cdot \dfrac{M_2}{M_1 + M_2}\\[.5pc]
            0
        \end{pmatrix}
    \equiv \begin{pmatrix}
        -R \alpha \\
        0
    \end{pmatrix}; \quad
    \vec{r}_2
        = \begin{pmatrix}
            R \cdot \dfrac{M_1}{M_1 + M_2}\\[.5pc]
            0
        \end{pmatrix}
    \equiv \begin{pmatrix}
        R \beta \\
        0
    \end{pmatrix}; \quad
    \vec{r}_3
        = \begin{pmatrix}
            x \\
            y
        \end{pmatrix}.
\end{equation*}
Координаты $(x, y)$ ненулевого вектора $\vec{r}_3$  и предстоит найти. Суть данного способа заключается в проецировании сил, действующих на пробную массу, находящуюся в точке $L_4$ на радиальную относительно центра масс ось и тангенциальную. Направление радиальной оси задается вектором $\vec{r}_3$, а тангенциальной~---  вектором ей ортогональным с координатами, например, $(y, -x)$. Обозначим его $\vec{r}_3^\perp$. Силы, принимаемые в расчет равны
\begin{gather*}
    \vec{F}_1
    = \dfrac{G M_1 m}{|\vec{r}_1 - \vec{r}_3 |^3} (\vec{r}_1 - \vec{r}_3)
    = \frac{G M_1 m}{\big[ (R \alpha + x)^2 + y^2 \big]^{3/2}}\begin{pmatrix}
        -R\alpha - x \\
        -y
    \end{pmatrix}
    \equiv \frac{G M_1 m}{A}\begin{pmatrix}
        -R\alpha - x \\
        -y
    \end{pmatrix},\\
    \vec{F}_2
        = \dfrac{G M_2 m}{|\vec{r}_2 - \vec{r}_3 |^3} (\vec{r}_2 - \vec{r}_3)
        = \frac{G M_2 m}{\big[ (R \beta - x)^2 + y^2 \big]^{3/2}}\begin{pmatrix}
            R\beta - x \\
            -y
        \end{pmatrix}
    \equiv \frac{G M_2 m}{B}\begin{pmatrix}
        R\beta - x \\
        -y
    \end{pmatrix},\\
    \vec{F}_\text{ц.б}
        = m\omega^2 \vec{r}_3
        = \frac{G (M_1 + M_2)m}{|\vec{r}_1 - \vec{r}_2|^3} \vec{r}_3
        = \frac{G(M_1 + M_2)m}{R^3} \begin{pmatrix}
            x \\
            y
        \end{pmatrix},
\end{gather*}
Запишем уравнение баланса этих сил в проекции на ось $\vec{r}_3^\perp$, используя координатное представление скалярного произведения:
\begin{gather*}
    \frac{ \left( \vec{r_3^\perp},  \vec{F}_1 \right)}{\left| \vec{r}_3^\perp \right|} + \frac{ \left( \vec{r_3^\perp}, \vec{F}_2 \right)}{\left| \vec{r}_3^\perp \right|} = 0,\\
%    \frac{ \left( \vec{r_3^\perp} \right)_x  F_1^x + \left( \vec{r_3^\perp} \right)_y F_1^y}{ \sqrt{\left( \vec{r_3^\perp} \right)_x^2 + \left( \vec{r_3^\perp} \right)_y^2}} + \frac{ \left( \vec{r_3^\perp} \right)_x F_2^x + \left( \vec{r_3^\perp} \right)_y  F_2^y}{ \sqrt{\left( \vec{r_3^\perp} \right)_x^2 + \left( \vec{r_3^\perp} \right)_y^2}} = 0,\\
%    \frac{ y F_1^x - x F_1^y}{ \sqrt{x^2 + y^2}} + \frac{ y F_2^x - x F_2^y}{ \sqrt{x^2 + y^2}} = 0,\\
    y F_1^x - x F_1^y + y F_2^x - x F_2^y = 0,\\
%    \begin{aligned}
%    \frac{1}{\sqrt{x^2 + y^2}} \Bigg(&\dfrac{-y M_1 (R\alpha + x )}{\big[ (R \alpha + x)^2 + y^2 \big]^{3/2}} + \dfrac{x y M_1}{\big[ (R \alpha + x)^2 + y^2 \big]^{3/2}}  + \\
%    & + \frac{y M_2(R\beta - x)}{\big[ (R \beta - x)^2 + y^2 \big]^{3/2}} + \frac{xy  M_2}{\big[ (R \beta - x)^2 + y^2 \big]^{3/2}} \Bigg)= 0,
%    \end{aligned}\\
%    \dfrac{-y M_1 (R\alpha + x ) + x y M_1}{\underbrace{\big[ (R \alpha + x)^2 + y^2 \big]^{3/2}}_A}     + \frac{y M_2(R\beta - x) + xy  M_2}{\underbrace{\big[ (R \beta - x)^2 + y^2 \big]^{3/2}}_B}= 0,\\
    \dfrac{-y M_1 (R\alpha + x ) + x y M_1}{A}    + \frac{y M_2(R\beta - x) + xy M_2}{B}= 0,\\
%    \dfrac{-y M_1 (R\alpha + x ) + x y M_1}{A}     + \frac{y M_2(R\beta - x) + xy  M_2}{B}= 0,\\
    yR \left( - \frac{M_1 \alpha}{A} + \frac{M_2 \beta}{B}\right) + xy \cancelto{0}{\left( - \frac{M_1}{A} + \frac{M_1}{A} - \frac{M_2}{B} + \frac{M_2}{B} \right)} = 0.
\end{gather*}
Подставим выражения для $\alpha$ и $\beta$:
\begin{gather*}
    \frac{yRM_1M_2}{M_1 + M_2} \left\{ \frac{1}{\big[ (R \alpha + x)^2 + y^2 \big]^{3/2}} - \frac{1}{\big[ (R \beta - x)^2 + y^2 \big]^{3/2}} \right\} = 0,\\
    \frac{\big[ (R \beta - x)^2 + y^2 \big]^{3/2} - \big[ (R \alpha + x)^2 + y^2 \big]^{3/2}}{\big[ (R \alpha + x)^2 + y^2 \big]^{3/2}\big[ (R \beta - x)^2 + y^2 \big]^{3/2}} = 0.
%    \big[ (R \beta - x)^2 + y^2 \big]^{3/2} = \big[ (R \alpha + x)^2 + y^2 \big]^{3/2},\\
\end{gather*}
Заметим, полученное уравнение верно лишь при равенстве нулю числителя, что равносильно условию $R \beta - x = R \alpha + x$, следовательно,
\begin{equation*}
    x = \frac{R}{2} \left( \beta - \alpha \right) = \frac{R}{2} \cdot \frac{M_1 - M_2}{M_1 + M_2}.
\end{equation*}
Важно отметить, при полученном значении $x$ выполняется равенство $A = B$, чем мы сейчас и воспользуемся. Найдём возможные значения $y$, при заданном значении $x$, полученном выше. Для этого запишем уравнения баланса сил  $\vec{F}_1$, $\vec{F}_2$ и $\vec{F}_\text{ц.б}$ на ось $\vec{r_3}$:
\begin{gather*}
    \frac{ \left( \vec{r_3},  \vec{F}_1 \right)}{\left| \vec{r}_3 \right|} + \frac{ \left( \vec{r_3}, \vec{F}_2 \right)}{\left| \vec{r}_3\right|} + \frac{m \omega^2 \left(\vec{r}_3, \vec{r}_3\right)}{\left| \vec{r}_3\right|} = 0,\\
    \frac{-x M_1 (R \alpha + x) - y^2 M_1}{A} + \frac{ x M_2 ( R \beta - x) - y^2 M_2}{B} + \frac{M_1 + M_2}{R^3} \cdot \left( x^2 + y^2 \right) = 0,\\
    \left( x^2 + y^2 \right) \left(   - \frac{M_1}{A} - \frac{M_2}{B} + \frac{M_1 + M_2}{R^3}\right) + xR \cancelto{0}{\left(  - \frac{M_1 \alpha}{A} + \frac{M_2 \beta}{B} \right)} = 0,\\
%    \left( x^2 + y^2 \right) \left(  \frac{M_1 + M_2}{R^3} - \frac{M_1 + M_2}{B} \right) + \cancelto{0}{\frac{xR M_1 M_2}{M_1 + M_2} \left( \frac{1}{B} - \frac{1}{B} \right)} = 0,\\
    \left( x^2 + y^2 \right)(M_1 + M_2) \left(  - \frac{1}{B} + \frac{1}{R^3} \right) = 0.
%    B = R^3,\\
%    y^2 = R^2 \left( 1 - \left(\frac{1}{2}\left(\beta - \alpha \right) - \beta \right)^2\right) = R^2 \left( 1 - \frac{1}{4}(\underbrace{\beta + \alpha}_{1})^2 \right) = \frac{3}{4}R^2,\\
\end{gather*}
Случай $x = y = 0$ нарушает предположение $\vec{r}_3 \not = \vec{0}$, значит приведенное выше равенство выполняется только при условии $R^3 = B$, иначе,
\begin{equation*}
(x - R \beta)
    ^2 + y^2 = R^2 \quad \Rightarrow y = \pm \frac{\sqrt{3}}{2} R.
\end{equation*}
Полученные координаты совпадают с найдеными ранее. Относятся, соответственно, к точкам $L_4$ и $L_5$.

