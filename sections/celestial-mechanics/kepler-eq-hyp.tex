\subsection{Гиперболическое уравнение Кеплера}

Для гиперболы тоже есть свой вариант уравнения Кеплера. Для того, чтобы его получить запишем выражение для момента импульса и уравнение гиперболы в полярных координатах:
\begin{equation*}
    l = r^2 \dot{\nu} = \sqrt{GMp}, \quad r = \frac{p}{1 + e \cos \nu}.
\end{equation*}
Подставив одно уравнение в другое можно получить такое дифференциальное соотношение:
\begin{equation*}
    \sqrt{GMp}\,dt=\frac{p^2 \,d\nu}{(1 + e \cos{\nu})^2}.
\end{equation*}
Вспомнив значение для частоты из третьего закона Кеплера и интегрируя с обеих сторон выходит:
\begin{equation*}
    \int\limits_{\nu_1}^{\nu_2}{\frac{d\nu}{(1+e\cos \nu)^2}} = \frac{\omega T}{\sqrt{(e^2 - 1)^3}}.
\end{equation*}
Далее сделаем замену в интеграле:
\begin{equation*}
    \cos \nu = \frac{e - \cosh H}{e \cosh H - 1}, \quad \sin \nu = \frac{\sqrt{e^2-1} \sinh H}{e \cosh H - 1}.
\end{equation*}
Дифференциал тогда:
\begin{gather*}
    -\sin \nu \, d \nu = -\frac{(e^2 - 1) \sinh H}{(e \cosh H - 1)^2}\,dH = -\frac{\sqrt{e^2-1}\sin \nu}{e \cosh H - 1}\,dH, \\
    d\nu = \frac{\sqrt{e^2-1}}{e \cosh H - 1}\,dH.
\end{gather*}
После замены интеграл упрощается до:
\begin{equation*}
    \int\limits_{H_1}^{H_2}{\frac{e \cosh H - 1}{\sqrt{(e^2-1)^3}} \,dH} = \frac{\omega T}{\sqrt{(e^2 - 1)^3}}.
\end{equation*}
Отсюда получаем \term{уравнение Кеплера для гиперболы}:
\begin{equation}
    \omega T = M = e \sinh H - H.
\end{equation}
