\subsection{Типы орбит}
\label{sec:orbit-types}

Вернемся к первому закону Кеплера, в частность к полученное уравнению орбиты \eqref{eq:first-kepler-law-conic-seq-eq}, определим, как знак полной механической энергии связан с видом орбиты. 

Заметим, в принятых обозначениях $s$~--- перицентр орбиты, так как истинная аномалия отсчитывается от перицентра, то есть минимальное расстояние от тела до гравитирующего центра. Согласно закону сохранения энергии на минимальном расстоянии достигается максимальная скорость $v_\text{макс}$. Также из соображения минимальности расстояния, в это момент скорость перпендикулярна радиус вектору, следовательно момент импульса тела $L = m\sqrt{GMh} =  m s v_\text{макс}$, то есть
 \begin{equation*}
     v_\text{макс} = \frac{\sqrt{Gmh}}{s}.
 \end{equation*}
 Запишем закон сохранения энергии:
 \begin{gather*}
     \frac{m v_\text{макс}^2}{2} - \frac{GMm}{s} = E_0,\\
     \frac{GMmh}{2s^2} - \frac{GMm}{s} = E_0,\\
     E_0 = GMm \cdot \frac{h - 2s}{2s^2}.
 \end{gather*}
 
 Таким образом, если движение финитно~--- $E_0 < 0$, то $h < 2s$ и эксцентриситет
 \begin{equation*}
     e = \frac{h-s}{s} < 1,
 \end{equation*}
 следовательно, орбита является \imp{эллипсом}. Пусть теперь $E_0=0$, тогда $h = 2s$ и $e = 1$, откуда орбита~--- \imp{парабола}. Остается рассмотреть случай $E_0 > 0$, тогда $h > 2s$, $e > 1$ и орбита является \imp{гиперболой}.
 
 Частным случаем эллиптической орбиты является \imp{окружность}, когда $s = h \equiv r$~--- радиус орбиты. В этом случае 
 \begin{equation*}
     E_0 = -\frac{GMm}{2r} = \frac{\Pi}{2},
 \end{equation*}
 откуда $K = - \Pi / 2$, следовательно,
 \begin{gather}
     \frac{m v_1^2}{2} = \frac{GMm}{2r}, \nonumber \\
     v_1 = \frac{GM}{r}.
     \label{eq:circle-speed}
 \end{gather} 
 Полученная скорость называется \term{первой космической} или \term{круговой} скоростью или и является минимальной скоростью, чтобы оставаться на орбите вокруг гравитирующего центра массы $M$ на расстоянии $r$ от него.
 
 