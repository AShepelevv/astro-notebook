\subsection{Закон сохранения энергии}

Эмпирически установлено, что в замкнутой системе энергия не берётся из ниоткуда и не исчезает в никуда, а лишь переходит из одной формы в другую. Именно этот принцип и называется \term{законом сохранения энергии}. Так для замкнутой системы двух тел сохраняется полная механическая энергия системы $E_0$~--- сумма потенциальной ($\Pi$) и кинетической ($K$) энергий.

Отсюда следует, что для движения тела c массой $m$ в гравитационном  в поле тела
с массой $M\gg m$ со скоростью $v$ на расстоянии $r$ от
гравитационного центра справедливо следующее соотношение\footnote{здесь не рассматривается вращательное движение тел}:
\begin{equation}
    \frac{m v^2}{2}-\frac{GM m }{r}=E_0,
\end{equation}
данное равенство принято называть \imp{законом сохранения энергии} тела, движущегося в поле консервативных (потенциальных) сил.

Определим, как знак полной механической энергии связан с характером движения пробного тела в гравитационном поле массивного. Пусть $E_0 < 0$, так как $v^2 \geqslant 0$, то $1/r > 0$, следовательно $r < \infty$, то есть движение ограничено (финитно).

При инфинитном движении в некоторый момент пробное тело удаляется бесконечно далеко от массивного, другими словами $r \rightarrow \infty$. В этом случае, согласно закону сохранения энергии, $E_0 = K \geqslant 0$, а значит, минимальное значение полной механической энергии, при котором движение неограниченно, равно нулю. При этом на бесконечном удалении от гравитирующего центра скорость пробного тела также будет равна нулю. Если же $E_0 > 0$, минимальная скорость пробного тела всегда больше нуля, а получение выражения для вычисления ее величины оставим читателю.

\subsubsection*{Вторая космическая скорость}

Найдем минимальную скорость $v_2$, необходимую пробному телу, чтобы удалиться от массивного тела бесконечно далеко. Как было показано выше, в этом случае $E_0 = 0$, следовательно,
\begin{gather}
    \frac{mv_2^2}{2} = \frac{GMm}{r}, \nonumber \\
    v_2 = \sqrt{\frac{2GM}{r}}.
\end{gather}
Полученная скорость $v_2$ называется \term{второй космической} или \term{пара\-бо\-ли\-ческой}\footnote{причина такого названия ясна из раздела \ref{sec:orbit-types}} скоростью и является минимальной необходимой скоростью, чтобы покинуть зону влияния гравитации тела с массой $M$, находясь от него расстоянии $r$.
