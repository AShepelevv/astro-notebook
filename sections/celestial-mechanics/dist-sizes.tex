\subsection{Расстояние и размеры}
\term{Астрономическая единица}~--- единица измерения расстояния в астрономии, равная большой полуоси орбиты Земли.
\begin{equation}
	1~\au = 149\:597\:870\:700~\text{м} \simeq 1.5 \times 10^{11}~\text{м}.
\end{equation}

\term{Годичный параллакс}\footnote{Важно отметить, здесь $\pi$~--- лишь обозначение, ничего общего с числом $\pi$ не имеющее.} ($\pi$) объекта~--- это угол, под которым видно
орбиту Земли из окрестностей данного объекта. Применяется к объектам вне
Солнечной системы.
\begin{equation}
	\tg \pi = \frac{a_\oplus}{r},
	\label{eq:parallax-sin}
\end{equation}
где $a_\oplus$~--- большая полуось орбиты Земли и $r$~--- расстояние до объекта
имеют одинаковые единицы измерений. Учитывая малость угла $\pi$, можно считать $\tg\pi \simeq \pi$ в \eqref{eq:parallax-sin}, тогда
\begin{equation}
	\pi = \frac{a_\oplus}{r}.
	\label{eq:parallax}
\end{equation}

\begin{wrapfigure}[10]{r}{0.45\tw}
	\centering
	\vspace{-2pc}
	\begin{tikzpicture}
%	    \tkzSetUpLabel[font=\footnotesize]
%	    \tkzSetUpPoint[fill=white, size=2]
%	    \tkzSetUpLine[line width=.4pt]
%	    \tkzSetUpArc[line width=.4pt]
%	    
	    \tkzDefPoint(0,0){Sun}
	    \tkzDefPoint(3,0){Star}
	    \tkzDefPoint(0,1){E}
%	    
	    \tkzLabelPoint[above](E){Земля}
	    \tkzLabelPoint[below](Sun){Солнце}
	    \tkzLabelPoint[below](Star){Звезда}
	    
	    \tkzDrawCircle[dashed, black, line width=.5pt](Sun,E)
	    \tkzDrawPolygon[thick](Sun,Star,E)
	    \tkzMarkRightAngle[size=0.2](Star,Sun,E)
	    \tkzMarkAngle(E,Star,Sun)
	    \tkzLabelAngle[font=\footnotesize, pos=1.3](E,Star,Sun){$\pi$}   
	    
	    \earth(E)
	    \sun(Sun)
	    \pointStar(Star)
	    
	    \tkzLabelSegment[below](Sun,Star){$r$}
	    \tkzLabelSegment[left=-2pt](Sun,E){$a_\oplus$}
	\end{tikzpicture}
	\caption{Схема годичного параллакса}
\end{wrapfigure}
Расстояние $r$, с которого большая полуось орбиты Земли $a_\oplus$ видна под углом $\pi = 1''$ называется \term[парсек]{1 парсеком}. Так как
\begin{equation}
	1~\text{рад} = \frac{180^\circ}{\pi} \simeq  3 438' \simeq 206265''
	\quad \Longrightarrow \quad \mathsf{1~\text{\sffamily пк} =
	206265~\text{\sffamily а.\,е.}},
\end{equation}
следовательно, записывая большую полуось орбиты Земли в \au, а расстояние до звезды в парсеках, получаем параллакс в секундах. Таким образом,
\begin{equation}
	r_\text{пк} = \frac{1~\au}{\pi''}.
\end{equation}

\term{Угловой размер объекта}~--- это угол, под которым видно объект. Для сферически симметричных объектов с радиусом $R$, угловой размер (диаметр) при наблюдении с расстояния $r$ определяется как
\begin{equation}
	\rho = 2 \arcsin \frac{R}{r}.
\end{equation}
В случае, когда $r\gg R$, можно считать, что $\sin \rho \simeq \rho$, тогда
\begin{equation}
	\rho \simeq \frac{2 R}{r}.
\end{equation}

\vspace{-1.5pc}
\begin{figure}[h!]
	\begin{minipage}[b]{0.5\tw}
	   \centering
        \begin{tikzpicture}
            \tkzSetUpLine[line width=.4pt]
            \tkzSetUpLabel[font=\footnotesize]
            \tkzSetUpPoint[fill=white, size=2]
        
            \tkzDefPoint(0,0){C}
            \tkzDefPoint(4,0){O}
            \tkzDefPointBy[homothety=center C ratio .25](O) \tkzGetPoint{U}
            
            \tkzDrawCircle[color=black, fill=gray!40, thick](C,U)
            \tkzDefLine[tangent from = O](C,U) \tkzGetPoints{I1}{I2}
            \tkzDrawSegments(C,I1 C,I2 C,O)
            \tkzDrawSegments[thick](O,I2 O,I1)
            \tkzMarkRightAngles[size=0.2](O,I1,C C,I2,O)
            \tkzMarkAngle[size=1.2](I2,O,C)
            \tkzMarkAngle[size=1.1](C,O,I1)
            
            \tkzLabelSegment[left](C,I1){$R$}
            \tkzLabelSegment[left](C,I2){$R$}
            \tkzLabelSegment[above](C,O){$r$}
            
            \DeclareCollectionInstance{angular-size-xfrac}{xfrac}{mathdefault}{math}{
                denominator-bot-sep = -1pt,
                slash-symbol = \scalebox{0.9}{/},
                numerator-bot-sep = 3pt,
                scaling= true,
                slash-right-mkern= -2 mu,
                slash-left-mkern= -1.5 mu
            }
            \UseCollection{xfrac}{angular-size-xfrac}
            
            \tkzLabelAngle[pos=1.5](I2,O,C){$\sfrac{\rho}{2}$}
            \tkzLabelAngle[pos=1.5](C,O,I1){$\sfrac{\rho}{2}$}
            
            \tkzDrawPoints(C, O, I1, I2)
        \end{tikzpicture}
		\captionof{figure}{Угловой размер}
	\end{minipage}
	\hfill
	\begin{minipage}[b]{0.5\tw}
		\centering
		\begin{tikzpicture}
            \tkzSetUpLine[line width=.4pt]
            \tkzSetUpLabel[font=\footnotesize]
            \tkzSetUpPoint[fill=white, size=2]
        
            \tkzDefPoint(0,0){O}
            \tkzDefPoint(4,0){S}
            \tkzDefPointBy[homothety=center O ratio .25](S) \tkzGetPoint{X}
            \tkzDefPointBy[rotation=center O angle -90](X) \tkzGetPoint{C}
            
            \tkzDrawCircle[color=black, fill=gray!40, thick](C,O)

            \tkzDrawSegments[thick](O,S C,S)
            \tkzDrawSegments(C,O)
            \tkzMarkRightAngles[size=0.2](C,O,S)
            \tkzMarkAngle(O,S,C)

            
            \tkzLabelSegment[below](C,S){$r$}
            \tkzLabelSegment[left=-2pt](C,O){$R_\oplus$}
            \tkzLabelAngle[pos=1.3, font=\footnotesize](O,S,C){$p$}
            
            \tkzDrawPoints(C, O, S)
        \end{tikzpicture}
		\captionof{figure}{Горизонтальный параллакс}
	\end{minipage}
\end{figure}

\term{Горизонтальный параллакс}~$(p)$~--- это угловой радиус Земли при наблюдении с объекта:
\begin{equation}
	\sin p=\frac{R_\oplus}{r}.
\end{equation}

\term{Правило Тициуса-Боде} --- эмпирическая формула, приблизительно описывающая
радиусы орбит планет в Солнечной системе:
\begin{equation}r=\frac{3\cdot 2^n+4}{10}, \quad n=-\infty, 0, 1, 2...
\end{equation}

