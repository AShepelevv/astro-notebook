\subsection{Собственное движение звёзд}
\term{Собственным движением} $(\mu)$ называется изменение координат звёзд на небесной сфере, вызванное относительным движением звёзд и Солнца в пространстве, измеряется обычно в mas/год. Пусть $v_\tau$~--- гелиоцентрическая (относительно Солнца) тангенциальная скорость звезды, $d$~---~расстояние до неё от Солнца, тогда её собственное движение
\begin{equation}
	\mu = \frac{v_\tau}{d}.
\end{equation}

Разделяют собственное движение по склонению~--- $\mu_\delta$ и собственное движение по прямому восхождению~--- $\mu_\alpha$, для которых справедливы следующие выражения:
\begin{gather}
	\mu_\delta(t) = \lim_{\Delta t \rightarrow 0} \frac{\delta(t + \Delta t) - \delta(t)}{\Delta t} = \delta'(t),\\
	\mu_\alpha(t) = \lim_{\Delta t \rightarrow 0} \frac{\alpha(t + \Delta t) - \alpha(t)}{\Delta t} = \alpha'(t).
\end{gather}

\begin{wrapfigure}[15]{r}{.4\tw}
	\begin{flushright}
		\vspace{-1pc}
		\begin{tikzpicture}[yscale=0.7]
			\footnotesize
			\draw [dashes] (0, 4) arc(90:0:3 and 4);
			\draw [dashes] (0, 4) arc(90:0:2 and 4);
			%
			\draw [dashes] (3.47, 2) arc(0:-70:3.47 and 1.16);
			\draw [dashes] (2.64, 3) arc(0:-70:2.64 and 0.88);
			%
			\draw [thick, -latex] (2.3, 2.55) arc(-34:-56:2.64 and 0.88);
			\draw [thick, -latex] (2.3, 2.55) arc(53:29:2 and 4);
			\draw [thick, -latex] (2.3, 2.55) .. controls (2.3, 1.9) and (2.1, 1.4) .. (1.93, 1.03);
			%
			\draw (.9, 2.2) node [anchor=south] {$\delta(t)$};
			\draw (1.2, .9) node [anchor=south] {$\delta(t + \Delta t)$};
			%
			\draw (2, 0) node [anchor=north] {$\alpha(t + \Delta t)$};
			\draw (3, 0) node [anchor=north] {$\alpha(t)$};
			%
			\draw [fill=white] (2.3, 2.55) circle (0.03);
			\draw [fill=white] (1.93, 1.03) circle (0.03);
			\draw [fill=white] (0, 4) circle (0.03);
			%
			\draw (0, 4) node [anchor=north] {$P$};
			%
			\draw (1.9, 2.4) node [anchor=south] {$\mu_\alpha$};
			\draw (2.6, 2.05) node [anchor=west] {$\mu_\delta$};
			\draw (2.06, 1.65) node [anchor=south] {$\mu$};
			%
		\end{tikzpicture}
	\end{flushright}
\end{wrapfigure}
Из определения видно, что $\mu_\alpha$ является угловой скоростью по малому кругу, а значит, зависит от $\delta$. Следовательно, полное собственное движение $\mu$ можно найти из такой формулы:
\begin{equation}
	\mu = \sqrt{\mu_\delta^2 + \mu_\alpha^2 \cos^2 \delta},
\end{equation}
потому что радиус малого круга, состоящего из точек со склонением~$\delta$, равен $R \cos \delta$, где $R$~--- радиус сферы, содержащей этот круг.


\begin{figure}[h!]
	\begin{subfigure}[b]{0.47\tw}
		\begin{tikzpicture}[scale=1.05]
			\footnotesize
			
			%	\foreach \x in {0, .1,...,4} {
			%		\draw [line width=.1pt] (\x - 1, 0) -- (\x - 1, 4);
			%	};
			%
			%	\foreach \x in {0, 1,...,4} {
			%		\draw [line width=.4pt] (\x - 1, 0) -- (\x - 1, 4);
			%	};
			%
			%	\foreach \y in {0, .1,...,4} {
			%		\draw [line width=.1pt] (-1, \y) -- (4, \y);
			%	};
			%
			%	\foreach \y in {0, 1,...,4} {
			%		\draw [line width=.4pt] (-1, \y) -- (4, \y);
			%	};
			
			\draw [double] (.21, .21) arc (45:104:.3);
			\draw (-.93, 3.71) arc (-76:-35:.3);
			
			\draw (0, 0) -- (-1, 4);
			\draw (0, 0) -- (2, 2);
			\draw (-1, 4) -- (2.6, 1.6);
			
			\draw [thick, -latex] (-1, 4) -- (0, 4.25);
			\draw [thick, -latex] (-1, 4) -- (-.6, 2.4);
			
			\draw [fill=white] (-1, 4) circle (.03);
			\draw [fill=white] (0, 0) circle (.03);
			\draw [fill=white] (2, 2) circle (.03);
			
			\draw (1, 1) node [anchor=north west] {$r$};
			\draw (-.45, 2.1) node [anchor=north east] {$r_0$};
			\draw (.5, 2.95) node [anchor=south west] {$v_0 \Delta t$};
			\draw (0, 0) node [anchor=north] {Солнце};
			\draw (-1, 4) node [anchor=south east] {Звезда};
			
			\draw (.1, .3) node [anchor=south] {$\xi$};
			\draw (-.9, 3.75) node [anchor=north west] {$\gamma$};
			
			\draw (-.5, 4.15) node [anchor=south] {$v_\tau$};
			\draw (-.75, 3.1) node [anchor=east] {$v_r$};
		\end{tikzpicture}
		\caption{}
		\label{pic:proper-motion-1}
	\end{subfigure}
	\hfill
	\begin{subfigure}[b]{0.47\tw}
		\begin{tikzpicture}[scale=0.9]
			\footnotesize
			
			\draw (.2, 4.86) arc (-45:-135:0.28);
			\draw [double] (-1.65, 1.51) arc (5:80:0.25);
			
			\draw (0, 5) .. controls (-1.5, 4) and (-2, 2) .. (-2, 0);
			\draw (0, 5) .. controls (1.5, 4) and (2, 2) .. (2, 0);
			\draw (-2, 0) .. controls (-1, -.5) and (1, -.5) .. (2, 0);
			\draw (-1.9, 1.5) .. controls (-1, 1.5) and (1, 2) .. (1.5, 3);
			
			\draw [fill=white] (0, 5) circle (.03);
			\draw [fill=white] (-2, 0) circle (.03);
			\draw [fill=white] (2, 0) circle (.03);
			\draw [fill=white] (-1.9, 1.5) circle (.03);
			\draw [fill=white] (1.5, 3) circle (.03);
			
			\draw (-2, .2) -- (-1.8, .11) -- (-1.8, -.09);
			\draw (2, .2) -- (1.8, .11) -- (1.8, -.09);
			
			\draw (0, 5) node [anchor=south] {$P$};
			\draw (0, 1.9) node [anchor=north] {$\xi$};
			\draw (-1.75, 1.55) node [anchor=south west] {$\psi$};
			\draw (0, -.4) node [anchor=south] {$\Delta \alpha$};
			\draw (0, 4.8) node [anchor=north] {$\Delta \alpha$};
			\draw (-1, 4) node [anchor=east] {$90^\circ - \delta$};
			\draw (.9, 4.2) node [anchor=west] {$90^\circ - (\delta + \Delta \delta)$};
			\draw (0, -.4) node [anchor=north] {Небесный экватор};
		\end{tikzpicture}
		\caption{}
		\label{pic:proper-motion-2}
	\end{subfigure}
	\caption{}
\end{figure}

Получим выражение для координат звезды через достаточно большой промежуток времени $\Delta t$. Пусть в начальный момент времени звезда имеет собственное движение $\mu = (\mu_\alpha, \mu_\delta)$, лучевую скорость $v_r$, параллакс $\pi_0$ и координаты $(\alpha, \delta)$. Найдем сначала тангенциальную скорость:
\begin{equation*}
	v_\tau =  \mu r_0 = r_0 \sqrt{ \mu_\delta^2 + \mu_\alpha^2 \cos^2 \delta} = \frac{a_\oplus \sqrt{ \mu_\delta^2 + \mu_\alpha^2 \cos^2 \delta}}{\pi_0},
\end{equation*}
где $r_0 = a_\oplus / \pi_0$~--- расстояние до звезды в начальный момент.
Определим угол между лучом зрения и полной скоростью звезды, исходя из Рис.\,\ref{pic:proper-motion-1}, имеем:
\begin{equation*}
	\tg \gamma = \frac{v_\tau}{v_r},
\end{equation*}
при этом полная скорость $v_0 = \sqrt{v_\tau^2 + v_r^2}$. 

Снова обратимся к Рис.\,\ref{pic:proper-motion-1}. Из теоремы косинусов найдём расстояние от Солнца до звезды через время $\Delta t$:
\begin{equation*}
	r = \sqrt{r_0^2 + (v_0 \Delta t)^2 - 2 r_0 r_0 \Delta t \cos \gamma}.
\end{equation*}
Далее из теоремы синусов получим угловое перемещения звезды:
\begin{equation*}
	\sin \xi = \frac{v_0 \Delta t \sin \gamma}{r}.
\end{equation*}
Через компоненты собственного движения нетрудно получить угол между направлением на полюс и вектором полного собственного движения в начальный момент (см.~Рис.\,\ref{pic:proper-motion-2}):
\begin{equation*}
	\tg \psi =  \frac{\mu_a \cos \delta}{\mu_\delta}.
\end{equation*}
Теперь с помощью сферической теоремы косинусов можно определить склонение звезды через время $\Delta t$:
\begin{equation*}
	\sin (\delta + \Delta \delta) = \cos \xi \sin \delta + \sin \xi \cos \delta \cos \psi.
\end{equation*}
Остается из сферической теоремы синусов получить выражение для изменения прямого восхождения за время $\Delta t$:
\begin{equation*}
	\sin \Delta \alpha = \frac{\sin \psi \sin \xi}{\cos (\delta + \Delta \delta)}.
\end{equation*}


