\subsection{Собственное движение звёзд}
\term{Собственным движением} $(\mu)$ называется изменение координат звёзд на небесной сфере, вызванное относительным движением звёзд и Солнца в пространстве, измеряется обычно в mas/год. Пусть $v_\tau$~--- гелиоцентрическая (относительно Солнца) тангенциальная скорость звезды, $d$~---~расстояние до неё от Солнца, тогда её собственное движение
\begin{equation}
	\mu = \frac{v_\tau}{d}.
\end{equation}

Разделяют собственное движение по склонению~--- $\mu_\delta$ и собственное движение по прямому восхождению~--- $\mu_\alpha$, для которых справедливы следующие выражения:
\begin{gather}
	\mu_\delta(t) = \lim_{\Delta t \rightarrow 0} \frac{\delta(t + \Delta t) - \delta(t)}{\Delta t} = \delta'(t),\\
	\mu_\alpha(t) = \lim_{\Delta t \rightarrow 0} \frac{\alpha(t + \Delta t) - \alpha(t)}{\Delta t} = \alpha'(t).
\end{gather}

Из определения видно, что $\mu_\alpha$ является угловой скоростью по малому кругу, а значит, зависит от $\delta$. Следовательно, полное собственное движение $\mu$ можно найти из такой формулы:
\begin{equation}
	\mu = \sqrt{\mu_\delta^2 + \mu_\alpha^2 \cos^2 \delta},
\end{equation}
потому что радиус малого круга, состоящего из точек со склонением~$\delta$, равен $R \cos \delta$, где $R$~--- радиус сферы, содержащей этот круг.


\begin{figure}[h!]
    \begin{subcaptionblock}{0.27\tw}
        \centering
        \tdplotsetmaincoords{70}{179}
        \begin{tikzpicture}[
            tdplot_main_coords, 
            tdplotCsFront/.style={},
            tdplotCsBack/.style={dashed, opacity=0, line cap=round}
        ]
            \footnotesize

            \def\r{2.5}
            \def\d{30}
            \def\dd{50}
            \def\l{120}
            \def\ll{150}

            \begin{scope}
                % some magic
                \clip (-2.6,-8.5) rectangle (0,-0.5);

                % Draw spherical grid
                \foreach \a in {\d,\dd}{
                    \tdplotCsDrawLatCircle[dashed]{\r}{\a}
                }
                \foreach \a in {90 + \l, 90 + \ll}{
                    \tdplotCsDrawLonCircle[dashed]{\r}{\a}
                }
            \end{scope}

            \draw [semithick, latex-] ({\r*cos(\dd)*cos(\l)}, {\r*cos(\dd)*sin(\l)}, {\r*sin(\dd)}) arc (\l:\ll:{\r*cos(\dd)});

            \tdplotsetrotatedcoords{\ll}{0}{0}
            \draw [tdplot_rotated_coords, canvas is xz plane at y = 0, semithick, -latex] ({\r*cos(\dd)}, {\r*sin(\dd)}) arc (\dd:\d:\r);

            \def\m{acos(sin(\d)*sin(\dd) + cos(\d)*cos(\dd)*cos(\ll-\l))}
            \def\p{acos((sin(\dd) - sin(\d)*cos(\m))/(cos(\d)*sin(\m)))}
            \def\g{acos(cos(\d)*sin(\p))}
            \def\y{asin(sin(\d)/sin(\g))}
            \def\z{asin(sin(\d) * sin(\p) / sin(\g))}

            \tdplotsetrotatedcoords{\l - 90 - \z}{\g}{90 + \y};
            \draw[tdplot_rotated_coords, semithick, latex-] (\r,0,0) arc (0:\m:\r);

            %%% Label arcs
            \tdplotCsLabelPoint{\r}{\ll}{90 - \d / 2 - \dd / 2}{$\mu_\delta$}{above right=-2pt}
            \tdplotCsLabelPoint{\r}{(\l + \ll) / 2 - 3}{90 - \dd}{$\mu_\alpha$}{above}
            \tdplotCsLabelPoint{\r}{(\l + \ll) / 2}{90 - \d / 2 - \dd / 2}{$\mu$}{below right=-5pt}
            \tdplotCsLabelPoint{\r}{\l - 20}{90 - \d}{$\delta(t\!+\!\Delta t)$}{above}
            \tdplotCsLabelPoint{\r}{\l - 20}{90 - \dd}{$\delta(t)$}{above}

            %%% Draw and label points
            \tdplotCsDrawPoint{\r}{0}{0}
            \tdplotCsLabelPoint{\r}{0}{0}{$P$}{above}

            \tdplotCsDrawPoint{\r}{\l}{90 - \d}
            \tdplotCsDrawPoint{\r}{\l}{90 - \dd}
            \tdplotCsDrawPoint{\r}{\ll}{90 - \d}
            \tdplotCsDrawPoint{\r}{\ll}{90 - \dd}

            \tdplotCsLabelPoint{\r}{\l-3}{90 - \d + 7}{$\alpha(t\!+\!\Delta t)$}{below}
            \tdplotCsLabelPoint{\r}{\ll+3}{90 - \d + 15.5}{$\alpha(t)$}{below}
        \end{tikzpicture}
		\caption{}
		\label{pic:proper-motion-components} 
    \end{subcaptionblock}
    \hfill
	\begin{subcaptionblock}[b]{0.21\tw}
        \centering
		\begin{tikzpicture}
			\tkzDefPoint(0,0){Sun}
			\tkzDefPoint(-0.5,2.5){Star}
			\tkzDefPoint(0.8,1.3){P}

			\tkzCalcLength(Sun,Star)\tkzGetLength{ss}

			\tkzDefPointBy[homothety=center Star ratio {0.7 / \ss}](Sun) \tkzGetPoint{Vr}

			\tkzDefPointBy[homothety=center Star ratio {0.5 / \ss}](Sun) \tkzGetPoint{x}
			\tkzDefPointBy[rotation=center Star angle 90](x) \tkzGetPoint{Vt}

			\tkzDrawSegments(Sun,Star Sun,P)
			\tkzDrawLine[add=0 and 0.2](Star,P)

			\tkzMarkAngle[line width = .3pt, size=0](P,Sun,Star)
			\tkzMarkAngle[arc=ll, size=0.3](P,Sun,Star)
			\tkzMarkAngle[size=0.3](Sun,Star,P)

			\tkzLabelAngle[pos=0.6](P,Sun,Star){\footnotesize$\xi$}
			\tkzLabelAngle[pos=0.5](Sun,Star,P){\footnotesize$\gamma$}

			\tkzDrawSegments[semithick, -latex](Star,Vr Star,Vt)
			\tkzLabelSegment[left](Star,Vr){$\vec{v}_r$}
			\tkzLabelSegment[above](Star,Vt){$\vec{v}_\tau$}
			\tkzLabelSegment[left](Star,Sun){$r_0$}
			\tkzLabelSegment[below right=-2pt](P,Sun){$r$}
			\tkzLabelSegment[above right=-2pt](Star,P){$v_0 \Delta t$}

			\sun(Sun)
			\pointStar(Star)
			\tkzDrawPoint(P)
	    \end{tikzpicture}
		\caption{}
		\label{pic:proper-motion-1}
	\end{subcaptionblock}
	\hfill
	\begin{subcaptionblock}[b]{0.42\tw}
	    \centering
		\tdplotsetmaincoords{70}{179}
        \begin{tikzpicture}[tdplot_main_coords]
            \footnotesize
        
            \def\r{2}
            \def\d{20}
            \def\dd{55}
            \def\l{40}
            \def\ll{130}

            % some magic
            \clip (-2.3,-6.1) rectangle (2.3,2.1);
    

            % Draw spherical grid
            \draw[tdplot_screen_coords,thin,black!30] (0,0,0) circle (\r);
            \foreach \a in {-60,-30,...,60}{
                \tdplotCsDrawLatCircle[thin,black!20]{\r}{\a}
            }
            \foreach \a in {0,30,...,150}{
                \tdplotCsDrawLonCircle[thin,black!20]{\r}{\a}
            }
        
        
            %%%% Draw great circles
            % Celestial equator
            \tdplotCsDrawLatCircle[tdplotCsFill/.style={opacity=0.00}]{\r}{0}
        
        
            %%% Label arcs
            \tdplotCsLabelPoint{\r}{\l}{45 - \d / 2}{$90 - \delta$}{left}
            \tdplotCsLabelPoint{\r}{\ll}{45 - \dd / 2}{$90 - \delta - \Delta \delta$}{above right=-3pt}
            \tdplotCsLabelPoint{\r}{\l}{90 - \d / 2}{$\delta$}{left=-1pt}
            \tdplotCsLabelPoint{\r}{\ll}{90 - \dd / 2}{$\delta + \Delta \delta$}{right}
            \tdplotCsLabelPoint{\r}{(\l + \ll) / 2 }{(\d + \dd) / 2}{$\xi$}{below left=5pt}
        
        
            \def\x{acos(sin(\d)*sin(\dd) + cos(\d)*cos(\dd)*cos(\ll - \l))}
            \def\p{acos((sin(\dd) - sin(\d)*cos(\x))/(cos(\d)*sin(\x)))}
            \def\g{acos(cos(\d)*sin(\p))}
            \def\y{asin(sin(\d)/sin(\g))}
            \def\z{asin(sin(\d) * sin(\p) / sin(\g))}
        
        
            %%% Mark angles
            \def\angleRadius{0.4}
            \tdplotsetrotatedcoords{\l}{-\d}{0}
            \draw [tdplot_rotated_coords, canvas is yz plane at x = \r, double, fill=none, line cap=butt] ({\angleRadius * cos(85)},{\angleRadius * sin(85)}) arc (85:{\p + 15}:\angleRadius);
            \tdplotCsLabelPoint{\r}{\l}{90 - \d}{\adjustbox{raise=30pt}{$\psi$}}{right=4pt}
        
            \tdplotsetrotatedcoords{0}{0}{0}
            \draw [tdplot_rotated_coords, canvas is xy plane at z = \r] ({\angleRadius * cos(\l+8)},{\angleRadius * sin(\l+8)}) arc ({\l+8}:{\ll-10}:\angleRadius);
            \tdplotCsLabelPoint{\r}{0}{0}{$\Delta\alpha$}{below=3pt}
        
        
            %%% Mark right angles
            \tdplotsetrotatedcoords{0}{0}{\l};
            \draw [tdplot_rotated_coords](\r,0.2,0.2) coordinate (c) (\r,0.2,0) coordinate (a1) -- (c) (\r,0,0.2) coordinate (a2) -- (c) pic [angle radius=0.2cm] {right angle=a1--c--a2};
        
            \tdplotsetrotatedcoords{0}{0}{\ll};
            \draw [tdplot_rotated_coords](\r,-0.2,0.2) coordinate (c) (\r,-0.2,0) coordinate (a1) -- (c) (\r,0,0.2) coordinate (a2) -- (c) pic [angle radius=0.2cm] {right angle=a1--c--a2};
        
        
            %%% Draw arcs
            \tdplotsetrotatedcoords{\ll - 90}{90}{90};
            \draw[tdplot_rotated_coords, semithick] (\r,0,0) arc (0:90:\r);
        
            \tdplotsetrotatedcoords{\l - 90}{90}{90};
            \draw[tdplot_rotated_coords, semithick] (\r,0,0) arc (0:90:\r);
        
            \tdplotsetrotatedcoords{0}{0}{\l};
            \draw[tdplot_rotated_coords, semithick] (\r,0,0) arc (0:{\ll - \l}:\r);
        
            \tdplotsetrotatedcoords{\l - 90 - \z}{\g}{90 + \y};
            \draw[tdplot_rotated_coords, semithick] (\r,0,0) arc (0:\x:\r);
    
            
            %%% Draw and label points
            \tdplotCsDrawPoint{\r}{0}{0}
            \tdplotCsLabelPoint{\r}{0}{0}{$P$}{above left=-3pt}
        
            \tdplotCsDrawPoint{\r}{\l}{90}
            \tdplotCsDrawPoint{\r}{\ll}{90}
            \tdplotCsDrawPoint{\r}{\ll}{90 - \dd}
            \tdplotCsDrawPoint{\r}{\l}{90 - \d}
        \end{tikzpicture}
		\caption{}
		\label{pic:proper-motion-2}
	\end{subcaptionblock}
	\caption{}
\end{figure}

Получим выражение для координат звезды через достаточно большой промежуток времени $\Delta t$. Пусть в начальный момент времени звезда имеет собственное движение $\mu = (\mu_\alpha, \mu_\delta)$, лучевую скорость $v_r$, параллакс $\pi_0$ и координаты $(\alpha, \delta)$. Найдем сначала тангенциальную скорость:
\begin{equation*}
	v_\tau =  \mu r_0 = r_0 \sqrt{ \mu_\delta^2 + \mu_\alpha^2 \cos^2 \delta} = \frac{a_\oplus \sqrt{ \mu_\delta^2 + \mu_\alpha^2 \cos^2 \delta}}{\pi_0},
\end{equation*}
где $r_0 = a_\oplus / \pi_0$~--- расстояние до звезды в начальный момент.
Определим угол между лучом зрения и полной скоростью звезды, исходя из Рис.\,\ref{pic:proper-motion-1}, имеем:
\begin{equation*}
	\tg \gamma = \frac{v_\tau}{v_r},
\end{equation*}
при этом полная скорость $v_0 = \sqrt{v_\tau^2 + v_r^2}$.

Снова обратимся к Рис.\,\ref{pic:proper-motion-1}. Из теоремы косинусов найдём расстояние от Солнца до звезды через время $\Delta t$:
\begin{equation*}
	r = \sqrt{r_0^2 + (v_0 \Delta t)^2 - 2 r_0 r_0 \Delta t \cos \gamma}.
\end{equation*}
Далее из теоремы синусов получим угловое перемещения звезды:
\begin{equation*}
	\sin \xi = \frac{v_0 \Delta t \sin \gamma}{r}.
\end{equation*}
Через компоненты собственного движения нетрудно получить угол между направлением на полюс и вектором полного собственного движения в начальный момент (см.~Рис.\,\ref{pic:proper-motion-2}):
\begin{equation*}
	\tg \psi =  \frac{\mu_a \cos \delta}{\mu_\delta}.
\end{equation*}
Теперь с помощью сферической теоремы косинусов можно определить склонение звезды через время $\Delta t$:
\begin{equation*}
	\sin (\delta + \Delta \delta) = \cos \xi \sin \delta + \sin \xi \cos \delta \cos \psi.
\end{equation*}
Остается из сферической теоремы синусов получить выражение для изменения прямого восхождения за время $\Delta t$:
\begin{equation*}
	\sin \Delta \alpha = \frac{\sin \psi \sin \xi}{\cos (\delta + \Delta \delta)}.
\end{equation*}


