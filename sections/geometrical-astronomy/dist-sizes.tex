\subsection{Расстояние и размеры}
В астрономии наравне с единицами Международной системы единиц, СИ, используются внесистемные единицы измерения длины.
Одна из них, \term{Астрономическая единица}~--- единица измерения расстояния в астрономии, 
\begin{equation}
    1~\au = 149\:597\:870\:700~\text{м} \simeq 1.5 \times 10^{11}~\text{м},
\end{equation}
величина которой близка к длине большой полуось орбиты Земли. Точное значение было привязано к метру на Генеральной Асамблеи Международного Астрономического Союза в 2012 году~\cite{au}.

Астрономическая единица используется для описания расстояний между объектами Солнечной системы и экзопланетных систем. Расстояния между звездами и размеры более крупных формирований имеют существенно большую величину, чем астрономическая единица, поэтому здесь она не используется.

\begin{wrapfigure}[7]{r}{0.45\tw}
    \centering
    \vspace{-1.3pc}
    \tikzsetnextfilename{scheme-of-parallax}
    \begin{tikzpicture}
    \tkzDefPoint(0,0){Sun}
    \tkzDefPoint(3,0){Star}
    \tkzDefPoint(0,1){E}

    \tkzLabelPoint[above](E){Земля}
    \tkzLabelPoint[below](Sun){Солнце}
    \tkzLabelPoint[below](Star){Звезда}

    \tkzDrawCircle[dashed, black, line width=.5pt](Sun,E)
    \tkzDrawPolygon[thick](Sun,Star,E)
    \tkzMarkRightAngle[size=0.2](Star,Sun,E)
    \tkzMarkAngle(E,Star,Sun)
    \tkzLabelAngle[font=\footnotesize, pos=1.2](E,Star,Sun){$p$}

    \earth(E)
    \sun(Sun)
    \pointStar(Star)

    \tkzLabelSegment[below](Sun,Star){$r$}
    \tkzLabelSegment[left=-2pt](Sun,E){1\,\text{а.\!\:е.}}
\end{tikzpicture}

    \caption{Схема годичного параллакса}
    \label{pic:scheme-of-parallax}
\end{wrapfigure}
Первым методом определения расстояния до звёзд был \imp{годичный параллакс}, основанный на видимом движении звезды в результате орбитального движения Земли. \term{Годичный параллакс} $p$ объекта~--- это угол, под которым отрезок длины 1~\au из окрестностей данного объекта, применяется к объектам вне
Солнечной системы.
\begin{equation}
    \tg p = \frac{1~\text{\au}}{r},
    \label{eq:parallax-sin}
\end{equation}
где $r$~--- расстояние до объекта. Учитывая малость угла $p$, можно считать $\tg p \simeq p$ в \eqref{eq:parallax-sin}, тогда
\begin{equation}
    p \simeq \frac{1~\text{\au}}{r}.
    \label{eq:parallax}
\end{equation}

С помощью параллакса вводится ещё одна внесистемная единица длина, используемая в астрономии,~--- \term{парсек}. Это такое расстояние~$r$, что, находясь на нём, произвольный объект имеет годичный параллакс~$p = 1''$. Напомним,
\begin{equation*}
    1~\text{рад} = \frac{180^\circ}{\pi} =  \frac{10800'}{\pi} = \frac{648000''}{\pi} \simeq 206265'',
\end{equation*}
следовательно, 
\begin{equation}
    1~\text{пк} = \frac{648000}{\pi}~\text{\au} \simeq 3.085677581 \cdot 10^{16}~\text{м} \simeq 206265~\au
\end{equation}
В силу определение парсека, справедливо соотношение
\begin{equation}
    r~\lbrack \text{пк} \rbrack
       = \frac{1~\lbrack \au \rbrack}{\pi~\lbrack '' \rbrack}.
\end{equation}


Достаточно часть в астрономии имеет значение \term{угловой размер объекта}~--- угол, под которым видно объект. Для несимметричных объектов понимается некая средняя величина, для объектов с осевой симметрией, например галактик, указывается два размера вдоль каждой из осей. Для сферически симметричных объектов с радиусом $R$, угловой размер (диаметр) при наблюдении с расстояния $r$, \lookPicRef{pic:angilar-size}, определяется как\\
\begin{wrapfigure}[9]{r}{0.45\tw}
        \centering
        \vspace{-2pc}
        \ifthenelse{\boolean{useLightPlotVersion}}{}{
            \tikzsetnextfilename{angilar-size}
            \begin{tikzpicture}
    \tkzDefPoint(0,0){C}
    \tkzDefPoint(4,0){O}
    \tkzDefPointBy[homothety=center C ratio .25](O) \tkzGetPoint{U}

    \tkzDrawCircle[color=black, fill=gray!40, thick](C,U)
    \tkzDefLine[tangent from = O](C,U) \tkzGetPoints{I1}{I2}
    \tkzDrawSegments(C,I1 C,I2 C,O)
    \tkzDrawSegments[thick](O,I2 O,I1)
    \tkzMarkRightAngles[size=0.2](O,I1,C C,I2,O)
    \tkzMarkAngle[size=1.2](I2,O,C)
    \tkzMarkAngle[size=1.1](C,O,I1)

    \tkzLabelSegment[left](C,I1){$R$}
    \tkzLabelSegment[left](C,I2){$R$}
    \tkzLabelSegment[above](C,O){$r$}

    \DeclareCollectionInstance{angular-size-xfrac}{xfrac}{mathdefault}{math}{
        denominator-bot-sep = -1pt,
        slash-symbol        = \scalebox{0.9}{/},
        numerator-bot-sep   = 3pt,
        scaling             = true,
        slash-right-mkern   = -2 mu,
        slash-left-mkern    = -1.5 mu
    }
    \UseCollection{xfrac}{angular-size-xfrac}

    \tkzLabelAngle[pos=1.5](I2,O,C){$\sfrac{\rho}{2}$}
    \tkzLabelAngle[pos=1.5](C,O,I1){$\sfrac{\rho}{2}$}

    \tkzDrawPoints(C, O, I1, I2)
\end{tikzpicture}

        }
        \caption{Угловой размер}
        \label{pic:angilar-size}
    \end{wrapfigure}
\vspace{-2pc}
\begin{equation}
    \rho = 2 \arcsin \frac{R}{r}.
\end{equation}
В случае, когда $r\gg R$, можно считать, что $\sin \rho \simeq \rho$, тогда
\begin{equation}
    \rho \simeq \frac{2 R}{r}.
\end{equation}

\begin{wrapfigure}[6]{r}{0.45\tw}
    \centering
    \vspace{-1.5pc}
    \tikzsetnextfilename{horizontal-parallax}
    \begin{tikzpicture}
    \tkzDefPoint(0,0){O}
    \tkzDefPoint(4,0){S}
    \tkzDefPointBy[homothety=center O ratio .25](S) \tkzGetPoint{X}
    \tkzDefPointBy[rotation=center O angle -90](X) \tkzGetPoint{C}

    \tkzDrawCircle[color=black, fill=gray!40, thick](C,O)

    \tkzDrawSegments[thick](O,S C,S)
    \tkzDrawSegments(C,O)
    \tkzMarkRightAngles[size=0.2](C,O,S)
    \tkzMarkAngle(O,S,C)

    \tkzLabelSegment[below](C,S){$r$}
    \tkzLabelSegment[left=-2pt](C,O){$R_\oplus$}
    \tkzLabelAngle[pos=1.3, font=\footnotesize](O,S,C){$p$}

    \tkzDrawPoints(C, O, S)
\end{tikzpicture}

    \caption{Горизонтальный параллакс}
    \label{pic:horizontal-parallax}
\end{wrapfigure}
Для определения расстояния до объектов Солнечной системы также используется метод параллакса. Измеряемая величина $p$ называется \term{горизонтальным параллаксом}, \lookPicRef{pic:horizontal-parallax},~--- это угловой радиус Земли при наблюдении из окрестностей объекта:
\begin{equation}
    \sin p =\frac{R_\oplus}{r}.
\end{equation}
Так, средний горизонтальный параллакс Луны $p_{\rightmoon}$ составляет около $1^\circ$, а Солнца~--- $p_\odot \approx 9''$.
