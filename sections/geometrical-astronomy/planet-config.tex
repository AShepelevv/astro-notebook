\subsection{Конфигурации планет}
\term[внутренняя планета]{Внутренними планетами} называются планеты, большая полуось орбиты $a$ которых меньше большой полуоси орбиты Земли $a_\oplus$. Отсюда следует, что для наблюдателя на Земле \imp{внутренними} планетами являются лишь Венера и Меркурий, остальные относятся к \term[внешняя планета]{внешним}~--- тем, у которых большая полуось орбиты больше большой полуоси орбиты Земли. Для таких планет выделяют три основные конфигурации: \imp{верхнее соединение}, \imp{нижнее соединение} и \imp{максимальная элонгация}. Различают две максимальные элонгации~--- \term[восточная элонгация]{восточную} и \term[западная элонгация]{западную}, когда планета наблюдается к востоку и к западу от Солнца соответственно.

Внутренняя планета находится в \term[верхнее соединение]{верхнем соединении}, когда Земля, Солнце и планета лежат на одной прямой, при этом планета и Земля располагаются по разные стороны от Солнца. Если пренебречь наклоном орбит планет к плоскости эклиптики, для наблюдателя на Земле планета находится точно за Солнцем.

\begin{wrapfigure}[18]{r}{0.55\tw}
	\centering
	\vspace{-1pc}
	\begin{tikzpicture}
	    \tkzDefPoint(0,0){S}
	    \tkzDefPoint(0,-2){E}
	    
	    
	    \tkzDefPointBy[homothety=center S ratio .72](E) \tkzGetPoint{LC}
	    \tkzDefPointBy[homothety=center S ratio -1](LC) \tkzGetPoint{UC}
	    \tkzDefLine[tangent from = E](S,LC) \tkzGetPoints{EE}{WE}
	    
	    \tkzDefPointBy[homothety=center S ratio 1.52](E) \tkzGetPoint{OP}
	    \tkzDefPointBy[homothety=center S ratio -1](OP) \tkzGetPoint{C}
	    
	    \tkzDefPointBy[homothety=center S ratio 0.4](E) \tkzGetPoint{Mer}
        \tkzDrawCircle[line width=.4pt, black](S,Mer)
	    
	    \tkzDefLine[tangent at=E](S) \tkzGetPoint{q}
	    \tkzInterLC(E,q)(S,OP) \tkzGetPoints{EQ}{WQ}
        
        \tkzDrawSegments[dashed](WQ,EQ OP,C E,WE E,EE EE,S WE,S)
        \tkzDrawCircles[line width=0.4pt, black](S,E S,LC S,OP)
        
        \tkzMarkRightAngles[size=0.2](E,EE,S S,WE,E S,E,EQ)
        \tkzMarkRightAngle[size=0.25](WQ,E,S)
        
        \sun(S)
        \earth(E)
        
        \newcommand{\pointAndLabel}[3]{
            \point(#2)
            \tkzLabelPoint[#1, align=center, fill=white, inner sep=1pt](#2){#3}
        }
        
        \pointAndLabel{below=0.05}{EQ}{Восточная \\ квадратура}
        \pointAndLabel{below=0.05}{WQ}{Западная \\ квадратура}
        \pointAndLabel{below=0.05}{OP}{Противостояние}
        \pointAndLabel{above=0.05}{C}{Соединение}
        
        \pointAndLabel{below left=0.04}{EE}{Восточная \\ элонгация}
        \pointAndLabel{below right=0.04}{WE}{Западная \\ элонгация}
        \pointAndLabel{above=0.15}{LC}{Нижнее \\ соединение}
        \pointAndLabel{below=0.15}{UC}{Верхнее \\ соединение}
        
        \tkzDefPointBy[rotation=center S angle 20](OP) \tkzGetPoint{M}
        \tkzLabelPoint[fill=white, inner sep=0](M){$\mars$}
        
        \tkzDefPointBy[rotation=center S angle 120](LC) \tkzGetPoint{V}
        \tkzLabelPoint[fill=white, inner sep=0](V){$\venus$}
        
        \tkzDefPointBy[rotation=center S angle -121](Mer) \tkzGetPoint{Mer}
        \tkzLabelPoint[fill=white, inner sep=0](Mer){$\mercury$}
	           
	\end{tikzpicture}
	\captionof{figure}{Конфигурации Венеры и Марса для земного наблюдателя}
\end{wrapfigure}
\term[нижнее сооединие]{Нижнее соединение} внутренней планеты происходит когда Земля, Солнце и планета, также как и в случае верхнего соединения, располагаются на одной прямой, но для нижнего соединения планета должна находиться между Солнцем и Землей. Если бы орбиты всех планет лежали в одной плоскости, тогда в момент каждого нижнего соединения внутренней планеты наблюдалось бы её прохождение по диску Солнца для наблюдателя на внешней планете.




\term{Элонгацией} планеты называется угол Солнце -- Земля -- планета, отсюда очевидно, что \imp{максимальная элонгация} внутренней планеты наблюдается в момент, когда прямая Земля -- планета является касательной к орбите планеты, то есть угол Солнце -- планета -- Земля является прямым.

\term[внешняя планета]{Внешними планетами} называются планеты, большая полуось орбиты $a$ которых больше большой полуоси орбиты Земли $a_\oplus$. Для таких планет также существуют три основные конфигурации: \imp{соединение}, \imp{противостояние} и \imp{квадратура}. Квадратура бывает \term{западная} и \term{восточная}, в какой именно квадратуре находится внешняя планета определяется аналогично максимальной элонгации.

\term{Соединение} внешней планеты, подобно верхнему соединению внутренней планеты, наблюдается в момент, когда Солнце, Земля и планета находятся на одной прямой, при этом Солнце находится между планетой и Землей. В этот момент для наблюдателя на внешней планете Земля, являясь внутренней планетой, наблюдается в верхнем соединении.

Аналогично, когда планета, Солнце и Земля располагаются на одной прямой, но Солнце и планета лежат по разные стороны от Земли, считается, что внешняя планета находится в \term[противостояние]{противостоянии}. Земля же находится в нижнем соединении для наблюдателя на внешней планете, наблюдаемой в противостоянии.

\term[квадратура]{Квадратурой} называется конфигурация, когда угол между направлениями на планету и Солнце (угол {\slshape Солнце -- Земля -- планета}) является прямым. Стоит заметить, что для наблюдателя на планете Земля будет наблюдаться в максимальной элонгации, причем если планета с Земли наблюдалась в восточной квадратуре, тогда Земля будет в западной максимальной элонгации и наоборот.

