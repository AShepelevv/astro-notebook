\subsection{Затмения}
\term{Тенью} называется область, откуда не виден рассматриваемый источник света. \term{Полутенью}~--- область, откуда источник света виден частично, из чего следует, что понятие \imp{полутени} не применимо к точечным источникам.

Рассмотрим тень, создаваемую сферическим объектом~--- ширмой, откуда не виден сферический источник света большего, чем ширма, размера, \lookPicRef{pic:shadow-length}. Границу тени формируют общие касательные в точках, лежащих по одну сторону от прямой, проходящей через центры источника и ширмы. Легко понять, что в этом случае тень имеет конечные размеры. Определим, чему равна \term{длина тени} $x$~--- расстояние от её вершины до центра ширмы.

\begin{figure}[h!]
    \centering
    \tikzsetnextfilename{shadow-length}
    \input{sections/geometrical-astronomy/eclipses.shadow-length.tikz}
    \caption{Тень от сферического объекта при сферическом источнике}
    \label{pic:shadow-length}
\end{figure}

Пусть $R$ и $r$~--- радиусы источника и ширмы соответственно, а~$d$~--- расстояние между их центрами. Тогда из подобия треугольников
\begin{equation*}
    \frac{R}{d + x} = \frac{r}{x},
\end{equation*}
откуда
\begin{equation}
    x = \frac{r d}{R - r}.
    \label{eq:eclipses-shadow-length}
\end{equation}
Так, например, длина тени Земли~--- области откуда не видно Солнца, составляет около $1.4 \times 10^{6}$~км.

\subsubsection*{Солнечные затмения}

Найдем теперь выражение для \term{размера тени} на поверхности~--- области поверхности, пересекающей тень, сферического объекта радиуса~$R$, находящегося на расстоянии  $D$ от сферического источника света радиуса $R_0$ и расстоянии $d$ от сферической ширмы радиуса $r$ так, что центры всех трёх объектов лежат на одной прямой, \lookPicRef{pic:shadow-size-on-surface}.

\begin{figure}[h!]
    \centering
    \tikzsetnextfilename{shadow-size-on-surface}
    \input{sections/geometrical-astronomy/eclipses.shadow-size-on-surface.tikz}
    \caption{Схема формирования тени на поверхности объекта}
    \label{pic:shadow-size-on-surface}    
\end{figure}

Как было получено ранее, длина тени
\begin{equation*}
    x = \frac{r (D - d)}{R_0 - r}.
\end{equation*}
Следовательно, высота конуса тени, находящегося за пересечением тени с поверхностью объекта, $h = x - d + R$. Заметим, что при $x < d - R$ на поверхности объекта тени вообще не будет. Пусть $x > d - D$, тогда из подобия треугольников радиус тени на поверхности объекта
\begin{multline}
    s 
        \overset{r \ll x}{\simeq} \frac{r h}{x} 
        = \frac{r(x - d + R)}{x} = \\
        = \frac{r (R_0 - r)}{r (D - d)} \cdot \frac{r(D - d) - (d - R)(R_0 - r)}{R_0 - r} = \\
        = \frac{r (D - R) - R_0 (d - R)}{D - d}.
\end{multline}

Среднее значение этой величины для системы Солнце~-- Земля~-- Луна составляет около 200~км, максимальное~--- около 215~км. При нецентральном затмении максимальный диаметр тени Луны на поверхности Земли может достигать 270~км. Это даёт оценку на продолжительность затмения, равную 7.5 минутам. Большинство полных затмений длятся 2\,--\,4~минуты.

Далее, рассмотрим сферический объект~$E$ радиуса~$R$, сферический источник света~---~$S$ радиуса~$R_0$, находящийся на расстоянии~$D$ от центра~$E$, и сферическую ширму~--- $M$ радиуса $r$ на расстоянии~$d$ от центра~$E$. Определим максимальное угловое расстояние~$\gamma$ между источником~$S$ и ширмой~$M$, при наблюдении из центра объекта~$E$, чтобы на его поверхности могло наблюдаться \term{полное затмение} источника~$S$ ширмой~$M$, \lookPicRef{pic:eclipse-vertical-distance}. То есть хотя бы одна точка поверхности~$E$ находится в тени~$M$.

\begin{figure}[h!]
    \centering
    \tikzsetnextfilename{full-eclipse-vertical-distance}
    \input{sections/geometrical-astronomy/eclipses.full-eclipse-vertical-distance.tikz}
    \caption{Схема расположение объектов в предельном для наблюдения полного затмения случае}
    \label{pic:eclipse-vertical-distance}
\end{figure}

Для этого найдем угол $\alpha$. Это угол между общей касательной и прямой, проходящей через центры $M$ и $E$. Из равенства вертикальных углов следует, 
\begin{equation}
    \alpha = \arcsin \frac{R + r}{d},
    \label{eq:eclipses-vertial-distance-alpha}
\end{equation}
из тех же соображений,
\begin{equation*}
    \beta = \arcsin \frac{R_0 + R}{D}.
\end{equation*}

По рисунку легко понять, что $\alpha$~--- внешний угол треугольника, в котором несмежными ему углами являются $\beta$ и $\gamma$, откуда
\begin{equation}
    \gamma = \alpha - \beta = \arcsin \frac{R + r}{d} - \arcsin \frac{R_0 + R}{D}.
\end{equation}
Для системы Солнце\,--\,Земля\,--\,Луна в приближении круговых орбит этот угол составляет около $1.21^\circ$.

\begin{figure}[h!]
    \centering
    \tikzsetnextfilename{partial-eclipse-vertical-distance}
    \input{sections/geometrical-astronomy/eclipses.partial-eclipse-vertical-distance.tikz}
    \caption{Схема расположение объектов в предельном для наблюдения частного затмения случае}
    \label{pic:partial-eclipse-vertical-distance}
\end{figure}

Найдём теперь максимальное угловое удаление~$\gamma$, при котором на поверхности объекта~$E$ может наблюдаться \term{частное затмение} источника~$S$ ширмой~$M$. Для этого обратимся к \picRef{pic:partial-eclipse-vertical-distance}, из него видно, что нас интересует угол $\gamma$~--- внешний угол треугольника, в котором несмежными ему углами являются $\alpha$ и $\beta$, откуда $\gamma = \alpha + \beta$. Где угол $\alpha$, как и в предыдущем случае, определяется выражением \eqref{eq:eclipses-vertial-distance-alpha}. А $\beta$~--- угловой радиус конуса тени, создаваемой объектом $E$, иначе, угол между общей касательной к $E$ и $S$ и прямой, проходящей через их центры.

Найдём выражение для $\beta$. Пусть $x$~--- длина тени от объекта $E$, тогда с использованием выражения для длины тени \eqref{eq:eclipses-shadow-length} можно записать, что 
\begin{equation*}
    \beta = \arcsin \frac{R_0}{D + x} = \arcsin \frac{R_0}{D + \frac{RD}{R_0 - R}} = \arcsin\frac{R_0 - R}{D}.
    \label{eq:shadow-radius}
\end{equation*}
Воспользуемся полученными результатами для записи окончательного выражения для $\gamma$:
\begin{equation*}
    \gamma = \arcsin \frac{R + r}{d} + \arcsin\frac{R_0 - R}{D}.
\end{equation*}
Для системы Солнце\,--\,Земля\,--\,Луна в приближении круговых орбит этот угол составляет около $1.47^\circ$. 

Полученное выражение также может быть применено для определения возможности наблюдения прохождения внутренних планет по диску Солнца или покрытия далёких звёзд Луной. Во втором случае значимым остается только первое слагаемое, так как второе стремится к нулю в силу того, что расстояние до звёзд много больше размеров звёзд и объектов солнечной системы.

\begin{wrapfigure}[9]{r}{0.37\tw}
    \centering
    \vspace{-1pc}
    \tikzsetnextfilename{eclipses-node-distance}
    \input{sections/geometrical-astronomy/eclipses.eclipses-node-distance.tikz}
    \caption{Область, в которой Луна должна находиться для того, чтобы наблюдать затмение}
    \label{pic:elipses-node-distance}    
\end{wrapfigure}
Найденное выше максимальное угловое расстояние центра ширмы от центра источника, при котором наблюдается частичное затмение, определяет ширину полосы вокруг эклиптики, \lookPicRef{pic:elipses-node-distance}, внутри которой должен оказаться центр Луны, чтобы была потенциальная возможность наблюдать солнечное затмение на Земле. Оказывается, что Луна должна располагаться не дальше, чем $l = \gamma / \sin i \simeq 16^\circ$ от узла своей орбиты\footnote{в случае круговой орбиты Луны}, где $i$~--- угол наклона орбиты Луны к эклиптике. 

\begin{figure}[h!]
    \begin{subcaptionblock}[t]{0.63\tw}
        \includegraphics[width=\tw]{eclipses-map-20.png}
        \caption{Карта полных фаз затмений в 20-ом веке}
        \label{pic:eclipse-20-century}
    \end{subcaptionblock}
    \hfill
    \begin{subcaptionblock}[t]{0.32\tw}
        \includegraphics[width=\tw]{eclipse-map-1999-08-11.png}
        % https://eclipse.gsfc.nasa.gov/SEplot/SEplot1951/SE1999Aug11T.GIF}
        \caption{Карта затмения 11 августа 1999 года}
        \label{pic:eclipse-1999-08-11}
    \end{subcaptionblock}
    \caption{}
\end{figure}

% 
% Покрывается схемами для определения углов
%
%\begin{figure}[h!]
%    \centering
%    \vspace{-.5pc}
%    \tikzsetnextfilename{}
%    \input{sections/geometrical-astronomy/eclipses.eclipses-full-solar-eslipse.tikz}
%    \caption{Полное солнечное затмение со стороны северного полюса эклиптики}
%    \label{fig:eclipses-full-solar-eslipse}
%\end{figure}
При \term[кольцеобразное солнечное затмение]{кольцеобразном солнечном затмении} Луна так расположена относительно Земли, что конус её тени не достаёт до поверхности планеты, и вокруг Луны можно наблюдать яркое кольцо незакрытой части солнечного диска.

% 
% Не особо информативная картинка
%
%\begin{figure}[h!]
%    \centering
%    \tikzsetnextfilename{eclipses-circle-solar-eslipse}
%    \input{sections/geometrical-astronomy/eclipses.eclipses-circle-solar-eslipse.tikz}
%    \caption{Кольцеобразное солнечное затмение со стороны северного полюса эклиптики}
%    \label{fig:eclipses-circle-solar-eslipse}
%\end{figure}
При особом расположении Луны и Земли возможны \term{гибридные} затмения, когда в разных пунктах Земли наблюдаются либо \imp{кольцеобразное}, либо \imp{полное} затмение. Причиной такого явления является шарообразность Земли.

\subsubsection*{Лунные затмения}
В отличие от солнечных затмений здесь ширмой будет объект $E$ радиуса $R$, и для простоты повествования будем считать, что $R$ больше радиуса $r$ объекта $M$. 

Найдем максимальное угловое удаление $\gamma$ объекта $M$ от прямой, проходящей через центры источника $S$ и ширмы $E$, при котором объект $M$ все ещё полностью находится в тени $E$, \lookPicRef{pic:eclipses-full-lunar-eslipse}.

\begin{figure}[h!]
    \centering
    \tikzsetnextfilename{eclipses-full-lunar-eclipse}
    \input{sections/geometrical-astronomy/eclipses.eclipses-full-lunar-eclipse.tikz}
    \caption{Схема расположение объектов в предельном для наблюдения полного лунного затмения случае}
    \label{pic:eclipses-full-lunar-eslipse}
\end{figure}

Пусть $\alpha$~--- угловой радиус тени от ширмы $E$, тогда он определяется выражением~\eqref{eq:shadow-radius}. Для угла $\beta$ справедливо,
\begin{equation}
    \beta = \arccos \frac{R - r}{d}.
    \label{eq:eclipses-lunar-beta}
\end{equation}
Отсюда,
\begin{equation*}
    \gamma = 90^\circ - \alpha - \beta = \arccos \frac{R_0 - R}{D} - \arccos \frac{R - r}{d}.
\end{equation*}

Для системы Солнце~-- Земля~-- Луна этот угол составляет около~$0.4^\circ$, что дает оценку на максимальную продолжительность \term{полного лунного затмения} около~1.5~часов, во время которого Луна полностью находится в тени Земли.

Определим теперь максимальное угловое расстояние от той же прямой до объекта $M$, при котором он заходится в полутени ширмы~$E$, \lookPicRef{pic:eclipses-semi-shadow-lunar-eclipse}. Пусть теперь $\alpha$~-- угол между другой парой общих касательных, тогда
\begin{equation*}
    \alpha = \arcsin \frac{R_0 + R}{D}.
\end{equation*}
Угол $\beta$ также определяется выражением~\eqref{eq:eclipses-lunar-beta}. Следовательно,
\begin{equation*}
    \gamma = 90^\circ + \alpha - \beta = \arcsin \frac{R_0 + R}{D} + \arcsin \frac{R - r}{d}.
\end{equation*}
Что для системы Солнце~-- Земля~-- Луна равно около $0.95^\circ$, что говорит о максимальной продолжительности \term{полутеневого лунного затмения} (от 2 до 4 контакта) порядка $3.8$~часов.

\begin{figure}[h!]
    \centering
    \tikzsetnextfilename{eclipses-semi-shadow-lunar-eclipse}
    \input{sections/geometrical-astronomy/eclipses.eclipses-semi-shadow-lunar-eclipse.tikz}
    \caption{Схема расположение объектов в предельном для наблюдения полного полутеневого лунного затмения случае}
    \label{pic:eclipses-semi-shadow-lunar-eclipse}    
\end{figure}

% 
% Покрывается схемами для определения углов
%
%\vspace{-1pc}
%\begin{figure}[h!]
%    \centering
%    \tikzsetnextfilename{moon-eclipse-scheme}
%    \input{sections/geometrical-astronomy/eclipses.moon-eclipse-scheme.tikz}
%    \caption{Схема лунного затмения со стороны северного полюса эклиптики}
%    \label{fig:moon-eclipse-scheme}
%\end{figure}
\term{Лунное затмение}, в отличие от солнечного, видно со всего ночного полушария Земли. Диаметр земной тени на расстоянии Луны превышает размер последней примерно в 2.5\,--\,3 раза. Бывают \term[частное затмение]{частные}, когда лишь часть Луны попадает в земную тень, \term[полное затмение]{полные}~--- Луна полностью погружается в тень Земли, и \term[полутеневое затмение]{полутеневые}~--- Луна проходит через полутень Земли, не затрагивая конус тени.

\term{Синодический месяц}~--- промежуток времени между одинаковыми фазами Луны, равен 29.53 суток.

\term{Драконический месяц}~--- промежуток времени между двумя последовательными прохождениями Луны через один и тот же узел орбиты,~--- 27.21 суток.

\term{Сарос}~--- промежуток  времени, по прошествии которого солнечные и лунные затмения повторяются в прежнем порядке. Происходит это из-за того, что каждый сарос Луна, орбита Луны и Солнце возвращаются в прежнее положение относительно далёких звёзд. Сарос длится ровно 242 драконических месяца, или 223 синодических месяца, или 18 лет 11 дней 8 часов.

В процессе затмения выделяют до пяти важных моментов. Моменты начала и конца затмения, иначе, первый и четвертый контакты. Когда затмевающий объект касается затмеваемого внешним образом. например, в случае солнечного затмения. Или когда затмеваемый объект касается тени, в которую входит, например, во время лунное затмения начинается и заканчивается касанием Луны и тени Земли. Первый и четвертый контакты во время любого затмения, кроме вырожденных, когда затмения по суть не происходит, лишь касание, тогда они совпадают.

Следующие два момента~--- начало и конец полной фазы затмения, второй и третий контакты, соответственно. В эти моменты затмевающий и затмеваемый объекты касаются внутренным образом. Во время частных затмений второго и третьего контактов не происходит.

Последний~--- момент максимальной фазы затмения. Применим ко всем типам затмения. Для частных в этот момент фаза меньше 1, для полных~--- больше.


\begin{wrapfigure}[15]{r}{.27\tw}
    \centering
    \vspace{-1pc}
    \begin{subcaptionblock}[t]{.27\tw}
        \centering
        \tikzsetnextfilename{part-eclipses-scheme1}
        \input{sections/geometrical-astronomy/eclipses.part-eclipses-scheme1.tikz}
        \caption{Частное}
        \label{pic:partial-esclipse-phase}
    \end{subcaptionblock}
    \begin{subcaptionblock}[t]{.27\tw}
        \centering
        \tikzsetnextfilename{part-eclipses-scheme2}
        \input{sections/geometrical-astronomy/eclipses.part-eclipses-scheme2.tikz}
        \caption{Полное}
        \label{pic:full-esclipse-phase}
    \end{subcaptionblock}
    \caption{Схемы затмений}
    \label{fig:part-eclipses-scheme}
\end{wrapfigure}
Важной характеристикой любого затмения является его \term{фаза}~--- для \imp{частных} и \imp{кольцеобразных} затмений: отношение закрытой части $x$ диаметра\footnote{Здесь имеется в виду \imp{угловой} диаметр} затмеваемого тела, проходящего через центр затмевающего тела, ко всему диаметру затмеваемого тела $D$; для \imp{полного}: единица плюс отношение расстояния\footnote{Расстояние между окружностями $l_1$ и $l_2$~--- это $\min |L_1L_2|$ по всем $L_1 \in l_1$ и $L_2 \in l_2$.} между краями дисков затмеваемого и затмевающего тел к диаметру затмеваемого тела $D$.
\begin{align}
    \Phi_{\text{част}} &= \frac{x}{D}< 1,\\
    \Phi_{\text{полн}} &= 1 + \frac{\min\{d_1, d_2\}}{D}> 1.
\end{align}
Иногда вводят такое понятие, как \term{площадная фаза затмения}, т.\,е. отношение площади закрытой части диска затмеваемого тела к полной площади его диска. Чаще всего площадную фазу используют применительно к двойным звёздам, когда считают падение блеска при затмении одной звезды другой.
