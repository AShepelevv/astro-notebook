\subsection{Эллипс}
\textbf{Эллипс} --- плоская замкнутая кривая, сумма расстояний от любой точки котрой до двух фиксированных точек, называемых фокусами, постоянна и равна удвоенной большой полуоси эллипса.
\begin{equation}F_1M+F_2M=const=2a
\end{equation}
Главные отрезки эллипса:
\begin{enumerate}
\item Большая полуось ($a$)
\item Малая полуось ($b$)
\item Фокусное расстояние ($c$)
\end{enumerate}
$a$, $b$ и $c$ связаны слейдующим образом: $b^2+c^2=a^2$, что несложно вывести из определения эллипса.
 Эксцентриситет ($e$) --- числовая характеристика, показывающая степень отклонения от окружности. В эллипсе $0<e<1$.
 \begin{center}
\includegraphics[width = 0.8\textwidth]{Ellips}
\begin{figure}[h!]
\caption{Эллипс}
\end{figure}
\end{center}
\textbf{Основные формулы для эллипса:}
Эксцетриситет
\begin{equation}
e=\frac{c}{a}=\sqrt{1-\frac{b^2}{a^2}
\end{equation}
Расстояние до апоцентра
\begin{equation}
r_{\text{а}}=a(1+e)
\end{equation}
Расстояние до перицентра
\begin{equation}
r_{\text{п}}=a(1-e)
\end{equation}
Фокальный параметр
\begin{equation}
p=\frac{b^2}{a}=a(1-e^2)=b\sqrt{1-e^2}
\end{equation}
Площадь эллипса
\begin{equation}
\pi ab
\end{equation}
Радиус кривизны дуги эллипса в зависимости от расстояния $x$ от фокуса:
\begin{enumerate}
R=\frac{(2ax-x^2)^{3/2}}{ab}
\end{enumerate}
 