\subsection{Звёздные величины}
Звёздная величина~--- безразмерная числовая характеристика яркости объекта. Принято, что увеличению светового потока в 100 раз соответствует уменьшение звёздной величины ровно на 5 единиц. Тогда уменьшение звёздной величины на одну единицу означает увеличение плотности светового потока в $\sqrt[5]{100}\approx 2.512$~раз, то есть звёздные величины являются логарифмической шкалой измерения плотности потока. Зависимость, связывающая отношение освещённостей $E_1$ и $E_2$ и разность звёздных величин $m_1$ и $m_2$ двух объектов, называется \term{формулой Погсона} и имеет вид
\begin{equation}
    \frac{E_1}{E_2} = 10^{0.4(m_2 - m_1)} \quad \text{или} \quad m_2 = m_1 + 2.5 \lg \frac{E_1}{E_2}.
    \label{eq:pogson-law}
\end{equation}

Широко используется понятие \term{абсолютной звёздной величины} $M$~--- это звёздная величина при наблюдении с установленного расстояния~$R_0$: для звёзд и объектов вне Солнечной системы~---~10~пк, для тел Солнечной системы~---~1~\au, причем считается, что тело находится в 1~\au~и от наблюдателя и от Солнца, а фаза равна единице, то есть можно считать, что наблюдатель находится в центре Солнца, а~тело~--- в~1~\au~от него.

Абсолютную звёздную величину объекта вне Солнечной системы можно получить по формуле Погсона \eqref{eq:pogson-law} из наблюдаемой звёздной величины~$m$ и расстояния~$r$ до него
\begin{equation}
    M = m + 2.5 \lg \frac{E}{E_\text{абс}} = m + 2.5 \lg \frac{L \cdot R_0^2}{r^2 \cdot L} = m + 5 \lg \frac{R_0}{r}.
    \label{eq:absolute-magnutude}
\end{equation}
Если принимать к рассмотрению межзвездное поглощение~$A$, то формулу  \eqref{eq:absolute-magnutude} необходимо уточнить:
\begin{equation}
    M = m + 5 \lg \frac{R_0}{r} - Ar.
\end{equation}

Важно определить понятие \term{болометрической звёздной величины}~$m_\text{bol}$~--- это звёздная величина, при расчёте которой учитывается полная мощность излучения источника во всем диапазоне длин электромагнитных волн.

Резолюция B2, принятая на Генеральной Асамблеи Международного астрономического союза в 2015 году\,\cite{bolometric-magnitude}, определяет нуль-пункт звёздных величин. Так \imp{абсолютную болометрическую звёздную величину}, равную $0^m$ имеет изотропный источник излучения с мощностью $L_0 = 3.0128 \cdot 10^{28}$~Вт. Данное значение выбрано таким образом, чтобы абсолютная болометрическая звёздная величина Солнца $M_{\text{Bol}\odot}$ составляла $4.74^m$. Такому источнику излучения соответствует наблюдаемая плотность потока
\begin{equation}
    E_0 = \frac{L_0}{4\pi R_0^2} = 2.518021002\ldots \cdot 10^{-8}~\frac{\text{Вт}}{\text{м}^2}.
\end{equation}
Исходя из этого, используя \imp{формулу Погсона}, можно определить болометрическую звёздную величину любого изотропного источника излучения, зная его светимость и расстояние до него.

\term{Фотометрическая звёздная величина}~--- звёздная величина источника в некотором фильтре $\xi$, для которого определена функция чувствительности приемника $S_\xi(\lambda): [0, +\infty] \rightarrow [0,1]$. Определяется выражением
\begin{equation}
    m_\xi = -2.5\lg \frac{\int E(\lambda) S_\xi(\lambda) \, d \lambda}{E_0} + C_\xi,
\end{equation}
где $E(\lambda)$~--- плотность потока излучения от источника на длине волны~$\lambda$, а $C_\xi$~--- нормировочная константа фильтра $\xi$. Несмотря на то, что интегралы определенные, звёздная величина в том или ином фильтре определяется с точностью до нормировки на определенный нуль-пункт.

\begin{wraptable}[15]{r}{0.43\tw}
    \vspace{-0.75pc}
    \footnotesize
    \centering
    \renewcommand{\arraystretch}{1.2}
    \begin{tabular}{|c|c|c|}
        \hline
        Фильтр & $\langle\lambda\rangle$, нм & FWHM, нм \\
        \hline
        $U$    & 365                         & 66       \\
        $B$    & 445                         & 94       \\
        $V$    & 551                         & 88       \\
        $R$    & 658                         & 138      \\
        $I$    & 806                         & 149      \\
        $Y$    & 1020                        & 120      \\
        $J$    & 1220                        & 213      \\
        $H$    & 1630                        & 307      \\
        $K$    & 2190                        & 390      \\
        $L$    & 3450                        & 472      \\
        $M$    & 4750                        & 460      \\
        $N$    & 10500                       & 2500     \\
        \hline
    \end{tabular}
    \caption{Параметры фильтров фотометрической системы Джонсона}
    \label{tbl:johnson-bands}
\end{wraptable}
Для стандартизации фотометрических звёздных величин описаны различные фотометрические системы, особенно распространена система, опубликованная в 1966 году\,\cite{johnson-photometry}, где авторы определяют 8 широкополосных фильтров $U\!BV\!RIJKL$ (позднее были добавлены фильтры $M$ и $N$), параметры которых приведены в Таблице\,\ref{tbl:johnson-bands}, а профили пропускания представлены на \picRef{pic:johnson-bands}. Нуль-пунктом отсчета звездной величины в каждом из фильтров является блеск звезды спектрального класса $A0V$~--- Веги ($\alpha$\,Lyr), которой принимается равным $0.03^m$ в каждом из фильтров.

\begin{figure}[t]
    \centering
    \begin{subcaptionblock}{\tw}
        \centering
        \begin{tikzpicture}
            \begin{axis} [
                width   =    \tw,
                height  =    3.5cm,
                xmax    =    1500,
                xmin    =    300,
                xlabel  =    {Длина волны $\lambda$,~нм},
                ylabel  =    {$S_\xi(\lambda)$}
            ]
                \addplot[black, smooth] table[x=l, y=U] {data/filters-data-U.txt} node at (axis cs:350, 0.5) {\scriptsize{$U$}};
                \addplot[black, smooth] table[x=l, y=B] {data/filters-data-B.txt} node at (axis cs:440, 0.5) {\scriptsize{$B$}};
                \addplot[black, smooth] table[x=l, y=V] {data/filters-data-V.txt} node at (axis cs:550, 0.5) {\scriptsize{$V$}};
                \addplot[black, smooth] table[x=l, y=R] {data/filters-data-R.txt} node at (axis cs:700, 0.5) {\scriptsize{$R$}};
                \addplot[black, smooth] table[x=l, y=I] {data/filters-data-I.txt} node at (axis cs:870, 0.5) {\scriptsize{$I$}};
                \addplot[black, smooth] table[x=l, y=J] {data/filters-data-J.txt} node at (axis cs:1250, 0.5) {\scriptsize{$J$}};
            \end{axis}
        \end{tikzpicture}
    \end{subcaptionblock}
    
    \vspace{0.5pc}
    \begin{subcaptionblock}{\tw}
        \centering
        \begin{tikzpicture}

            \begin{axis} [
                width   =    \tw,
                height  =    3.5cm,
                xmax    =    15,
                xmin    =    1,
                xlabel  =    {Длина волны $\lambda$,~мкм},
                ylabel  =    {$S_\xi(\lambda)$}
            ]
                \addplot[black, smooth] table[x=l, y=K] {data/filters-data-K.txt} node at (axis cs:2.2, 0.5) {\scriptsize{$K$}};
                \addplot[black, smooth] table[x=l, y=L] {data/filters-data-L.txt} node at (axis cs:3.5, 0.5) {\scriptsize{$L$}};
                \addplot[black, smooth] table[x=l, y=M] {data/filters-data-M.txt} node at (axis cs:5, 0.5) {\scriptsize{$M$}};
                \addplot[black, smooth] table[x=l, y=N] {data/filters-data-N.txt} node at (axis cs:10, 0.5) {\scriptsize{$N$}};
            \end{axis}
        \end{tikzpicture}
    \end{subcaptionblock}
    \caption{Профили пропускания фотометрических фильтров системы $U\!BV\!RIJKLMN$}
    \label{pic:johnson-bands}
\end{figure}

Важно отметить, когда речь идёт звёздной величине без какой-либо кон\-кретизации, обычно имеется в виду видимая звёздная величина, другими словами~--- звёздная величина в фильтре~$V$, и обозначается как~$m_V$ или просто~$m$.

Разность между болометрической и фотометрической звёздными величинами называется \term{болометрической поправкой} (B.C.), которая отличается для разных спектральных классов звёзд и разных фильтров. Болометрическая поправка для фильтра $\xi$ может быть найдена из определения по формуле
\begin{equation}
    \text{B.C.}_\xi = M_\text{Bol} - M_\xi = m_\text{Bol} - m_\xi = 2.5\lg \frac{\int E(\lambda) S_\xi(\lambda) \, d \lambda}{\int E(\lambda)\, d \lambda} + C_\xi - C_\text{Bol},
\end{equation}
где $C_\text{Bol}=0$, если $C_\xi$ была определена с использованием стандарта~$E_0$.

