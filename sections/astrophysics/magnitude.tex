\subsection{Звёздные величины. Световой поток. Альбедо}

Звёздную величину различают на \textit{видимую}($m$) и \textit{абсолютную}($M$). \textit{Абсолютная звёздная величина} --- видимая звёздная величина с установленного расстояния от Солнца. Для звёзд --- 10 пк, для астероидов и комет  --- 1 а.е. Также звёздная величина может быть \textit{болометрической}($m_{Bol}$). Это звёздная величина, при расчёте которой учитывается полное излучение во всех диапазонах электромагнитных волн. Найти болометрическую величину можно, зная болометрическую поправку:
\begin{equation}
m+BC=m_{bol}
\end{equation}

Где $BC$ --- болометрическая поправка.

Абсолютную звёздную величину звезды можно вычислить по следующей формуле:
\begin{equation}
M=m+5-5\lg(r)=m+5+5\lg(\pi)
\end{equation}

Где $M$ --- абсолютная звёздная величина, $m$ --- видимая звёздная величина, $r$ --- расстояние до звезды в парсеках, $\pi$ --- параллакс звезды.

Освещённость --- количество света, падающего на квадратный метр площади. Освещённость связана с расстоянием до объекта следующим соотношением:
\begin{equation}
E\sim \frac{1}{r^2}
\end{equation}

Где $E$ --- освещённость от объекта

Светимость --- количество света, исходящего от объекта. Светимость вычисляется по следующей формуле:
\begin{equation}
E=\frac{L}{4\pi r^2}
\end{equation}

Где $L$ --- светимость объекта.

Альбедо --- характеристика отражательной способности поверхности какого-либо объекта. Альбедо является отношением отражённого светового потока к падающему на поверхность объекта.

Звёздную величину и освещённость объекта связывает \textit{формула Погсона}:
\begin{equation}
\frac{E_1}{E_2}=10^{0.4(m_2-m_1)}
\end{equation}
Эту формулу можно записать по-другому:
\begin{equation}
m_1-m_2=-2.5\lg\left(\frac{E_1}{E_2}\right)
\end{equation}

Где $E_1$ и $E_2$ --- освещённость от объекта, $m_1$ и $m_2$ --- звёздная величина объекта.