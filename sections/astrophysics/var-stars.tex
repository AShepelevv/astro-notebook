\subsection{Переменные звёзды}
\textit{Переменные звёзды} --- звёзды, у которых наблюдаются колебания блеска.   Для отнесения звезды к разряду переменных достаточно, чтобы блеск звезды хотя бы однажды претерпел изменение.

Переменные звёзды делятся на две большие группы: \textit{затменные} и \textit{физические}, причём физические подразделяются на \textit{пульсирующие} и \textit{эруптивные}.

К \textit{пульсирующим} переменным  относят те звёзды, переменность которых вызвана процессами, происходящими в их недрах. Эти процессы приводят к периодическому изменению блеска звезды, а вместе с ним и других характеристик звезды --- температуры поверхности, радиуса фотосферы и пр. Период переменности заключён в пределе от нескольких долей суток до нескольких лет. Абсолютную звёздную величину  и период связывает следующая формула:
\begin{equation}
M=-1.01^m-2.97\lg(T),
\end{equation}
где $T$ --- период, выраженный в сутках, а $M$  --- абсолютная звёздная величина.

Классический пример пульсирующих переменных звёзд --- цефеиды.

\begin{figure}[h!]
\begin{center}
\includegraphics[width=0.7\textwidth]{var-stars}
\caption{Кривая блеска $\delta$ Цефея}
\end{center}
\end{figure}

К \textit{эруптивным} переменным звёздам относятся звёзды, меняющие свой блеск нерегулярно или единожды за время наблюдений. Все изменения блеска эруптивных звёзд связывают с бурными процессами и вспышками в их хромосферах и коронах.

\textit{Затменно-переменные} звёзды --- системы из двух звёзд, суммарный блеск которых периодически изменяется с течением времени. Причиной изменения блеска могут быть затмения звёзд друг другом, или изменение их формы взаимной гравитацией в тесных системах.

На всех кривых блеска (Рис.\ref{var-stars}) заметны два минимума: глубокий (главный), соответствующий затмению главной звезды спутником, и слабый (вторичный), возникающий, когда главная звезда затмевает спутник.

\begin{figure}[h!]
\begin{center}
\includegraphics[width=0.8\textwidth]{var-stars2}
\caption{Кривые блеска некоторых затменно-переменных звёзд}\label{var-stars}
\end{center}
\end{figure}



