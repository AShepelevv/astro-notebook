\subsection{Вырожденные звёзды}
\textit{Вырожденные звезды} --- звезды в которых гравитации противостоит давление вырожденного газа. К ним относятся \textit{белые карлики} и \textit{нейтронные звезды}. 

\textit{Белый карлик} --- проэволюционировавшие звёзды с массой, не превышающей предел Чандрасекара (максимальная масса, при которой звезда может существовать как белый карлик), лишённые собственных источников термоядерной энергии. Масса белого карлика меняется в диапазоне от $0.6M_{\odot}$ до $1.44 M_{\odot}$, а радиус  примерно в 100 раз меньше солнечного, т.е. сравним с радиусом Земли. Плотность белых карликов состовляет $10^8$ --- $10^{12}$ $\text{кг}/\text{м}^3$.

\textit{Нейтронная звезда} --- сверхплотная звезда, образующаяся в результате взрыва Сверхновой. Вещество нейтронной звезды состоит в основном из нейтронов.

Масса нейтронной звезды лежит в пределах от $1.44M_{\odot}$ до $2.5M_{\odot}$ (предел Оппенгеймера-Волкова). Размер данной звезды состовляет лишь $10$ --- $20$~км, а плотность $10^{16}$ --- $10^{18}$ $\text{кг}/\text{м}^3$.  Дальнейшему гравитационному сжатию нейтронной звезды препятствует давление ядерной материи, возникающее за счёт взаимодействия нейтронов.

Так как нейтронные звёзды образуются в результате  коллапса массивных звёзд, то из-за сохранения момента импульса скорость их вращения очень велика --- максимальная скорость может достигать $10^5$~км/с.