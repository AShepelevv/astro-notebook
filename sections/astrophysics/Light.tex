\subsection{Световой поток. Альбедо}
Освещённость (плотность потока) --- мощность излучения, приходящаяся на единичную площадь. Освещённость обратно пропорционально квадрату расстояния до объекта:
\begin{equation}
E\sim \frac{1}{r^2},
\end{equation}

Где $E$ --- освещённость (плотность потока) от объекта, $r$ --- расстояние до объекта.

Светимость --- мощность излучения, испускаемая с единичной площади поверхности объекта. Светимость вычисляется по следующей формуле:
\begin{equation}
E=\frac{L}{4\pi r^2},
\end{equation}

Где $L$ --- полная светимость объекта.

Альбедо($A$) --- характеристика отражательной способности поверхности какого-либо объекта. Альбедо является отношением отражённого светового потока к падающему на поверхность объекта. Тогда для нахождения поглощённой части излучения используется следующее соотношение:
\begin{equation}
E_{\text{п}}=E_0\cdot (1-A),
\end{equation}

Где $E_{\text{п}}$ --- поглощённая часть излучения, $E_0$ --- приходящее излучение, $A$ --- альбедо.