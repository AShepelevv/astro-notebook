\subsection{Световой поток. Альбедо}
Освещённость (плотность потока) --- мощность излучения, приходящаяся на единичную площадь. Освещённость обратно пропорционально квадрату расстояния до объекта:
\begin{equation}
E\sim \frac{1}{r^2},
\end{equation}
где $E$ --- освещённость (плотность потока) от объекта, $r$ --- расстояние до объекта.

Светимость --- мощность излучения, испускаемая с единичной площади поверхности объекта. Светимость вычисляется по следующей формуле:
\begin{equation}
E=\frac{L}{4\pi r^2},
\end{equation}
где $L$ --- полная светимость объекта.

Прежде всего световой поток является частным случаем \textit{теоремы Гаусса}. Общая формулировка:  поток излучения равен мощности, переносимой оптическим излучением через какую-либо поверхность. 
\begin{equation}
\Phi_e=\oint_SJ\cdot dS=\frac{dQ_e}{dt},
\end{equation}
где $J$ --- мощность светового потока, $dQ_e$~--- энергия излучения, переносимая через поверхность за время $dt$.


Альбедо($A$) --- характеристика отражательной способности поверхности какого-либо объекта. Альбедо является отношением отражённого светового потока к падающему на поверхность объекта. Тогда для нахождения поглощённой части излучения используется следующее соотношение:
\begin{equation}
E_{\text{п}}=E_0\cdot (1-A),
\end{equation}
где $E_{\text{п}}$ --- поглощённая часть излучения, $E_0$ --- приходящее излучение, $A$ --- альбедо.

А для отражённой части излучения можно использовать следующую формулу:
\begin{equation}
E_{\text{отр}}=A\cdot E_0,
\end{equation}
где $E_{\text{отр}}$ --- отражённая часть излучения.

Существует несколько видов альбедо --- \textit{геометрическое}, \textit{сферическое} и \textit{бондовское}. \textit{Геометрическое альбедо} равно отношению освещённости у Земли, создаваемой планетой в полной фазе, к освещённости, которую создал бы плоский абсолютно белый экран того же размера, что и планета, расположенный на её месте перпендикулярно лучу зрения и солнечным лучам. \textit{Сферическое альбедо} определяется как отношение светового потока, рассеянного телом во всех направлениях, к потоку, падающему на это тело. Может быть определено и для некоторого диапазона длин волн, и для всего спектра. Сферическое альбедо для всего спектра излучения называется \textit{альбедо Бонда}.