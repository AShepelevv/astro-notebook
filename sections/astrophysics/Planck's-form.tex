\subsection{Формула Планка}

\textit{Формула Планка} --- выражение для спектральной плотности мощности излучения абсолютно чёрного тела, применяемое  для интервала частот излучения  $[\nu, \nu + d \nu)$, которое было получено Максом Планком для равновесной плотности излучения $B(\nu,T)$. Полученное выражение записывается следующим образом, где $\nu$ --- частота излучения, $T$ --- температура, $h$ --- постоянная Планка, $k$ --- постоянная Больцмана, $c$ --- скорость света:
\begin{equation}\label{Planck's formula}
B(\nu,T)=\frac{2h\nu^3}{c^2}\cdot \frac{h\nu}{\exp\left(\frac{h\nu}{kT}\right)-1}
\end{equation}

Если записать закон излучения Планка (\ref{Planck's formula}) для длин волн, то функция примет следующий вид:
\begin{equation}\label{Planck's formula2}
B(\lambda,T)=\frac{2hc^2}{\lambda^5} \cdot \frac{1}{\exp\left(\frac{hc}{\lambda kT}\right)-1}
\end{equation}

Стоит заметить, что при переходе функции к длинам волн меняется не только частота на длину волны, но и интервал. 

\begin{figure}[h!]
\begin{center}
 \begin{tikzpicture}
  \begin{axis}[
  		ymax=4.35e+14,
  		xmax=1.5,
  		axis x line=bottom,
		axis y line=left, 
		xlabel={Длина волны $\lambda$,~мкм}, 
		ylabel={Мощность излучения $B(\lambda)$,~Вт/$\text{м}^2\cdot$~нм},
		width=7.5cm, 
		height=7.5cm, 
		no markers, 
		minor x tick num = 1,
		minor y tick num = 1,
		grid = both,
		line width=.7pt,
		cycle list = {
			{smooth,green!50!black,solid},
			{smooth,red,solid},
			{smooth,blue,solid},
			{smooth,brown,solid},
			{smooth,black,solid},
			{smooth,red,solid},
			{smooth,brown,solid}
		}
]
   \addplot table[x=l, y=t10] {data2.txt}
   node at (axis cs:0.29, 4.2e+14)
{$5500$K};
   \addplot table[x=l, y=t9] {data2.txt}node at (axis cs:0.33, 2.55e+14)
{$5000$K};
   \addplot table[x=l, y=t8] {data2.txt}node at (axis cs:0.364, 1.475e+14)
{$4500$K};
   \addplot table[x=l, y=t7] {data2.txt}node at (axis cs:0.415, 8.3e+13)
{$4000$K};
   \addplot table[x=l, y=t6] {data2.txt}node at (axis cs:0.44, 4.5e+13)
{$3500$K};
  \end{axis}
 \end{tikzpicture}
\end{center}
\caption{Кривые спектральной плотности потока излучения абсолютно чёрных тел с разной температурой}\label{pic:wien-law}
\end{figure}

Формула Планка появилась тогда, как стало ясно, что формула Рэлея — Джинса удоволетворительно описывает излучение только в области длинных волн, а с убыванием длин волн даёт сильные расхождения с эмпирическими данными. Формула Рэлея-Джинса для длин волн записывается таким образом:
\begin{equation}\label{Ray-Jean}
B(\lambda,T)=\frac{2ckT}{\lambda^4}
\end{equation}

Если формулу Рэлея — Джинса записать для частоты излучения, то формула примет данный вид:
\begin{equation}\label{Ray-Jean2}
B(\nu,T)=\frac{2\nu^2 kT}{c^2}
\end{equation}

Также формулы (\ref{Ray-Jean}) и (\ref{Ray-Jean2}) можно записать для коротковолновой и высокочастотной области:
\begin{equation}
U(\lambda,T)\approx\frac{2hc^2}{\lambda^5}\exp\left(-\frac{hc}{\lambda kT}\right)
\end{equation}
\begin{equation}
U(\nu,T)\approx\frac{2h\nu^3}{c^2}\exp\left(-\frac{h\nu}{kT}\right)
\end{equation}