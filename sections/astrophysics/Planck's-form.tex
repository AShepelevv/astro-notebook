\subsection{Формула Планка}

\textit{Формула Планка} --- выражение для спектральной плотности мощности излучения абсолютно чёрного тела, применяемое  для интервала частот излучения  $[\nu, \nu + d \nu)$, которое было получено Максом Планком для равновесной плотности излучения $B(\nu,T)$. Полученное выражение записывается следующим образом, где $\nu$ --- частота излучения, $T$ --- температура, $h$ --- постоянная Планка, $k$ --- постоянная Больцмана, $c$ --- скорость света:
\begin{equation}\label{Planck's formula}
B(\nu,T)=\frac{2h\nu^3}{c^2}\cdot \frac{h\nu}{\exp\left(\frac{h\nu}{kT}\right)-1}
\end{equation}

Если записать закон излучения Планка (\ref{Planck's formula}) для длин волн, то функция примет следующий вид:
\begin{equation}\label{Planck's formula2}
B(\lambda,T)=\frac{2hc^2}{\lambda^5} \cdot \frac{1}{\exp\left(\frac{hc}{\lambda kT}\right)-1}
\end{equation}

Стоит заметить, что при переходе функции к длинам волн меняется не только частота на длину волны, но и интервал. 

Формула Планка появилась тогда, как стало ясно, что формула Рэлея — Джинса удоволетворительно описывает излучение только в области длинных волн, а с убыванием длин волн даёт сильные расхождения с эмпирическими данными. Формула Рэлея-Джинса для длин волн записывается таким образом:
\begin{equation}
B(\lambda,T)=\frac{2ckT}{\lambda^4}
\end{equation}

Если формулу Рэлея — Джинса записать для частоты излучения, то формула примет данный вид:
\begin{equation}
B(\nu,T)=\frac{2\nu^2 kT}{c^2}
\end{equation}