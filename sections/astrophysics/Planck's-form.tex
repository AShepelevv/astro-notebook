\subsection{Формула Планка}\label{subsection:Planck's-form}

\textit{Формула Планка} --- выражение для спектральной плотности мощности излучения абсолютно чёрного тела, которое было получено Максом Планком для равновесной плотности излучения $u(\omega,T)$. Полученное выражение записывается следующим образом:
\begin{equation}
u(\omega,T)=\frac{\omega^2}{\pi^2c^3}\cdot \frac{h\omega}{exp\left(\frac{h\omega}{kT}\right)-1}
\end{equation}

Где $\omega$ --- частота излучения, $T$ --- температура, $h$ --- постоянная Планка, $k$ --- постоянная Больцмана, $c$ --- скорость света.

Если записать закон излучения Планка для длин волн, то функция примет следующий вид:
\begin{equation}
B(\lambda,T)=\frac{2hc}{\lambda^5 \left(exp\left(\frac{hc}{\lambda kT}\right)-1\right)}
\end{equation}