\subsubsection{Дифракция света в атмосфере}
\label{sec:atmospheric-diffraction}

Крошечные капельки воды в облаках и тумане создают интересные оптические эффекты,  в основном это яркоокрашенные кольца или дуги. Геометрическая оптика имеет малое значения в этих эффектах, так как капли воды настолько малы, что дифракция и интерференция излучения преобладают над классическими законами прямолинейного распространения света.

\paragraph{Корона}
Результат дифракции прямого солнечного или отраженного Луной света на частицах в атмосфере Земли. Как известно из раздела~\ref{sec:diffraction}, размер дифракционного пятна от круглого экрана диаметра $d$ имеет радиус $1.22 \lambda / d$, где $\lambda$~--- длина волны наблюдаемого излучения. Таким образом, чем меньше размер экрана или диафрагмы, тем больше диаметр дифракционного пятна и шаг дифракционных колец. Поэтому корона наблюдается только при наличии достаточно мелких частиц~--- капель воды, кристаллов льда, пыли, в атмосфере.

Оценим максимальных размер $d_\text{макс}$ дифрагирующих частиц для наблюдения короны. Существенная часть энергии приходится на главный и вторичный дифракционные пики. Будем считать, что в случае наложения второго дифракционного пика на диск Солнца, корона не наблюдается, тогда
\begin{equation*}
    d_\text{макс} = \frac{2\lambda}{\rho_\odot} \simeq 250~\text{мкм}
\end{equation*}
при $\lambda = 550$~мкм и угловом радиусе Солнца $\rho_\odot \simeq 0.25^\circ$. При меньшем размере частиц корона (второй дифракционный максимум короны) находится вне солнечного диска и, следовательно, корона может наблюдаться. 

Также нужно отметить важность единообразия размеров частиц. При высокой дисперсии размеров частиц корона не наблюдается, так как дифракционные максимумы от частиц одного размера приходятся на дифракционные минимумы от частиц другого размера и наоборот. В итоге волновой фронт выравнивается и корона не наблюдается.~\cite{Les_Cowley_2005}

