\paragraph{Паргелический круг}

Также в силу горизонтальной ориентации пластинчатых кристаллов происходит следующее. Солнечный свет, попадая в пластинчатый кристалл через верхнюю торцевую грань, один или более раз отражаясь от боковых граней, выходит из кристалла через нижнюю торцевую грань под тем же углом к горизонту, однако, имея отличное по азимуту направление. Таким образом равномерно распределенные в воздухе пластинчатые кристаллы отражают солнечный свет во всех направлениях в горизонтальной плоскости, сохраняя угол следования лучей к горизонту.

В результате описанных выше отражений при подходящих метеорологических условиях и высоте Солнца наблюдатель может увидеть полную окружность увеличения яркости, параллельную горизонту, проходящую через Солнце.