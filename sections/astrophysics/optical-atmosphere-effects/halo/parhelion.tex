\paragraph{Паргелий}

Когда пластинчатые кристаллы дрейфуют вниз под действием силы тяжести, из-за сопротивления воздуха и своей формы они ориентируются определённым образом: их основания почти горизонтальны. Такая конфигурация стабильна, то есть небольшие отклонения создают корректирующие силы, возвращающие кристаллы в близкое горизонтальному положение. Таким образом у пластинчатых кристаллов только одна степень свободы~--- вращение в горизонтальное плоскости.

При небольшой высоте Солнца над горизонтом такие кристаллы увеличивают яркость $22^\circ$ гало на высоте, равной высоте Солнца над горизонтом. Однако, начиная с некоторой высоты, торцевые грани располагаются под углом к лучам солнечного света, в силу чего происходят внутренние отражения от торцевых граней, и не достигается минимальный угол преломления~--- $22^\circ$. В такой ситуации \imp{паргелий} наблюдается на большем угловом расстоянии от Солнца.