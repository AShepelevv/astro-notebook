\paragraph{$\mathbf{22^\circ}$ гало}

\begin{wrapfigure}[11]{r}{0.25\tw}
    \vspace{-1pc}
    \centering
    \tikzsetnextfilename{22-galo-sheme}
    \begin{tikzpicture}
    
        \tkzDefPoint(0,0){C}

        \def\R{2}
        \def\f{0}
        \def\n{1.31}

        \tkzInit[
           xmin={-0.2*\R},
           xmax={1.1*\R},
           ymin={-1.1*\R},
           ymax={1.3*\R},
        ]
        \tkzClip


        \tkzDefShiftPoint[C](\R,0){x}
        \tkzDefPointBy[rotation=center C angle \f](x)  \tkzGetPoint{R1}
        \tkzDefPointBy[rotation=center C angle 60](R1) \tkzGetPoint{R2}
        \tkzDefPointBy[rotation=center C angle 60](R2) \tkzGetPoint{R3}
        \tkzDefPointBy[rotation=center C angle 60](R3) \tkzGetPoint{R4}
        \tkzDefPointBy[rotation=center C angle 60](R4) \tkzGetPoint{R5}
        \tkzDefPointBy[rotation=center C angle 60](R5) \tkzGetPoint{R6}

        \tkzDrawPolygon[thick](R1,R2,R3,R4,R5,R6)

        \def\al{30}
        \def\x{0.4}

        \tkzDefPointBy[homothety=center R2 ratio \x](R3) \tkzGetPoint{I}
        \tkzDefLine[perpendicular=through I](R3,R2) \tkzGetPoint{I'}

        \def\bet{asin(sin(\al / 180 * pi) / \n)}
        \def\b{\R * (sqrt(3) / 2  + (0.5 - \x) * tan(pi / 2 - \bet))}
        \def\xo{(\b + sqrt(3) * \R) / (tan(pi / 2 - \bet) + sqrt(3))}
        \def\yo{sqrt(3) * \xo - sqrt(3) * \R}

        \tkzDefPoint(\xo,\yo){O}
        \tkzDefLine[perpendicular=through O](R1,R6) \tkzGetPoint{O'}

        \tkzInterLL(I,I')(O,O') \tkzGetPoint{T}

        \tkzDefPointBy[rotation=center I angle {-180 + \al}](T) \tkzGetPoint{x}
        \tkzDefPointBy[homothety=center I ratio 0.4](x) \tkzGetPoint{A}

        \tkzFindAngle(I,O,T) \tkzGetAngle{gam}
        \def\del{asin(\n * sin(\gam / 180 * pi)) * 180 / pi}

        \tkzDefPointBy[rotation=center O angle {180 - \del}](T) \tkzGetPoint{x}
        \tkzDefPointBy[homothety=center O ratio 0.5](x) \tkzGetPoint{D}

        \tkzDrawSegment[arrow={latex}{0.3}](A,I)
        \tkzDrawSegment[arrow={latex}{0.5}](I,O)
        \tkzDrawSegment[arrow={latex}{0.9}](O,D)

        \tkzDrawLines[dashed, add=0.5cm and 0.5cm](I,T O,T)

        \tkzMarkRightAngles[size=0.2](R3,I,T T,O,R6)

        \tkzMarkAngle[arc=lll, size=0.5](T,I,O)
        \tkzLabelAngle[pos=0.75](T,I,O){\footnotesize$\beta$}

        \tkzMarkAngle[arc=lll, mark=|, size=0.4, mksize=2](I',I,A)
        \tkzLabelAngle[pos=0.6](I',I,A){\footnotesize$\alpha$}

        \tkzMarkAngle[line width = .3pt, size=0.01](I,O,T)
        \tkzMarkAngle[arc=ll, size=0.3](I,O,T)
        \tkzLabelAngle[pos=0.5](I,O,T){\footnotesize$\gamma$}

        \tkzMarkAngle[arc=ll, mark=|, size=0.3, mksize=2](D,O,O')
        \tkzLabelAngle[pos=0.5](D,O,O'){\footnotesize$\delta$}

        \tkzMarkAngles[size=0.2](O,T,I R2,R1,R6 R3,R2,R1 R1,R6,R5)
        \tkzLabelAngle[pos=0.5](O,T,I){\scriptsize$120^\circ$}

        \tkzDrawPoints(I, T, O, R1, R2, R6)
    \end{tikzpicture}
    \caption{}
    \label{pic:22-galo-scheme}
\end{wrapfigure}
Яркая окружность вокруг Солнца, угловой радиус которой оставляет около $22^\circ$ в зависимости от длины волны излучения. Формируется в произвольно расположенных кристаллах в результате рефракции света на двух, следующих через одну, боковых гранях.

В зависимости от угла падения света на первую грань меняется угол итогового преломления, минимальная величина которого~--- примерно $22^\circ$. В силу экстремальности этого значения, наибольшее число кристаллов преломляют солнечный свет именно под этим углом.

Детально расмотрим геометрию формирования $22^\circ$~гало. Пусть~$\alpha$~--- угол падения луча на боковую грань $\mathcal{A}$ в проекции на нормальную к оси кристалла плоскость, а $n$~--- коэффициент преломления льда. Тогда, исходя из закона Снеллиуса, угол преломления $\beta$ определяется соотношением $\sin \alpha = n \sin \beta$. Выберем место падения луча и угол $\alpha$ такими, чтобы преломленный луч упал на несмежную и непротивоположную боковую грань $\mathcal{B}$. Обозначим угол падения на эту грань как $\gamma$. Учитывая геометрию кристаллов льда, несложно показать, что угол между нормалями к $\mathcal{A}$ и $\mathcal{B}$ составляет $120^\circ$, таким образом $\gamma = 60^\circ - \beta$. Снова применяя закон Снеллиуса, получаем, $\sin \delta = n \sin \gamma$. Окончательно, угол преломления исходного луча кристаллом льда $\rho =  \alpha + \delta - 60^\circ$.

Из полученных соотношений имеем зависимость $\rho(\alpha)$. Максимум плотности излучения наблюдается близ экстремума данной функции, то есть при $\rho'(\alpha) = 0$. Что соответствует $\alpha_0 = 40.9^\circ$ и $\rho(\alpha_0) = 21.8^\circ$ при $n=1.31$.
