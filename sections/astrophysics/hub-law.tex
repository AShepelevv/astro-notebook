\section{Закон Хаббла}
\textit{Закон Хаббла} --- это эмпирический закон, связывающий красное смещение галактик и расстояние до них линейным образом, и описывающий расширение вселенной. Этот закон записывается следующим образом:
\begin{equation}
V=HD,
\end{equation}
где $V$ --- скорость удаления галактики, $H$ --- постоянная Хаббла, среднее значение которой принимается как $H=68$~$\text{(км/с)}/\text{Мпк}$, $D$ --- расстояние до галактики. 

Из Доплеровского эффекта выполняется следующее соотношение:
\begin{equation}
V=cz,
\end{equation}\label{speed}
где $c$~--- скорость света, $z$~--- красное смещение.

\textit{Красное смещение}~--- сдвиг спектральных линий в длинноволновую (красную) сторону. Параметр смещения определяется так:
\begin{equation}
z=\frac{\lambda-\lambda_0}{\lambda}
\end{equation}

Красное смещение показывает скорость удаления и расстояние до галактик или других объектов.

Равенство (\ref{speed})  справедливо только при $z\ll1$, а при больших значениях $z$ из-за релятивизма используется следующая формула:
\begin{equation}
V=c\frac{(1+z)^2-1}{(1+z)^2+1}
\end{equation}