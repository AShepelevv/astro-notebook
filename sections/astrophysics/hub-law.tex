\section{Закон Хаббла}
\textit{Закон Хаббла} --- это эмпирический закон, связывающий красное смещение галактик и расстояние до них линейным образом, и описывающий расширение вселенной. Этот закон записывается следующим образом:
\begin{equation}
V=cz=HD,
\end{equation}
где $V$ --- скорость удаления галактики, $c$ --- скорость света, $z$ --- красное смещение, $H$ --- постоянная Хаббла, среднее значение которой принимается как $H=68$~$\text{(км/с)}/\text{Мпк}$, $D$ --- расстояние до галактики. 

Но данное равенство справедливо только при $z\ll1$, а при больших значениях $z$ используется следующее соотношение:
\begin{equation}
V=c\frac{(1+z)^2-1}{(1+z)^2+1}=HD
\end{equation}