\subsection{Закон смещения Вина}
\textit{Закон смещения Виина} --- закон, устанавливающий зависимость длины волны от температуры чёрного тела, на которой поток излучения энергии чёрного тела достигает своего максимума (Рис.\ref{pic:wien-law}).

\begin{figure}[h!]
\begin{center}
\includegraphics[scale=0.35]{Wien-law}
\end{center}
\caption{Кривые спектральной плотности потока излучения абсолютно чёрных тел с разной температурой}\label{pic:wien-law}
\end{figure}

Длину волны, на которой интенсивность излучения абсолютно чёрного тела достигает своего максимума, можно определить по следующей формуле:
\begin{equation}
\lambda_{max}\approx\frac{b}{T}
\end{equation}

Где $b$ --- постоянная Вина равная $b\approx0.0029 $ $\text{м} \cdot \text{К}$

Данная формула получается путём нахождения экстремума \textit{функции Планка} для абсолютно чёрного тела, записанного для длин волн (\ref{Planck's formula2}).

