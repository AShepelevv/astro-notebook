\subsection{Закон смещения Вина}
\textit{Закон смещения Вина} --- закон, устанавливающий зависимость длины волны от температуры чёрного тела, на которой поток излучения энергии чёрного тела достигает своего максимума (Рис.\ref{pic:wien-law}).

\begin{figure}[h!]
\begin{center}
 \begin{tikzpicture}
  \begin{axis}[width=7.5cm, height=7.5cm,legend pos = outer north east, no markers,minor x tick num = 1,minor y tick num = 1,grid = both,line width=1pt,cycle list = {
{smooth,green!50!black,solid},
{smooth,blue!50!green,solid},
{smooth,blue,solid},
{smooth,green!50!white,solid},
{smooth,black,solid},
{smooth,red,solid},
{smooth,brown,solid}
}
]
   \addplot table[x=l,y=t10] {data.txt};
   \addplot table[x=l,y=t9] {data.txt};
   \addplot table[x=l,y=t8] {data.txt};
   \addplot table[x=l,y=t7] {data.txt};
   \addplot table[x=l,y=t6] {data.txt};
   \addplot table[x=l,y=t5] {data.txt};
   \addplot table[x=l,y=t4] {data.txt};
   \legend{5500K, 5000K, 4500K, 4000K, 3500K, 3000K, 2500K}
  \end{axis}
 \end{tikzpicture}
\end{center}
\caption{Кривые спектральной плотности потока излучения абсолютно чёрных тел с разной температурой}\label{pic:wien-law}
\end{figure}

Длину волны, на которой интенсивность излучения абсолютно чёрного тела достигает своего максимума, можно определить по следующей формуле:
\begin{equation}
\lambda_{max}\approx\frac{b}{T}
\end{equation}

Где $b$ --- постоянная Вина равная $b\approx0.0029 $ $\text{м} \cdot \text{К}$

Данная формула получается путём нахождения экстремума \textit{функции Планка} для абсолютно чёрного тела, записанного для длин волн (\ref{Planck's formula2}).

