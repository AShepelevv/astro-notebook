\subsection{Солнце}
\subsubsection*{Общая информация}

Солнце --- центральное тело Солнечной системы. В нём сосредоточено 99,866\%  массы Солнечной системы. Солнце состоит из водорода (73\% от массы), гелия (25\%) и других элементов с меньшим содержанием: железа, никеля, азота, кислорода, кремния, серы, магния, углерода, неона, кальция, хрома и др.

По спектральной классификации Солнце --- звезда типа G2V (жёлтая звезда главной последовательности). Температура поверхности Солнца  примерно $5 800$~К, поэтому Солнце светит почти в белом свете, но прямой свет Солнца у поверхности Земли приобретает жёлтый оттенок из-за рассеяния и поглощения коротковолновой части спектра в атмосфере.

Солнце вырабатывает энергию путём термоядерного синтеза. Каждую секунду в ядре около 4 млн тонн веществе превращается в лучистую энергию.

\subsubsection*{Строение Солнца}

В центре Солнца находится ядро радиусом $150$ --- $180$ тыс. км, в котором идут термоядерные реакции. Плотность  вещества в ядре $1.5\times 10^5~\text{кг}/\text{м}^3$, температура в центре ядра около $1.5\times 10^7$~К.