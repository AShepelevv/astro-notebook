\subsection{Атмосфера Земли}
Применим знания о МКТ для поиска характерной высоты атмосферы Земли для двух её моделей: изотермической и адиабатической. Прежде чем мы приступим к поиску, несколько фактов об атмосфере Земли. Давление у поверхности Земли в среднем составляет 
\begin{equation*}
	p_0 = 1~\text{атм} = 101325~\text{Па} = 760~\text{мм рт. ст}.
\end{equation*}
Наблюдаемые отклонения от среднего значения незначительны для последующих выкладок, так как составляют не более $5\%$. 

Средняя температура воздуха у поверхности $T_0 \approx 12^\circ\text{C} = 285~\text{K}$. Молярная масса воздуха, средневзвешенное молярных масс его компонент, $\mu_0 = 29~\text{г}/\text{моль}$. 

\paragraph{Изотермическая атмосфера} Падение давления воздуха с увеличением высоты математически можно описать с помощью баланса трёх сил. Для этого рассмотрим тонкий слой воздуха с площадью $dS$ и толщиной $dh$. Пусть масса воздуха в этом слое равна $dm$. Можно представить, что воздух в этом слое окружен невесомой тонкой оболочкой, не перемешивается с окружающим воздухом и находится в равновесии. Которое достигается благодаря равенству нулю равдействующей трёх сил: силы давления воздуха снизу $F_\uparrow = p\d S$, силы давления воздуха сверху $F_\downarrow = -(p + dp)\d S$ и силы тяжести $F_g = -g\d m$, действующей на выделенный слой. Здесь $dp < 0$, так как с высотой давления, очевидно, уменьшается, потому что все меньший столб воздуха оказывает давление на выделенный слой. Итак,
\begin{gather*}
	F_\uparrow + F_\downarrow + F_g = 0,\\
	p \d S - (p + dp) \d S - g \d m = 0,\\
	- dp\d S = g \rho \d V = g \rho \d h \d S,\\
	dp = - g \rho \d h.
\end{gather*}

Из уравнения состояния идеального газа \eqref{eq:mkt-p}, считая воздух таковым, выразим плотность воздуха $\rho$:
\begin{equation*}
	\rho = \frac{p \mu}{RT}.
\end{equation*}
Использую данный результат в выкладках выше, получим
\begin{equation}
	dp = - \frac{p \mu g}{RT} \d h.
	\label{eq:earth-atm-dp}
\end{equation}

Разделим переменные и проинтегрируем левую и правую часть от нуля до высоты $h$:
\begin{gather*}
	\int\limits_{p_0}^{p(h)} \frac{dp}{p} = -\int\limits_0^h \frac{\mu g}{RT} dh,\\
	\ln p(h) - \ln p_0 = -\frac{\mu g h}{RT},\\
	p(h) = p_0 \exp \left[ -\frac{\mu g h}{RT} \right]. \tag{\theequation} 
	\label{eq:earth-atm-isot}
\end{gather*}
Полученное равенство называется \imp{барометрической формулой для изотермической атмосферы}. Из нее получаем, что характерная (давление уменьшилось в $e$ раз) высота изотермической атмосферы
\begin{equation*}
	H_{T=\const} = \frac{RT}{\mu g} \simeq 8.3~\text{км}.
\end{equation*}

\paragraph{Адиабатическая атмосфера}
Конечно, считать атмосферу изотермической достаточно грубо. Рассмотрим модель, в рамках которой будем считать, что выполняется равенство $dQ = 0$ для рассматриваемого слоя воздуха. 

Такая модель хорошо описывает наблюдаемый процесс. Наибольший нагрев воздуха происходит у поверхности Земли. И чем выше, тем воздух холоднее. Это хорошо заметно в горах, где воздух холоднее чем в нижележащих долинах. В ходе конвекции слои теплого воздуха поднимаются и расширяются, так как давление окружающего воздуха с высотой уменьшается. Работа газа по его расширению уменьшает его же внутреннюю энергии~--- газ остывает. В таком подходе как раз можно считать, что $dQ$ для рассматриваемого слоя воздуха равно нулю.

Приступим к выводу барометрической формулы для адиабатической атмосферы. Начнем с выражения для работы газа и изменения его потенциальной энергии:
\begin{gather*}
	\delta A = p \d V,\\
	dU = C_V \nu \d T.
\end{gather*}
Запишем первое начало термодинамики для рассмотренного выше слоя воздуха:
\begin{gather}
	dQ = 0 = dU + \delta A = p \d V + C_V \nu \d T,\notag\\
	dT = - \frac{p}{C_V \nu} dV.
	\label{eq:earth-atm-dt}
\end{gather}
Из уравнения состояния идеального газа \eqref{eq:mkt-idial-gas} выразим объём и перейдем к дифференциалам:
\begin{equation}
	dV = d \left(\frac{\nu R T}{p} \right) = \frac{\nu R}{p} dT - \frac{\nu R T}{p^2} dp.
	\label{eq:earth-atm-dv}
\end{equation}
Подставив \eqref{eq:earth-atm-dv} в уравнение \eqref{eq:earth-atm-dt}, получим
\begin{gather*}
	dT = - \frac{p}{C_V \nu} \left( \frac{\nu R}{p} dT - \frac{\nu R T}{p^2} dp \right),\\
%	\left( 1 + \frac{R}{C_V} \right)\frac{dT}{T} = \frac{R}{C_V} \frac{dp}{p},\\
	\therefore \frac{dT}{T} = \frac{R}{C_V + R} \frac{dp}{p} = \frac{R}{C_p} \frac{dp}{p}.
\end{gather*}

Подставим сюда сюда выражение для $dp$ \eqref{eq:earth-atm-dp}:
\begin{gather*}
	\frac{d T}{T} = -\frac{R \mu g}{C_P R T} dh,\\
	\frac{dT}{dh} =  -\frac{\mu g}{C_P} \equiv \Gamma.
\end{gather*}
Данное выражение показывает, что в случае адиабатической атмосферы градиент температуры $\Gamma$ является постоянной величиной. Поэтому температура воздуха $T(h) = T_0 + \Gamma h$, где $T_0$~--- температура у поверхности. Для воздуха молярная теплоёмкость при постоянном давлении $C_p = 29.15$~Дж/моль, поэтому градиент температуры $\Gamma$ составляет $-0.83$~К на 100~м. 

Вернемся к уравнению \eqref{eq:earth-atm-dp} и подставим в него полученное выражение для температуры, чтобы получить барометрическую формулу:
\begin{gather*}
	dp = - \frac{p \mu g}{RT(h)} dh,\\
	\int\limits_{p_0}^p \frac{dp}{p} = - \frac{\mu g}{R} \int\limits_0^H \frac{dh}{T_0 + \Gamma h}= - \frac{\mu g}{R\Gamma} \int\limits_0^H \frac{d(T_0 + \Gamma h)}{T_0 + \Gamma h},\\
	\ln \frac{p}{p_0} = - \frac{\mu g}{R\Gamma} \ln \frac{T_0 + \Gamma H}{T_0},\\
	p(H) = p_0 \left(1 + \frac{\Gamma H}{T_0} \right)^{ - \mu g /R\Gamma} =  p_0 \left(1 + \frac{\Gamma H}{T_0} \right)^{ - C_p/R}.
\end{gather*}

Считая понижение давление в $e$ раз характерным, получаем, что высота атмосферы при адиабатической модели
\begin{equation*}
	H_{dQ=0} = \left(\exp \left[ \frac{R}{C_p} \right] - 1 \right) \cdot  \frac{T_0}{\Gamma} = 9.6~\text{км}.
\end{equation*}

\begin{figure}[h]
	\centering
	\begin{tikzpicture}
	\begin{axis}[
		height	=	6cm,
		width	=	8cm,
		xlabel	=	{$H$, км},
		ylabel	=	{$p(H)$, км},
%		ylabel shift	= -1 cm,
		ymax=1,
		ymin=0,
		xmin=0,
		xmax=10,
		legend cell align=left,
			legend style={
			draw=none,
			fill=none,
			font=\scriptsize,
			at={(axis cs:1, 0.1)}, anchor=south west,
			row sep=.5pc,
			},
		]
		
		\addplot[smooth, domain=0:10, samples=200] {1 * exp(-0.029*9.81*x*1000/8.31/285) };
		\addplot[smooth, domain=0:10, samples=200, gray] {1 * (1 + 10*x/285)^(-0.029*9.81/8.31/0.00975)};
		\legend{
			$T=\const$,
			$dQ = 0$
		};
	\end{axis}
\end{tikzpicture}
\end{figure}


















