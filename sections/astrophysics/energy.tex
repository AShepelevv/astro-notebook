\subsection{Энергия излучения}
Согласно ГОСТ 7601---78 \term{энергия излучения}~--- энергия, переносимая излучением, обозначается как $Q_e$. А \term{поток излучения}~--- мощность излучения, определяемая отношением энергии, переносимой излучением, ко времени переноса,  значительно превышающему период электромагнитных колебаний, обозначается $\Phi_e$ и согласно определения вычисляется как
\begin{equation}
	\Phi_e = \frac{\partial Q_e}{\partial t} = [\text{Вт}].
\end{equation}

Одним из параметров потока излучения является \term{спектральная плотность потока излучения}~--- величина потока излучения в расчёте на единицу длины или частоты:
\begin{gather*}
    \Phi_{e, \lambda}
    = \frac{\partial \Phi_e}{\partial \lambda} 
    = \frac{\partial^2 Q_e}{\partial t \, \partial \lambda}
    = [\text{Вт/м}],\\
    \Phi_{e, \nu} 
    = \frac{\partial \Phi_e}{\partial \nu} 
    = \frac{\partial^2 Q_e}{\partial t \, \partial \nu}
    = \frac{\lambda^2}{c}\Phi_{e, \lambda}
    = [\text{Вт/Гц}].
\end{gather*}

%Также вводится понятие \term{объемной плотности энергии излучения}~--- это отношение энергии излучения к объему, который оно заполняет,
%\begin{equation}
%	U_e = \frac{d Q_e}{dV} = \left[\frac{\text{Дж}}{\text{м}^3}\right].
%\end{equation}

В ГОСТ 26148---84 вводятся такие понятия, как \term{светимость}~--- физическая величина, определяемая отношением потока излучения, исходящего от малого участка по­верхности, содержащего рассматри­ваемую точку, к площади этого участка:
\begin{equation}
	M_e = \frac{\partial \Phi_e}{\partial S},
\end{equation}
здесь $S$~--- площадь поверхности объекта, испускающего энергию.

\term{Освещенность (облученность)}~--- величина, равная отношению светового потока, падающего на малый участок поверхности, к его площади~--- поверхностная плотность потока
\begin{equation}
	E_e = \frac{d\Phi_e}{dS_2} \sim \frac{1}{r^2},
\end{equation}
здесь $S_2$~--- площадь поверхности приёмника, $r$~--- расстояние от источника.

\term{Яркость}~--- световой поток, приходящийся на единичный телесный угол, в расчёте на единичную площадку проекции излучающей поверхности на картинную плоскость,
\begin{equation}
	L_e = \frac{d^2 \Phi_e}{d \Omega \,d S_1 \cos \varepsilon},
\end{equation}
где $\varepsilon$~--- угол между направлением потока излучения и нормалью к плоскости излучающей поверхности.

\term{Интегральная яркость}~--- интеграл яркости по видимой поверхности источника. Показывает количество энергии, пришедшее от источника за единицу времени.
\begin{equation}
	\Lambda_e = \int \limits_S L_e(\vec{r}) \,d s.
\end{equation}

