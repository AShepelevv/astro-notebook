\subsection{Чёрные дыры}
\textit{Чёрная дыра} (ЧД) --- область пространства-времени, гравитационное притяжение которой настолько велико, что покинуть её не могут даже объекты, движущиеся со скоростью света. Граница этой области называется \textit{горизонтом событий}, а её характерный размер --- \textit{гравитационным радиусом}, который вычисляется по следующей формуле:
\begin{equation}
r_G=\frac{2GM}{c^2}
\end{equation}

Минимальная масса ЧД составляет около $2.5M_{\odot}$. Разделив массу ЧД на её объём, можно получить среднюю плотность ЧД:
\begin{equation}
\rho=\frac{3c^6}{32\pi M^2G^3},
\end{equation}
где $M$ --- масса ЧД, $c$ --- скорость света.

Эффект излучения (испарения) Хокинга --- эффект, при котором гравитационное поле черной дыры поляризует вакуум, в результате чего возможно образование не только виртуальных, но и реальных пар частица~--античастица. Одна из частиц, оказавшаяся чуть ниже горизонта событий, падает внутрь чёрной дыры, а другая, оказавшаяся чуть выше горизонта, улетает, унося энергию (то есть часть массы) чёрной дыры. Для мощности излучения ЧД справедлива следующая формула:
\begin{equation}
L=\frac{hc^6}{30720\pi^2G^2M^2},
\end{equation}
где $h$ --- постоянная Планка.

Спектр хокинговского излучения для безмассовых полей оказался строго совпадающим с излучением абсолютно чёрного тела, что позволило приписать ЧД температуру
\begin{equation}
T=\frac{hc^3}{16\pi^2kGM},
\end{equation}
где $k$ --- постоянная Больцмана.