\subsection{Закон Стефана-Больцмана}
\term{Закон Стефана~--- Больцмана} определяет зависимость плотности мощности излучения абсолютно чёрного тела (АЧТ) $u$ от его температуры $T$:
\begin{equation}
	u = a T^4,
\end{equation}
где $a$~--- некая универсальная константа.
Отсюда полная светимость АЧТ с площадью поверхности $S$
\begin{equation}
	L = S \sigma T^4,
	\label{eq:steff-bol-law}
\end{equation}
константа $\sigma$ называется \term{постоянной Стефана-Больцмана}.

Важно отметить, что \imp{закон Стефана-Больцмана}~--- прямое следствие формулы Планка (\ref{eq:plancks-law-nu} -- \ref{eq:plancks-law-lambda}), так как\change{, исходя из физического смысла формулы Планка, справедливы равенства}
\begin{multline}
	\sigma T^4 = \int\limits^\infty_0 B(\lambda, T)\,d\lambda \int\limits_0^{\pi/2} \sin \varphi\, d\varphi \int\limits_0^{2\pi} \cos \varphi\, d\theta = \pi \int\limits^\infty_0 B(\lambda, T)\,d\lambda =\\
	\change{= \int\limits^\infty_0 B(\nu, T)\,d\nu \int\limits_0^{\pi/2} \sin \varphi\, d\varphi \int\limits_0^{2\pi} \cos \varphi\, d\theta = \pi \int\limits^\infty_0 B(\nu, T)\,d\nu.}
\end{multline}

\change{Здесь интегрирование ведется в сферических координатах $(\varphi, \theta)$ по телесному углу $d\Omega = d\varphi \cos \varphi d\theta$. А $\sin \varphi$ во втором интеграле отвечает за проекцию единичной площадки на направление излучения. Вычислим данный интеграл, чтобы получить значение постоянной Стефана-Больцмана:
\begin{equation*}
	\sigma T^4 = \pi \int\limits_0^{\infty} \frac{2h\nu^3}{c^2}\cdot \frac{1}{\exp\left(\frac{h\nu}{kT}\right)-1} \,d\nu.
\end{equation*}
Сделаем замену $x = \frac{h \nu}{k T}$, так как $dx = \frac{h}{k T} d\nu$, то
\begin{equation*}
	\sigma T^4 = \frac{2\pi}{c^2}  \int\limits_0^{\infty} \underbrace{\frac{k^3 T^3}{h^2} x^3}_{h\nu^3} \cdot \frac{1}{e^x - 1} \cdot \underbrace{\frac{kT}{h} \,dx}_{d\nu} = \frac{2 \pi k^4 T^4}{c^2 h^3} \int\limits_0^{\infty} \frac{x^3 \, dx}{e^x - 1}.
\end{equation*}
Значение табличного интеграла $\int\limits_0^{\infty} \frac{x^3 \, dx}{e^x - 1}$ равно $\frac{\pi^4}{15}$, откуда
\begin{equation*}
	\sigma = \frac{2 \pi^5 k^4}{15 c^2 h^3} = 5.67 \cdot 10^{-8}~\frac{\text{Вт}}{\text{м}^2 \cdot \text{К}^4}.
\end{equation*}
}
Для звёзд главной последовательности выполняется соотношение $L \sim M^{\alpha}$, где~$\alpha$~--- коэффициент пропорциональности, который зависит от массы звезды следующим образом:
\begin{align*}
	\alpha &= 2.5,\quad M < 0.5 M_\odot; \\
	\alpha &= 4, \quad 0.5 M_\odot < M < 8 M_\odot;\\
	\alpha &= 2.5, \quad  M > 8 M_\odot.
	%\alpha &= 1, \quad M > 20 M_\odot.
\end{align*}
Также существует примерная зависимость светимости звёзды от её радиуса, имеющая вид  $L\sim R^{5.2}$.
