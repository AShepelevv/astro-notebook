\subsection{Оптические атмосферные эффекты}

В этом разделе будут рассмотрены известные оптические эффекты, возникающие в атмосфере Земли. Их можно разделить на три группы по природе происхождения. Так, \term{радуга} (см.~\ref{sec:rainbow}) формируется в ходе рефракции и отражения света в достаточно крупных каплях воды, распыленных в воздухе. Различные виды \term{гало}~--- \imp{гало} (см.~\ref{sec:halo}), \imp{зенитная дуга} (см.~\ref{sec:circumzenithal-arc}), \imp{округло-горизонтальная дуга} (см.~\ref{sec:circumhorizon-arc}) и \imp{паргелий} (см.~\ref{sec:parhelion}), формируются в результате прохождения света через кристаллы льда, находящиеся в воздухе. А такие эффекты как \imp{глория} (см.~\ref{sec:glory}), \imp{венец (корона)} (см.~\ref{sec:corona}) и \imp{радужные облака} (см.~\ref{sec:iridenscent-clouds}) формируются в силу  диффракции света на мелких каплях или кристаллах льда, или других частицах, например, частицы дыма или мелкий песок.

\subsubsection{Радуга}
\label{sec:rainbow}

\newcommand{\drawRainbow}[2][1]{
    \tkzInit[xmin=-0.4, ymin=-1.5*#1, xmax=2.1, ymax=1.1*#1]
    \tkzClip
    
    \tkzDefPoint(0,0){R}
    \tkzDefPoint(1,0){C}
    
    \foreach \x in {-0.02,-0.04,...,-1} {
%        \tkzSetUpLine[gray]

        \tkzDefPoint(-0.5,\x){A}
        \tkzDefPoint(0,\x){B}
        
        \tkzInterLC(B,A)(C,R) 
        \tkzGetPoints{R1}{x}
        
        \tkzDrawSegment[gray](A,R1)
        
        \tkzFindAngle(A,R1,C) 
        \tkzGetAngle{ABC}
        
        \pgfmathparse{\ABC - 180}
        \pgfmathsetmacro\ALPHA{\pgfmathresult}
        
        \pgfmathparse{asin(sin(\ALPHA)/1.333)}
        \pgfmathsetmacro\BETA{\pgfmathresult}
        
        \tkzDefPointBy[homothety=center R1 ratio 1](R1) 
        \tkzGetPoint{R2}
        \foreach \i in {0,1,...,#2} {
            \tkzDefPointBy[rotation=center R1 angle -\BETA](C) 
            \tkzGetPoint{R1'}
            
            \tkzInterLC(R1,R1')(C,R) 
            \tkzGetPoints{x}{R2}
            
            \tkzDrawSegment[gray](R1,R2)
            
            \tkzDefPointBy[homothety=center R1 ratio 1](R2) 
            \tkzGetPoint{R1}
        }
        
        \tkzDefPointBy[rotation=center R1 angle -180 + \ALPHA](C) 
        \tkzGetPoint{L'}
        
        \tkzDefPointBy[homothety=center R1 ratio 1.5](L') 
        \tkzGetPoint{L}
        
        \tkzDrawSegment[gray](R2,L)
    }
        
    \tkzDefPoint(-0.5,0){A0}
    \tkzDefPoint(2,0){B0}
    \tkzDrawSegment[dotted, thick](A0,B0))
        
    \tkzDrawCircle[thick, black](C,R)
}
\newcommand{\drawRainbowDispersion}[2][1]{
    \tkzInit[xmin=-0.4, ymin=-1.5*#1, xmax=2.1, ymax=1.1*#1]
    \tkzClip
    
    \tkzDefPoint(0,0){R}
    \tkzDefPoint(1,0){C}
    
    \foreach \n in {1.332,1.334,...,1.345} {

        \pgfmathparse{sqrt((1 + #2)^2 -\n^2)/sqrt(2 * #2 + #2 * #2)}
        \pgfmathsetmacro\x{\pgfmathresult}

    
        \tkzDefPoint(-0.5,-\x){A}
        \tkzDefPoint(0,-\x){B}
        
        \tkzInterLC(B,A)(C,R) 
        \tkzGetPoints{R1}{x}
        
        \tkzDrawSegment[gray](A,R1)
        
        \tkzFindAngle(A,R1,C) 
        \tkzGetAngle{abc}
        
        \pgfmathparse{\abc - 180}
        \pgfmathsetmacro\ALPHA{\pgfmathresult}
            
        \pgfmathparse{asin(sin(\ALPHA)/\n)}
        \pgfmathsetmacro\BETA{\pgfmathresult}
        
        \tkzDefPointBy[homothety=center R1 ratio 1](R1)
        \tkzGetPoint{R2}
        
        \foreach \i in {0,1,...,#2} {
            \tkzDefPointBy[rotation=center R1 angle -\BETA](C) 
            \tkzGetPoint{R1'}
            
            \tkzInterLC(R1,R1')(C,R) 
            \tkzGetPoints{x}{R2}
            
            \tkzDrawSegment[gray](R1,R2)
            
            \tkzDefPointBy[homothety=center R1 ratio 1](R2) 
            \tkzGetPoint{R1}
        }
                
        \pgfmathparse{-180+\ALPHA}
        \pgfmathsetmacro\r{\pgfmathresult}
        
        \tkzDefPointBy[rotation=center R1 angle \r](C) 
        \tkzGetPoint{L'}
        
        \tkzDefPointBy[homothety=center R1 ratio 1](L') 
        \tkzGetPoint{L}
        
        \tkzDrawSegment[gray](R1,L)
    }
        
    \tkzDefPoint(-0.5,0){A0}
    \tkzDefPoint(2,0){B0}
    \tkzDrawSegment[dotted, thick](A0,B0))
        
    \tkzDrawCircle[thick, black](C,R)
}

Радуга~--- событие проявления преломления (рефракции) и отражения солнечного света, в каплях воды, распыленных в воздуха. Наблюдается в виде ярких разноцветных колец, перпендикулярных прямой \textsl{Солце -- наблюдатель}, с центром на ней. 

Стоит обратить внимание на возможность обобщения данной формулировки. Излучение может быть монохромным, тогда радуга будет также монохромна. Для появления радуги необязательно наличие воздуха. Данного эффекта можно достичь, распыляя воду в космосе~--- без атмосферы и гравитации. Также радуга может наблюдаться во взвеси стеклянных шариков или капель любой другой прозрачной для рассматриваемого излучения жидкости.

Рассмотрим детальнее, как формируются радужные кольца. Для упрощения расчетов солнечный свет можно считать параллельным фронтом с выделенным направлением, а капли воды с достаточной точностью сферическими. 

\begin{wrapfigure}[13]{r}{0.47\tw}
    \vspace{-1.5pc}
    \tikzsetnextfilename{rainbow-scheme}
\begin{tikzpicture}[
    scale=2,
    arrow/.style 2 args={
        postaction=decorate,
        decoration={
            markings, 
            mark=at position #2 with {\arrow{#1}}
        } 
    }
]
    \tkzInit[xmin=-0.55, ymin=-1.2, xmax=2.15, ymax=1.15]
    \tkzClip
    
    \tkzDefPoint(0,0){R}
    \tkzDefPoint(1,0){C}

    \tkzDefPoint(-0.5,0.8){A}
    \tkzDefPoint(0,0.8){B}
    
    \tkzInterLC(B,A)(C,R) 
    \tkzGetPoints{x}{R1}
    
    \tkzFindAngle(A,R1,C) 
    \tkzGetAngle{ABC}
    
    \pgfmathparse{180 - \ABC}
    \pgfmathsetmacro\ALPHA{\pgfmathresult}
    
    \pgfmathparse{asin(sin(\ALPHA)/1.333)}
    \pgfmathsetmacro\BETA{\pgfmathresult}
    
    \tkzDefPointBy[rotation=center R1 angle \BETA](C) 
    \tkzGetPoint{R1'}
    \tkzInterLC(R1,R1')(C,R) 
    \tkzGetPoints{R2}{x}
    
    \tkzDefPointBy[rotation=center R2 angle \BETA](C) 
    \tkzGetPoint{R2'}
    \tkzInterLC(R2,R2')(C,R) 
    \tkzGetPoints{R3}{x}
    
    \tkzDefPointBy[rotation=center R3 angle \BETA](C) 
    \tkzGetPoint{R3'}
    \tkzInterLC(R3,R3')(C,R) 
    \tkzGetPoints{R4}{x}
    
    \tkzDefPointBy[rotation=center R4 angle 180-\ALPHA](C) 
    \tkzGetPoint{L}
    
    \tkzDefPoint(-0.5,0){A0}
    \tkzDefPoint(2,0){B0}
    \tkzDrawSegment[dash dot, thick](A0,B0)
%    
    \tkzDrawSegments[arrow={latex}{0.5}](R1,R2 R2,R3 R3,R4 R4,L)
    \tkzDrawSegments[arrow={latex}{0.7}](A,R1)
    \tkzDrawLines[dashed](C,R2 C,R3)
    
    \tkzMarkAngle[line width = 0.15pt, size=0.01](C,R1,R2) % костыль для расстояния между дугами
    \tkzMarkAngles[size = 0.25](C,R1,R2 C,R2,R3 C,R3,R4)
    \tkzMarkAngles[size = 0.3](R1,R2,C R2,R3,C R3,R4,C)
    \tkzLabelAngles[pos=0.4, fill=white, inner sep=1pt](C,R1,R2 C,R2,R3 C,R3,R4 R1,R2,C R2,R3,C R3,R4,C){\footnotesize $\beta$}
    
    
    \tkzDefPointBy[homothety=center R1 ratio -0.5](C) \tkzGetPoint{C1}
    \tkzDefPointBy[homothety=center R2 ratio -0.5](C) \tkzGetPoint{C2}
    \tkzDefPointBy[homothety=center R3 ratio -0.5](C) \tkzGetPoint{C3}
    \tkzDefPointBy[homothety=center R4 ratio -.15](C) \tkzGetPoint{C4}
    
    \tkzDrawLines[dashed](C,C1 C,C4)
    \tkzMarkAngles[size=0.15, arc=ll](C1,R1,A L,R4,C4)
    \tkzLabelAngles[pos=0.3,font=\footnotesize](C1,R1,A L,R4,C4){$\alpha$}
    
    \tkzDefPointBy[rotation=center R1 angle -90](C) \tkzGetPoint{T1}
    \tkzDefPointBy[rotation=center R2 angle  90](C) \tkzGetPoint{T2}
    \tkzDefPointBy[rotation=center R3 angle  90](C) \tkzGetPoint{T3}
    \tkzDefPointBy[rotation=center R4 angle  90](C) \tkzGetPoint{T4}
    
    
    \tkzMarkRightAngles[size=0.1](T1,R1,C T2,R2,C2 T3,R3,C3 C,R4,T4)
    
    \tkzInterLL(R4,L)(A,R1)
    \tkzGetPoint{W}
    \tkzMarkAngle[line width = 0.2pt, size=0.01](C,R1,R2) % костыль для расстояния между дугами
    \tkzMarkAngle[arc=lll, size=0.15](R1,W,L)
    \tkzLabelAngle[pos=0.3, font=\footnotesize](R1,W,L){$\omega$}
    
    \tkzDefMidPoint(A,R1)  \tkzGetPoint{M0}
    \tkzDefMidPoint(R1,R2) \tkzGetPoint{M1}
    \tkzDefMidPoint(R2,R3) \tkzGetPoint{M2}
    \tkzDefMidPoint(R3,R4) \tkzGetPoint{M3}
    \tkzDefMidPoint(R4,L)  \tkzGetPoint{M4}
    
    \tkzDefPointBy[homothety=center A ratio 0.1](R1) \tkzGetPoint{A'}
    \tkzDefPointBy[translation=from A to A'](A0)    \tkzGetPoint{A0'}
    \tkzDrawSegment[latex-latex](A',A0')
    \tkzLabelSegment[left, font=\footnotesize](A',A0'){$\rho$}
    
    \tkzDrawCircle[thick, black](C,R)
    
    \tkzDrawPoints(C, R1, R2, R3, R4)
\end{tikzpicture}

    \caption{Схема следования луча в капле при $k=2$}
\end{wrapfigure}
Представим такую каплю и отметим выделенное направление падения света пунктиром (\lookPicRef{pic:rainbow}). Рассмотрим луч, падающий на каплю на расстоянии $\rho \in [0,1]$ радиусов капли от её оси. В таком случае угол падения луча на поверхность капли $\alpha = \arcsin \rho$. Пусть $n$~--- коэффициент преломления воды, тогда по закону Снеллиуса угол преломления
\begin{equation*}
    \beta = \arcsin \frac{\sin{\alpha}}{n} = \arcsin \frac{\rho}{n}.
\end{equation*}
Так как $\beta$~--- угол между радиусом и преломленным лучем, то преломленный луч вновь пересечется с поверхностью капли на угловом расстоянии $180^\circ - 2\beta$ в сторону движения луча. 

В точке пересечения преломленного луча с поверхностью капли часть света преломится и выйдет наружу, а остальная часть отразится от внутренней поверхности под углом $\beta$ согласно закону отражения. И~на угловом расстоянии $180^\circ - 2\beta$ вновь частично преломится и покинет каплю, а частично отразится и продолжит свой путь внутри капли.

Определим \imp{порядок} радуги $k$ как число отражений луча от внутренней поверхности капли. Тогда при формировании радуги $k$-ого порядка луч света, проходящий на расстоянии $\rho$ радиусов капли от её оси с точностью до целого числа полных оборотов поворачивается на угол 
\begin{multline*}
    \omega 
        = 2(\alpha - \beta) + k (\pi - 2\beta) 
        = 2\alpha - 2\beta(k + 1) + \pi (k \!\!\!\! \mod 2) = \\
        = 2\arcsin \rho - 2(k + 1) \arcsin \frac{\rho}{n} + \pi (k \!\!\!\!  \mod 2). 
\end{multline*}

Радуга выделяется на фоне за счет своей яркости, которая достигается из-за увеличения плотности потока излучения, выходящего из капли после $k$ отражения, в конкретном направлении. Что соответствует равенству нулю производной $\partial\omega/\partial\rho$. Определим значение~$\rho$, при котором это происходит, чтобы найти угловой радиус радуги $k$-ого порядка.
\begin{gather*}
    \frac{\partial \omega}{\partial \rho} = \frac{2}{\sqrt{1 - \rho^2}} - \frac{k + 1}{\sqrt{n^2 - \rho^2}} = 0,\quad \rho \in [0,1];\\
%    \sqrt{n^2 - \rho^2} - (k + 1) \sqrt{1 - \rho^2} = 0,\quad \rho \in [0,1);\\
%    n^2 - \rho^2 = (k+1)^2(1 - \rho^2), \quad \rho \in [0,1) ;\\
    \rho = \sqrt{\frac{(k+1)^2 - n^2}{(k+1)^2 - 1}},\quad \rho \in [0, 1).
\end{gather*}

Коэффициент преломления воды в оптическом диапазоне зависит от длины волны практически линейно, принимая значения от $n_\text{к} = 1.331$ для красного света ($\lambda = 700$~нм) до $n_\text{ф} = 1.344$ для фиолетового ($\lambda = 400$~нм). Таким образом внешний край \imp{первичной} радуги имеет радиус~$42.4^\circ$ и красный цвет, а внутренний~---~$40.5^\circ$ и фиолетовый цвет,  \lookPicRef{pic:rainbow-disp-1}.
\imp{Вторичная} радуга имеет размер от~$50.4^\circ$ для красного до~$53.7^\circ$ для фиолетового, \lookPicRef{pic:rainbow-disp-2}. Важно отметить, из-за дополнительного отражения цвета вторичной радуги идут в обратном порядке относительно первичной. 

\begin{figure}[t]
    \begin{subcaptionblock}{0.3\tw}
        \centering
        \tikzsetnextfilename{rainbow-1-scheme}
        \begin{tikzpicture}[scale=1.3]
            \begin{scope}[yscale=-1]
                \drawRainbow[-1]{1}
            \end{scope}
        \end{tikzpicture}
        \caption{$k=1$}
        \label{pic:rainbow-1}
    \end{subcaptionblock}
    \hfill
    \begin{subcaptionblock}{0.3\tw}
        \centering
        \tikzsetnextfilename{rainbow-2-scheme}
        \begin{tikzpicture}[scale=1.3]
            \drawRainbow{2}
        \end{tikzpicture}
        \caption{$k=2$}
        \label{pic:rainbow-2}
    \end{subcaptionblock}
    \hfill
    \begin{subcaptionblock}{0.3\tw}   
        \centering
        \tikzsetnextfilename{rainbow-3-scheme}
        \begin{tikzpicture}[scale=1.3]
            \drawRainbow{3}
        \end{tikzpicture}
        \caption{$k=3$}
        \label{pic:rainbow-3}
    \end{subcaptionblock}
    \caption{Схема хода лучей при формировании радуги $k$-ого порядка}
    \label{pic:rainbow}
\end{figure}
\begin{figure}[h!]
    \begin{subcaptionblock}{0.3\tw}
        \centering
        \tikzsetnextfilename{rainbow-1-dispersion}
        \begin{tikzpicture}[scale=1.3]
            \begin{scope}[yscale=-1]
                \drawRainbowDispersion[-1]{1}
            \end{scope}
            \tkzDefPoint(1.2,-1.15){Label}
            \tkzLabelPoint[below=-1pt](Label){\tiny$42.4^\circ$}
            \tkzLabelPoint[left=6pt](Label){\tiny$40.5^\circ$}
        \end{tikzpicture}
        \caption{$k=1$}
        \label{pic:rainbow-disp-1}
    \end{subcaptionblock}
    \hfill
    \begin{subcaptionblock}{0.3\tw}
        \centering
        \tikzsetnextfilename{rainbow-2-dispersion}
        \begin{tikzpicture}[scale=1.3]
            \begin{scope}
                \drawRainbowDispersion{2}
            \end{scope}
            \tkzDefPoint(-0.2,-0.3){Label}
            \tkzLabelPoint[below](Label){\adjustbox{right=18pt, raise=12pt}{\tiny$53.7^\circ$}}
            \tkzLabelPoint[above](Label){\adjustbox{raise=18pt}{\tiny$50.4^\circ$}}
        \end{tikzpicture}
        \caption{$k=2$}
        \label{pic:rainbow-disp-2}
    \end{subcaptionblock}
    \hfill
    \begin{subcaptionblock}{0.3\tw}
        \centering
        \tikzsetnextfilename{rainbow-3-dispersion}
        \begin{tikzpicture}[scale=1.3]
            \drawRainbowDispersion{3}
            \tkzDefPoint(1.1,-1.3){Label}
            \tkzLabelPoint[above right=-2pt](Label){\tiny${37.7^\circ}$}
            \tkzLabelPoint[below left=-1pt](Label){\adjustbox{raise=15pt}{\tiny${42.5^\circ}$}}
        \end{tikzpicture}
        \caption{$k=3$}
        \label{pic:rainbow-disp-3}
    \end{subcaptionblock}
    \caption{Схема дисперсии лучей при формировании радуги $k$-ого порядка}
\end{figure}

Обе радуги, несложно заключить, расположены напротив Солнца и их радиус отсчитывается от противосолнечной точки. Отсюда следует важное замечание, чем ниже Солнце расположено над горизонтом, тем выше радуги первого и второго порядков. Напротив, \imp{третичная} радуга расположена со стороны Солнца на расстоянии от~$37.7^\circ$ для фиолетового до~$42.5^\circ$ для красного цветов, \lookPicRef{pic:rainbow-disp-3}.

В заключение необходимо объяснить применимость геометрической оптики в изложенных выше рассуждениях. Чаще всего радуга наблюдается до или после дождя, капли которого существенно больше капель тумана или облаков и достигают нескольких миллиметров в диаметре. Это больше характерной длины волны видимого глазом излучения на три--четыре порядка, что позволяет не учитывать волновые свойства света. В силу последних капли существенно меньшего размера могут вообще не сформировать радугу. 

\subsubsection{Гало}
\label{sec:halo}

Как было сказано выше, различные виды \imp{гало} наблюдаются в результате рефракции солнечного света в кристаллах льда, составляющих перистые облака. Эти облака в среднем находятся на высоте от 5 до 10~км, имеют температуру около $-40^\circ$C, почему состоят исключительно из кристаллов льда.


\begin{wrapfigure}[6]{r}{0.3\tw}
    \centering
    \vspace{-0.7pc}
    \newcommand{\drawPrizm}[2]{
        \tkzDefPoint(0,0){C}
	    
	    \def\R{#1}
	    \def\h{#2}
	    \def\f{0}
	    
	    \tkzDefShiftPoint[C](\R,0){x}
	    \tkzDefPointBy[rotation=center C angle \f](x)  \tkzGetPoint{R1}
	    \tkzDefPointBy[rotation=center C angle 60](R1) \tkzGetPoint{R2}
	    \tkzDefPointBy[rotation=center C angle 60](R2) \tkzGetPoint{R3}
	    \tkzDefPointBy[rotation=center C angle 60](R3) \tkzGetPoint{R4}
	    \tkzDefPointBy[rotation=center C angle 60](R4) \tkzGetPoint{R5}
	    \tkzDefPointBy[rotation=center C angle 60](R5) \tkzGetPoint{R6}
	    
	    \tkzGetPointCoord(R1){r}
	    \tkzGetPointCoord(R2){R}
	    \tkzGetPointCoord(R3){rr}
	    \tkzGetPointCoord(R4){rR}
	    \tkzGetPointCoord(R5){Rr}
	    \tkzGetPointCoord(R6){RR}
	    
	    \draw[dashed] (\rx,\ry,0) -- (\Rx,\Ry,0) -- (\rrx,\rry,0) -- (\rRx,\rRy,0);
	    \draw (\rRx,\rRy,0) -- (\Rrx,\Rry,0) -- (\RRx,\RRy,0) -- (\rx,\ry,0);
	    
	    \draw (\rx,\ry,\h) -- (\Rx,\Ry,\h) -- (\rrx,\rry,\h) -- (\rRx,\rRy,\h) -- (\Rrx,\Rry,\h) -- (\RRx,\RRy,\h) -- cycle; 
	    
	    \draw (\rx,\ry,0) -- (\rx,\ry,\h);
	    \draw[dashed] (\Rx,\Ry,0) -- (\Rx,\Ry,\h);
	    \draw[dashed] (\rrx,\rry,0) -- (\rrx,\rry,\h);
	    \draw (\rRx,\rRy,0) -- (\rRx,\rRy,\h);
	    \draw (\Rrx,\Rry,0) -- (\Rrx,\Rry,\h);
	    \draw (\RRx,\RRy,0) -- (\RRx,\RRy,\h);
    }

    \tdplotsetmaincoords{70}{165}
    \begin{tikzpicture}[tdplot_main_coords]
        \drawPrizm{0.4}{1.5}
	\end{tikzpicture}
	\tdplotsetmaincoords{70}{185}
    \begin{tikzpicture}[tdplot_main_coords]
        \drawPrizm{0.9}{0.3}
	\end{tikzpicture}
	\caption{}
	\label{}    
\end{wrapfigure}

В условия формирования перистых облаков кристаллы льда имеют шестиугольную симметрию и являются прямоугольными шестиугольными призмами. Их боковые грани могут иметь разные пропорции, но угол между соседними всегда составляет $60^\circ$. В дальнейшем будем разделять кристаллы на два вида: пластинчатые~--- размер оснований много больше высоты призмы, и колончатые~--- наоборот, высота призмы существенно больше размеров основания.

\paragraph{$\mathbf{22^\circ}$ гало}
\begin{wrapfigure}[11]{r}{0.25\tw}
    \vspace{-1pc}
    \centering
    \begin{tikzpicture}[
        arrow/.style 2 args={
            postaction=decorate,
                decoration={
                    markings, 
                    mark=at position #2 with {\arrow{#1}}
                } 
            }
    ]
    
	    \tkzDefPoint(0,0){C}
	    
	    \def\R{2}
	    \def\f{0}
	    \def\n{1.31}
	    
	    \tkzInit[
	       xmin={-0.2*\R},
	       xmax={1.1*\R},
	       ymin={-1.1*\R},
	       ymax={1.3*\R},
	    ]
	    \tkzClip
	    
	    
	    \tkzDefShiftPoint[C](\R,0){x}
	    \tkzDefPointBy[rotation=center C angle \f](x)  \tkzGetPoint{R1}
	    \tkzDefPointBy[rotation=center C angle 60](R1) \tkzGetPoint{R2}
	    \tkzDefPointBy[rotation=center C angle 60](R2) \tkzGetPoint{R3}
	    \tkzDefPointBy[rotation=center C angle 60](R3) \tkzGetPoint{R4}
	    \tkzDefPointBy[rotation=center C angle 60](R4) \tkzGetPoint{R5}
	    \tkzDefPointBy[rotation=center C angle 60](R5) \tkzGetPoint{R6}
	    
	    \tkzDrawPolygon(R1,R2,R3,R4,R5,R6)
	    
	    \def\al{30}
	    \def\x{0.4}
	    
	    \tkzDefPointBy[homothety=center R2 ratio \x](R3) \tkzGetPoint{I}
	    \tkzDefLine[perpendicular=through I](R3,R2) \tkzGetPoint{I'}
	    
	    \def\bet{asin(sin(\al / 180 * pi) / \n)}
	    \def\b{\R * (sqrt(3) / 2  + (0.5 - \x) * tan(pi / 2 - \bet))}
	    \def\xo{(\b + sqrt(3) * \R) / (tan(pi / 2 - \bet) + sqrt(3))}
	    \def\yo{sqrt(3) * \xo - sqrt(3) * \R}
	    
	    \tkzDefPoint(\xo,\yo){O}
	    \tkzDefLine[perpendicular=through O](R1,R6) \tkzGetPoint{O'}
	    
	    \tkzInterLL(I,I')(O,O') \tkzGetPoint{T}
	    
	    \tkzDefPointBy[rotation=center I angle {-180 + \al}](T) \tkzGetPoint{x}
	    \tkzDefPointBy[homothety=center I ratio 0.4](x) \tkzGetPoint{A}
	    
	    \tkzFindAngle(I,O,T) \tkzGetAngle{gam}
	    \def\del{asin(\n * sin(\gam / 180 * pi)) * 180 / pi}
	    
	    \tkzDefPointBy[rotation=center O angle {180 - \del}](T) \tkzGetPoint{x}
	    \tkzDefPointBy[homothety=center O ratio 0.5](x) \tkzGetPoint{D}
	    
	    \tkzDrawSegments[semithick, arrow={latex}{0.3}](A,I)
	    \tkzDrawSegments[semithick, arrow={latex}{0.5}](I,O)
	    \tkzDrawSegments[semithick, arrow={latex}{0.9}](O,D)
	     
	    \tkzDrawLines[dashed, add=0.5cm and 0.5cm](I,T O,T)
	    
	    \tkzMarkRightAngles[size=0.2](R3,I,T T,O,R6)
	    
	    \tkzMarkAngle[arc=lll, size=0.5](T,I,O)
	    \tkzLabelAngle[pos=0.75](T,I,O){\footnotesize$\beta$}
	    
	    \tkzMarkAngle[arc=lll, mark=|, size=0.4, mksize=2](I',I,A)
	    \tkzLabelAngle[pos=0.6](I',I,A){\footnotesize$\alpha$}
	    
	    \tkzMarkAngle[line width = .3pt, size=0.01](I,O,T)
	    \tkzMarkAngle[arc=ll, size=0.3](I,O,T)
	    \tkzLabelAngle[pos=0.5](I,O,T){\footnotesize$\gamma$}
	    
	    \tkzMarkAngle[arc=ll, mark=|, size=0.3, mksize=2](D,O,O')
	    \tkzLabelAngle[pos=0.5](D,O,O'){\footnotesize$\delta$}
	    
	    \tkzMarkAngles[size=0.2](O,T,I R2,R1,R6 R3,R2,R1 R1,R6,R5)
	    \tkzLabelAngle[pos=0.5](O,T,I){\scriptsize$120^\circ$}
	   	    
	    \tkzDrawPoints(I, T, O, R1, R2, R6)
	\end{tikzpicture}
	\caption{}
	\label{}
\end{wrapfigure}
Яркая окружность вокруг Солнца, угловой радиус которой оставляет около $22^\circ$ в зависимости от длины волны излучения. Формируется в произвольно расположенных кристаллах в результате рефракции света на двух, следующих через одну, боковых гранях.

В зависимости от угла падения света на первую грань меняется угол итогового преломления, минимальная величина которого~--- примерно $22^\circ$. В силу экстремальности этого значения, наибольшее число кристаллом преломляют солнечный свет именно под этим углом.

Детально расмотрим геометрию формирования $22^\circ$ гало. Пусть~$\alpha$~--- угол падения луча на боковую грань $\mathcal{A}$ в проекции на нормальную к оси кристалла плоскость, а $n$~--- коэффициент преломления льда. Тогда, исходя из закона Снеллиуса, угол преломления $\beta$ определяется соотношением $\sin \alpha = n \sin \beta$. Выберем место падения луча и угол $\alpha$ такими, чтобы преломленный луч упал на несмежную и непротивоположную боковую грань $\mathcal{B}$. Обозначим угол падения на эту грань как $\gamma$. Учитывая геометрию кристаллов льда, несложно показать, что угол между нормалями к $\mathcal{A}$ и $\mathcal{B}$ составляет $120^\circ$, таким образом $\gamma = 60^\circ - \beta$. Снова применяя закон Снеллиуса, получаем, $\sin \delta = n \sin \gamma$. Окончательно, угол преломления исходного луча кристаллом льда $\rho =  \alpha + \delta - 60^\circ$.

Из полученных соотношений имеем зависимость $\rho(\alpha)$. Максимум плотности излучения наблюдается близ экстремума данной функции, то есть при $\rho'(\alpha) = 0$. Что соответствует $\alpha_0 = 40.9^\circ$ и $\rho(\alpha_0) = 21.8^\circ$ при $n=1.31$. 

\paragraph{Паргелий}

Когда пластинчатые кристаллы дрейфуют вниз под действием силы тяжести, из-за сопротивления воздуха и своей формы они ориентируются определённым образом: их основания почти горизонтальны. Такая конфигурация стабильна, то есть небольшие отклонения создают корректирующие силы, возвращающие кристаллы в близкое горизонтальному положение. Таким образом у пластинчатых кристаллов только одна степень свободы~--- вращение в горизонтальное плоскости.

Эта особенность пластинчатых кристаллов проявляется в виде увеличения яркости $22^\circ$ гало на высоте в окрестности высоты Солнца над горизонтом. Данное явление считают отдельным видом гало~--- \imp{паргелием}. А 

\paragraph{Паргелический круг}

В силу горизонтальной ориентации пластинчатых кристаллов происходит следующее. Солнечный свет, попадая в пластинчатый кристалл через верхнюю торцевую грань, один или более раз отражаясь от боковых граней, выходит из кристалла через нижнюю торцевую грань под тем же углом к горизонту, однако, имея отличное по азимуту направление. Таким образом равномерно распределенные в воздухе пластинчатые кристаллы отражают солнечный свет во всех направлениях в горизонтальной плоскости, сохраняя угол следования лучей к горизонту.

В результате описанных выше отражений при подходящих метеорологических условиях и высоте Солнца наблюдатель может увидеть полную окружность, параллельную горизонту, проходящую через Солнце.

\begin{tikzpicture}
		\begin{axis}[
			height	=	7cm,
			width	=	7cm,
			xmin = -2.1,
			xmax = 2.1,
			ymin = -2.1,
			ymax = 2.1,
			grid=none,
			axis line style={draw=none},
			every tick/.style	=	{ draw=none},
			yticklabels={,,},
			xticklabels={,,} 
			]
			
			\addplot+[smooth, dashes, gray] table[x=x, y=y] {data/tanget_arc_grid-80.txt};
			\addplot+[smooth, dashes, gray] table[x=x, y=y] {data/tanget_arc_grid-70.txt};
			\addplot+[smooth, dashes, gray] table[x=x, y=y] {data/tanget_arc_grid-60.txt};
			\addplot+[smooth, dashes, gray] table[x=x, y=y] {data/tanget_arc_grid-50.txt};
			\addplot+[smooth, dashes, gray] table[x=x, y=y] {data/tanget_arc_grid-40.txt};
			\addplot+[smooth, dashes, gray] table[x=x, y=y] {data/tanget_arc_grid-30.txt};
			\addplot+[smooth, dashes, gray] table[x=x, y=y] {data/tanget_arc_grid-20.txt};
			\addplot+[smooth, dashes, gray] table[x=x, y=y] {data/tanget_arc_grid-10.txt};
			\addplot[smooth, gray] table[x=x, y=y] {data/tanget_arc_grid0.txt};
			\addplot+[smooth, dashes, gray] table[x=x, y=y] {data/tanget_arc_grid10.txt};
			\addplot+[smooth, dashes, gray] table[x=x, y=y] {data/tanget_arc_grid20.txt};
			\addplot+[smooth, dashes, gray] table[x=x, y=y] {data/tanget_arc_grid30.txt};
			\addplot+[smooth, dashes, gray] table[x=x, y=y] {data/tanget_arc_grid40.txt};
			\addplot+[smooth, dashes, gray] table[x=x, y=y] {data/tanget_arc_grid50.txt};
			\addplot+[smooth, dashes, gray] table[x=x, y=y] {data/tanget_arc_grid60.txt};
			\addplot+[smooth, dashes, gray] table[x=x, y=y] {data/tanget_arc_grid70.txt};
			\addplot+[smooth, dashes, gray] table[x=x, y=y] {data/tanget_arc_grid80.txt};
			\addplot[smooth] table[x=x, y=y] {data/tanget_arc_grid_border.txt};
			
			\addplot+[only marks, mark = o, mark options={scale=0.2, black}] table[x=x, y=y] {data/tanget_arc_h45.txt};
			\addplot+[only marks, mark = o, mark options={scale=0.2, darkgray}] table[x=x, y=y] {data/tanget_arc_h30.txt};
			\addplot+[only marks, mark = o, mark options={scale=0.2, gray}] table[x=x, y=y] {data/tanget_arc_h15.txt};
			\addplot+[only marks, mark = o, mark options={scale=0.2, lightgray}] table[x=x, y=y] {data/tanget_arc_h0.txt};
			
		\end{axis}
	\end{tikzpicture}

\paragraph{Зенитная дуга}

Радужная дуга недалеко от зенита~--- результат рефракции солнечного света в пластинчатых кристаллах льда, расположенных над наблюдателем. 


% todo: Как пересечение 22 гало и паргелического круга


\subsubsection{Зенитная дуга}
\label{sec:circumzenithal-arc}
\subsubsection{Округло-горизонтальная дуга}
\label{sec:circumhorizon-arc}
\subsubsection{Паргелий}
\label{sec:parhelion}
\subsubsection{Радужные облака}
\label{sec:iridenscent-clouds}
\subsubsection{Венец (корона)}
\label{sec:corona}
\subsubsection{Глория}
\label{sec:glory}