\subsection{Давление излучения}
\term{Давление электромагнитного излучения}, падающего на поверхность тела, в отсутствии рассеяния выражается формулой
\begin{equation}
    p = \frac{I}{c} \cdot (1 - k + A) \cdot \cos \beta,
\end{equation}
здесь $I$~--- поток падающего излучения, $c$~--- скорость света, $k$~--- коэффициент пропускания, $A$~--- коэффициент отражения, а $\beta$~--- угол падения излучения.

\term{Плотность энергии фотонного газа} — величина, отражающая количество энергии излучения в некотором объёме пространства. Рассмотрим пробную  площадку $dS$ в пространстве с фотонным газом. В одну сторону за время $dt$ она излучает исходя из закона Стефана-Больцмана \eqref{eq:steff-bol-law}:
\begin{equation*}
	d E=\sigma T^4 d t
\end{equation*}
С другой стороны количество частиц столкнувшихся с площадкой находящейся в идеальном газе за время $dt$ задаётся формулой \eqref{eq:mean-count-mxwl}:
\begin{equation*}
d N=\frac{1}{4} n \cdot\langle v\rangle \, d t=\frac{1}{4} n c \, d t
\end{equation*}
Если $d E=d N \cdot E_0$, где $E_0$~--- энергия одного фотона, а с другой стороны $d E=\sigma T^4 d t$, то:
\begin{equation*}
\sigma T^4=\frac{1}{4} n c E_0 \rightarrow E_0=\frac{4 \sigma T^4}{c n}
\end{equation*}
Таким образом плотность энергии:
\begin{equation}
	u=\frac{d W}{d V}=\frac{n \cdot E_0 d V}{d V}=\frac{4 \sigma T^4}{c},
\end{equation}
где $dW$~---- полная энергия в некотором объеме $dV$.

\term{Давление фотонного газа} определяется соотношением
\begin{equation}
    p_\text{ф.г.} = \frac{u}{3} = \frac{4 \sigma T^4}{3c},
\end{equation}
Возможными областями применения являются солнечный парус, а в отдалённом будущем~--- фотонный двигатель.
