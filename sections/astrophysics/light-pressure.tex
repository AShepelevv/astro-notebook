\subsection{Давление излучения}
\term{Давление электромагнитного излучения}, падающего на поверхность тела, в отсутствии рассеяния выражается формулой
\begin{equation}
    p = \frac{I}{c} \cdot (1 - k + A) \cdot \cos \beta,
\end{equation}
здесь $I$~--- поток падающего излучения, $c$~--- скорость света, $k$~--- коэффициент пропускания, $A$~--- коэффициент отражения, а $\beta$~--- угол падения излучения.

\term{Плотность энергии фотонного газа} — величина, отражающая количество энергии излучения в некотором объёме пространства. Рассмотрим пробную  площадку площади $dS$ в пространстве с фотонным газом. В одну сторону за время $dt$ она излучает, исходя из закона Стефана-Больцмана \eqref{eq:steff-bol-law}:
\begin{equation*}
	d E = \sigma T^4 \, d t.
\end{equation*}

С другой стороны, фотонный газ удовлетворяет определению идеального газа. Значит, количество частиц, столкнувшихся с площадкой за время $dt$, задаётся формулой \eqref{eq:mean-count-mxwl}:
\begin{equation*}
d N = \frac{1}{4} n \langle v \rangle \, d t = \frac{1}{4} n c \, d t,
\end{equation*}
так как скорость движения фотонов равна скорости света~$c$.

Пусть $E_0$~--- энергия одного фотона, тогда $dE = E_0\,dN$, следовательно,
\begin{gather*}
    \sigma T^4 \, dt = \frac{1}{4} n c E_0 \, dt,\\
    E_0 = \frac{4 \sigma T^4}{c n}.
\end{gather*}
Таким образом плотность энергии
\begin{equation}
	u = n E_0 = \frac{4 \sigma T^4}{c}.
\end{equation}

\term{Давление фотонного газа} определяется соотношением
\begin{equation}
    p_\text{ф.г.} = \frac{u}{3} = \frac{4 \sigma T^4}{3c},
\end{equation}
Возможными областями применения являются солнечный парус, а в отдалённом будущем~--- фотонный двигатель.
