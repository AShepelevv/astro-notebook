\subsection{Давление излучения}
\term{Давление электромагнитного излучения}, падающего на поверхность тела, в отсутствии рассеяния выражается формулой
\begin{equation}
    p = \frac{I}{c} \cdot (1 - k + A) \cdot \cos \beta,
\end{equation}
здесь $I$~--- поток падающего излучения, $c$~--- скорость света, $k$~--- коэффициент пропускания, $A$~--- коэффициент отражения, а $\beta$~--- угол падения излучения.

\term{Плотность энергии фотонного газа} — величина отражающая количество энергии излучения в некотором объёме пространства. Рассмотрим область пространства, занимаемую излучением, температурой~$T$ с излучательной способностью на стерадиан $I$. Для того чтобы определить плотность энергии излучения в равновесном состоянии, представим излучающую в обе стороны площадку. Излучая энергию $dE$ на поверхность $dS$, объем занимаемый излучением будет расти как $dS\cdot c \, dt$. Отсюда связь плотности энергии и излучательной способности:
\begin{equation*}
	u = \frac{dE}{dV} = \frac{dE}{dS \cdot c \, dt} = \frac{1}{c} \cdot \frac{dE}{dS\cdot dt} = \frac{I}{c}.
\end{equation*}
Исходя из закона Стефана-Больцмана излучательная способность площадки в полупространство равна $\sigma T^4$, однако наша площадка излучает в обе стороны, поэтому излучательная способность $I$ в нашем случае будет равна $2 \sigma T^4$. Теперь добавим условие термодинамического равновесия. Так как искусственная площадка излучает энергию из ниоткуда, для равновесия системы со всех прочих направлений в данную точку должна приходить та же самая энергия, что излучает площадка, отсюда полная $I$ в рассматриваемой точке равна $4 \sigma T^4$, следовательно для плотности энергии имеем:
\begin{equation}
	u = \frac{4\sigma T^4}{c}.
\end{equation}

\term{Давление фотонного газа} определяется соотношением
\begin{equation}
    p_\text{ф.г.} = \frac{u}{3} = \frac{4 \sigma T^4}{3c},
\end{equation}
Возможными областями применения являются солнечный парус, а в отдалённом будущем~--- фотонный двигатель.
