\subsection{Давление и температура в центре звезды}
В данном разделе найдем оценку для давления и температуры в центре звезды. Для этого рассмотрим такую модель, пусть звезда~--- однородное газовое облако радиусом $R$. Пусть плотность звезды равна $\rho$, а вещество~--- водородоподобная плазма, такое допущение справедливо для звёзд на главной последовательности. Тогда средняя масса частиц газа составляется примерно $m = 0.5m_p$, потому вещество состоит из одинакового числа протонов и электронов, так как в целом оно нейтрально, а $m_e \ll m_p$.

Аналогично поиску высоты атмосферы рассмотрим тонкий слой вещества высоты $dr$ с площадью сечения $dS$ на расстоянии $r$ от центра звезды. Пусть его масса  равна $dm$. На слой действуют три силы: сила давления газа снизу $F_\uparrow = p(r)\d S$, сила давления газа сверху $F_\downarrow = -p(r + \d r)\d S$ и сила гравитационного притяжения $F_g = -GM(r)/r^2\d m$ от массы, находящейся внутри сферы радиуса $r$, так как суммарное воздействие от внешней оболочка равно нулю. Стационарность звезды обеспечивается балансом этих сил, значит
\begin{gather*}
	F_\uparrow + F_\downarrow + F_g = 0,\\
	p(r)\d S - p(r + \d r)\d S - \frac{GM(r)}{r^2}\d m = 0,\\
	p\d S - (p + dp)\d S - \frac{4\pi G \rho r^3}{3r^2} \cdot \rho \d r \d S = 0,\\
%	-dp = \frac{4\pi G \rho^2}{3} \cdot r\d r,\\  
	\int\limits_{p_0}^0 -dp = \int\limits_{0}^R \frac{4\pi G \rho^2}{3} \cdot r\d r,\\
 	p_0 = \frac{4\pi G \rho^2 R^2}{6} = \frac{GM\rho}{R}.
\end{gather*}

Приняв вещество звезды за идеальный газ, получим:
\begin{gather*}
	p_0 = \frac{G M \rho}{R} = n k T_0,\\
	T_0 = \frac{G M \rho}{n k R} = \frac{G M \rho m}{\rho k R} = \frac{G M m_p}{2k R}.
\end{gather*}
Для Солнца имеем $p_0 \sim 10^{14}$~Па, $T_0 \sim 10^7$. В действительности же давление в центре Солнца оценивается на два порядка большим значением, однако температуры рассмотренная модель дает похожий на действительность результат.