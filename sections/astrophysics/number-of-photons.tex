\subsection{Количество фотонов. Средняя Энергия}
Если разделить формулу Планка на энергию одного фотона, то можно получить функцию распределения количества фотонов по энергиям
\begin{equation*}
	\frac{\pi B_{\lambda}}{E_{\gamma}} = \frac{2 \pi c}{\lambda^4} \cdot \frac{1}{\exp\left(\frac{hc}{\lambda k T}\right)-1} = f_{\lambda}.
\end{equation*}
Интегрируя от 0 до $\infty$ можно получить полное количество фотонов излучаемое АЧТ:
\begin{equation*}
	N = \int_{0}^{\infty}{\pi f_{\lambda}d\lambda} = \int_{0}^{\infty}{\frac{2\pi c}{\lambda^4} \cdot \frac{d\lambda}{\exp\left(\frac{hc}{\lambda k T}\right)-1}}.
\end{equation*}
Сделав замену $x = \frac{hc}{\lambda kt}$ интеграл сводится к
\begin{equation*}
	N = \frac{2\pi k^3 T^3}{h^3 c^2}\int_{0}^{\infty}{\frac{x^2 dx}{e^x-1}}.
\end{equation*}
Последний интеграл вычисляется аналочично интегралу, что появляется при выводе закона Стефана-Больцмана и равен $2\zeta(3)$, отсюда
\begin{equation}
	N = \frac{4\pi k^3 \zeta(3)}{h^3 c^2} \cdot T^3 = \beta T^3.
\end{equation}
Константа пропорциональности $\beta$ равна примерно $1.52 \cdot 10^{15}$. Если поделить полную энергию излучаемое АЧТ на полное количество фотонов, то можно получить \term{среднюю энергию чернотельного фотона}:
\begin{equation}
	\overline{E_{\gamma}} = \frac{\sigma T^4}{\beta T^3} \simeq 2.7 k T.
\end{equation}