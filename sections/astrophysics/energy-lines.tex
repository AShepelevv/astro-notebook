\subsection{Линии излучения и поглощения}

\term{Планетарная модель атома (Резерфорда)} описывает модель атома, состоящего из маленького ядра, в котором сосредоточена почти вся масса атома, и электронов, обращающихся по круговым орбитам вокргу ядра. \\

\term{Модель атома Бора} содержит в основе модель атома Резерфорда, но в данной модели электроны могут обращаться только по строго определённым орбитам.\\

\term{Певый постулат Бора.} Электроны в атоме могут двигаться только по определенным (стационарным) орбитам, находясь на которых они не излучают энергию.  Причём, стационарными являются лишь те орбиты, при движении по которым момент количества движения электрона равен целому числу постоянных Планка:
\begin{equation}
    m_e v r_n = \frac{n h}{2 \pi} \equiv n \hbar,~n \in \mathbb{N} \backslash \lbrace 0 \rbrace,
\end{equation}
где $m_e$~--- масса электрона, $v$~--- скорость электрона, $r_n$~--- радиус $n$-ой орбиты, $h$~--- постоянная Планка.

Рассмотрим электрон, находящийся на произвольной орбите вокруг ядра некоторого атома. Запишем для него равенство центробежной и Кулоновской силы:
\begin{equation}
    \frac{m_e v^2}{r_n} = \frac{Z^2 e^2}{4 \pi \varepsilon_0 r_n^2},
\end{equation}
где $Z$~--- зарядовое число, $e$~--- заряд электрона, $\varepsilon_0$~--- электрическая постоянная. Из данного уравнения и первого постулата Бора можно получить выражение для радиуса $n$-ой орбиты:
\begin{equation}
    r_n = \frac{\varepsilon_0 n^2 h^2}{\pi m_e Z^2 e^2}.
\end{equation}
Отсюда можно получить значение боровского радиуса~--- радиус первой орбиты для атома водорода $(Z = 1,~n = 1)$. $a_0 \approx 5,291769241 \times 10^{-11}~\text{м}$.

Считая, что полная энергия электрона равна кинетической, взятой со знаком минус, можно получить выражение для полной энергии электрона на некоторой орбите:
\begin{equation}
    E_n = -\frac{m_e Z^2 e^2}{8 n^2 h^2 \varepsilon_0}.
\end{equation}

Посчитаем, как изменяется энергия эектрона при переходе с уровня $n$ на уровень $m$:
\begin{equation}
    \Delta E_{n, m} = \frac{m_e Z^2 e^2}{8 n^2 h^2 \varepsilon_0} \left( \frac{1}{n^2} - \frac{1}{m^2}\right) = h \nu.
\end{equation}
Получается, что при переходе с нижнего уровня на верхний может поглотиться фотон с определенной частотой, а при ообратном переходе~--- излучиться. Частоту фотона можно найти из следущего выражения:
\begin{equation}
    \nu_{n, m} =
    \frac{\Delta E_{n, m}}{h} = \frac{m_e Z^2 e^2}{8 n^2 h^3 \varepsilon_0} \left( \frac{1}{n^2} - \frac{1}{m^2}\right) = R Z^2 \left(\frac{1}{n^2} - \frac{1}{m^2} \right),
\end{equation}
где $R$~--- постоянная Ридберга $(10973731~\text{м}^{-1})$.

Из-за того, что энергетических уровней в атоме лишь счетное число, то и возможные длины частот поглощения и излучения принимают только счетное количество значений.
