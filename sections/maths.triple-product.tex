\subsection{Смешанное произведение}
Векторное произведение определяет вектор площади параллелограмма, построенного на двух векторах, а скалярное произведение~--- величину проекции одного вектора на другой. Рассмотрим такую операцию:
\begin{equation}
	\triple{a}{b}{c} = \scalar{a}{\cross{b}{c}}.
\end{equation}

Разберем, что является результатом данной операции: $\cross{b}{c} = \vec{n}$~--- вектор нормали к плоскости векторов $\vec{b}$ и $\vec{c}$ такой, что $|\vec{n}| = |b||c| \sin \widehat{\vec{b}\vec{c}} \equiv S$~--- площадь параллелограмма.

Идем дальше, $\scalar{a}{n} = |a||n|\cos \widehat{\vec{a} \vec{n}}$~--- произведение длины вектора $\vec{n}$ на длину проекции $\pr_{\vec{n}} \vec{a}$. Значит величина смешанного произведения есть объем параллелепипеда, построенного на них.

В матричной форме смешанное произведение можно записать, как
\begin{equation}
	\triple{a}{b}{c} = \det
	\begin{pmatrix}
		\vec{a}^{\T}\\
		\vec{b}^{\T}\\
		\vec{c}^{\T}
	\end{pmatrix} =
	\begin{vmatrix}
		a_1 & a_2 & a_3\\
		b_1 & b_2 & b_3\\
		c_1 & c_2 & c_3
	\end{vmatrix},
\end{equation}
то есть определитель матрицы $n \times n$~--- ориентированные объем $n$-мерного параллелепипеда, построенного на $n$ векторах.

Практическое значение смешанного произведения основано на его свойстве: если проекция вектора $\vec{a}$ на вектор $\vec{n}$~--- $\pr_\vec{n} \vec{a} = 0$, значит, векторы $\vec{a}$, $\vec{b}$ и $\vec{c}$ лежат в одной плоскости, либо хотя бы один из них нулевой, что также означает, что эти три вектора лежат в одной плоскости.

Отсюда можно сделать вывод, что равенство нулю смешанного произведения трех векторов означает из компланарность или, что тоже самое, равенство нулю объема параллелепипеда, построенного на них.
