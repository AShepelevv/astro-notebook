
\begin{table}[h!]
\footnotesize
	\renewcommand{\arraystretch}{1.5}
\renewcommand{\tabcolsep}{2pt}
\centering
\begin{tabularx}{\tw}{|C{.13}|C{.125}|C{.14}|C{.125}|C{.115}|C{.15}|C{.13}|}
\hline 
\quad\quad\quad Объект &  Большая полуось  $\mathbf{a}$,~а.\,е. & Сиде\-ри\-чес\-кий пе\-риод $\mathbf{T}$,~год & Эксцен\-триситет $\mathbf{e}$ & Накло\-нение $\mathbf{i}$,~$~^\circ$ & Долгота восходящего узла $\mathbf{\Omega}$,~$~^\circ$ & Аргумент перицентра $\boldsymbol{\omega} $, $~^\circ$\\
\hline 
Меркурий & 0.387 & 0.241 & 0.206 & 7.00  & 48.3  & 29.1\\ 
 
Венера	 & 0.723 & 0.615 & 0.007 & 3.39  & 76.7  & 54.9\\

Земля    & 1.000 & 1.000 & 0.017 & 0.00    & 348.7 & 114.2\\

Марс     & 1.524 & 1.881  & 0.093 & 1.85  & 49.6  & 286.5\\

Церера   & 2.765 & 4.601  & 0.079 & 10.6 & 80.4  & 2.83\\

Юпитер   & 5.204 & 11.86 & 0.048 & 1.31  & 100.6 & 275.1\\

Сатурн   & 9.582 & 29.46 & 0.056 & 2.49  & 113.6 & 336.0\\

Уран     & 19.23 & 84.01 & 0.044 & 0.77  & 74.0  & 96.5\\

Нептун   & 30.06  & 164.8 & 0.011 & 1.77  & 131.8 & 265.6\\

Плутон   & 39.48 & 247.9 & 0.249 & 17.1 & 110.2 & 113.8\\ 
 \hline
\end{tabularx}
\caption{Параметры орбит больших тел Солнечной системы}
\end{table}
\begin{table}[h!]
\footnotesize
	\renewcommand{\arraystretch}{1.5}
\renewcommand{\tabcolsep}{2pt}
\centering
\begin{tabularx}{\tw}{|C{0.14}|C{0.16}|C{0.17}|C{0.17}|C{0.17}|C{0.2}|}
\hline  
 Объект & Символ & Масса $\mathbf M$,~кг & Радиус $\mathbf R$,~м & Период $\mathbf T$,~ч & Наклон оси $\mathbf i$,~$~^\circ$\\
\hline 
Солнце & $\odot$ & $1.99 \times 10^{30}$ & $6.97 \times 10^8$ & 609.1 &7.25\\

Меркурий & $\mercury$ & $3.33 \times 10^{23}$ & $2.44 \times 10^6$ & 1408. & 0.035\\ 

Венера   &  $\venus$  & $4.87 \times 10^{24}$ & $6.05 \times 10^6$ & 5833. & 177.4\\

Земля    & $\oplus$   & $5.97 \times 10^{24}$ & $6.37 \times 10^6$ & 23.93 & 23.44 \\
Луна	&	$\rightmoon$ & $7.35 \times 10^{22}$ & $1.74 \times 10^6$ &  655.7 & 1.54\\ 

Марс    & $\mars$ & $6.42 \times 10^{23} $ & $3.39 \times 10^6 $  & 24.62 & 25.19 \\

Церера  &  & $9.39 \times 10^{20}$ & $4.63 \times 10^{5}$  & 9.077 & 3 \\
 
Юпитер   &$\jupiter$ & $1.90 \times 10^{27}$ & $7.00 \times 10^{7}$ & 9.925 & 3.13 \\
 
Сатурн   &$\saturn$& $5.68 \times 10^{26}$ & $5.82 \times 10^{7}$ & 10.53 & 26.73  \\

Уран    & $\uranus$& $8.68 \times 10^{25}$ & $2.54 \times 10^7$ & 17.24 & 97.77  \\

Нептун   &$\neptune$& $1.02 \times 10^{26}$  & $2.46 \times 10^7$ & 15.97 & 28.32  \\

Плутон   &$\pluto$& $1.30 \times 10^{22}$ & $1.19 \times 10^6 $ & 153.3 & 119.6 \\ 
\hline 
\end{tabularx}
\caption{Физические характеристики больших тел Солнечной системы}
\end{table}
Светимость Солнца $L_\odot$ \hfill $3.828 \times 10^{26}$~Вт\\
Видимая звёздная величина Солнца $m_\odot$ \hfill $-26.74^m$\\
Абсолютная звёздная величина Солнца $M_\odot$ \hfill $+4.83^m$\\
Показатель цвета Солнца $(B - V)_\odot$ \hfill $+0.67^m$\\
Эффективная температура Солнца $T_\odot$ \hfill $5778$~К\\
Большая полуось орбиты Луны $a_{\moon}$ \hfill $384399$~км\\
Эксцентриситет орбиты Луны $e_{\moon}$ \hfill $0.055$\\
Наклонение плоскости орбиты Луны к эклиптике $i_{\moon}$ \hfill $5.15^\circ$\\
Сидерический период Луны $T_{\moon}$ \hfill $27.3217$~сут\\
Геометрическое альбедо Луны $A_{\moon}$ \hfill $0.12$\\
Видимая звёздная величина Луны в полнолунии $m_{\odot}$ \hfill $-12.7^m$\\
Геометрическое Альбедо Земли $A_\oplus$ \hfill $0.37$\\[-5pt]
\rule{\tw}{.7pt}\\
Заряд электрона $e$ \hfill $-1.6 \times 10^{-19}$~Кл\\
Постоянная Планка $h$ \hfill $6.626 \times 10^{-34}~\text{Дж}\cdot\text{с}$\\
Постоянная Стефана-Больцмана $\sigma$ \hfill $5.670 \times 10^{-8}~\text{Вт} \cdot \text{м}^{-2} \cdot \text{К}^{-4}$
Гравитационная постоянная $G$ \hfill $6.672 \times 10^{-11}~\text{м}^3 \cdot \text{с}^{-2} \cdot \text{кг}^{-1}$\\
Постоянная Больцмана $k$ \hfill $1.381 \times 10^{-23}~\text{Дж} \cdot \text{К}^{-1}$\\
Постоянная Хаббла $H$ \hfill $67~\text{км} \cdot \text{с}^{-1} \cdot \text{Мпк}^{-1}$


























