\subsection{Закон Стефана-Больцмана}
Светимость абсолютно чёрного тела
	\begin{equation}
	L = S \sigma T^4,
	\label{eq:steff-bol-law}
\end{equation}
 где $S$~--- площадь поверхности тела, $T$~--- температура тела, $\sigma$~--- постоянная Стефана-Больцмана.
  
Важно отметить, что \imp{закон Стефана-Больцмана}~--- прямое следствие формулы Планка \eqref{Planck's formula}, т.\,к.\begin{equation}
	\sigma T^4 = \int\limits^\infty_0 B(\lambda, T)\,d\lambda,
\end{equation}
отсюда $\sigma = (2\pi^5k^4)/(15c^2h^3) = 5.67 \cdot 10^{-8}~\text{Вт}/(\text{м}^2\cdot \text{К}^4)$.


Для тел сферической формы формула~\eqref{eq:steff-bol-law} принимает следующий вид:
\begin{equation}
L=4\pi R^2\sigma T^4
\end{equation}
Для звёзд главной последовательности в окрестности Солнца выполняются пропорциональности
\begin{equation}
L\sim M^{3.9}
\end{equation}
\begin{equation}
L\sim R^{5.2}
\end{equation}