\subsection{Закон Стефана-Больцмана}
Светимость абсолютно чёрного тела, где $S$ --- площадь поверхности тела, $T$ --- температура тела, $\sigma=5.67\cdot 10^{-8} \text{Вт}/(\text{м}^2\cdot \text{К}^4)$ --- постоянная Стефана-Больцмана, которая определяется как $\sigma=(2\pi^5k^4)/(15c^2h^3)$
\begin{equation}
L=S\sigma T^4
\end{equation}
Для тел сферической формы формула принимает следующий вид:
\begin{equation}
L=4\pi R^2\sigma T^4
\end{equation}
Полезные соотношения для звёзд главной последовательности:
\begin{equation}
L\sim M^{3.9}
\end{equation}
\begin{equation}
L\sim R^{5.2}
\end{equation}
Для нормальных звёзд и гигантов:
\begin{flushleft}
Время жизни звезды
\end{flushleft}
\begin{equation}
t\sim M^{-3}
\end{equation}
\paragraph{Зависимость жизни звезды от массы:}
\begin{flushleft}
\textbf{Возможные значения массы звезды при рождении:}
\end{flushleft}
\begin{equation}
0.08M_{sun}<M<150M_{sun}
\end{equation}
Сверхновая
\begin{equation}
M>8M_{sun}
\end{equation}
Белый карлик
\begin{equation}
M<8M_{sun}
\end{equation}
\textbf{На конечных стадиях:}
\begin{flushleft}
Белый карлик (предел Чандрасекара)
\end{flushleft}
\begin{equation}
M<1.44M_{sun}
\end{equation}
Нейтронная звезда
\begin{equation}
1.44M_{sun}<M<2.5M_{sun}
\end{equation}
