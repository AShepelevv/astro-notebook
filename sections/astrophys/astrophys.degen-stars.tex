\subsection{Вырожденные звёзды}
\term{Вырожденные звезды}~--- звезды, в которых силам гравитации противостоят силы давление вырожденного газа. К ним относятся \imp{белые карлики} и \imp{нейтронные звезды}. 

\term{Белые карлики}~--- проэволюционировавшие звёзды лишённые собственных источников термоядерной энергии. 

Масса белого карлика находится в диапазоне от $0.6M_{\odot}$ до $1.44 M_{\odot}$. Верхняя границы массы белого карлика называется пределом Чандрасекара, звезда с массой больше данного предела не может существовать как белый карлик. Радиус белых карликов примерно в $10^2$ раз меньше солнечного, т.е. можно считать, что $R_\text{БК} \simeq R_\oplus$. Плотность белых карликов лежит в диапазоне $10^7$~---~$10^{10}$~$\text{кг}/\text{м}^3$.

\term{Нейтронная звезда}~--- сверхплотная звезда, образующаяся в результате взрыва Сверхновой. Вещество нейтронной звезды состоит в основном из нейтронов. 

Масса нейтронной звезды лежит в пределах от $1.44M_{\odot}$ до $2.5M_{\odot}$ (предел Оппенгеймера-Волкова). Размер данной звезды составляет лишь $10$~--- $20$~км, а плотность составляет $10^{16}$~--- $10^{18}$ $\text{кг}/\text{м}^3$.  Дальнейшему гравитационному сжатию нейтронной звезды препятствует давление ядерной материи, возникающее за счёт взаимодействия нейтронов. Так как нейтронные звёзды образуются в результате  коллапса массивных звёзд, то из-за сохранения момента импульса скорость их вращения очень велика --- максимальная скорость может достигать $10^5$~км/с.