\subsection{Эддингтоновский предел}
\term{Предел Эддингтона (эддингтоновский предел)}~--- величина мощности электромагнитного излучения, исходящего из недр звезды, при которой его давления достаточно для компенсации веса оболочек звезды, которые окружают зону термоядерных реакций, то есть звезда находится в состоянии равновесия: не сжимается и не расширяется.\\
Сила тяжести $F_g$, действующая со стороны тела массы $M$ на протон, находящийся на расстоянии $r$ от источника, равна:
\begin{equation}
F_g = \frac{G M m_p}{r^2}
\end{equation}
$m_p$~--- масса протона.
Поток излучения $I$ на этом расстоянии:
\begin{equation}
I=\frac{L}{4\pi r^2}
\end{equation}
$L$~--- светимость источника.
Тогда сила $F_r$ действующая на электрон вследствие томсоновского рассеяния фотонов на электронах, равна:
\begin{equation}
F_r = \frac{I \sigma_T}{c}
\end{equation}
$\sigma_T$~--- томсоновское сечение рассеяния фотона на электроне:
\begin{equation}
\sigma_T = \left(\frac{8\pi}{3}\right)\left(\frac{e^2}{m_e c^2}\right)^2 = 6.65 \times 10^{-29} \text{ м}^2
\end{equation}
Таким образом, так как $F_g=F_r$, то:
\begin{equation}
L_{edd} = \frac{4\pi G M m_p c}{\sigma_T}
\end{equation}
\begin{equation}
L_{edd} = 10^{31} \frac{M}{M_\odot} \text{ Вт}
\end{equation}