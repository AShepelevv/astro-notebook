\subsection{МКТ}
\term{Молекулярно кинетическая теория}~--- теория, изучающая процессы, происходящие в макроскопических телах, на основе предположения о том, что вещество состоит из атомов и молеул, движение которых подчиняется законам механики Ньютона.\\
Рассмотрим некоторые следствия из этой теории, которые связывают давление $P$, объем $V$ и температуру $T$ идеального газа.\\
Уравнение Менделеева-Клайперона:
\begin{equation}
PV = \nu R T 
\end{equation}
$\nu$~--- количество газа в молях, $R$~--- универсальная газовая постоянная.
\begin{equation}
P = nkT =  \frac{\rho}{\mu}RT
\end{equation}
$n$~--- концентрация, $k$~--- постоянная Больцмана, $\rho$~--- плотность газаб $\mu$~--- молярная масса.
\begin{equation}
\langle E \rangle = \frac{3}{2}kT
\end{equation}
$\langle E \rangle$~--- средняя кинетическая энергия частиц.
\begin{equation}
\langle v^2 \rangle = \frac{3kT}{m}
\end{equation}
$v$~--- скорость движения молекул.
\begin{equation}
v_\text{нв} = \sqrt{\frac{2kT}{m}}
\end{equation}
$v_\text{нв}$~--- наиболее вероятная скорость движения молекул.\\
Первое начало термодинамики:
\begin{equation}
Q = \Delta U + A
\end{equation}
$Q$~--- подведенная теплота, $\Delta U$~--- изменение внутренней энергии, $A$~--- работа.\\
\begin{equation}
U = \frac{i}{2}RT
\end{equation}
$i$~--- количество степеней свободы.\\
Изотерма:
\begin{equation}
PV = const
\end{equation}
Изохора:
\begin{equation}
\frac{P}{T} = const
\end{equation}
\begin{equation}
C_v = \frac{i}{2}R
\end{equation}
$C_v$~--- теплоемкость при постоянном объеме.\\
Изобара:
\begin{equation}
\frac{V}{T} = const
\end{equation}
\begin{equation}
C_p = \frac{i+2}{2}R
\end{equation}
$C_p$~--- теплоемкость при постоянном давлении.\\
Адиабата:
\begin{equation}
Q=0
\end{equation}
\begin{equation}
P V^\alpha = const
\end{equation}
\begin{equation}
\alpha = \frac{C_p}{C_v}
\end{equation}