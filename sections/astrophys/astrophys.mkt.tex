\subsection{МКТ и термодинамика}
\term{Молекулярно-кинетическая теория} изучает процессы, происходящие в макроскопических телах, на основе предположения о том, что вещество состоит из частиц, движение которых подчиняется законам механики Ньютона.

\term{Уравнение Менделеева-Клайперона} является следствием этой теории и~связывает давление $(P)$, объем $(V)$ и температуру $(T)$ идеального газа:
\begin{equation}
PV = \nu R T,
\end{equation}
здесь $R = 8.31~\text{м}^2 \cdot \text{кг} \cdot \text{с}^{-2} \cdot \text{К}^{-1} \cdot \text{моль}^{-1}$~--- \imp{универсальная газовая постоянная}, а $\nu$~--- количество вещества идеального газа. Иначе, можно записать 
\begin{equation}
P = nkT =  \frac{\rho}{\mu}RT,
\end{equation}
где $n$~--- концентрация, $k$~--- постоянная Больцмана, $\rho$~--- плотность газа, $\mu$~--- молярная масса.

Частицы газа имеют равномерное распределение в предоставленном пространстве и распределение Максвелла по энергиям, то есть вероятность, что частица обладает энергией $E$ равна
\begin{equation}
	 f_E(E) = \frac {2\pi \sqrt{E}}{\sqrt{(\pi kT)^3}}  \exp\left(\frac {-E} {kT} \right). 
	 \label{eq:maxwell}
\end{equation} 
После некоторых математических преобразований несложно получить выражение для средней энергии $\langle E \rangle$ частиц идеального газа:
\begin{equation}
\langle E \rangle = \frac{3}{2}kT,
\end{equation}
следовательно, \term{среднеквадратичная скорость} определяется, как
\begin{equation}
\langle v^2 \rangle = \frac{3kT}{m}.
\end{equation}
Однако, исходя из распределения \eqref{eq:maxwell}, \imp{наиболее вероятная скорость} частиц 
\begin{equation}
v_\text{н.в.} = \sqrt{\frac{2kT}{m}}.
\end{equation}

\term{Первое начало термодинамики} гласит:
\begin{equation}
\delta Q = dU + \delta A,
\end{equation}
где $\delta Q$~--- подведенная теплота, $d U$~--- изменение внутренней энергии газа, $\delta A$~--- работа газа.

\imp{Внутренняя энергия} газа определяется соотношением
\begin{equation}
U = \frac{i}{2}RT,
\end{equation}
здесь $i$~--- количество степеней свободы частиц газа.

\term{Теплоёмкость идеального газа}~--- отношение количества теплоты, сообщённого газу $\Delta Q$, к изменению температуры $\Delta T$, которое при этом произошло, в расчёте на один моль:
\begin{equation}
C = \frac{\delta Q}{dT}
\end{equation} 

\term{Изотермический} процесс~--- при котором температура газа не изменяется:
\begin{equation}
PV = \const, \quad\quad C_T = +\infty,
\end{equation}
индекс теплоёмкости соответствует величине, постоянной в течение процесса.

\term{Изохорический} процесс~--- когда объём, занимаемый газом, не изменяется:
\begin{equation}
\frac{P}{T} = \const, \quad\quad C_V = \frac{i}{2}R.
\end{equation}

\term{Изобарический} процесс~--- тот, при котором давление газа не изменяется:
\begin{equation}
\frac{V}{T} = \const, \quad\quad C_P = \frac{i+2}{2}R = C_V + R.
\end{equation}

\term{Адиабатический} процесс~--- процесс, в котором количество подведенной теплоты $\delta Q$ равно нулю:
\begin{equation}
P V^\alpha = \const, \quad\quad C_Q = 0,
\end{equation}
где $\alpha = C_P/C_V$~--- показатель адиабаты.