\subsection{Альбедо}
\term{Альбедо}~($A$)~--- характеристика отражательной способности поверхности какого-либо объекта. Альбедо является отношением отражённого светового потока к падающему на поверхность объекта. Тогда для нахождения поглощённой части излучения используется соотношение
\begin{equation}
E_\text{п} = E_0 \cdot (1-A),
\end{equation}
где $E_{\text{п}}$~--- поглощённая часть излучения, $E_0$~--- пришедшее излучение, $A$~--- альбедо. А для отражённой части излучения $E_{\text{отр}}$ можно использовать формулу
\begin{equation}
	E_{\text{отр}}= A \cdot E_0.
\end{equation}

Существует несколько видов альбедо: \imp{геометрическое}, \imp{сферическое} и \imp{бондовское}. \term{Геометрическое альбедо} равно отношению освещённости у Земли, создаваемой планетой в полной фазе, к освещённости, которую создал бы плоский абсолютно белый экран того же размера, что и планета, расположенный на её месте перпендикулярно лучу зрения и солнечным лучам. \term{Сферическое альбедо} определяется как отношение светового потока, рассеянного телом во всех направлениях, к потоку, падающему на это тело. Может быть определено и для некоторого диапазона длин волн, и для всего спектра. Сферическое альбедо для всего спектра излучения называется \imp{альбедо Бонда}.