\subsection{Закон Хаббла}
\term{Закон Хаббла}~--- эмпирический закон, связывающий скорость удаления галактик и расстояние до них линейным образом. Описывается выражением \begin{equation}
V = H R,
\end{equation}
где $V$~--- скорость удаления галактики, $H=68$~км/$(\text{c} \cdot \text{Мпк})$~--- постоянная Хаббла, $R$~--- расстояние до галактики. 

При $v \ll c$ можно использовать приближение эффекта Доплера, тогда
\begin{equation}
V = c z,
\label{eq:hubble-speed}
\end{equation}
где $c$~--- скорость света, $z$~--- красное смещение.

Равенство \eqref{eq:hubble-speed} справедливо только при $z \ll 1$, а при б\'{o}льших значениях $z$ из-за эффектов СТО используется более сложная формула \begin{equation}
V = c \cdot \frac{(1 + z)^2 - 1}{(1 + z)^2 + 1}
\end{equation}