\subsection{Закон Хаббла}
\term{Закон Хаббла}~--- эмпирический закон, связывающий скорость удаления галактик $V$ и расстояние $R$ до них линейным образом: 
\begin{equation}
	V = H R,
\end{equation}
величина $H=68~\text{км/c} \cdot \text{Мпк})$ называется \imp{постоянной Хаббла}.

При $v \ll c$ можно использовать приближение эффекта Доплера, тогда
\begin{equation}
	V = c z.
\label{eq:hubble-speed}
\end{equation}

Равенство \eqref{eq:hubble-speed} справедливо только при $z \ll 1$, а при б\'{o}льших значениях $z$ космологическое красное смещение нельзя связывать с эффектом Доплера, поэтому можно пользоваться только формулой 
\begin{equation}
	\frac{dz}{dt} = - H(z)(1+z),
\end{equation}
где постоянная Хаббла введена как функция красного смещения.