\subsection{Закон Хаббла}
\term{Закон Хаббла}~--- эмпирический закон, связывающий скорость удаления галактик $V$ и расстояние $R$ до них линейным образом. 
\begin{equation}
	V = H R
\end{equation}
Величина $H=68$~км/$(\text{c} \cdot \text{Мпк})$ называется \imp{постоянная Хаббла}.

При $v \ll c$ можно использовать приближение эффекта Доплера, тогда
\begin{equation}
	V = c z.
\label{eq:hubble-speed}
\end{equation}

Равенство \eqref{eq:hubble-speed} справедливо только при $z \ll 1$, а при б\'{o}льших значениях $z$ из-за эффектов СТО необходимо использовать более сложную формулу 
\begin{equation}
	V = c \cdot \frac{(1 + z)^2 - 1}{(1 + z)^2 + 1}
\end{equation}