\subsection{Энергия и импульс фотона}
\term{Фотон}~--- материальная, электрически нейтральная частица, квант электромагнитного поля (переносчик электромагнитного взаимодействия). В силу корпускулярно-волнового дуализма фотон можно рассматривать либо как частицу, либо как волну. Фотон не имеет массы, однако обладает энергией
\begin{equation}
	E = h \nu = \frac{h c}{\lambda}
\end{equation}
и импульсом, определяемом как
\begin{equation}
	p = \frac{E}{c} = \frac{h}{\lambda},
\end{equation}
где коэффициент пропорциональности $h = 6.63 \times 10^{-34}~\text{Дж}\cdot\text{с}$ называется \imp{постоянной Планка}.