\subsection{Эффект Доплера. Красное смещение} 
\term{Эффект Доплера} --- эффект изменения частоты и длины волны электромагнитного излучения, регистрируемого приёмником, вызванное относительным движением источника и приёмника (Рис.~\ref{doppler-ef}).

\begin{figure}[h!]
	\centering
	\includegraphics[width=.5\tw]{doppler-ef}
	\caption{Эффект Доплера}
	\label{doppler-ef}
\end{figure}

При $\Delta \lambda \ll \lambda_0$ с большой точностью выполняется следующее важное соотношение:\begin{equation}
	\frac{v}{c} = \frac{\lambda - \lambda_0}{\lambda_0} \equiv \frac{\Delta \lambda}{\lambda_0},
	\label{eq:dopler-ef-simple}
\end{equation}
где $\lambda_0$~--- лабораторная длина волны излучения источника, а $\lambda$~--- наблюдаемая. 
В действительности же имеет место более общий случай: \imp{релятивистский эффект Доплера}, обусловленный проявлением СТО при $v \simeq c$, для которого формула~\eqref{eq:dopler-ef-simple} усложняется и принимает вид \begin{equation}
	\nu = \nu_0 \cdot \frac{\sqrt{1 - \frac{v^2}{c^2}}}{1 - \frac{v}{c}\cdot \cos\theta},
	\label{eq:dopler-ef-rel}
\end{equation}
где $\nu$~--- частота, с которой наблюдатель принимает волны, $\nu_0$~--- частота, с которой источник испускает волны, $v$~--- скорость источника, $\theta$~--- угол между направлением на источник и вектором скорости в системе отсчёта приёмника. Если источник радиально удаляется от наблюдателя, то $\theta = 0$, если приближается, то $\theta =\pi$. 

Стоит отметить, что~\eqref{eq:dopler-ef-simple} напрямую следует из \eqref{eq:dopler-ef-rel} при $\Delta \lambda \ll \lambda_0$.

\term{Красное смещение}~--- явление сдвига спектральных линий химических элементов в красную (длинноволновую) сторону обусловленное удалением галактик друг от друга. Параметр красного смещения определяется из наблюдаемой и лабораторной длин волн, как
\begin{equation}
	z = \frac{\lambda - \lambda_0}{\lambda_0}
\end{equation}

Доплеровское смещение длины волны в спектре источника, движущегося с лучевой скоростью $v_{r}$ и полной скоростью $v$,
\begin{equation}
z = \frac{1 + \frac{v_r}{c}}{\sqrt{1 - \left(\frac{v}{c}\right)^2}}.
\end{equation}

\term{Гравитационное красное смещение}~--- проявление эффекта изменения частоты излучения испущенного некоторым источником по мере удаления от массивных объектов, таких как звёзды и чёрные дыры. Наблюдается как сдвиг спектральных линий в излучении источников, близких к массивным телам, в красную область спектра. Гравитационное красное смещение определяется из формулы, выведенной Эйнштейном:
\begin{equation}
z_G=\frac{GM}{c^2 R}-\frac{GM}{c^2 r},
\label{eq:grav-red-shift}
\end{equation}
где $M$~--- масса гравитирующего тела, $R$~--- радиальное расстояние от центра масс тела до точки излучения (радиус источника), $r$~---  радиальное расстояние от центра масс тела до точки наблюдения. В случае, когда наблюдатель находится много дальше радиуса источника, т.\,е. выполняется соотношение $r \gg R$ формула~\eqref{eq:grav-red-shift} упрощается до \begin{equation}
	z_G \simeq \frac{GM}{c^2 R},
\end{equation}
