\subsection{Давление света}
\term{Давление электромагнитного излучения, давление света}~--- давление, которое оказывает световое (и вообще электромагнитное) излучение, падающее на поверхность тела.\\
Для вычисления давления света при нормальном падении излучения и отсутствии рассеяния можно воспользоваться формулой
\begin{equation}
p = \frac{I}{c} \cdot (1-k+\rho),
\end{equation}
где $I$~--- интенсивность падающего излучения; $c$~--- скорость света, $k$~--- коэффициент пропускания, $\rho$~--- коэффициент отражения.
Поток излучения $I$ на расстоянии $r$ от источника со светимостью $L$, определяется, как
\begin{equation}
I=\frac{L}{4\pi r^2}.
\end{equation}

\term{Давление фотонного газа} $p$ определяется соотношением
\begin{equation}
p = \frac{u}{3} = \frac{4 \sigma T^4}{3c},
\end{equation}
где $u$~--- плотность энергии фотонного газа, $T$~--- температура фотонного газа.

Возможными областями применения являются солнечный парус, а в более отдалённом будущем~--- фотонный двигатель.