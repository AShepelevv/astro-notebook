\subsection{Давление света}
\term{Давление электромагнитного излучения, давление света}~--- давление, которое оказывает световое (и вообще электромагнитное) излучение, падающее на поверхность тела.\\
Для вычисления давления света при нормальном падении излучения и отсутствии рассеяния можно воспользоваться следующей формулой:
\begin{equation}
p = \frac{I}{c}(1-k+\rho)
\end{equation}
$I$~--- интенсивность падающего излучения; $c$~--- скорость света, $k$~--- коэффициент пропускания, $\rho$~--- коэффициент отражения.
Поток излучения $I$ на расстоянии $r$ от источника:
\begin{equation}
I=\frac{L}{4\pi r^2}
\end{equation}
$L$~--- светимость источника.\\
\term{Давление фотонного газа}:
\begin{equation}
p = \frac{u}{3} = \frac{4}{3c}\sigma T^4
\end{equation}
$u$~--- плотность энергии фотонного газа, $T$~--- температура фотонного газа.
Применение:\\
Космические двигатели: возможными областями применения являются солнечный парус и разделение газов, а в более отдалённом будущем~--- фотонный двигатель,\\
Ядерная физика: в настоящее время широко обсуждается возможность ускорения световым давлением, создаваемым сверхсильными лазерными импульсами, тонких (толщиной от 5 до 10 нм) металлических плёнок с целью получения высокоэнергичных протонов.
