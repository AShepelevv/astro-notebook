\subsection{Звёздные величины}

Звёздная величина~--- безразмерная числовая характеристика яркости объекта. Известно, что увеличению светового потока в 100 раз соответствует уменьшение видимой звёздной величины ровно на 5 единиц. Тогда уменьшение звёздной величины на одну единицу означает увеличение светового потока в $\sqrt[5]{100}\approx 2.512$, то есть звёздные величины являются логарифмической шкалой измерения плотности потока. Зависимость, связывающая отношение освещённостей $E_1$ и $E_2$ и разность звёздных величин $m_1$ и $m_2$ двух объектов называется \term{формулой Погсона} и имеет вид
\begin{equation}
	\frac{E_1}{E_2} = 10^{0.4(m_2 - m_1)} \quad \Longleftrightarrow \quad m_2 = m_1 + 2.5 \lg \frac{E_1}{E_2}.
	\label{eq:Pogson-law}
\end{equation}

Широко используется понятие \term{абсолютной звёздной величины} $M$~--- это видимая звёздная величина $m$ при наблюдении с установленного расстояния: для звёзд~---~10~пк, для тел Солнечной системы~---~1~\au, причем считается, что тело находится в 1~\au~и от наблюдателя и от Солнца, а фаза равна единице, то есть можно считать, что наблюдатель находится в центре Солнца, а~тело в~1~\au~от него. 

Кроме этого, часто используется понятие \term{болометрической звёздной величины} $m_\text{bol}$~--- это звёздная величина, при расчёте которой учитывается полная мощность излучения источника во всех диапазонах электромагнитных волн. Обычная звёздная величина или видимая учитывает излучение лишь в видимой части спектра от примерно 390~нм до примерно~770~нм.

Разность между болометрической и видимой звёздными величинами называется \term{болометрической поправкой} ($BC$), своей для каждого спектральноого класса звёзд. Из определения болометрическая поправка может быть найдена по формуле
\begin{equation}
	BC = m_\text{bol} - m.
\end{equation}


Абсолютная звёздная величина звезды может быть получена по формуле Погсона \eqref{eq:Pogson-law} из видимой звёздной величины $m$ и расстояния $r$ до неё в парсеках
\begin{equation}
	M = m + 2.5 \lg \frac{E}{E_\text{абс}} = m + 2.5 \lg \frac{(10~\text{пк})^2}{r^2} = m + 5 - 5\lg r.
\end{equation}