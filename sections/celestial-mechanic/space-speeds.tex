\subsection{Первая, вторая и третья космические скорости} 

\bfseries Первая космическая скорость \mdseries --- минимальная скорость, необходимая для того, чтобы маломассивное тело стало искусственным спутником центрального тела.
\begin{equation}v_1=\sqrt{\frac{GM}{R}}
\end{equation}
Где $M$ --- масса массивного тела.

\bfseries Вторая космическая скорость \mdseries --- минимальная скорость, необходимая для того, чтобы маломассивное тело преодолело гравитационное притяжение центрального тела и покинуло замкнутую орбиту вокруг последнего. 
\begin{equation}v_2=v_p=\sqrt{2gR}=\sqrt{\frac{2GM}{R}}=\sqrt{2}v_1
\end{equation}
$v_1$ и $v_2$ на некоторых телах Солнечной системы:
\begin{table}[h!]
\centering
\begin{tabular}{|c|c|c|}
\hline
\textbf{Планета} & $\mathbf{v_1}$,\textbf{км/c} & $\mathbf{v_2}$,\textbf{км/c}\\
\hline
Солнце & 436,8 & 617,7\\
\hline
Меркурий & 3,0 & 4,3\\
\hline
Венера & 7,4 & 10,5\\
\hline
Земля & 7,9 & 11,2\\
\hline
Луна & 1,7 & 2,4\\
\hline
Марс & 3,5 & 5,0\\
\hline
Юпитер & 42,0 & 59,5\\
\hline
Сатурн & 25,1 & 35,5\\
\hline
Уран & 15,0 & 21,3\\
\hline
Нептун & 16,6 & 23,5\\
\hline
\end{tabular}
\end{table}

Скорость искусственного небесного тела на высоте $h$.\begin{equation}v_h=\sqrt{\frac{GМ}{R+h}}=\sqrt{\frac{gR^2}{R+h}}
\end{equation}

\bfseries Третья космическая скорость \mdseries --- минимальная скорость, которую необходимо придать находящемуся вблизи поверхности Земли телу, что-бы оно могло преодолеть гравитационное притяжение Земли и Солнца и покинуть пределы Солнечной системы.
\begin{equation}v_3=\sqrt{(\sqrt{2}-1)^2v^2_1+v^2_2}
\end{equation} 