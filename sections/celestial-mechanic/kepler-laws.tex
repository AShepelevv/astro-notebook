\subsection{Законы Кеплера}
{\bfseries I-ый закон:} Все планеты движутся по 
эллиптическим орбитам, в одном из фокусов которых 
находится Солнце.
\begin{figure}[h!]
\centering
\includegraphics[scale=0.2]{first-kepler}
\caption{Первый закон Кеплера \label{pic:1st-kep-law}}
\end{figure}

{\bfseries II-ой закон:} Радиус-вектор планеты за 
равные промежутки времени заметает равные площади.
\begin{equation}
\frac{dS}{dt}=const
\end{equation}
\begin{figure}[h!]
\centering
\includegraphics[scale=0.4]{second-kepler}
\caption {Второй закон Кеплера}
\end{figure}

{\bfseries 3-ий закон:} Квадраты периодов обращения планет 
относятся, как кубы больших полуосей их орбит.
\begin{equation}
\frac{T^2_1}{T^2_2}=\frac{a^3_1}{a^3_2},
\end{equation}
где $a$ --- большая полуось, $T$ --- период обращения.
Обобщённый Ньютоном III-ий закон имеет следующий вид:
\begin{equation}
\frac{T^2_1(M_1+m_1)}{T^2_2(M_2+m_2)}=\frac{a^3_1}{a^3_2}
\end{equation}
или, что эквивалентно, \begin{equation}
	\frac{T^2}{a^3}=\frac{4\pi^2}{G(M+m)},
\end{equation}
где $M_1$ и $M_2$ --- массы центральных тел, $m_1$ и 
$m_2$ --- массы обращающихся вокруг них тел. Так как массы планет 
$m$ много меньше массы звезды $M$, то $M + m \simeq M$.