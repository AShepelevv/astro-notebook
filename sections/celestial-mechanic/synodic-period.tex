\subsection{Синодический период}

{\sffamily \bfseries Синодический период} --- промежуток времени между двумя последовательными одноимёнными конфигурациями планеты или Луны.

\textit{Относительная угловая скорость $\mathit{\omega_s}$} планеты равна разности скоростей углового перемещения планеты ($360/T$) и Земли ($360/E$) по орбите. Из определения относительной угловой скорости выводится общая формула для синодического периода:

\begin{equation}
\frac1S=\left|\frac1E-\frac1T\right|
\end{equation}

Для внешних и внутренних планет соответственно формула может принимать следующий вид:
\begin{equation}\frac1S=\frac1E-\frac1T
\end{equation}

\begin{equation}\frac1S=\frac1T-\frac1E
\end{equation}

Где $S$ --- синодический период, $T$ --- сидерический период, $E$ --- сидерический период обращения Земли.

В случае, если тело обращается по орбите в протвоположную сторону, то связь между синодическим и сидерическим периодами тела выглядит следующим образом:
\begin{equation}\frac1S=\frac1E+\frac1T
\end{equation}
Синодический период планет или их спутников является периодом смены фаз. 