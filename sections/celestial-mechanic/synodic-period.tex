\subsection{Синодический период}

\textbf{Синодический период} --- промежуток времени между двумя последовательными одноимёнными конфигурациями планеты или Луны. Период смены фаз Луны равен её синодическому периоду.

$S$ --- синодический период. $T$ --- сидерический период. $E$ --- сидерический период обращения Земли.

\textit{Относительная угловая скорость} планеты или скорость углового смещения равна разности скоростей углового перемещения планет по орбите. Отсюда выводятся следующие формулы:

Для внешних планет:
\begin{equation}\frac1S=\frac1E-\frac1T
\end{equation}

Для внутренних планет:
\begin{equation}\frac1S=\frac1T-\frac1E
\end{equation}

В случае, если тело обращается по орбите в протвоположную сторону, то связь между синодическим и сидерическим периодами тела выглядит следующим образом:
\begin{equation}\frac1S=\frac1E+\frac1T
\end{equation}