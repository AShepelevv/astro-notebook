\subsection{Закон всемирного тяготения}
Согласно закону всемирного тяготения, сила притяжения 
между двумя точечными телами с массами $M$ и $m$,
находящимися на расстоянии $R$ выражается следующим
образом:\begin{equation}
	F=G\frac{Mm}{R^2}, \label{eq:grav-law}
\end{equation}
где $G\simeq 6.67\cdot 10^{-11}~\text{м}^3 / 
\left( \text{кг} \cdot \text{с}^2 \right)$ --- 
{\itshape гравитационная постоянная}.

{\itshape Гравитационный потенциал} поля точечной (или сферически 
симметричной) массы $M$ на расстоянии $R$ от нее равен
работе, которую необходимо затратить, чтобы принести
единичную массу с бесконечности в данную точку. Так как
гравитационные силы между двумя массами --- это силы 
притяжения, то эта работа отрицательна. Данная
величина также является {\itshape потенциальной энергией} точечной
массы на расстоянии $R$ от массы $M$, а выражение для нее имеет 
следующий вид:\begin{equation}
U=-\frac{GM}{r}
\end{equation}

Напряженность гравитационного поля часто называют 
{\itshape ускорением свободного падения} $g$, где\begin{equation}
	g = G \frac{M}{R^2}
\end{equation}
Тогда (\ref{eq:grav-law}) можно переписать, как \begin{equation}
	F = mg
\end{equation}
\begin{table}[h!]
\centering
\begin{tabular}{|c|c|c|c|}
\hline 
{\bfseries Планета} & $\mathbf{g}$, 
{\bfseries м/$\text{\bfseries c}^2$} 
& {\bfseries Планета} & $\mathbf{g}$, 
{\bfseries м/$\text{\bfseries c}^2$}\\
\hline
Солнце & 275.779 & Марс & 3.729\\
\hline
Меркурий & 3.734 & Юпитер & 25.93\\
\hline
Венера & 8.871 & Сатурн & 11.189\\
\hline
Земля & 9.821 & Уран & 9.009\\
\hline
Луна & 1.625 & Нептун & 11.273\\
\hline
\end{tabular}
\caption{Ускорение свободного падения на поверхности тел 
солнечной системы}
\end{table}