\subsection{Закон всемирного тяготения}
Сила притяжения между двумя телами с массами M и m, 
где $G\approx\ 6.67\cdot \cdot10^{-11}\frac{\text{Н}
\cdot \text{м}^2}{\text{кг}^2}$ --- гравитационная 
постоянная.\begin{equation}
	F=G\frac{Mm}{R^2}
\end{equation}
Потенциал точечной (или сферически симметричной) массы 
$M$ в точке $r$; он равен энергии единичной массы, 
принесенной из бесконечности в эту точку.\begin{equation}
U=-\frac{GM}{r}
\end{equation}
Ускорение свободного падения.\begin{equation}
	g = G \frac{M}{R^2}
\end{equation}
Ускорение свободного падения для тел солнечной системы:
\begin{table}[h!]
\centering
\begin{tabular}{|c|c|c|c|}
\hline 
\textbf{Планета} & $\mathbf{g}$, \textbf{м/c$~^2$} 
& \textbf{Планета} & $\mathbf{g}$, \textbf{м/c$~^2$}\\
\hline
Солнце & 274 & Марс & 3,7\\
\hline
Меркурий & 3,7 & Юпитер & 24,8\\
\hline
Венера & 8,9 & Сатурн & 10,4\\
\hline
Земля & 9,8 & Уран & 8,8\\
\hline
Луна & 1,6 & Нептун & 11,2\\
\hline
\end{tabular}
\end{table}