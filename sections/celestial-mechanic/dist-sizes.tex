\subsection{Расстояние и размеры}
\begin{equation}r=\frac{1}{\pi}
\end{equation}
Где $r$ --- расстояние до звезды, $\pi$ --- годовой параллакс звезды.
\begin{equation}r=\frac{R_{\text{З}}}{\sin p_0}=\frac{3438'}{p_0'}R_{\text{З}}=\frac{206265''}{p_0''}R_{\text{З}}
\end{equation}
Где $R_{\text{З}}$ --- радиус Земли, $p_0$ --- горизонтальный экваториальный параллакс.

\textbf{Правило Тициуса-Боде} --- эмпирическая формула приблизительно описывающая радиусы орбит планет от Солнца:
\begin{equation}r=\frac{n+4}{10}
\end{equation}
Где $n=0, 3 ,6, 12, 24, 48...$ или
\begin{equation}r=\frac{3\cdot 2^n+4}{10}
\end{equation}
Где $n=-\infty, 0, 1, 2...$
\begin{equation}R=r\frac{\rho'}{3238'}=r\frac{\rho''}{206265''}
\end{equation}
Где $R$ --- радиус объекта, $\rho$ --- угловые размеры объекта.
