\subsection{Специальная теория относительности}

\textbf{Аберрация} --- явление, состоящее в том, что движущийся наблюдатель видит светило не в том направлении, в котором он видел бы его в тот же момент, если бы находился в покое. Происходит это из-за изменения системы отсчёта для наблюдателя, причём смещается светило в сторону его движения. 
Угол аберрационного смещения можно найти по слейдующей формуле:
\begin{equation}\sigma=\frac{v}{c}\sin\theta
\end{equation}
Где $v$ --- скорость наблюдателя, $\theta$ --- угол между направлением вектора скорости наблюдателя и направлением на объект.