\subsection{Движение по эллиптической орбите}

(Численно) Для тел солнечной системы.\begin{equation}T^2_{\text{год}}=a^3_{\text{а.е.}}
\end{equation}
Средняя скорость планет Солнечной системы.\begin{equation}v_{\text{орб}}=\sqrt{\frac{GM}{a}}\approx \frac{29,8}{\sqrt{a}}
\end{equation}
Скорость в апоцентре, $e$ --- эксцентриситет.\begin{equation}v_{\text{аф}}=v_{\text{орб}}\sqrt{\frac{1-e}{1+e}}
\end{equation}
Скорость в перицентре, $e$ --- эксцентриситет.\begin{equation}v_{\text{пер}}=v_{\text{орб}}\sqrt{\frac{1+e}{1-e}}
\end{equation}
Скорость в точке орбиты, удалённой на расстояние $r$ от центрального тела, $M$ --- масса центрального тела.\begin{equation}v=\sqrt{GM\left(\frac2r - \frac1a\right)}
\end{equation}
Скорость в точке орбиты, для которой истинная аномалия $\nu$, $p$ --- фокальный параметр.\begin{equation}v=\sqrt{\frac{GM}{p}\cdot(1+2e\nu+e^2)}
\end{equation}