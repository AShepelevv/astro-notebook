\subsection{Движение по орбите}

\textit{Закон сохранения момента импульса} ---  векторная сумма всех моментов 
импульса относительно выбранной оси для замкнутой системы тел, которая остается 
постоянной, пока на систему не воздействуют внешние силы:
\begin{equation}
\vec{r} \cdot m\vec{v}=const  
\end{equation}

Закон сохранения момента импульса справедлив как для эллипса, так и для 
гиперболы и параболы. Следствием этого закона и закона сохранения энергиии 
является        \textit{ интеграл энергии} (скорость в точке орбиты, удалённой 
на расстояние $r$ от центрального тела, где $M$ --- масса центрального тела):
\begin{equation}v=\sqrt{GM\left(\frac2r - \frac1a\right)}
\end{equation}

При подстановке расстояний ($r$) апоцентра или перицентра интеграл энергии 
принимает следующий вид:
\begin{equation}v_{\text{аф}}=\sqrt{\frac{GM}{a}} \sqrt{\frac{1-e}{1+e}}
\end{equation}
\begin{equation}v_{\text{пер}}=\sqrt{\frac{GM}{a}}\sqrt{\frac{1+e}{1-e}}
\end{equation}

При использовании уравнения эллипса в полярных координатах значение скорости 
можно определить по следующей формуле, где $\nu$ --- истинная аномалия, $p$ --- 
фокальный параметр:
\begin{equation}v=\sqrt{\frac{GM}{p}\cdot(1+2e\nu+e^2)}
\end{equation}
