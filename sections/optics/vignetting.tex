\subsection{Виньетирование}
\imp{Виньетирование} (от франц. vignette~--- заставка), явление частичного ограничения (затенения) различными диафрагмами оптической системы наклонных (по отношению к оптической оси) пучков световых лучей.\footnote{{\itshape Фото и кинотехника}: <<Советская энциклопедия>>, 1981.}

В однолинзовых рефракторах перекрытия световых пучков не происходит, однако виньетирование наблюдается. Связано это с изменение эффективной площадью $S'$ собирающей поверхности в зависимости от угла $\alpha$ между центральным лучом пучка и оптической осью. Пусть площадь линзы равна $S$, тогда $S' = S\cos \alpha$. Отсюда получаем зависимость яркости изображения от линейного расстояния $x$ от оптическое оси в фокальной плоскости:
\begin{equation*}
    \frac{I}{I_0} = \cos \frac{x}{F},
\end{equation*}
где $F$~---  фокусное расстояние линзы, а $I_0$~--- яркость на оптической оси.

Однако в телескопах-рефракторах этот эффект ничтожно мал в силу малости поля зрения, а значит, и угла $\alpha$. Так как при $\alpha \ll 1$ с хорошей точностью справедливо приближение $\cos \alpha \simeq 1$. С другой стороны виньетирование сильно проявляется в широкоугольных объективах, угол $\alpha$ там может быть вплоть до $90^\circ$. Поэтому, но не только, фотографические объективы состоят из большого числа линз.

