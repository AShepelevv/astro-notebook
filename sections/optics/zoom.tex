\subsection{Параметры телескопа}
Из геометрической оптики известно, что лучи, проходящие через оптический центр линзы, не преломляются. Отсюда следует важное соотношение для линейного размера $l$ изображения объекта с угловым размером $\rho$ в фокальной плоскости:
\begin{equation}
	l = \rho F,
\end{equation}
здесь $F$~--- фокусное расстояние используемого телескопа.
Так как при помощи окуляра наблюдатель фактически смотрит на фокальную плоскость с расстояния $f$~--- фокусного расстояния окуляра, то угловой размер изображения, которое видит наблюдатель равно
\begin{equation}
	\alpha = \frac{l}{f}.
	\label{eq:zoom2}
\end{equation}
Следовательно, \term{увеличение телескопа}~$\Gamma$~--- отношение наблюдаемого и реального угловых размеров объекта равно отношению фокусных расстояний телескопа и окуляра. Из подобных треугольников также следует, что
\begin{equation}
	\Gamma =\frac{F}{f} = \frac{D}{d},
	\label{eq:zoom1}
\end{equation}
где $D$~--- диаметр входного зрачка (телескопа), $d$~--- диаметр выходного зрачка (окуляра). Важно отметить, что диаметры выходного и входного зрачка~--- диаметры пучков света, а не самих линз.

Увеличение называется \imp{равнозрачковым}, если диаметр выходного зрачка равен диаметру глазного зрачка наблюдателя, то есть
\begin{equation}
	\Gamma_\text{р.з.} = \frac{D}{d_\text{г}},
\end{equation}
где $d_\text{г}$~--- диаметр человеческого зрачка, обычно в тёмное время суток принимается за~6~мм.

Однако при росте увеличения детальность наблюдаемого изображения не улучшается. Происходит это в силу волновой природы света и, как следствие, явления \imp{дифракции} на входном отверстии телескопа. Наименьший угловой размер ещё различимых деталей определяется \term{разрешающей способностью} телескопа~--- это наименьшее угловое расстояние между двумя точечными объектами, при котором в телескоп ещё можно различить их раздельно. Предельное разрешение телескопа определяется формулой
\begin{equation}
	\beta = \frac{1.22\lambda}{D},
\end{equation}
где $\lambda$ --- длина волны наблюдений, при визуальных наблюдениях $\lambda \approx 550$~нм.

Кроме того, чем больше увеличение, тем меньше \term{поле зрения}~--- множество направлений, доступных для наблюдения. Из соотношений \eqref{eq:zoom2} и \eqref{eq:zoom1} следует зависимость \imp{поле зрения телескопа} $\alpha_\text{т}$ от \imp{поля зрения окуляра} $\alpha_\text{ок}$:
\begin{equation}
	\alpha_\text{т} = \frac{\alpha_\text{ок}}{\text{Г}},
\end{equation}
поле зрения стандартного окуляра составляет $45^\circ$.

Также, поле зрения телескопа можно вычислить, зная время $\tau$, за которое звезда со склонением $\delta$ пересекает поле зрения через его центр:
\begin{equation}
	\alpha_\text{т} = \frac{\tau \cos\delta}{4}.
\end{equation}

\term{Масштаб}~--- отношение углового размера объекта к линейному размеру его изображения на фокальной плоскости, следовательно
\begin{equation}
	\mu = \frac{\rho}{l} = \frac{\rho}{\rho F} = \frac{1}{F}=\left[\frac{\text{рад}}{\text{м}}\right].
\end{equation}

\term{Относительное отверстие}~--- геометрический параметр телескопа, равный отношению диаметра телескопа к его фокусному расстоянию
\begin{equation}
	\forall=\frac{D}{F}.
\end{equation}

\term{Светосила}~--- отношение освещенностей входного отверстия и фокальной плоскости, равна квадрату относительного отверстия.
\begin{equation}
	A=\forall^2=\frac{D^2}{F^2}.
\end{equation}

Пожалуй, самой важной характеристикой телескопа является то, насколько слабые объекты можно зафиксировать с его помощью. Эта величина называется \term{проницающей способностью телескопа}~--- это предельная звёздная величина объектов, которые доступны для наблюдения в данный телескоп. Чаще всего проницающая способность определяется из формулы Погсона \eqref{eq:pogson-law} в ходе сравнения телескопа с глазом ($m_\text{г} \simeq 6^m$), однако здесь нужно учесть:
\begin{enumerate}
	\item Отношение площадей собирающих поверхностей.
	\item Диаметр выходного пучка может быть больше размера приёмника, тогда часть света теряется.
	\item Отношение времён экспозиций (выдержка глаза составляет около 0.3~сек).
	\item Отношение квантовых эффективностей\footnote{\term{Квантовая эффективность}~--- отношение мощности регистрируемого излучения к мощности подающего для единицы площади приемника.}.
	\item Прочие эффекты, возникающие в ходе фотографических наблюдений.
\end{enumerate}
При учёте только первых двух пунктов проницающая способность равна
\begin{equation}
	m_\text{т} = m_\text{г} + 5\lg\frac{D}{d_\text{г}} - 5\lg\max \left\{ \frac{d}{d_\text{г}}, 1\right\}.
\end{equation}
