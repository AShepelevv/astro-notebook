\subsection{Энергия}
По аналогии с импульсом малое изменение энергии $dE$ есть элементарная работа силы $F$ на перемещении $dx$. Тогда расписывая силу $F$ через ускорение в лабораторной СО:
\begin{equation*}
	dE_K = F dx = F v dt = m a' v dt = m a \gamma^{3} v dt = m \gamma^{3} v dv.
\end{equation*}
Интегрируя от 0 до $v$ можно найти выражение для кинетической энергии тела:
\begin{equation*}
	E_K = \int\limits_0^{E} dE_K = \int\limits_0^{v} m \gamma^{3}v \,dv= \int\limits_0^{v} \frac{m v \, dv}{\left(1 - \frac{v^2}{c^2}\right)^{3/2}}.
\end{equation*}
Сделав замену переменной аналогично \eqref{eq:sr-imp-trigsub} можно получить:
\begin{equation*}
	\int \frac{m c^2 \sin x \cdot \cos x}{\cos ^{3} x} d x=m c^2 \int \frac{\sin x}{\cos ^2 x} d x.
\end{equation*}
Последний интеграл после замены $t = \cos x$ сводится к:
\begin{equation*}
	\frac{m c^2}{\cos \alpha}=\frac{m c^2}{\sqrt{1-v^2 / c^2}}.
\end{equation*}
Далее по формуле Ньютона-Лейбница:
\begin{equation}
E_K=\left.\frac{m c^2}{\sqrt{1-v^2 / c^2}}\right|_0 ^v=\frac{m c^2}{\sqrt{1-v^2 / c^2}}-m c^2=(\gamma-1) m c^2
\end{equation}
Помимо кинетической энергии, тело обладает также и энергией покоя~--- той, которой оно обладает независимо от скорости движения, эта энергия зависит только от его массы. Её значение определяется формулой:
\begin{equation}
	E_0 = m c^2.
\end{equation}
Полная энергия тела состоит из его энергии покоя и кинетической:
\begin{equation}
	E=E_0+E_K=\gamma m c^2.
\end{equation}
Фотоны, кванты света~--- безмассовые частицы, следовательно их энергия покоя равна нулю, вся их энергия состоит из кинетической. Но прямой подстановкой значений $m$ и $\gamma$ в формулу для кинетической энергии мы не можем ее получить, так как $m=0$, а $\gamma \rightarrow \infty$ при $v \rightarrow c$. Энергия фотона в действительности зависит от его частоты и равна $E=h \nu$, где $h$~--- коэффициент пропорциональности, называемый постоянной Планка.