\subsection{Аксиомы}
СТО, как и всякая другая теория строится на определенных постулатах~--- утверждениях, принимаемых без доказательства:
\begin{enumerate}
	\item Пространство однородно: его свойства и законы физики не зависят от точки в пространстве;
	\item Пространство изотропно: в нем нет какого-либо выделенного направления;
	\item Время однородно: все моменты времени равноправны, законы физики не меняются со временем;
	\item Пространство плоское~--- это означает справедливость в пространстве законов евклидовой геометрии, в частности утверждения, что сумма углов треугольника равна $180^{\circ}$;
	\item Все инерциальные системы отсчета равноправны~--- законы физики одинаковы во всех ИСО;
	\item Скорость света постоянна и не зависит от скорости движения источника и приемника света~--- принцип инвариантности скорости света.
\end{enumerate}
Из первых трех приведенных выше утверждений следует, что при проведении любого эксперимента мы можем перенести экспериментальную установку в любое другое место, развернуть ее в произвольном направлении или провести эксперимент в другое время, при этом результаты эксперимента не должны измениться.