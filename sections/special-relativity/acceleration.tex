\subsection{Ускорение}
Рассмотрим релятивистски равноускоренное движение~--- такое, при котором ускорение тела в мгновенно сопутствующей ему системе отсчета остается постоянным. Это происходит, если на тело действует постоянная (по модулю и направлению) сила $F$. Пусть масса тела при этом $m$. Рассмотрим малое изменение скорости тела $dv'$ в мгновенно сопутствующей ему СО. Тогда по определению
\begin{equation*}
	a' 
	   = \frac{F}{m} 
	   = \frac{dv'}{dt'} 
	   = \text{const}.
\end{equation*}
По закону сложения скоростей в лабораторной СО скорость тела изменится на $dv$:
\begin{equation*}
	dv = \frac{v+dv'}{1+\frac{v dv'}{c^2}} - v = \frac{v+d v^{\prime}-v-d v^{\prime}\left(\frac{v^2}{c^2}\right)}{1+\frac{v d v^{\prime}}{c^2}}=d v^{\prime}\left(1-v^2 / c^2\right).
\end{equation*}
При этом из эффекта замедления времени
\begin{equation*}
	dt = \frac{dt'}{\sqrt{1-v^2/c^2}}.
\end{equation*}
Отсюда ускорение $a$ в лабораторной СО соотностся с ускорением в сопутствующей СО как
\begin{equation}
	a = \frac{dv}{dt} = a' \gamma^{-3}.
\end{equation}