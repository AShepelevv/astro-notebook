\subsection{Специальная теория относительности. Аберрация}

Обычно в СТО рассматриваются две инерциальные системы $S$ и $S'$. Время и координаты некоторого события, измеренные относительно системы $S$, обозначаются как $(t, x, y, z)$, а координаты и время этого же события, измеренные относительно системы $S'$, как $(t', x', y', z')$. Удобно считать, что координатные оси систем параллельны друг другу, и система $S'$ движется вдоль оси $x$ системы $S$ со скоростью $v$. Одной из задач СТО является поиск соотношений, связывающих $(t', x', y', z')$ и $(t, x, y, z)$, которые называются \textit{преобразованиями Лоренца}. Общий вид \term{преобразований Лоренца} в векторном виде, когда относительная скорость систем отсчёта имеет произвольное направление:
\begin{equation}
	t'=\gamma\cdot \left(t-\frac{(\vec{r},\vec{v})}{c^2}\right),
\end{equation}
\begin{equation}
	\vec{r'} = \vec{r} - \gamma \vec{v} t + (\gamma - 1) \cdot \frac{(\vec{r},\vec{v})\vec{v}}{v^2},
\end{equation}
где  $\gamma = \dfrac{1}{\sqrt{1 - \frac{v^2}{c^2}}} \equiv \dfrac{1}{\sqrt{1-\beta^2}}$~--- Лоренц фактор, являющийся коэффициентом масштабирования; $\vec{r}$ и $\vec{r'}$ --- радиус-векторы события относительно систем отсчёта $S$ и $S'$.

Если сориентировать координатные оси по направлению относительного движения инерциальных систем и выбрать это направление в качестве оси $x$, то преобразования Лоренца примут вид:
\begin{equation}
	t'=\gamma \left(t - \frac{v x}{c^2} \right),\quad\quad x'= \gamma \left( x-vt \right), \quad\quad y'=y,\quad\quad z'=z.
\end{equation}

При скоростях много меньше скорости света $(v\ll c)$ преобразования Лоренца переходят в \textit{преобразования Галилея}:
\begin{equation}
	t'=t,\quad\quad x'=x-vt,\quad\quad  y'=y,\quad\quad  z'=z.
\end{equation}

\term{Аберрация} --- явление, состоящее в том, что движущийся наблюдатель видит светило не в том направлении, в котором он видел бы его в тот же момент, если бы находился в покое. Причём светило  смещается в сторону движения наблюдателя. Это происходит из-за конечности скорости света и из-за изменения системы отсчёта для наблюдателя.
Угол аберрационного смещения можно найти по следующей формуле
\begin{equation}\sigma=\frac{v}{c}\sin\theta,
\end{equation}
где $v$ --- скорость наблюдателя, $\theta$ --- угол между направлением вектора скорости наблюдателя и направлением на объект.
