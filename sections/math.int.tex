\subsection{Интеграл}
\term{Неопределенным интегралом} функции $f(x)$ называется такая функция $F(x)$, производная которой равна $f(x)$.
\begin{equation}
	F(x) = \int f(x)dx,\quad F^\prime(x)=f(x).
\end{equation}
\term{Определенный интеграл} характеризуется верхним и нижним пределом интегрирования. Значение определенного интеграла численно равно площади под графиком функции на данном промежутке.
\begin{equation}
	\int\limits^b_a f(x)\,dx = F(x) \biggr|^b_a = F(b) - F(a)
\end{equation}
Правила интегрирования:
\begin{align*}
	\int c f(x) \,dx &= c \int f(x) \,dx;  &&&&&\int f(ax + b) \,dx &= \dfrac{1}{a}F(ax + b) + C;\\
	\int f \,dg &= fg - \int g \,df; &&&&& \int \bigl[f(x) + g(x)\bigr] \,dx &= \int f(x) \,dx + \int g(x) \,dx;
\end{align*}
Таблица интегралов:
\begin{align*}
	\int  x^a \,dx &= \dfrac{x^{a+1}}{a+1} + C,\quad a \neq -1;\quad &
	\int \dfrac{dx}{\sqrt{a^2 - x^2}} &= \arcsin\dfrac{x}{a} + C;\\
	\int \frac{dx}{x} &= \ln x + C; &
	\int \dfrac{dx}{-\sqrt{a^2 - x^2}} &= \arccos\dfrac{x}{a} + C;\\
	\int a^x \,dx &= \dfrac{a^x}{\ln a} + C; &
	\int \dfrac{dx}{x^2 + a^2} &= \dfrac{1}{a} \arctg \dfrac{x}{a} + C; \\
	\int \cos x \,dx &= \sin x + C; &
	\int \dfrac{dx}{x^2 - a^2} &= \dfrac{1}{2a} \ln \dfrac{|x - a|}{|x + a|} + C;\\
	\int \sin x \,dx &= -\cos x + C; &
	\int \dfrac{dx}{\sqrt{x^2 + a}} &= \ln \left| x + \sqrt{x^2 + a} \right| + C.
\end{align*}
