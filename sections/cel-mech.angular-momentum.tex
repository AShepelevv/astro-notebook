\subsection{Закон сохранения момента импульса}
\change{
\term{Момент импульса} ($\vec{L}$)~--- векторная величина, определяющая момент количества движение (импульса) относительно выбранной оси.}

\begin{figure}[h!]
	\begin{subfigure}[b]{0.47\tw}
		\centering
		\begin{tikzpicture}[scale=1.3]
			\footnotesize
			
			\draw [-latex] (0, 0) -- (-2, 0);
			\draw [-latex] (-2, 0) -- (-3, -1);
			
			\draw [dashes, -latex] (-2, 0) -- (-2, -1);
			\draw [dashes] (-2, 0) -- (-3, 0);
			
			\draw (-2.3, 0) arc(180:225:.3);
			\draw (-1.8, 0) -- (-1.8, -.2) -- (-2, -.2);
			
			\draw (0, 0) node {$\odot$};
			
			\draw [fill=white] (-2, 0) circle(0.03);
			
			\draw (-1, 0) node[anchor=south west] {$\vec{r}$};
			\draw (-2.95, -.9) node[anchor=south] {$\vec{p}$};
			\draw (-2, -.75) node[anchor=west] {$\vec{p}_\perp$};
			\draw (-2.25, -.17) node[anchor=east] {$\alpha$};
			\draw (0, 0) node[anchor=south west] {$\vec{L}$};
			
		\end{tikzpicture}
		\caption{}
	\end{subfigure}
	\hfill
	\begin{subfigure}[b]{0.47\tw}
		\centering
		\begin{tikzpicture}[scale=1]
			\footnotesize
			
			\draw [-latex] (0, 0) -- (0, 2);
			\draw [-latex] (0, 0) -- (2.2, -.6);
			\draw [-latex] (2.2, -.6) -- (3.6, 0);
			\draw [dashes, -latex] (0, 0) -- (1.4, 0.6);
			
			%	\draw [dashes, -latex] (-2, 0) -- (-2, -1);
			\draw [dashes] (2.2, -.6) -- (3, -.82);
			\draw (0, -1) .. controls (2, -1) and (3, -.25) .. (3.5, .4);
			
			\draw (0, .2) -- (.2, 0.15) -- (0.2, -.05);
			\draw (0, .25) -- (.25, 0.35) -- (0.25, .1);
			
			\draw [fill=white] (2.2, -.6) circle (.03);
			
			%	\draw (2.55, -.7) arc(-16:22:.38);
			\draw (2.5, -.68) .. controls (2.65, -.6) and (2.65, -.5) .. (2.5, -.47);
			
			\draw (1.1, -.3) node[anchor=south] {$\vec{r}$};
			\draw (3.4, -.1) node[anchor=north] {$\vec{p}$};
			\draw (1.15, 0.55) node[anchor=south] {$\vec{p}$};
			%	\draw (-2, -.75) node[anchor=west] {$\vec{p}_\perp$};
			\draw (2.6, -.55) node[anchor=west] {$\alpha$};
			\draw (0, 1) node[anchor=east] {$\vec{L}$};
			
			%	\draw (-1.8, 0) -- (-1.8, -.2) -- (-2, -.2);
			%
			%	\draw (0, 0) node {$\odot$};
			%
			%	\draw [fill=white] (-2, 0) circle(0.03);
			%
			%	\draw (-1, 0) node[anchor=south west] {$\vec{r}$};
			%	\draw (-2.95, -.9) node[anchor=south] {$\vec{p}$};
			%	\draw (-2, -.75) node[anchor=west] {$\vec{p}_\perp$};
			%	\draw (-2.25, -.17) node[anchor=east] {$\alpha$};
			%	\draw (0, 0) node[anchor=south west] {$\vec{L}$};
			
			%	\foreach \x in {0, .1,...,5} {
			%		\draw [line width=.1pt] (\x, -3) -- (\x, 3);
			%	};
			%
			%	\foreach \y in {-3, -2.9,...,3} {
			%		\draw [line width=.1pt] (0, \y) -- (5, \y);
			%	};
			
		\end{tikzpicture}
		\caption{}
	\end{subfigure}
	\caption{}
\end{figure}

\begin{equation}
	\vec{L} = \cross{r}{p}.
\end{equation}
\change{
Здесь $\cross{\,\cdot}{\cdot\,}$~--- векторное произведение, $\vec{r}$~--- радиус вектор материальной точки с импульсом $\vec{p}$. Иными словами в терминах материальной точки, это произведение длины её радиус-вектора на ортогональную составляющую импульса этой точки. В геометрической интерпретации величина момента импульса равна площади параллелограмма, построенного на радиус-векторе материальной точки и векторе её импульса. Причем направление момента импульса определяется правилом правой руки: сначала рука идёт по радиус-векторы, потом по вектору импульса, направление большого пальца определяет направление вектора момента импульса. С помощью векторного произведения можно записать так:
\begin{equation}
	\vec{L} = \cross{r}{p}.
\end{equation}
}
\change{В скалярном виде момент импульса можно записать, как $L = r p \sin \alpha$, где $\alpha \equiv \widehat{\vec{r}\vec{p}}$~--- угол поворота от радиус вектора материальной точки к вектору ее импульса.
}

Полный момент импульса системы тел, суммарный момент $\vec{M}$ внешних сил, действующих на которую, равен нулю сохраняется. Данное утверждение носит название \term{закона сохранения момента импульса}, на математическом языке можно записать так:
\begin{equation}
	\vec{L} \equiv \sum\limits_{i=1}^{n} \left[ \vec{r}_i \times m_i \vec{v}_i \right] = \const.
\end{equation}

\change{Докажем сначала ЗСМИ в случае движения постоянной пробной массы в гравитационном поле массивного тела.
\begin{multline}
	\frac{d\vec{L}}{dt} = \frac{d\cross{r}{p}}{dt} = \left[ \frac{d \vec{r}}{dt} \times \vec{p} \right] + \left[\vec{r} \times \frac{d\vec{p}}{dt} \right] = m \underbrace{\cross{v}{v}}_0 + \left[\vec{r} \times m \frac{d\vec{v}}{dt} \right] = \\
	= m \cross{r}{a} = m \cross{r}{g(r)} = m \left[\vec{r} \times \frac{G M \vec{r}}{r^3} \right] =  \frac{G M m}{r^3} \left[\vec{r} \times \vec{r}\right] = 0.
\end{multline}
Докажем это, доказав равенство нулю производной по времени:
\begin{multline*}
	\frac{d\vec{L}}{dt} =  \sum\limits_{i=1}^n \frac{[d\vec{r}_i \times \vec{p}_i]}{dt} \overset{m_i=\const}{=} \sum\limits_{i=1}^n \left[\frac{d\vec{r}_i}{dt} \times m_i\vec{v}_i \right] + \sum\limits_{i=1}^n \left[\vec{r}_i \times m_i \frac{d\vec{v}_i}{dt} \right] = \\
	= \sum\limits_{i=1}^n \underbrace{\left[\vec{v}_i \times m_i\vec{v}_i \right]}_{=\vec{0}} + \sum\limits_{i=1}^n \left[\vec{r}_i \times m_i \vec{a}_i \right] = \\
	= \sum\limits_{i=1}^n \left[ \vec{r}_i \times m_i \sum\limits_{j=1}^n \frac{G m_j}{|\vec{r}_j - \vec{r}_i|^3} (\vec{r}_j - \vec{r}_i) \right] = \\
	= \sum\limits_{i,j=1}^n \frac{G m_i m_j}{|\vec{r}_j - \vec{r}_i|^3} \big[\vec{r}_i \times (\vec{r}_j - \vec{r}_i) \big] = \sum\limits_{i,j=1}^n \frac{G m_i m_j}{|\vec{r}_j - \vec{r}_i|^3} [\vec{r}_i \times \vec{r}_j] \equiv \sum\limits_{i,j=1}^n \vec{x}_{ij}.
\end{multline*}
Можно заметить, что $\vec{x}_{ij} = -\vec{x}_{ji}$, так как
\begin{equation*}
	\vec{x}_{ij} \equiv \frac{G m_i m_j}{|\vec{r}_j - \vec{r}_i|^3} [\vec{r}_i \times \vec{r}_j] = - \frac{G m_j m_i}{|\vec{r}_i - \vec{r}_j|^3} [\vec{r}_j \times \vec{r}_i] \equiv \vec{x}_{ji}.
\end{equation*}
Отсюда сразу следует равенство нулю производной по времени полного момента импульса, что заканчивает доказательство его сохранения.
}

