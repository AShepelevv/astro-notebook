\subsection{Закон смещения Вина}
\term{Закон смещения Вина} --- закон, устанавливающий зависимость длины волны~$\lambda_\text{макс}$, на которой спектральная плотность излучения $B_\lambda(\lambda, T)$ абсолютно чёрного тела достигает своего максимума, от температуры $T$ этого тела:
\begin{equation}
	\lambda_\text{макс} \approx \frac{b}{T} \equiv \frac{0.0029~\text{м} \cdot \text{К}}{T}.
\end{equation}
Закон является следствием исследования функции Планка (\ref{sec:planck-law}) на экстремальность. 
\change{Проведем это исследование. Зафиксируем температуру черного тела и введем обозначения: $\alpha 2 h c^2$ и $\beta = h c / k T$. Тогда задача сводится к поиску экстремума функции
	\begin{equation*}
		B(\lambda) = \frac{\alpha \lambda^{-5}}{\exp \left(\beta \lambda^{-1} \right) - 1}
	\end{equation*}
	Продифференцируем $B(\lambda)$:
	\begin{align*}
		B'(\lambda) &= \frac{-5 \alpha \lambda^{-6}}{\exp \left(\beta \lambda^{-1} \right) - 1} + \frac{-\alpha \lambda^{-5}}{\big(\exp \left(\beta \lambda^{-1} \right) - 1\big)^2} \cdot \exp \left( \beta \lambda^{-1} \right) \cdot \left(-\beta \lambda^{-2}\right) = \\
		&= \frac{\alpha \lambda^{-7}}{\big(\exp \left(\beta \lambda^{-1} \right) - 1\big)^2} \left[ -5\lambda \left( \exp \frac{\beta}{\lambda} - 1 \right) + \beta \exp \frac{\beta}{\lambda} \right]c	\end{align*}
	Остается приравнять выражения в квадратных скобках к нулю, и найти корни получившегося уравнения.
	\begin{gather*}
		\beta \exp \frac{\beta}{\lambda} - 5\lambda \left( \exp \frac{\beta}{\lambda} - 1 \right) = 0;\\
		\frac{\beta}{\lambda} \exp \frac{\beta}{\lambda} = 5 \left( \exp \frac{\beta}{\lambda} - 1 \right) = 0,~~~\text{пусть}~x = \frac{\beta}{\lambda};\\
		\frac{x e^x}{e^x - 1} = 5.
	\end{gather*}
	К сожалению решение полученного уравнения не выражается в элементарных функциях. Численные методы дают ответ $x = 4.9651...$ Возвращаясь к исходным переменным, имеем:
	\begin{equation*}
		\frac{h c}{k T \lambda} = x \approx 4.9651.
	\end{equation*}
	Выражая отсюда $\lambda$, окончательно получаем следующее:
	\begin{equation*}
		\lambda \approx \frac{h c}{4.9651 k T} = \frac{2.898... \times 10^{-3}~\text{м}\cdot\text{К}}{T} \approx  \frac{0.0029~\text{м} \cdot \text{К}}{T}.
	\end{equation*}
 }