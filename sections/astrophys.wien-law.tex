\subsection{Закон смещения Вина}
\term{Закон смещения Вина} --- закон, устанавливающий зависимость длины волны~$\lambda_\text{макс}$, на которой спектральная плотность излучения $B_\lambda(\lambda, T)$ абсолютно чёрного тела достигает своего максимума, от температуры $T$ этого тела:
\begin{equation}
	\lambda_\text{макс} \approx \frac{b}{T} \equiv \frac{0.0029~\text{м} \cdot \text{К}}{T}.
\end{equation}
Закон является следствием исследования функции Планка (\ref{sec:planck-law}) на экстремальность.