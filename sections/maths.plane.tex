\subsection{Плоскость}

Аналогично предыдущему разделу, рассмотрим условие принадлежности точки с радус-вектором $\vec{r}$ плоскости $\Pi$ в $\R^3$. Пусть неколлинеарные ($ \cross{a}{b} \not = 0$) векторы  $\vec{a}$ и $\vec{b}$~--- направляющие векторы плоскости $\Pi$, а точка $O$ с радиус-вектором $\vec r_0$ такова, что $\vec{r}_0 \in \Pi$. Тогда
\begin{equation}
	\vec{r} = \vec{r}_0 + \lambda\vec{a} + \mu\vec{b} \quad \Leftrightarrow \quad \begin{cases}
	x - x_0 = \lambda x_a + \mu x_b,\\
	y - y_0 = \lambda y_a + \mu y_b,\\
	z - z_0 = \lambda z_a + \mu z_b;
\end{cases}\quad \lambda, \mu \in \R.
\end{equation}
Преобразуем полученную систему уравнений:
\begin{align*}
& \left\{
\begin{aligned}
	\lambda &= \dfrac{x - x_0 - \mu x_b}{x_a},\\
	y - y_0 &= \dfrac{x - x_0 - \mu x_b}{x_a} \cdot y_a + \mu y_b,\\
	z - z_0 &= \lambda z_a + \mu z_b.
\end{aligned}\right.\\
& \left\{
\begin{aligned}
	\lambda &= \dfrac{x - x_0 - \mu x_b}{x_a},\\
	y - y_0 &= (x - x_0) \cdot \dfrac{y_a}{x_a} + \mu \left(y_b - \dfrac{x_b y_a}{x_a} \right),\\
	z - z_0 &= \lambda z_a + \mu z_b.
\end{aligned}\right.\\
&\left\{
\begin{aligned}
	\lambda &= \dfrac{x - x_0 - \mu x_b}{x_a},\\
	\mu &= \dfrac{x_a y - x_a y_0 - (x - x_0) \cdot y_a}{x_a y_b - x_b y_a},\\
	z - z_0 &= \lambda z_a + \mu z_b.
\end{aligned}\right.\\
\end{align*}
Подставим выражения для $\lambda$ и $\mu$ в третье уравнение:
\begin{gather*}
z - z_0 = \dfrac{x - x_0 - \mu x_b}{x_a} \cdot z_a + \mu z_b = (x - x_0) \cdot \dfrac{z_a}{x_a} + \mu \left( z_b - \dfrac{x_b z_a}{x_a} \right),\\
z - z_0 = (x - x_0) \cdot \dfrac{z_a}{x_a} + \dfrac{x_a y - x_a y_0 - (x - x_0) \cdot y_a}{x_a y_b - x_b y_a} \cdot \left( z_b - \dfrac{x_b z_a}{x_a} \right),
\end{gather*}
Приведя подобные слагаемые с $x$, $y$ и $z$, получим:
\begin{multline*}
z = x \cdot \underbrace{\left( \dfrac{z_a}{x_a} - \dfrac{y_a}{x_a y_b - x_b y_a} \left( z_b - \dfrac{x_b z_a}{x_a} \right) \right)}_A +\\
+ y \cdot \underbrace{\dfrac{x_a}{x_a y_b - x_b y_a} \cdot \left( z_b - \dfrac{x_b z_a}{x_a} \right)}_B +\\
+ \underbrace{z_0 - \dfrac{x_0 z_a}{x_a} - \dfrac{x_a y_0 - x_0 y_a}{x_a y_b - x_b y_a} \cdot \left( z_b - \dfrac{x_b z_a}{x_a} \right)}_D.
\end{multline*}
Сделаем такую замену
\begin{align*}
&
\begin{aligned}
	A \equiv  \dfrac{z_a}{x_a} &- \dfrac{y_a}{x_a y_b - x_b y_a} \left( z_b - \dfrac{x_b z_a}{x_a} \right) = \\
	&\quad\quad= \frac{x_a y_b z_a - x_b y_a z_a}{x_a (x_a y_b - x_b y_a)} - \frac{x_a y_a z_b - x_b y_a z_a}{x_a (x_a y_b - x_b y_a)} = \frac{y_b z_a - y_a z_b}{x_a y_b - x_b y_a},
\end{aligned}\\
&
B \equiv \dfrac{x_a}{x_a y_b - x_b y_a} \cdot \left( z_b - \dfrac{x_b z_a}{x_a} \right) = \frac{x_a (x_a z_b - x_b z_a)}{x_a (x_a y_b - x_b y_a} = \frac{x_a z_b - x_b z_a}{x_a y_b - x_b y_a},\\
&
D \equiv z_0 - \dfrac{x_0 z_a}{x_a} - \dfrac{x_a y_0 - x_0 y_a}{x_a y_b - x_b y_a} \cdot \left( z_b - \dfrac{x_b z_a}{x_a} \right).
\end{align*}
В результате такой замены получим уравнение
\begin{equation}
Ax + By - z + D = 0.
\end{equation}
Домножим в нём обе части на $\xi \equiv x_a y_b - x_b y_a$ и сделаем ещё одну замену:
\begin{equation*}
A'x + B'y + C'z + D' = 0,
\end{equation*}
где $D' = \xi D$, а
\begin{equation*}
\begin{cases}
	A' = y_b z_a - y_a z_b = \det
	\begin{pmatrix}
		y_b & z_b\\
		y_a & z_a
	\end{pmatrix}\\[1pc]
	B' = x_a z_b - x_b z_a = \det
	\begin{pmatrix}
		x_a & z_a\\
		x_b & z_b
	\end{pmatrix}\\[1pc]
	C' = x_b y_a - x_a y_b = \det
	\begin{pmatrix}
		x_b & y_b\\
		x_a & y_a
	\end{pmatrix}
\end{cases}
\end{equation*}
Легко видеть, что
\begin{equation}
\begin{pmatrix}
	A'\\
	B'\\
	C'
\end{pmatrix} =
\cross{b}{a} \equiv \vec{n}.
\end{equation}
Из свойств векторного произведения вектор $\vec n$ является \imp{вектором нормали} к плоскости $\Pi$.

Теперь  самое первое условие можно записать так:
\begin{equation}
\scalar{r}{n} = \triple{r}{b}{a} = -D' = (\vec{r}_0, \vec{b}, \vec{a}).
\end{equation}
