\subsection{Годичное движение Солнца}
В течение сидерического года Земля совершает полный оборот вокруг Солнца. В следствие этого Солнце движется относительно далёких звёзд для наблюдателя на Земле. Это движение совершается по большому кругу небесной сферы, называемому \term{эклиптикой} и совпадающему с плоскостью орбиты Земли. Однако, в силу прецессии земной оси с периодом около 25765~лет, период такого движения равен \imp{тропического году}, который длиннее сидерического года примерно на 20~мин~25~сек.

\begin{wrapfigure}[12]{r}{0.5\tw}
	\centering
	\vspace{-.9pc}
 	\begin{tikzpicture}
 		\begin{axis}[
 						width	=	.5\tw,
 						height	=	4.5cm,
 						xlabel	=	{Прямое восхождение $\alpha^h$}, 
 						ylabel	=	{Склонение $\delta^{\circ}$},
 						extra y ticks	=	{23.44, -23.44},
 						ytick = {-20, -10, 0, 10, 20},
 						ymax	=	25,
 						ymin	=	-25,
 						xmax	=	24,
 						xmin	=	0,
 						xtick	=	{0, 4, 8, 12, 16, 20, 24},
 						x dir = reverse
 					]
 			\addplot [domain=0:24, samples=100] {atan(sin(x*15)*tan(23.44))}; 
 		\end{axis}
 	\end{tikzpicture}
 	\caption{График зависимости склонения Солнца от его прямого восхождения}
\end{wrapfigure}
В моменты, когда Солнце находится в \imp{точке весеннего равноденствия}  (20~марта, реже~21) его координаты: $\alpha=0^h$, $\delta=0^{\circ}$. Во время прохождения этой точки обе координаты Солнца растут. Так происходит до момента, пока Солнце не пройдет \imp{точку летнего солнцестояния} (21~июня, реже~20), после этого склонение Солнца начинает уменьшаться. В момент прохождения \imp{точки осеннего равноденствия} (22~или 23~сентября), координаты Солнца составляют $\alpha=12^h$, $\delta=0^{\circ}$. После прохождения \imp{точки зимнего солнцестояния} (22~или 21~декабря) склонение Солнца начинает увеличиваться.

Пренебрегая сферическими искажениями, годичный путь Солнца по небесной сфере можно считать синусоидой, откуда 
\begin{equation}
\delta=\varepsilon\cdot\sin \frac{2 \pi t}{T},
\end{equation}
где $t$~--- время, прошедшее с момента весеннего равноденствия, $T$~--- тропический год.

Более точная формула следует из сферической тригонометрии и имеет вид
\begin{equation}
\delta=\arcsin\left(\sin\varepsilon\cdot\sin \frac{2 \pi t}{T}\right).
\label{eq:delta-sun}
\end{equation}

Известно, что движение Солнца по эклиптике происходит неравномерно, поэтому данные формулы не являются абсолютно точными.

Прямое восхождение Солнца связано со склонением формулой
\begin{equation}
\sin\alpha=\frac{\tg\delta}{\tg\varepsilon}.
\label{eq:sin-alpha}
\end{equation}

Выражения \eqref{eq:delta-sun} и \eqref{eq:sin-alpha} следуют из формул перехода между экваториальной и эклиптической системами координат, получаемых из сферической тригонометрии.
