\subsection{Закон всемирного тяготения}
Согласно \imp{закону всемирного тяготения}, сила притяжения
между двумя точечными телами с массами $M$ и $m$,
находящимися на расстоянии $r$, равна
\begin{equation}
	F=\frac{GMm}{r^2}, \label{eq:grav-law}
\end{equation}\nopagebreak где $G\simeq 6.67\cdot 10^{-11}~\text{м}^3 /
\left( \text{кг} \cdot \text{с}^2 \right)$~---
\term{гравитационная постоянная}.

\term{Гравитационный потенциал} поля точечной (или сферически
симметричной) массы $M$ на расстоянии $r$ от нее равен
работе, которую необходимо затратить, чтобы принести
единичную массу с бесконечности в данную точку. Так как
гравитационные силы между двумя массами --- это силы
притяжения, то эта работа отрицательна. Данная
величина также является \term{потенциальной энергией} точечной
массы на расстоянии $r$ от массы $M$, а выражение для нее имеет
следующий вид:
\begin{equation}
	U=-\frac{GM}{r}.
\end{equation}

Напряженность гравитационного поля $dU/dr$ часто называют
\term{ускорением свободного падения} $g$, она вычисляется по формуле
\begin{equation}
	g = \frac{GM}{r^2}.
	\label{eq:g}
\end{equation}
Тогда (\ref{eq:grav-law}) можно записать как
\begin{equation}
	F = mg.
\end{equation}
