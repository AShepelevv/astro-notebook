\documentclass[10pt,a5paper]{article}
\usepackage[utf8]{inputenc}
\usepackage[english,russian]{babel}
\usepackage[OT1]{fontenc}
\usepackage{amsmath}
\unitlength=1mm
\usepackage{amsfonts}
\usepackage{amssymb}
\usepackage[left=1.2 cm,right=1.2 cm,top=1.8 cm,bottom=1.8 cm]{geometry}
\usepackage{graphicx}
\graphicspath{img/}
\title{Астрадь}
\begin{document}

\begin{titlepage}
	
	\begin{center}
	\vspace*{5 cm}
		{\Huge \bfseries \scshape Астрадь}
	\end{center}
\end{titlepage}

\tableofcontents
\newpage

\input{Элементы небесной механики}
\input{Закон всемирного тяготения}
\input{Закон сохранения энергии и типы орбит}
\input{Законы Кеплера}
\input{Первая, вторая и третья космические скорости}
\input{Движение по эллиптической орбите}
\input{Синодический период}
\input{Конфигурации планет}
\input{Кеплеровы элементы орбиты}
\input{Фазы планет и спутников}
\input{Аберрация}
\input{Приливы и отливы}
\input{Солнечные и лунные затмения. Сарос}
\input{Прецессия}
\input{Точки Лагранжа}
\subsection{Расстояние и размеры}
$$r=\frac{1}{\pi}$$
Где $r$ --- расстояние до звезды, $\pi$ --- годовой параллакс звезды.
$$r=\frac{R_{\text{З}}}{\sin p_0}=\frac{3438'}{p_0'}R_{\text{З}}=\frac{206265''}{p_0''}R_{\text{З}}$$
Где $R_{\text{З}}$ --- радиус Земли, $p_0$ --- горизонтальный экваториальный параллакс.

\textbf{Правило Тициуса-Боде} --- эмпирическая формула приблизительно описывающая радиусы орбит планет от Солнца:
$$r=\frac{n+4}{10}$$
Где $n=0, 3 ,6, 12, 24, 48...$ или
$$r=\frac{3\cdot 2^n+4}{10}$$
Где $n=-\infty, 0, 1, 2...$
$$R=r\frac{\rho'}{3238'}=r\frac{\rho''}{206265''}$$
Где $R$ --- радиус объекта, $\rho$ --- угловые размеры объекта.

\input{Конические сечения}
\input{Эллипс}
\end{document}